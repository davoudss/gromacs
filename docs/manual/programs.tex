%
% This file is part of the GROMACS molecular simulation package.
%
% Copyright (c) 2013,2014,2015,2016, by the GROMACS development team, led by
% Mark Abraham, David van der Spoel, Berk Hess, and Erik Lindahl,
% and including many others, as listed in the AUTHORS file in the
% top-level source directory and at http://www.gromacs.org.
%
% GROMACS is free software; you can redistribute it and/or
% modify it under the terms of the GNU Lesser General Public License
% as published by the Free Software Foundation; either version 2.1
% of the License, or (at your option) any later version.
%
% GROMACS is distributed in the hope that it will be useful,
% but WITHOUT ANY WARRANTY; without even the implied warranty of
% MERCHANTABILITY or FITNESS FOR A PARTICULAR PURPOSE.  See the GNU
% Lesser General Public License for more details.
%
% You should have received a copy of the GNU Lesser General Public
% License along with GROMACS; if not, see
% http://www.gnu.org/licenses, or write to the Free Software Foundation,
% Inc., 51 Franklin Street, Fifth Floor, Boston, MA  02110-1301  USA.
%
% If you want to redistribute modifications to GROMACS, please
% consider that scientific software is very special. Version
% control is crucial - bugs must be traceable. We will be happy to
% consider code for inclusion in the official distribution, but
% derived work must not be called official GROMACS. Details are found
% in the README & COPYING files - if they are missing, get the
% official version at http://www.gromacs.org.
%
% To help us fund GROMACS development, we humbly ask that you cite
% the research papers on the package. Check out http://www.gromacs.org.

\chapter{Run parameters and Programs}
\label{ch:programs}

\section{Online documentation}
\index{online documentation}
More documentation is available online from the {\gromacs} web site,
\url{http://manual.gromacs.org/documentation}.

In addition, we install standard UNIX man pages for all the programs. If
you have sourced the {\tt GMXRC} script in the {\gromacs} binary directory for
your host they should already be present in your {\tt MANPATH} environment variable, and you
should be able to type {\eg} {\tt man gmx-grompp}. You can also use
the {\tt -h} flag on the command line (e.g. {\tt gmx grompp -h}) to
see the same information, as well as {\tt gmx help grompp}.
The list of all programs are available from {\tt gmx help}.

\section{File types\swapindexquiet{file}{type}}
\label{sec:fileformats}
\tabref{form} lists the file types used by {\gromacs} along with
a short description, and you can find a more detail description for
each file in your HTML reference, or in our online version.

{\gromacs} files written in \normindex{XDR} format can be read on any
architecture with {\gromacs} version 1.6 or later if the configuration
script found the XDR libraries on your system. They should always be
present on UNIX since they are necessary for NFS support.

\section{\-File \-List}
\-Here is a list of all files with brief descriptions\-:\begin{DoxyCompactList}
\item\contentsline{section}{cmake/\hyperlink{TestAtomics_8c}{\-Test\-Atomics.\-c} }{\pageref{TestAtomics_8c}}{}
\item\contentsline{section}{cmake/\hyperlink{TestAVX_8c}{\-Test\-A\-V\-X.\-c} }{\pageref{TestAVX_8c}}{}
\item\contentsline{section}{cmake/\hyperlink{TestAVXMaskload_8c}{\-Test\-A\-V\-X\-Maskload.\-c} }{\pageref{TestAVXMaskload_8c}}{}
\item\contentsline{section}{cmake/\hyperlink{TestCatamount_8c}{\-Test\-Catamount.\-c} }{\pageref{TestCatamount_8c}}{}
\item\contentsline{section}{cmake/\hyperlink{TestClangVersion_8c}{\-Test\-Clang\-Version.\-c} }{\pageref{TestClangVersion_8c}}{}
\item\contentsline{section}{cmake/\hyperlink{TestFFTW2_8c}{\-Test\-F\-F\-T\-W2.\-c} }{\pageref{TestFFTW2_8c}}{}
\item\contentsline{section}{cmake/\hyperlink{TestFileOffsetBits_8c}{\-Test\-File\-Offset\-Bits.\-c} }{\pageref{TestFileOffsetBits_8c}}{}
\item\contentsline{section}{cmake/\hyperlink{TestFloatFormat_8c}{\-Test\-Float\-Format.\-c} }{\pageref{TestFloatFormat_8c}}{}
\item\contentsline{section}{cmake/\hyperlink{TestInline_8c}{\-Test\-Inline.\-c} }{\pageref{TestInline_8c}}{}
\item\contentsline{section}{cmake/\hyperlink{TestInlineASM__gcc__x86_8c}{\-Test\-Inline\-A\-S\-M\-\_\-gcc\-\_\-x86.\-c} }{\pageref{TestInlineASM__gcc__x86_8c}}{}
\item\contentsline{section}{cmake/\hyperlink{TestMKL_8c}{\-Test\-M\-K\-L.\-c} }{\pageref{TestMKL_8c}}{}
\item\contentsline{section}{cmake/\hyperlink{TestMPI_8c}{\-Test\-M\-P\-I.\-c} }{\pageref{TestMPI_8c}}{}
\item\contentsline{section}{cmake/\hyperlink{TestPipes_8c}{\-Test\-Pipes.\-c} }{\pageref{TestPipes_8c}}{}
\item\contentsline{section}{cmake/\hyperlink{TestQPX_8c}{\-Test\-Q\-P\-X.\-c} }{\pageref{TestQPX_8c}}{}
\item\contentsline{section}{cmake/\hyperlink{TestRestrict_8c}{\-Test\-Restrict.\-c} }{\pageref{TestRestrict_8c}}{}
\item\contentsline{section}{cmake/\hyperlink{TestRetSigType_8c}{\-Test\-Ret\-Sig\-Type.\-c} }{\pageref{TestRetSigType_8c}}{}
\item\contentsline{section}{cmake/\hyperlink{TestSIGUSR1_8c}{\-Test\-S\-I\-G\-U\-S\-R1.\-c} }{\pageref{TestSIGUSR1_8c}}{}
\item\contentsline{section}{cmake/\hyperlink{TestVMD_8c}{\-Test\-V\-M\-D.\-c} }{\pageref{TestVMD_8c}}{}
\item\contentsline{section}{cmake/\hyperlink{TestWindowsFSeek_8c}{\-Test\-Windows\-F\-Seek.\-c} }{\pageref{TestWindowsFSeek_8c}}{}
\item\contentsline{section}{cmake/\hyperlink{TestXDR_8c}{\-Test\-X\-D\-R.\-c} }{\pageref{TestXDR_8c}}{}
\item\contentsline{section}{\-C\-Make\-Files/\hyperlink{TestLargeFiles_8c}{\-Test\-Large\-Files.\-c} }{\pageref{TestLargeFiles_8c}}{}
\item\contentsline{section}{\-C\-Make\-Files/\-Check\-Type\-Size/\hyperlink{CMAKE__SIZEOF__UNSIGNED__SHORT_8c}{\-C\-M\-A\-K\-E\-\_\-\-S\-I\-Z\-E\-O\-F\-\_\-\-U\-N\-S\-I\-G\-N\-E\-D\-\_\-\-S\-H\-O\-R\-T.\-c} }{\pageref{CMAKE__SIZEOF__UNSIGNED__SHORT_8c}}{}
\item\contentsline{section}{\-C\-Make\-Files/\-Check\-Type\-Size/\hyperlink{gid__t_8c}{gid\-\_\-t.\-c} }{\pageref{gid__t_8c}}{}
\item\contentsline{section}{\-C\-Make\-Files/\-Check\-Type\-Size/\hyperlink{off__t_8c}{off\-\_\-t.\-c} }{\pageref{off__t_8c}}{}
\item\contentsline{section}{\-C\-Make\-Files/\-Check\-Type\-Size/\hyperlink{size__t_8c}{size\-\_\-t.\-c} }{\pageref{size__t_8c}}{}
\item\contentsline{section}{\-C\-Make\-Files/\-Check\-Type\-Size/\hyperlink{SIZEOF__BOOL_8c}{\-S\-I\-Z\-E\-O\-F\-\_\-\-B\-O\-O\-L.\-c} }{\pageref{SIZEOF__BOOL_8c}}{}
\item\contentsline{section}{\-C\-Make\-Files/\-Check\-Type\-Size/\hyperlink{SIZEOF__INT_8c}{\-S\-I\-Z\-E\-O\-F\-\_\-\-I\-N\-T.\-c} }{\pageref{SIZEOF__INT_8c}}{}
\item\contentsline{section}{\-C\-Make\-Files/\-Check\-Type\-Size/\hyperlink{SIZEOF__LONG__INT_8c}{\-S\-I\-Z\-E\-O\-F\-\_\-\-L\-O\-N\-G\-\_\-\-I\-N\-T.\-c} }{\pageref{SIZEOF__LONG__INT_8c}}{}
\item\contentsline{section}{\-C\-Make\-Files/\-Check\-Type\-Size/\hyperlink{SIZEOF__LONG__LONG__INT_8c}{\-S\-I\-Z\-E\-O\-F\-\_\-\-L\-O\-N\-G\-\_\-\-L\-O\-N\-G\-\_\-\-I\-N\-T.\-c} }{\pageref{SIZEOF__LONG__LONG__INT_8c}}{}
\item\contentsline{section}{\-C\-Make\-Files/\-Check\-Type\-Size/\hyperlink{SIZEOF__OFF__T_8c}{\-S\-I\-Z\-E\-O\-F\-\_\-\-O\-F\-F\-\_\-\-T.\-c} }{\pageref{SIZEOF__OFF__T_8c}}{}
\item\contentsline{section}{\-C\-Make\-Files/\-Check\-Type\-Size/\hyperlink{SIZEOF__VOIDP_8c}{\-S\-I\-Z\-E\-O\-F\-\_\-\-V\-O\-I\-D\-P.\-c} }{\pageref{SIZEOF__VOIDP_8c}}{}
\item\contentsline{section}{\-C\-Make\-Files/\-Check\-Type\-Size/\hyperlink{uid__t_8c}{uid\-\_\-t.\-c} }{\pageref{uid__t_8c}}{}
\item\contentsline{section}{\-C\-Make\-Files/\-Compiler\-Id\-C/\hyperlink{CMakeFiles_2CompilerIdC_2CMakeCCompilerId_8c}{\-C\-Make\-C\-Compiler\-Id.\-c} }{\pageref{CMakeFiles_2CompilerIdC_2CMakeCCompilerId_8c}}{}
\item\contentsline{section}{include/\hyperlink{include_23dview_8h}{3dview.\-h} }{\pageref{include_23dview_8h}}{}
\item\contentsline{section}{include/\hyperlink{include_2assert_8h}{assert.\-h} }{\pageref{include_2assert_8h}}{}
\item\contentsline{section}{include/\hyperlink{include_2atomprop_8h}{atomprop.\-h} }{\pageref{include_2atomprop_8h}}{}
\item\contentsline{section}{include/\hyperlink{include_2bondf_8h}{bondf.\-h} }{\pageref{include_2bondf_8h}}{}
\item\contentsline{section}{include/\hyperlink{include_2calcgrid_8h}{calcgrid.\-h} }{\pageref{include_2calcgrid_8h}}{}
\item\contentsline{section}{include/\hyperlink{include_2calch_8h}{calch.\-h} }{\pageref{include_2calch_8h}}{}
\item\contentsline{section}{include/\hyperlink{include_2calcmu_8h}{calcmu.\-h} }{\pageref{include_2calcmu_8h}}{}
\item\contentsline{section}{include/\hyperlink{include_2centerofmass_8h}{centerofmass.\-h} \\*\-A\-P\-I for calculation of centers of mass/geometry }{\pageref{include_2centerofmass_8h}}{}
\item\contentsline{section}{include/\hyperlink{include_2chargegroup_8h}{chargegroup.\-h} }{\pageref{include_2chargegroup_8h}}{}
\item\contentsline{section}{include/\hyperlink{include_2checkpoint_8h}{checkpoint.\-h} }{\pageref{include_2checkpoint_8h}}{}
\item\contentsline{section}{include/\hyperlink{include_2confio_8h}{confio.\-h} }{\pageref{include_2confio_8h}}{}
\item\contentsline{section}{include/\hyperlink{include_2constr_8h}{constr.\-h} }{\pageref{include_2constr_8h}}{}
\item\contentsline{section}{include/\hyperlink{include_2copyrite_8h}{copyrite.\-h} }{\pageref{include_2copyrite_8h}}{}
\item\contentsline{section}{include/\hyperlink{include_2coulomb_8h}{coulomb.\-h} }{\pageref{include_2coulomb_8h}}{}
\item\contentsline{section}{include/\hyperlink{include_2displacement_8h}{displacement.\-h} \\*\-A\-P\-I for on-\/line calculation of displacements }{\pageref{include_2displacement_8h}}{}
\item\contentsline{section}{include/\hyperlink{include_2disre_8h}{disre.\-h} }{\pageref{include_2disre_8h}}{}
\item\contentsline{section}{include/\hyperlink{include_2do__fit_8h}{do\-\_\-fit.\-h} }{\pageref{include_2do__fit_8h}}{}
\item\contentsline{section}{include/\hyperlink{include_2domdec_8h}{domdec.\-h} }{\pageref{include_2domdec_8h}}{}
\item\contentsline{section}{include/\hyperlink{include_2domdec__network_8h}{domdec\-\_\-network.\-h} }{\pageref{include_2domdec__network_8h}}{}
\item\contentsline{section}{include/\hyperlink{include_2ebin_8h}{ebin.\-h} }{\pageref{include_2ebin_8h}}{}
\item\contentsline{section}{include/\hyperlink{include_2edsam_8h}{edsam.\-h} }{\pageref{include_2edsam_8h}}{}
\item\contentsline{section}{include/\hyperlink{include_2enxio_8h}{enxio.\-h} }{\pageref{include_2enxio_8h}}{}
\item\contentsline{section}{include/\hyperlink{include_2ffscanf_8h}{ffscanf.\-h} }{\pageref{include_2ffscanf_8h}}{}
\item\contentsline{section}{include/\hyperlink{include_2filenm_8h}{filenm.\-h} }{\pageref{include_2filenm_8h}}{}
\item\contentsline{section}{include/\hyperlink{include_2force_8h}{force.\-h} }{\pageref{include_2force_8h}}{}
\item\contentsline{section}{include/\hyperlink{include_2futil_8h}{futil.\-h} }{\pageref{include_2futil_8h}}{}
\item\contentsline{section}{include/\hyperlink{include_2gbutil_8h}{gbutil.\-h} }{\pageref{include_2gbutil_8h}}{}
\item\contentsline{section}{include/\hyperlink{include_2gen__ad_8h}{gen\-\_\-ad.\-h} }{\pageref{include_2gen__ad_8h}}{}
\item\contentsline{section}{include/\hyperlink{include_2genborn_8h}{genborn.\-h} }{\pageref{include_2genborn_8h}}{}
\item\contentsline{section}{include/\hyperlink{include_2gmx__ana_8h}{gmx\-\_\-ana.\-h} }{\pageref{include_2gmx__ana_8h}}{}
\item\contentsline{section}{include/\hyperlink{include_2gmx__arpack_8h}{gmx\-\_\-arpack.\-h} \\*\-Selected routines from \-A\-R\-P\-A\-C\-K }{\pageref{include_2gmx__arpack_8h}}{}
\item\contentsline{section}{include/\hyperlink{include_2gmx__avx__double_8h}{gmx\-\_\-avx\-\_\-double.\-h} }{\pageref{include_2gmx__avx__double_8h}}{}
\item\contentsline{section}{include/\hyperlink{include_2gmx__avx__single_8h}{gmx\-\_\-avx\-\_\-single.\-h} }{\pageref{include_2gmx__avx__single_8h}}{}
\item\contentsline{section}{include/\hyperlink{include_2gmx__blas_8h}{gmx\-\_\-blas.\-h} \\*\-Header definitions for the standard \-B\-L\-A\-S library }{\pageref{include_2gmx__blas_8h}}{}
\item\contentsline{section}{include/\hyperlink{include_2gmx__cpuid_8h}{gmx\-\_\-cpuid.\-h} }{\pageref{include_2gmx__cpuid_8h}}{}
\item\contentsline{section}{include/\hyperlink{include_2gmx__cyclecounter_8h}{gmx\-\_\-cyclecounter.\-h} }{\pageref{include_2gmx__cyclecounter_8h}}{}
\item\contentsline{section}{include/\hyperlink{include_2gmx__detect__hardware_8h}{gmx\-\_\-detect\-\_\-hardware.\-h} }{\pageref{include_2gmx__detect__hardware_8h}}{}
\item\contentsline{section}{include/\hyperlink{include_2gmx__fatal_8h}{gmx\-\_\-fatal.\-h} }{\pageref{include_2gmx__fatal_8h}}{}
\item\contentsline{section}{include/\hyperlink{include_2gmx__fatal__collective_8h}{gmx\-\_\-fatal\-\_\-collective.\-h} }{\pageref{include_2gmx__fatal__collective_8h}}{}
\item\contentsline{section}{include/\hyperlink{include_2gmx__fft_8h}{gmx\-\_\-fft.\-h} }{\pageref{include_2gmx__fft_8h}}{}
\item\contentsline{section}{include/\hyperlink{include_2gmx__ga2la_8h}{gmx\-\_\-ga2la.\-h} }{\pageref{include_2gmx__ga2la_8h}}{}
\item\contentsline{section}{include/\hyperlink{include_2gmx__hash_8h}{gmx\-\_\-hash.\-h} }{\pageref{include_2gmx__hash_8h}}{}
\item\contentsline{section}{include/\hyperlink{include_2gmx__header__config_8h}{gmx\-\_\-header\-\_\-config.\-h} }{\pageref{include_2gmx__header__config_8h}}{}
\item\contentsline{section}{include/\hyperlink{include_2gmx__lapack_8h}{gmx\-\_\-lapack.\-h} \\*\-Header definitions for the standard \-L\-A\-P\-A\-C\-K library }{\pageref{include_2gmx__lapack_8h}}{}
\item\contentsline{section}{include/\hyperlink{include_2gmx__math__x86__avx__128__fma__double_8h}{gmx\-\_\-math\-\_\-x86\-\_\-avx\-\_\-128\-\_\-fma\-\_\-double.\-h} }{\pageref{include_2gmx__math__x86__avx__128__fma__double_8h}}{}
\item\contentsline{section}{include/\hyperlink{include_2gmx__math__x86__avx__128__fma__single_8h}{gmx\-\_\-math\-\_\-x86\-\_\-avx\-\_\-128\-\_\-fma\-\_\-single.\-h} }{\pageref{include_2gmx__math__x86__avx__128__fma__single_8h}}{}
\item\contentsline{section}{include/\hyperlink{include_2gmx__math__x86__avx__256__double_8h}{gmx\-\_\-math\-\_\-x86\-\_\-avx\-\_\-256\-\_\-double.\-h} }{\pageref{include_2gmx__math__x86__avx__256__double_8h}}{}
\item\contentsline{section}{include/\hyperlink{include_2gmx__math__x86__avx__256__single_8h}{gmx\-\_\-math\-\_\-x86\-\_\-avx\-\_\-256\-\_\-single.\-h} }{\pageref{include_2gmx__math__x86__avx__256__single_8h}}{}
\item\contentsline{section}{include/\hyperlink{include_2gmx__math__x86__sse2__double_8h}{gmx\-\_\-math\-\_\-x86\-\_\-sse2\-\_\-double.\-h} }{\pageref{include_2gmx__math__x86__sse2__double_8h}}{}
\item\contentsline{section}{include/\hyperlink{include_2gmx__math__x86__sse2__single_8h}{gmx\-\_\-math\-\_\-x86\-\_\-sse2\-\_\-single.\-h} }{\pageref{include_2gmx__math__x86__sse2__single_8h}}{}
\item\contentsline{section}{include/\hyperlink{include_2gmx__math__x86__sse4__1__double_8h}{gmx\-\_\-math\-\_\-x86\-\_\-sse4\-\_\-1\-\_\-double.\-h} }{\pageref{include_2gmx__math__x86__sse4__1__double_8h}}{}
\item\contentsline{section}{include/\hyperlink{include_2gmx__math__x86__sse4__1__single_8h}{gmx\-\_\-math\-\_\-x86\-\_\-sse4\-\_\-1\-\_\-single.\-h} }{\pageref{include_2gmx__math__x86__sse4__1__single_8h}}{}
\item\contentsline{section}{include/\hyperlink{include_2gmx__matrix_8h}{gmx\-\_\-matrix.\-h} }{\pageref{include_2gmx__matrix_8h}}{}
\item\contentsline{section}{include/\hyperlink{include_2gmx__omp_8h}{gmx\-\_\-omp.\-h} }{\pageref{include_2gmx__omp_8h}}{}
\item\contentsline{section}{include/\hyperlink{include_2gmx__omp__nthreads_8h}{gmx\-\_\-omp\-\_\-nthreads.\-h} }{\pageref{include_2gmx__omp__nthreads_8h}}{}
\item\contentsline{section}{include/\hyperlink{include_2gmx__parallel__3dfft_8h}{gmx\-\_\-parallel\-\_\-3dfft.\-h} }{\pageref{include_2gmx__parallel__3dfft_8h}}{}
\item\contentsline{section}{include/\hyperlink{include_2gmx__random_8h}{gmx\-\_\-random.\-h} }{\pageref{include_2gmx__random_8h}}{}
\item\contentsline{section}{include/\hyperlink{include_2gmx__simd__macros_8h}{gmx\-\_\-simd\-\_\-macros.\-h} }{\pageref{include_2gmx__simd__macros_8h}}{}
\item\contentsline{section}{include/\hyperlink{include_2gmx__sort_8h}{gmx\-\_\-sort.\-h} }{\pageref{include_2gmx__sort_8h}}{}
\item\contentsline{section}{include/\hyperlink{include_2gmx__statistics_8h}{gmx\-\_\-statistics.\-h} }{\pageref{include_2gmx__statistics_8h}}{}
\item\contentsline{section}{include/\hyperlink{include_2gmx__system__xdr_8h}{gmx\-\_\-system\-\_\-xdr.\-h} }{\pageref{include_2gmx__system__xdr_8h}}{}
\item\contentsline{section}{include/\hyperlink{include_2gmx__thread__affinity_8h}{gmx\-\_\-thread\-\_\-affinity.\-h} }{\pageref{include_2gmx__thread__affinity_8h}}{}
\item\contentsline{section}{include/\hyperlink{include_2gmx__wallcycle_8h}{gmx\-\_\-wallcycle.\-h} }{\pageref{include_2gmx__wallcycle_8h}}{}
\item\contentsline{section}{include/\hyperlink{include_2gmx__x86__avx__128__fma_8h}{gmx\-\_\-x86\-\_\-avx\-\_\-128\-\_\-fma.\-h} }{\pageref{include_2gmx__x86__avx__128__fma_8h}}{}
\item\contentsline{section}{include/\hyperlink{include_2gmx__x86__avx__256_8h}{gmx\-\_\-x86\-\_\-avx\-\_\-256.\-h} }{\pageref{include_2gmx__x86__avx__256_8h}}{}
\item\contentsline{section}{include/\hyperlink{include_2gmx__x86__simd__double_8h}{gmx\-\_\-x86\-\_\-simd\-\_\-double.\-h} }{\pageref{include_2gmx__x86__simd__double_8h}}{}
\item\contentsline{section}{include/\hyperlink{include_2gmx__x86__simd__single_8h}{gmx\-\_\-x86\-\_\-simd\-\_\-single.\-h} }{\pageref{include_2gmx__x86__simd__single_8h}}{}
\item\contentsline{section}{include/\hyperlink{include_2gmx__x86__sse2_8h}{gmx\-\_\-x86\-\_\-sse2.\-h} }{\pageref{include_2gmx__x86__sse2_8h}}{}
\item\contentsline{section}{include/\hyperlink{include_2gmx__x86__sse4__1_8h}{gmx\-\_\-x86\-\_\-sse4\-\_\-1.\-h} }{\pageref{include_2gmx__x86__sse4__1_8h}}{}
\item\contentsline{section}{include/\hyperlink{include_2gmxcomplex_8h}{gmxcomplex.\-h} }{\pageref{include_2gmxcomplex_8h}}{}
\item\contentsline{section}{include/\hyperlink{include_2gmxcpp_8h}{gmxcpp.\-h} }{\pageref{include_2gmxcpp_8h}}{}
\item\contentsline{section}{include/\hyperlink{include_2gmxfio_8h}{gmxfio.\-h} }{\pageref{include_2gmxfio_8h}}{}
\item\contentsline{section}{include/\hyperlink{include_2gpp__atomtype_8h}{gpp\-\_\-atomtype.\-h} }{\pageref{include_2gpp__atomtype_8h}}{}
\item\contentsline{section}{include/\hyperlink{include_2gpp__nextnb_8h}{gpp\-\_\-nextnb.\-h} }{\pageref{include_2gpp__nextnb_8h}}{}
\item\contentsline{section}{include/\hyperlink{include_2gpu__utils_8h}{gpu\-\_\-utils.\-h} }{\pageref{include_2gpu__utils_8h}}{}
\item\contentsline{section}{include/\hyperlink{include_2grompp_8h}{grompp.\-h} }{\pageref{include_2grompp_8h}}{}
\item\contentsline{section}{include/\hyperlink{include_2gstat_8h}{gstat.\-h} }{\pageref{include_2gstat_8h}}{}
\item\contentsline{section}{include/\hyperlink{include_2hackblock_8h}{hackblock.\-h} }{\pageref{include_2hackblock_8h}}{}
\item\contentsline{section}{include/\hyperlink{include_2histogram_8h}{histogram.\-h} \\*\-A\-P\-I for calculation of histograms with error estimates }{\pageref{include_2histogram_8h}}{}
\item\contentsline{section}{include/\hyperlink{include_2index_8h}{index.\-h} }{\pageref{include_2index_8h}}{}
\item\contentsline{section}{include/\hyperlink{include_2indexutil_8h}{indexutil.\-h} \\*\-A\-P\-I for handling index files and index groups }{\pageref{include_2indexutil_8h}}{}
\item\contentsline{section}{include/\hyperlink{include_2inputrec_8h}{inputrec.\-h} }{\pageref{include_2inputrec_8h}}{}
\item\contentsline{section}{include/\hyperlink{include_2invblock_8h}{invblock.\-h} }{\pageref{include_2invblock_8h}}{}
\item\contentsline{section}{include/\hyperlink{include_2macros_8h}{macros.\-h} }{\pageref{include_2macros_8h}}{}
\item\contentsline{section}{include/\hyperlink{include_2main_8h}{main.\-h} }{\pageref{include_2main_8h}}{}
\item\contentsline{section}{include/\hyperlink{include_2mainpage_8h}{mainpage.\-h} \\*\-Dummy header for \-Doxygen documentation }{\pageref{include_2mainpage_8h}}{}
\item\contentsline{section}{include/\hyperlink{include_2maths_8h}{maths.\-h} }{\pageref{include_2maths_8h}}{}
\item\contentsline{section}{include/\hyperlink{include_2matio_8h}{matio.\-h} }{\pageref{include_2matio_8h}}{}
\item\contentsline{section}{include/\hyperlink{include_2md5_8h}{md5.\-h} }{\pageref{include_2md5_8h}}{}
\item\contentsline{section}{include/\hyperlink{include_2md__logging_8h}{md\-\_\-logging.\-h} }{\pageref{include_2md__logging_8h}}{}
\item\contentsline{section}{include/\hyperlink{include_2md__support_8h}{md\-\_\-support.\-h} }{\pageref{include_2md__support_8h}}{}
\item\contentsline{section}{include/\hyperlink{include_2mdatoms_8h}{mdatoms.\-h} }{\pageref{include_2mdatoms_8h}}{}
\item\contentsline{section}{include/\hyperlink{include_2mdebin_8h}{mdebin.\-h} }{\pageref{include_2mdebin_8h}}{}
\item\contentsline{section}{include/\hyperlink{include_2mdrun_8h}{mdrun.\-h} }{\pageref{include_2mdrun_8h}}{}
\item\contentsline{section}{include/\hyperlink{include_2mpelogging_8h}{mpelogging.\-h} }{\pageref{include_2mpelogging_8h}}{}
\item\contentsline{section}{include/\hyperlink{include_2mshift_8h}{mshift.\-h} }{\pageref{include_2mshift_8h}}{}
\item\contentsline{section}{include/\hyperlink{include_2mtop__util_8h}{mtop\-\_\-util.\-h} }{\pageref{include_2mtop__util_8h}}{}
\item\contentsline{section}{include/\hyperlink{include_2mtxio_8h}{mtxio.\-h} }{\pageref{include_2mtxio_8h}}{}
\item\contentsline{section}{include/\hyperlink{include_2mvdata_8h}{mvdata.\-h} }{\pageref{include_2mvdata_8h}}{}
\item\contentsline{section}{include/\hyperlink{include_2names_8h}{names.\-h} }{\pageref{include_2names_8h}}{}
\item\contentsline{section}{include/\hyperlink{include_2nbnxn__cuda__data__mgmt_8h}{nbnxn\-\_\-cuda\-\_\-data\-\_\-mgmt.\-h} }{\pageref{include_2nbnxn__cuda__data__mgmt_8h}}{}
\item\contentsline{section}{include/\hyperlink{include_2nbsearch_8h}{nbsearch.\-h} \\*\-A\-P\-I for neighborhood searching }{\pageref{include_2nbsearch_8h}}{}
\item\contentsline{section}{include/\hyperlink{include_2network_8h}{network.\-h} }{\pageref{include_2network_8h}}{}
\item\contentsline{section}{include/\hyperlink{include_2nonbonded_8h}{nonbonded.\-h} }{\pageref{include_2nonbonded_8h}}{}
\item\contentsline{section}{include/\hyperlink{include_2nrama_8h}{nrama.\-h} }{\pageref{include_2nrama_8h}}{}
\item\contentsline{section}{include/\hyperlink{include_2nrjac_8h}{nrjac.\-h} }{\pageref{include_2nrjac_8h}}{}
\item\contentsline{section}{include/\hyperlink{include_2nrnb_8h}{nrnb.\-h} }{\pageref{include_2nrnb_8h}}{}
\item\contentsline{section}{include/\hyperlink{include_2ns_8h}{ns.\-h} }{\pageref{include_2ns_8h}}{}
\item\contentsline{section}{include/\hyperlink{include_2nsgrid_8h}{nsgrid.\-h} }{\pageref{include_2nsgrid_8h}}{}
\item\contentsline{section}{include/\hyperlink{include_2oenv_8h}{oenv.\-h} }{\pageref{include_2oenv_8h}}{}
\item\contentsline{section}{include/\hyperlink{include_2orires_8h}{orires.\-h} }{\pageref{include_2orires_8h}}{}
\item\contentsline{section}{include/\hyperlink{include_2partdec_8h}{partdec.\-h} }{\pageref{include_2partdec_8h}}{}
\item\contentsline{section}{include/\hyperlink{include_2pbc_8h}{pbc.\-h} }{\pageref{include_2pbc_8h}}{}
\item\contentsline{section}{include/\hyperlink{include_2pdb2top_8h}{pdb2top.\-h} }{\pageref{include_2pdb2top_8h}}{}
\item\contentsline{section}{include/\hyperlink{include_2pdbio_8h}{pdbio.\-h} }{\pageref{include_2pdbio_8h}}{}
\item\contentsline{section}{include/\hyperlink{include_2perf__est_8h}{perf\-\_\-est.\-h} }{\pageref{include_2perf__est_8h}}{}
\item\contentsline{section}{include/\hyperlink{include_2physics_8h}{physics.\-h} }{\pageref{include_2physics_8h}}{}
\item\contentsline{section}{include/\hyperlink{include_2pmalloc__cuda_8h}{pmalloc\-\_\-cuda.\-h} }{\pageref{include_2pmalloc__cuda_8h}}{}
\item\contentsline{section}{include/\hyperlink{include_2pme_8h}{pme.\-h} }{\pageref{include_2pme_8h}}{}
\item\contentsline{section}{include/\hyperlink{include_2poscalc_8h}{poscalc.\-h} \\*\-A\-P\-I for structured and optimized calculation of positions }{\pageref{include_2poscalc_8h}}{}
\item\contentsline{section}{include/\hyperlink{include_2position_8h}{position.\-h} \\*\-A\-P\-I for handling positions }{\pageref{include_2position_8h}}{}
\item\contentsline{section}{include/\hyperlink{include_2princ_8h}{princ.\-h} }{\pageref{include_2princ_8h}}{}
\item\contentsline{section}{include/\hyperlink{include_2pull_8h}{pull.\-h} }{\pageref{include_2pull_8h}}{}
\item\contentsline{section}{include/\hyperlink{include_2pull__rotation_8h}{pull\-\_\-rotation.\-h} }{\pageref{include_2pull__rotation_8h}}{}
\item\contentsline{section}{include/\hyperlink{include_2qmmm_8h}{qmmm.\-h} }{\pageref{include_2qmmm_8h}}{}
\item\contentsline{section}{include/\hyperlink{include_2random_8h}{random.\-h} }{\pageref{include_2random_8h}}{}
\item\contentsline{section}{include/\hyperlink{include_2rbin_8h}{rbin.\-h} }{\pageref{include_2rbin_8h}}{}
\item\contentsline{section}{include/\hyperlink{include_2readinp_8h}{readinp.\-h} }{\pageref{include_2readinp_8h}}{}
\item\contentsline{section}{include/\hyperlink{include_2resall_8h}{resall.\-h} }{\pageref{include_2resall_8h}}{}
\item\contentsline{section}{include/\hyperlink{include_2rmpbc_8h}{rmpbc.\-h} }{\pageref{include_2rmpbc_8h}}{}
\item\contentsline{section}{include/\hyperlink{include_2selection_8h}{selection.\-h} \\*\-A\-P\-I for handling selection (the {\ttfamily \hyperlink{structgmx__ana__selection__t}{gmx\-\_\-ana\-\_\-selection\-\_\-t}} structure and related functions) }{\pageref{include_2selection_8h}}{}
\item\contentsline{section}{include/\hyperlink{include_2selmethod_8h}{selmethod.\-h} \\*\-A\-P\-I for handling selection methods }{\pageref{include_2selmethod_8h}}{}
\item\contentsline{section}{include/\hyperlink{include_2selparam_8h}{selparam.\-h} \\*\-A\-P\-I for handling parameters used in selections }{\pageref{include_2selparam_8h}}{}
\item\contentsline{section}{include/\hyperlink{include_2selvalue_8h}{selvalue.\-h} \\*\-Declaration of {\ttfamily \hyperlink{structgmx__ana__selvalue__t}{gmx\-\_\-ana\-\_\-selvalue\-\_\-t}} }{\pageref{include_2selvalue_8h}}{}
\item\contentsline{section}{include/\hyperlink{include_2sfactor_8h}{sfactor.\-h} }{\pageref{include_2sfactor_8h}}{}
\item\contentsline{section}{include/\hyperlink{include_2shellfc_8h}{shellfc.\-h} }{\pageref{include_2shellfc_8h}}{}
\item\contentsline{section}{include/\hyperlink{include_2shift_8h}{shift.\-h} }{\pageref{include_2shift_8h}}{}
\item\contentsline{section}{include/\hyperlink{include_2sighandler_8h}{sighandler.\-h} }{\pageref{include_2sighandler_8h}}{}
\item\contentsline{section}{include/\hyperlink{include_2sim__util_8h}{sim\-\_\-util.\-h} }{\pageref{include_2sim__util_8h}}{}
\item\contentsline{section}{include/\hyperlink{include_2smalloc_8h}{smalloc.\-h} }{\pageref{include_2smalloc_8h}}{}
\item\contentsline{section}{include/\hyperlink{include_2sortwater_8h}{sortwater.\-h} }{\pageref{include_2sortwater_8h}}{}
\item\contentsline{section}{include/\hyperlink{include_2sparsematrix_8h}{sparsematrix.\-h} }{\pageref{include_2sparsematrix_8h}}{}
\item\contentsline{section}{include/\hyperlink{include_2split_8h}{split.\-h} }{\pageref{include_2split_8h}}{}
\item\contentsline{section}{include/\hyperlink{include_2splitter_8h}{splitter.\-h} }{\pageref{include_2splitter_8h}}{}
\item\contentsline{section}{include/\hyperlink{include_2statutil_8h}{statutil.\-h} }{\pageref{include_2statutil_8h}}{}
\item\contentsline{section}{include/\hyperlink{include_2strdb_8h}{strdb.\-h} }{\pageref{include_2strdb_8h}}{}
\item\contentsline{section}{include/\hyperlink{include_2string2_8h}{string2.\-h} \\*\-Generic string handling functions }{\pageref{include_2string2_8h}}{}
\item\contentsline{section}{include/\hyperlink{include_2symtab_8h}{symtab.\-h} }{\pageref{include_2symtab_8h}}{}
\item\contentsline{section}{include/\hyperlink{include_2sysstuff_8h}{sysstuff.\-h} }{\pageref{include_2sysstuff_8h}}{}
\item\contentsline{section}{include/\hyperlink{include_2tables_8h}{tables.\-h} }{\pageref{include_2tables_8h}}{}
\item\contentsline{section}{include/\hyperlink{include_2tgroup_8h}{tgroup.\-h} }{\pageref{include_2tgroup_8h}}{}
\item\contentsline{section}{include/\hyperlink{include_2thread__mpi_8h}{thread\-\_\-mpi.\-h} \\*\-Convenience header file for non-\/\-M\-P\-I compatibility }{\pageref{include_2thread__mpi_8h}}{}
\item\contentsline{section}{include/\hyperlink{include_2tmpi_8h}{tmpi.\-h} \\*\-Convenience header file for \-M\-P\-I compatibility }{\pageref{include_2tmpi_8h}}{}
\item\contentsline{section}{include/\hyperlink{include_2topsort_8h}{topsort.\-h} }{\pageref{include_2topsort_8h}}{}
\item\contentsline{section}{include/\hyperlink{include_2toputil_8h}{toputil.\-h} }{\pageref{include_2toputil_8h}}{}
\item\contentsline{section}{include/\hyperlink{include_2tpxio_8h}{tpxio.\-h} }{\pageref{include_2tpxio_8h}}{}
\item\contentsline{section}{include/\hyperlink{include_2trajana_8h}{trajana.\-h} \\*\-Main \-A\-P\-I of the trajectory analysis library }{\pageref{include_2trajana_8h}}{}
\item\contentsline{section}{include/\hyperlink{include_2trnio_8h}{trnio.\-h} }{\pageref{include_2trnio_8h}}{}
\item\contentsline{section}{include/\hyperlink{include_2txtdump_8h}{txtdump.\-h} }{\pageref{include_2txtdump_8h}}{}
\item\contentsline{section}{include/\hyperlink{include_2typedefs_8h}{typedefs.\-h} }{\pageref{include_2typedefs_8h}}{}
\item\contentsline{section}{include/\hyperlink{include_2update_8h}{update.\-h} }{\pageref{include_2update_8h}}{}
\item\contentsline{section}{include/\hyperlink{include_2vcm_8h}{vcm.\-h} }{\pageref{include_2vcm_8h}}{}
\item\contentsline{section}{include/\hyperlink{include_2vec_8h}{vec.\-h} }{\pageref{include_2vec_8h}}{}
\item\contentsline{section}{include/\hyperlink{include_2version_8h}{version.\-h} }{\pageref{include_2version_8h}}{}
\item\contentsline{section}{include/\hyperlink{include_2viewit_8h}{viewit.\-h} }{\pageref{include_2viewit_8h}}{}
\item\contentsline{section}{include/\hyperlink{include_2visibility_8h}{visibility.\-h} }{\pageref{include_2visibility_8h}}{}
\item\contentsline{section}{include/\hyperlink{include_2vsite_8h}{vsite.\-h} }{\pageref{include_2vsite_8h}}{}
\item\contentsline{section}{include/\hyperlink{include_2warninp_8h}{warninp.\-h} }{\pageref{include_2warninp_8h}}{}
\item\contentsline{section}{include/\hyperlink{include_2wgms_8h}{wgms.\-h} }{\pageref{include_2wgms_8h}}{}
\item\contentsline{section}{include/\hyperlink{include_2wman_8h}{wman.\-h} }{\pageref{include_2wman_8h}}{}
\item\contentsline{section}{include/\hyperlink{include_2writeps_8h}{writeps.\-h} }{\pageref{include_2writeps_8h}}{}
\item\contentsline{section}{include/\hyperlink{include_2xdrf_8h}{xdrf.\-h} }{\pageref{include_2xdrf_8h}}{}
\item\contentsline{section}{include/\hyperlink{include_2xtcio_8h}{xtcio.\-h} }{\pageref{include_2xtcio_8h}}{}
\item\contentsline{section}{include/\hyperlink{include_2xvgr_8h}{xvgr.\-h} }{\pageref{include_2xvgr_8h}}{}
\item\contentsline{section}{include/thread\-\_\-mpi/\hyperlink{include_2thread__mpi_2atomic_8h}{atomic.\-h} }{\pageref{include_2thread__mpi_2atomic_8h}}{}
\item\contentsline{section}{include/thread\-\_\-mpi/\hyperlink{include_2thread__mpi_2barrier_8h}{barrier.\-h} }{\pageref{include_2thread__mpi_2barrier_8h}}{}
\item\contentsline{section}{include/thread\-\_\-mpi/\hyperlink{include_2thread__mpi_2collective_8h}{collective.\-h} \\*\-Collective functions }{\pageref{include_2thread__mpi_2collective_8h}}{}
\item\contentsline{section}{include/thread\-\_\-mpi/\hyperlink{include_2thread__mpi_2event_8h}{event.\-h} \\*\-Event notification wait and signaling functions }{\pageref{include_2thread__mpi_2event_8h}}{}
\item\contentsline{section}{include/thread\-\_\-mpi/\hyperlink{include_2thread__mpi_2hwinfo_8h}{hwinfo.\-h} \\*\-C\-P\-U/core/\-H\-T count function }{\pageref{include_2thread__mpi_2hwinfo_8h}}{}
\item\contentsline{section}{include/thread\-\_\-mpi/\hyperlink{include_2thread__mpi_2list_8h}{list.\-h} \\*\-Lock-\/free list data structures }{\pageref{include_2thread__mpi_2list_8h}}{}
\item\contentsline{section}{include/thread\-\_\-mpi/\hyperlink{include_2thread__mpi_2lock_8h}{lock.\-h} }{\pageref{include_2thread__mpi_2lock_8h}}{}
\item\contentsline{section}{include/thread\-\_\-mpi/\hyperlink{include_2thread__mpi_2mpi__bindings_8h}{mpi\-\_\-bindings.\-h} \\*\-M\-P\-I bindings for thread\-\_\-mpi/tmpi.\-h }{\pageref{include_2thread__mpi_2mpi__bindings_8h}}{}
\item\contentsline{section}{include/thread\-\_\-mpi/\hyperlink{include_2thread__mpi_2mutex_8h}{mutex.\-h} \\*\-Mutex objects with \-C++11 \-A\-P\-I compatibility }{\pageref{include_2thread__mpi_2mutex_8h}}{}
\item\contentsline{section}{include/thread\-\_\-mpi/\hyperlink{include_2thread__mpi_2numa__malloc_8h}{numa\-\_\-malloc.\-h} \\*\-N\-U\-M\-A aware memory allocators }{\pageref{include_2thread__mpi_2numa__malloc_8h}}{}
\item\contentsline{section}{include/thread\-\_\-mpi/\hyperlink{include_2thread__mpi_2system__error_8h}{system\-\_\-error.\-h} \\*\-A \-C++11 compatible system\-\_\-error class for reporting exceptions }{\pageref{include_2thread__mpi_2system__error_8h}}{}
\item\contentsline{section}{include/thread\-\_\-mpi/\hyperlink{include_2thread__mpi_2threads_8h}{threads.\-h} }{\pageref{include_2thread__mpi_2threads_8h}}{}
\item\contentsline{section}{include/thread\-\_\-mpi/\hyperlink{include_2thread__mpi_2tmpi_8h}{tmpi.\-h} \\*\-Partial implementation of \-M\-P\-I using only threads }{\pageref{include_2thread__mpi_2tmpi_8h}}{}
\item\contentsline{section}{include/thread\-\_\-mpi/\hyperlink{include_2thread__mpi_2visibility_8h}{visibility.\-h} \\*\-Visibility macros }{\pageref{include_2thread__mpi_2visibility_8h}}{}
\item\contentsline{section}{include/thread\-\_\-mpi/\hyperlink{include_2thread__mpi_2wait_8h}{wait.\-h} }{\pageref{include_2thread__mpi_2wait_8h}}{}
\item\contentsline{section}{include/thread\-\_\-mpi/atomic/\hyperlink{include_2thread__mpi_2atomic_2cycles_8h}{cycles.\-h} }{\pageref{include_2thread__mpi_2atomic_2cycles_8h}}{}
\item\contentsline{section}{include/thread\-\_\-mpi/atomic/\hyperlink{include_2thread__mpi_2atomic_2gcc_8h}{gcc.\-h} }{\pageref{include_2thread__mpi_2atomic_2gcc_8h}}{}
\item\contentsline{section}{include/thread\-\_\-mpi/atomic/\hyperlink{include_2thread__mpi_2atomic_2gcc__ia64_8h}{gcc\-\_\-ia64.\-h} }{\pageref{include_2thread__mpi_2atomic_2gcc__ia64_8h}}{}
\item\contentsline{section}{include/thread\-\_\-mpi/atomic/\hyperlink{include_2thread__mpi_2atomic_2gcc__intrinsics_8h}{gcc\-\_\-intrinsics.\-h} }{\pageref{include_2thread__mpi_2atomic_2gcc__intrinsics_8h}}{}
\item\contentsline{section}{include/thread\-\_\-mpi/atomic/\hyperlink{include_2thread__mpi_2atomic_2gcc__ppc_8h}{gcc\-\_\-ppc.\-h} }{\pageref{include_2thread__mpi_2atomic_2gcc__ppc_8h}}{}
\item\contentsline{section}{include/thread\-\_\-mpi/atomic/\hyperlink{include_2thread__mpi_2atomic_2gcc__spinlock_8h}{gcc\-\_\-spinlock.\-h} }{\pageref{include_2thread__mpi_2atomic_2gcc__spinlock_8h}}{}
\item\contentsline{section}{include/thread\-\_\-mpi/atomic/\hyperlink{include_2thread__mpi_2atomic_2gcc__x86_8h}{gcc\-\_\-x86.\-h} }{\pageref{include_2thread__mpi_2atomic_2gcc__x86_8h}}{}
\item\contentsline{section}{include/thread\-\_\-mpi/atomic/\hyperlink{include_2thread__mpi_2atomic_2msvc_8h}{msvc.\-h} }{\pageref{include_2thread__mpi_2atomic_2msvc_8h}}{}
\item\contentsline{section}{include/thread\-\_\-mpi/atomic/\hyperlink{include_2thread__mpi_2atomic_2suncc-sparc_8h}{suncc-\/sparc.\-h} }{\pageref{include_2thread__mpi_2atomic_2suncc-sparc_8h}}{}
\item\contentsline{section}{include/thread\-\_\-mpi/atomic/\hyperlink{include_2thread__mpi_2atomic_2xlc__ppc_8h}{xlc\-\_\-ppc.\-h} }{\pageref{include_2thread__mpi_2atomic_2xlc__ppc_8h}}{}
\item\contentsline{section}{include/types/\hyperlink{include_2types_2atoms_8h}{atoms.\-h} }{\pageref{include_2types_2atoms_8h}}{}
\item\contentsline{section}{include/types/\hyperlink{include_2types_2block_8h}{block.\-h} }{\pageref{include_2types_2block_8h}}{}
\item\contentsline{section}{include/types/\hyperlink{include_2types_2commrec_8h}{commrec.\-h} }{\pageref{include_2types_2commrec_8h}}{}
\item\contentsline{section}{include/types/\hyperlink{include_2types_2constr_8h}{constr.\-h} }{\pageref{include_2types_2constr_8h}}{}
\item\contentsline{section}{include/types/\hyperlink{include_2types_2energy_8h}{energy.\-h} }{\pageref{include_2types_2energy_8h}}{}
\item\contentsline{section}{include/types/\hyperlink{include_2types_2enums_8h}{enums.\-h} }{\pageref{include_2types_2enums_8h}}{}
\item\contentsline{section}{include/types/\hyperlink{include_2types_2fcdata_8h}{fcdata.\-h} }{\pageref{include_2types_2fcdata_8h}}{}
\item\contentsline{section}{include/types/\hyperlink{include_2types_2filenm_8h}{filenm.\-h} }{\pageref{include_2types_2filenm_8h}}{}
\item\contentsline{section}{include/types/\hyperlink{include_2types_2force__flags_8h}{force\-\_\-flags.\-h} }{\pageref{include_2types_2force__flags_8h}}{}
\item\contentsline{section}{include/types/\hyperlink{include_2types_2forcerec_8h}{forcerec.\-h} }{\pageref{include_2types_2forcerec_8h}}{}
\item\contentsline{section}{include/types/\hyperlink{include_2types_2genborn_8h}{genborn.\-h} }{\pageref{include_2types_2genborn_8h}}{}
\item\contentsline{section}{include/types/\hyperlink{include_2types_2globsig_8h}{globsig.\-h} }{\pageref{include_2types_2globsig_8h}}{}
\item\contentsline{section}{include/types/\hyperlink{include_2types_2graph_8h}{graph.\-h} }{\pageref{include_2types_2graph_8h}}{}
\item\contentsline{section}{include/types/\hyperlink{include_2types_2group_8h}{group.\-h} }{\pageref{include_2types_2group_8h}}{}
\item\contentsline{section}{include/types/\hyperlink{include_2types_2hw__info_8h}{hw\-\_\-info.\-h} }{\pageref{include_2types_2hw__info_8h}}{}
\item\contentsline{section}{include/types/\hyperlink{include_2types_2idef_8h}{idef.\-h} }{\pageref{include_2types_2idef_8h}}{}
\item\contentsline{section}{include/types/\hyperlink{include_2types_2ifunc_8h}{ifunc.\-h} }{\pageref{include_2types_2ifunc_8h}}{}
\item\contentsline{section}{include/types/\hyperlink{include_2types_2inputrec_8h}{inputrec.\-h} }{\pageref{include_2types_2inputrec_8h}}{}
\item\contentsline{section}{include/types/\hyperlink{include_2types_2interaction__const_8h}{interaction\-\_\-const.\-h} }{\pageref{include_2types_2interaction__const_8h}}{}
\item\contentsline{section}{include/types/\hyperlink{include_2types_2ishift_8h}{ishift.\-h} }{\pageref{include_2types_2ishift_8h}}{}
\item\contentsline{section}{include/types/\hyperlink{include_2types_2iteratedconstraints_8h}{iteratedconstraints.\-h} }{\pageref{include_2types_2iteratedconstraints_8h}}{}
\item\contentsline{section}{include/types/\hyperlink{include_2types_2matrix_8h}{matrix.\-h} }{\pageref{include_2types_2matrix_8h}}{}
\item\contentsline{section}{include/types/\hyperlink{include_2types_2mdatom_8h}{mdatom.\-h} }{\pageref{include_2types_2mdatom_8h}}{}
\item\contentsline{section}{include/types/\hyperlink{include_2types_2membedt_8h}{membedt.\-h} }{\pageref{include_2types_2membedt_8h}}{}
\item\contentsline{section}{include/types/\hyperlink{include_2types_2nb__verlet_8h}{nb\-\_\-verlet.\-h} }{\pageref{include_2types_2nb__verlet_8h}}{}
\item\contentsline{section}{include/types/\hyperlink{include_2types_2nblist_8h}{nblist.\-h} }{\pageref{include_2types_2nblist_8h}}{}
\item\contentsline{section}{include/types/\hyperlink{include_2types_2nbnxn__cuda__types__ext_8h}{nbnxn\-\_\-cuda\-\_\-types\-\_\-ext.\-h} }{\pageref{include_2types_2nbnxn__cuda__types__ext_8h}}{}
\item\contentsline{section}{include/types/\hyperlink{include_2types_2nbnxn__pairlist_8h}{nbnxn\-\_\-pairlist.\-h} }{\pageref{include_2types_2nbnxn__pairlist_8h}}{}
\item\contentsline{section}{include/types/\hyperlink{include_2types_2nlistheuristics_8h}{nlistheuristics.\-h} }{\pageref{include_2types_2nlistheuristics_8h}}{}
\item\contentsline{section}{include/types/\hyperlink{include_2types_2nrnb_8h}{nrnb.\-h} }{\pageref{include_2types_2nrnb_8h}}{}
\item\contentsline{section}{include/types/\hyperlink{include_2types_2ns_8h}{ns.\-h} }{\pageref{include_2types_2ns_8h}}{}
\item\contentsline{section}{include/types/\hyperlink{include_2types_2nsgrid_8h}{nsgrid.\-h} }{\pageref{include_2types_2nsgrid_8h}}{}
\item\contentsline{section}{include/types/\hyperlink{include_2types_2oenv_8h}{oenv.\-h} }{\pageref{include_2types_2oenv_8h}}{}
\item\contentsline{section}{include/types/\hyperlink{include_2types_2pbc_8h}{pbc.\-h} }{\pageref{include_2types_2pbc_8h}}{}
\item\contentsline{section}{include/types/\hyperlink{include_2types_2qmmmrec_8h}{qmmmrec.\-h} }{\pageref{include_2types_2qmmmrec_8h}}{}
\item\contentsline{section}{include/types/\hyperlink{include_2types_2shellfc_8h}{shellfc.\-h} }{\pageref{include_2types_2shellfc_8h}}{}
\item\contentsline{section}{include/types/\hyperlink{include_2types_2simple_8h}{simple.\-h} }{\pageref{include_2types_2simple_8h}}{}
\item\contentsline{section}{include/types/\hyperlink{include_2types_2state_8h}{state.\-h} }{\pageref{include_2types_2state_8h}}{}
\item\contentsline{section}{include/types/\hyperlink{include_2types_2symtab_8h}{symtab.\-h} }{\pageref{include_2types_2symtab_8h}}{}
\item\contentsline{section}{include/types/\hyperlink{include_2types_2topology_8h}{topology.\-h} }{\pageref{include_2types_2topology_8h}}{}
\item\contentsline{section}{include/types/\hyperlink{include_2types_2trx_8h}{trx.\-h} }{\pageref{include_2types_2trx_8h}}{}
\item\contentsline{section}{share/template/\hyperlink{template_8c}{template.\-c} \\*\-Template code for writing analysis programs }{\pageref{template_8c}}{}
\item\contentsline{section}{share/template/\hyperlink{template__doc_8c}{template\-\_\-doc.\-c} \\*\-Doxygen documentation source for template.\-c }{\pageref{template__doc_8c}}{}
\item\contentsline{section}{share/template/gromacs/\hyperlink{share_2template_2gromacs_23dview_8h}{3dview.\-h} }{\pageref{share_2template_2gromacs_23dview_8h}}{}
\item\contentsline{section}{share/template/gromacs/\hyperlink{share_2template_2gromacs_2assert_8h}{assert.\-h} }{\pageref{share_2template_2gromacs_2assert_8h}}{}
\item\contentsline{section}{share/template/gromacs/\hyperlink{share_2template_2gromacs_2atomprop_8h}{atomprop.\-h} }{\pageref{share_2template_2gromacs_2atomprop_8h}}{}
\item\contentsline{section}{share/template/gromacs/\hyperlink{share_2template_2gromacs_2bondf_8h}{bondf.\-h} }{\pageref{share_2template_2gromacs_2bondf_8h}}{}
\item\contentsline{section}{share/template/gromacs/\hyperlink{share_2template_2gromacs_2calcgrid_8h}{calcgrid.\-h} }{\pageref{share_2template_2gromacs_2calcgrid_8h}}{}
\item\contentsline{section}{share/template/gromacs/\hyperlink{share_2template_2gromacs_2calch_8h}{calch.\-h} }{\pageref{share_2template_2gromacs_2calch_8h}}{}
\item\contentsline{section}{share/template/gromacs/\hyperlink{share_2template_2gromacs_2calcmu_8h}{calcmu.\-h} }{\pageref{share_2template_2gromacs_2calcmu_8h}}{}
\item\contentsline{section}{share/template/gromacs/\hyperlink{share_2template_2gromacs_2centerofmass_8h}{centerofmass.\-h} \\*\-A\-P\-I for calculation of centers of mass/geometry }{\pageref{share_2template_2gromacs_2centerofmass_8h}}{}
\item\contentsline{section}{share/template/gromacs/\hyperlink{share_2template_2gromacs_2chargegroup_8h}{chargegroup.\-h} }{\pageref{share_2template_2gromacs_2chargegroup_8h}}{}
\item\contentsline{section}{share/template/gromacs/\hyperlink{share_2template_2gromacs_2checkpoint_8h}{checkpoint.\-h} }{\pageref{share_2template_2gromacs_2checkpoint_8h}}{}
\item\contentsline{section}{share/template/gromacs/\hyperlink{share_2template_2gromacs_2confio_8h}{confio.\-h} }{\pageref{share_2template_2gromacs_2confio_8h}}{}
\item\contentsline{section}{share/template/gromacs/\hyperlink{share_2template_2gromacs_2constr_8h}{constr.\-h} }{\pageref{share_2template_2gromacs_2constr_8h}}{}
\item\contentsline{section}{share/template/gromacs/\hyperlink{share_2template_2gromacs_2copyrite_8h}{copyrite.\-h} }{\pageref{share_2template_2gromacs_2copyrite_8h}}{}
\item\contentsline{section}{share/template/gromacs/\hyperlink{share_2template_2gromacs_2coulomb_8h}{coulomb.\-h} }{\pageref{share_2template_2gromacs_2coulomb_8h}}{}
\item\contentsline{section}{share/template/gromacs/\hyperlink{share_2template_2gromacs_2displacement_8h}{displacement.\-h} \\*\-A\-P\-I for on-\/line calculation of displacements }{\pageref{share_2template_2gromacs_2displacement_8h}}{}
\item\contentsline{section}{share/template/gromacs/\hyperlink{share_2template_2gromacs_2disre_8h}{disre.\-h} }{\pageref{share_2template_2gromacs_2disre_8h}}{}
\item\contentsline{section}{share/template/gromacs/\hyperlink{share_2template_2gromacs_2do__fit_8h}{do\-\_\-fit.\-h} }{\pageref{share_2template_2gromacs_2do__fit_8h}}{}
\item\contentsline{section}{share/template/gromacs/\hyperlink{share_2template_2gromacs_2domdec_8h}{domdec.\-h} }{\pageref{share_2template_2gromacs_2domdec_8h}}{}
\item\contentsline{section}{share/template/gromacs/\hyperlink{share_2template_2gromacs_2domdec__network_8h}{domdec\-\_\-network.\-h} }{\pageref{share_2template_2gromacs_2domdec__network_8h}}{}
\item\contentsline{section}{share/template/gromacs/\hyperlink{share_2template_2gromacs_2ebin_8h}{ebin.\-h} }{\pageref{share_2template_2gromacs_2ebin_8h}}{}
\item\contentsline{section}{share/template/gromacs/\hyperlink{share_2template_2gromacs_2edsam_8h}{edsam.\-h} }{\pageref{share_2template_2gromacs_2edsam_8h}}{}
\item\contentsline{section}{share/template/gromacs/\hyperlink{share_2template_2gromacs_2enxio_8h}{enxio.\-h} }{\pageref{share_2template_2gromacs_2enxio_8h}}{}
\item\contentsline{section}{share/template/gromacs/\hyperlink{share_2template_2gromacs_2ffscanf_8h}{ffscanf.\-h} }{\pageref{share_2template_2gromacs_2ffscanf_8h}}{}
\item\contentsline{section}{share/template/gromacs/\hyperlink{share_2template_2gromacs_2filenm_8h}{filenm.\-h} }{\pageref{share_2template_2gromacs_2filenm_8h}}{}
\item\contentsline{section}{share/template/gromacs/\hyperlink{share_2template_2gromacs_2force_8h}{force.\-h} }{\pageref{share_2template_2gromacs_2force_8h}}{}
\item\contentsline{section}{share/template/gromacs/\hyperlink{share_2template_2gromacs_2futil_8h}{futil.\-h} }{\pageref{share_2template_2gromacs_2futil_8h}}{}
\item\contentsline{section}{share/template/gromacs/\hyperlink{share_2template_2gromacs_2gbutil_8h}{gbutil.\-h} }{\pageref{share_2template_2gromacs_2gbutil_8h}}{}
\item\contentsline{section}{share/template/gromacs/\hyperlink{share_2template_2gromacs_2gen__ad_8h}{gen\-\_\-ad.\-h} }{\pageref{share_2template_2gromacs_2gen__ad_8h}}{}
\item\contentsline{section}{share/template/gromacs/\hyperlink{share_2template_2gromacs_2genborn_8h}{genborn.\-h} }{\pageref{share_2template_2gromacs_2genborn_8h}}{}
\item\contentsline{section}{share/template/gromacs/\hyperlink{share_2template_2gromacs_2gmx__ana_8h}{gmx\-\_\-ana.\-h} }{\pageref{share_2template_2gromacs_2gmx__ana_8h}}{}
\item\contentsline{section}{share/template/gromacs/\hyperlink{share_2template_2gromacs_2gmx__arpack_8h}{gmx\-\_\-arpack.\-h} \\*\-Selected routines from \-A\-R\-P\-A\-C\-K }{\pageref{share_2template_2gromacs_2gmx__arpack_8h}}{}
\item\contentsline{section}{share/template/gromacs/\hyperlink{share_2template_2gromacs_2gmx__avx__double_8h}{gmx\-\_\-avx\-\_\-double.\-h} }{\pageref{share_2template_2gromacs_2gmx__avx__double_8h}}{}
\item\contentsline{section}{share/template/gromacs/\hyperlink{share_2template_2gromacs_2gmx__avx__single_8h}{gmx\-\_\-avx\-\_\-single.\-h} }{\pageref{share_2template_2gromacs_2gmx__avx__single_8h}}{}
\item\contentsline{section}{share/template/gromacs/\hyperlink{share_2template_2gromacs_2gmx__blas_8h}{gmx\-\_\-blas.\-h} \\*\-Header definitions for the standard \-B\-L\-A\-S library }{\pageref{share_2template_2gromacs_2gmx__blas_8h}}{}
\item\contentsline{section}{share/template/gromacs/\hyperlink{share_2template_2gromacs_2gmx__cpuid_8h}{gmx\-\_\-cpuid.\-h} }{\pageref{share_2template_2gromacs_2gmx__cpuid_8h}}{}
\item\contentsline{section}{share/template/gromacs/\hyperlink{share_2template_2gromacs_2gmx__cyclecounter_8h}{gmx\-\_\-cyclecounter.\-h} }{\pageref{share_2template_2gromacs_2gmx__cyclecounter_8h}}{}
\item\contentsline{section}{share/template/gromacs/\hyperlink{share_2template_2gromacs_2gmx__detect__hardware_8h}{gmx\-\_\-detect\-\_\-hardware.\-h} }{\pageref{share_2template_2gromacs_2gmx__detect__hardware_8h}}{}
\item\contentsline{section}{share/template/gromacs/\hyperlink{share_2template_2gromacs_2gmx__fatal_8h}{gmx\-\_\-fatal.\-h} }{\pageref{share_2template_2gromacs_2gmx__fatal_8h}}{}
\item\contentsline{section}{share/template/gromacs/\hyperlink{share_2template_2gromacs_2gmx__fatal__collective_8h}{gmx\-\_\-fatal\-\_\-collective.\-h} }{\pageref{share_2template_2gromacs_2gmx__fatal__collective_8h}}{}
\item\contentsline{section}{share/template/gromacs/\hyperlink{share_2template_2gromacs_2gmx__fft_8h}{gmx\-\_\-fft.\-h} }{\pageref{share_2template_2gromacs_2gmx__fft_8h}}{}
\item\contentsline{section}{share/template/gromacs/\hyperlink{share_2template_2gromacs_2gmx__ga2la_8h}{gmx\-\_\-ga2la.\-h} }{\pageref{share_2template_2gromacs_2gmx__ga2la_8h}}{}
\item\contentsline{section}{share/template/gromacs/\hyperlink{share_2template_2gromacs_2gmx__hash_8h}{gmx\-\_\-hash.\-h} }{\pageref{share_2template_2gromacs_2gmx__hash_8h}}{}
\item\contentsline{section}{share/template/gromacs/\hyperlink{share_2template_2gromacs_2gmx__header__config_8h}{gmx\-\_\-header\-\_\-config.\-h} }{\pageref{share_2template_2gromacs_2gmx__header__config_8h}}{}
\item\contentsline{section}{share/template/gromacs/\hyperlink{share_2template_2gromacs_2gmx__lapack_8h}{gmx\-\_\-lapack.\-h} \\*\-Header definitions for the standard \-L\-A\-P\-A\-C\-K library }{\pageref{share_2template_2gromacs_2gmx__lapack_8h}}{}
\item\contentsline{section}{share/template/gromacs/\hyperlink{share_2template_2gromacs_2gmx__math__x86__avx__128__fma__double_8h}{gmx\-\_\-math\-\_\-x86\-\_\-avx\-\_\-128\-\_\-fma\-\_\-double.\-h} }{\pageref{share_2template_2gromacs_2gmx__math__x86__avx__128__fma__double_8h}}{}
\item\contentsline{section}{share/template/gromacs/\hyperlink{share_2template_2gromacs_2gmx__math__x86__avx__128__fma__single_8h}{gmx\-\_\-math\-\_\-x86\-\_\-avx\-\_\-128\-\_\-fma\-\_\-single.\-h} }{\pageref{share_2template_2gromacs_2gmx__math__x86__avx__128__fma__single_8h}}{}
\item\contentsline{section}{share/template/gromacs/\hyperlink{share_2template_2gromacs_2gmx__math__x86__avx__256__double_8h}{gmx\-\_\-math\-\_\-x86\-\_\-avx\-\_\-256\-\_\-double.\-h} }{\pageref{share_2template_2gromacs_2gmx__math__x86__avx__256__double_8h}}{}
\item\contentsline{section}{share/template/gromacs/\hyperlink{share_2template_2gromacs_2gmx__math__x86__avx__256__single_8h}{gmx\-\_\-math\-\_\-x86\-\_\-avx\-\_\-256\-\_\-single.\-h} }{\pageref{share_2template_2gromacs_2gmx__math__x86__avx__256__single_8h}}{}
\item\contentsline{section}{share/template/gromacs/\hyperlink{share_2template_2gromacs_2gmx__math__x86__sse2__double_8h}{gmx\-\_\-math\-\_\-x86\-\_\-sse2\-\_\-double.\-h} }{\pageref{share_2template_2gromacs_2gmx__math__x86__sse2__double_8h}}{}
\item\contentsline{section}{share/template/gromacs/\hyperlink{share_2template_2gromacs_2gmx__math__x86__sse2__single_8h}{gmx\-\_\-math\-\_\-x86\-\_\-sse2\-\_\-single.\-h} }{\pageref{share_2template_2gromacs_2gmx__math__x86__sse2__single_8h}}{}
\item\contentsline{section}{share/template/gromacs/\hyperlink{share_2template_2gromacs_2gmx__math__x86__sse4__1__double_8h}{gmx\-\_\-math\-\_\-x86\-\_\-sse4\-\_\-1\-\_\-double.\-h} }{\pageref{share_2template_2gromacs_2gmx__math__x86__sse4__1__double_8h}}{}
\item\contentsline{section}{share/template/gromacs/\hyperlink{share_2template_2gromacs_2gmx__math__x86__sse4__1__single_8h}{gmx\-\_\-math\-\_\-x86\-\_\-sse4\-\_\-1\-\_\-single.\-h} }{\pageref{share_2template_2gromacs_2gmx__math__x86__sse4__1__single_8h}}{}
\item\contentsline{section}{share/template/gromacs/\hyperlink{share_2template_2gromacs_2gmx__matrix_8h}{gmx\-\_\-matrix.\-h} }{\pageref{share_2template_2gromacs_2gmx__matrix_8h}}{}
\item\contentsline{section}{share/template/gromacs/\hyperlink{share_2template_2gromacs_2gmx__omp_8h}{gmx\-\_\-omp.\-h} }{\pageref{share_2template_2gromacs_2gmx__omp_8h}}{}
\item\contentsline{section}{share/template/gromacs/\hyperlink{share_2template_2gromacs_2gmx__omp__nthreads_8h}{gmx\-\_\-omp\-\_\-nthreads.\-h} }{\pageref{share_2template_2gromacs_2gmx__omp__nthreads_8h}}{}
\item\contentsline{section}{share/template/gromacs/\hyperlink{share_2template_2gromacs_2gmx__parallel__3dfft_8h}{gmx\-\_\-parallel\-\_\-3dfft.\-h} }{\pageref{share_2template_2gromacs_2gmx__parallel__3dfft_8h}}{}
\item\contentsline{section}{share/template/gromacs/\hyperlink{share_2template_2gromacs_2gmx__random_8h}{gmx\-\_\-random.\-h} }{\pageref{share_2template_2gromacs_2gmx__random_8h}}{}
\item\contentsline{section}{share/template/gromacs/\hyperlink{share_2template_2gromacs_2gmx__simd__macros_8h}{gmx\-\_\-simd\-\_\-macros.\-h} }{\pageref{share_2template_2gromacs_2gmx__simd__macros_8h}}{}
\item\contentsline{section}{share/template/gromacs/\hyperlink{share_2template_2gromacs_2gmx__sort_8h}{gmx\-\_\-sort.\-h} }{\pageref{share_2template_2gromacs_2gmx__sort_8h}}{}
\item\contentsline{section}{share/template/gromacs/\hyperlink{share_2template_2gromacs_2gmx__statistics_8h}{gmx\-\_\-statistics.\-h} }{\pageref{share_2template_2gromacs_2gmx__statistics_8h}}{}
\item\contentsline{section}{share/template/gromacs/\hyperlink{share_2template_2gromacs_2gmx__system__xdr_8h}{gmx\-\_\-system\-\_\-xdr.\-h} }{\pageref{share_2template_2gromacs_2gmx__system__xdr_8h}}{}
\item\contentsline{section}{share/template/gromacs/\hyperlink{share_2template_2gromacs_2gmx__thread__affinity_8h}{gmx\-\_\-thread\-\_\-affinity.\-h} }{\pageref{share_2template_2gromacs_2gmx__thread__affinity_8h}}{}
\item\contentsline{section}{share/template/gromacs/\hyperlink{share_2template_2gromacs_2gmx__wallcycle_8h}{gmx\-\_\-wallcycle.\-h} }{\pageref{share_2template_2gromacs_2gmx__wallcycle_8h}}{}
\item\contentsline{section}{share/template/gromacs/\hyperlink{share_2template_2gromacs_2gmx__x86__avx__128__fma_8h}{gmx\-\_\-x86\-\_\-avx\-\_\-128\-\_\-fma.\-h} }{\pageref{share_2template_2gromacs_2gmx__x86__avx__128__fma_8h}}{}
\item\contentsline{section}{share/template/gromacs/\hyperlink{share_2template_2gromacs_2gmx__x86__avx__256_8h}{gmx\-\_\-x86\-\_\-avx\-\_\-256.\-h} }{\pageref{share_2template_2gromacs_2gmx__x86__avx__256_8h}}{}
\item\contentsline{section}{share/template/gromacs/\hyperlink{share_2template_2gromacs_2gmx__x86__simd__double_8h}{gmx\-\_\-x86\-\_\-simd\-\_\-double.\-h} }{\pageref{share_2template_2gromacs_2gmx__x86__simd__double_8h}}{}
\item\contentsline{section}{share/template/gromacs/\hyperlink{share_2template_2gromacs_2gmx__x86__simd__single_8h}{gmx\-\_\-x86\-\_\-simd\-\_\-single.\-h} }{\pageref{share_2template_2gromacs_2gmx__x86__simd__single_8h}}{}
\item\contentsline{section}{share/template/gromacs/\hyperlink{share_2template_2gromacs_2gmx__x86__sse2_8h}{gmx\-\_\-x86\-\_\-sse2.\-h} }{\pageref{share_2template_2gromacs_2gmx__x86__sse2_8h}}{}
\item\contentsline{section}{share/template/gromacs/\hyperlink{share_2template_2gromacs_2gmx__x86__sse4__1_8h}{gmx\-\_\-x86\-\_\-sse4\-\_\-1.\-h} }{\pageref{share_2template_2gromacs_2gmx__x86__sse4__1_8h}}{}
\item\contentsline{section}{share/template/gromacs/\hyperlink{share_2template_2gromacs_2gmxcomplex_8h}{gmxcomplex.\-h} }{\pageref{share_2template_2gromacs_2gmxcomplex_8h}}{}
\item\contentsline{section}{share/template/gromacs/\hyperlink{share_2template_2gromacs_2gmxcpp_8h}{gmxcpp.\-h} }{\pageref{share_2template_2gromacs_2gmxcpp_8h}}{}
\item\contentsline{section}{share/template/gromacs/\hyperlink{share_2template_2gromacs_2gmxfio_8h}{gmxfio.\-h} }{\pageref{share_2template_2gromacs_2gmxfio_8h}}{}
\item\contentsline{section}{share/template/gromacs/\hyperlink{share_2template_2gromacs_2gpp__atomtype_8h}{gpp\-\_\-atomtype.\-h} }{\pageref{share_2template_2gromacs_2gpp__atomtype_8h}}{}
\item\contentsline{section}{share/template/gromacs/\hyperlink{share_2template_2gromacs_2gpp__nextnb_8h}{gpp\-\_\-nextnb.\-h} }{\pageref{share_2template_2gromacs_2gpp__nextnb_8h}}{}
\item\contentsline{section}{share/template/gromacs/\hyperlink{share_2template_2gromacs_2gpu__utils_8h}{gpu\-\_\-utils.\-h} }{\pageref{share_2template_2gromacs_2gpu__utils_8h}}{}
\item\contentsline{section}{share/template/gromacs/\hyperlink{share_2template_2gromacs_2grompp_8h}{grompp.\-h} }{\pageref{share_2template_2gromacs_2grompp_8h}}{}
\item\contentsline{section}{share/template/gromacs/\hyperlink{share_2template_2gromacs_2gstat_8h}{gstat.\-h} }{\pageref{share_2template_2gromacs_2gstat_8h}}{}
\item\contentsline{section}{share/template/gromacs/\hyperlink{share_2template_2gromacs_2hackblock_8h}{hackblock.\-h} }{\pageref{share_2template_2gromacs_2hackblock_8h}}{}
\item\contentsline{section}{share/template/gromacs/\hyperlink{share_2template_2gromacs_2histogram_8h}{histogram.\-h} \\*\-A\-P\-I for calculation of histograms with error estimates }{\pageref{share_2template_2gromacs_2histogram_8h}}{}
\item\contentsline{section}{share/template/gromacs/\hyperlink{share_2template_2gromacs_2index_8h}{index.\-h} }{\pageref{share_2template_2gromacs_2index_8h}}{}
\item\contentsline{section}{share/template/gromacs/\hyperlink{share_2template_2gromacs_2indexutil_8h}{indexutil.\-h} \\*\-A\-P\-I for handling index files and index groups }{\pageref{share_2template_2gromacs_2indexutil_8h}}{}
\item\contentsline{section}{share/template/gromacs/\hyperlink{share_2template_2gromacs_2inputrec_8h}{inputrec.\-h} }{\pageref{share_2template_2gromacs_2inputrec_8h}}{}
\item\contentsline{section}{share/template/gromacs/\hyperlink{share_2template_2gromacs_2invblock_8h}{invblock.\-h} }{\pageref{share_2template_2gromacs_2invblock_8h}}{}
\item\contentsline{section}{share/template/gromacs/\hyperlink{share_2template_2gromacs_2macros_8h}{macros.\-h} }{\pageref{share_2template_2gromacs_2macros_8h}}{}
\item\contentsline{section}{share/template/gromacs/\hyperlink{share_2template_2gromacs_2main_8h}{main.\-h} }{\pageref{share_2template_2gromacs_2main_8h}}{}
\item\contentsline{section}{share/template/gromacs/\hyperlink{share_2template_2gromacs_2mainpage_8h}{mainpage.\-h} \\*\-Dummy header for \-Doxygen documentation }{\pageref{share_2template_2gromacs_2mainpage_8h}}{}
\item\contentsline{section}{share/template/gromacs/\hyperlink{share_2template_2gromacs_2maths_8h}{maths.\-h} }{\pageref{share_2template_2gromacs_2maths_8h}}{}
\item\contentsline{section}{share/template/gromacs/\hyperlink{share_2template_2gromacs_2matio_8h}{matio.\-h} }{\pageref{share_2template_2gromacs_2matio_8h}}{}
\item\contentsline{section}{share/template/gromacs/\hyperlink{share_2template_2gromacs_2md5_8h}{md5.\-h} }{\pageref{share_2template_2gromacs_2md5_8h}}{}
\item\contentsline{section}{share/template/gromacs/\hyperlink{share_2template_2gromacs_2md__logging_8h}{md\-\_\-logging.\-h} }{\pageref{share_2template_2gromacs_2md__logging_8h}}{}
\item\contentsline{section}{share/template/gromacs/\hyperlink{share_2template_2gromacs_2md__support_8h}{md\-\_\-support.\-h} }{\pageref{share_2template_2gromacs_2md__support_8h}}{}
\item\contentsline{section}{share/template/gromacs/\hyperlink{share_2template_2gromacs_2mdatoms_8h}{mdatoms.\-h} }{\pageref{share_2template_2gromacs_2mdatoms_8h}}{}
\item\contentsline{section}{share/template/gromacs/\hyperlink{share_2template_2gromacs_2mdebin_8h}{mdebin.\-h} }{\pageref{share_2template_2gromacs_2mdebin_8h}}{}
\item\contentsline{section}{share/template/gromacs/\hyperlink{share_2template_2gromacs_2mdrun_8h}{mdrun.\-h} }{\pageref{share_2template_2gromacs_2mdrun_8h}}{}
\item\contentsline{section}{share/template/gromacs/\hyperlink{share_2template_2gromacs_2mpelogging_8h}{mpelogging.\-h} }{\pageref{share_2template_2gromacs_2mpelogging_8h}}{}
\item\contentsline{section}{share/template/gromacs/\hyperlink{share_2template_2gromacs_2mshift_8h}{mshift.\-h} }{\pageref{share_2template_2gromacs_2mshift_8h}}{}
\item\contentsline{section}{share/template/gromacs/\hyperlink{share_2template_2gromacs_2mtop__util_8h}{mtop\-\_\-util.\-h} }{\pageref{share_2template_2gromacs_2mtop__util_8h}}{}
\item\contentsline{section}{share/template/gromacs/\hyperlink{share_2template_2gromacs_2mtxio_8h}{mtxio.\-h} }{\pageref{share_2template_2gromacs_2mtxio_8h}}{}
\item\contentsline{section}{share/template/gromacs/\hyperlink{share_2template_2gromacs_2mvdata_8h}{mvdata.\-h} }{\pageref{share_2template_2gromacs_2mvdata_8h}}{}
\item\contentsline{section}{share/template/gromacs/\hyperlink{share_2template_2gromacs_2names_8h}{names.\-h} }{\pageref{share_2template_2gromacs_2names_8h}}{}
\item\contentsline{section}{share/template/gromacs/\hyperlink{share_2template_2gromacs_2nbnxn__cuda__data__mgmt_8h}{nbnxn\-\_\-cuda\-\_\-data\-\_\-mgmt.\-h} }{\pageref{share_2template_2gromacs_2nbnxn__cuda__data__mgmt_8h}}{}
\item\contentsline{section}{share/template/gromacs/\hyperlink{share_2template_2gromacs_2nbsearch_8h}{nbsearch.\-h} \\*\-A\-P\-I for neighborhood searching }{\pageref{share_2template_2gromacs_2nbsearch_8h}}{}
\item\contentsline{section}{share/template/gromacs/\hyperlink{share_2template_2gromacs_2network_8h}{network.\-h} }{\pageref{share_2template_2gromacs_2network_8h}}{}
\item\contentsline{section}{share/template/gromacs/\hyperlink{share_2template_2gromacs_2nonbonded_8h}{nonbonded.\-h} }{\pageref{share_2template_2gromacs_2nonbonded_8h}}{}
\item\contentsline{section}{share/template/gromacs/\hyperlink{share_2template_2gromacs_2nrama_8h}{nrama.\-h} }{\pageref{share_2template_2gromacs_2nrama_8h}}{}
\item\contentsline{section}{share/template/gromacs/\hyperlink{share_2template_2gromacs_2nrjac_8h}{nrjac.\-h} }{\pageref{share_2template_2gromacs_2nrjac_8h}}{}
\item\contentsline{section}{share/template/gromacs/\hyperlink{share_2template_2gromacs_2nrnb_8h}{nrnb.\-h} }{\pageref{share_2template_2gromacs_2nrnb_8h}}{}
\item\contentsline{section}{share/template/gromacs/\hyperlink{share_2template_2gromacs_2ns_8h}{ns.\-h} }{\pageref{share_2template_2gromacs_2ns_8h}}{}
\item\contentsline{section}{share/template/gromacs/\hyperlink{share_2template_2gromacs_2nsgrid_8h}{nsgrid.\-h} }{\pageref{share_2template_2gromacs_2nsgrid_8h}}{}
\item\contentsline{section}{share/template/gromacs/\hyperlink{share_2template_2gromacs_2oenv_8h}{oenv.\-h} }{\pageref{share_2template_2gromacs_2oenv_8h}}{}
\item\contentsline{section}{share/template/gromacs/\hyperlink{share_2template_2gromacs_2orires_8h}{orires.\-h} }{\pageref{share_2template_2gromacs_2orires_8h}}{}
\item\contentsline{section}{share/template/gromacs/\hyperlink{share_2template_2gromacs_2partdec_8h}{partdec.\-h} }{\pageref{share_2template_2gromacs_2partdec_8h}}{}
\item\contentsline{section}{share/template/gromacs/\hyperlink{share_2template_2gromacs_2pbc_8h}{pbc.\-h} }{\pageref{share_2template_2gromacs_2pbc_8h}}{}
\item\contentsline{section}{share/template/gromacs/\hyperlink{share_2template_2gromacs_2pdb2top_8h}{pdb2top.\-h} }{\pageref{share_2template_2gromacs_2pdb2top_8h}}{}
\item\contentsline{section}{share/template/gromacs/\hyperlink{share_2template_2gromacs_2pdbio_8h}{pdbio.\-h} }{\pageref{share_2template_2gromacs_2pdbio_8h}}{}
\item\contentsline{section}{share/template/gromacs/\hyperlink{share_2template_2gromacs_2perf__est_8h}{perf\-\_\-est.\-h} }{\pageref{share_2template_2gromacs_2perf__est_8h}}{}
\item\contentsline{section}{share/template/gromacs/\hyperlink{share_2template_2gromacs_2physics_8h}{physics.\-h} }{\pageref{share_2template_2gromacs_2physics_8h}}{}
\item\contentsline{section}{share/template/gromacs/\hyperlink{share_2template_2gromacs_2pmalloc__cuda_8h}{pmalloc\-\_\-cuda.\-h} }{\pageref{share_2template_2gromacs_2pmalloc__cuda_8h}}{}
\item\contentsline{section}{share/template/gromacs/\hyperlink{share_2template_2gromacs_2pme_8h}{pme.\-h} }{\pageref{share_2template_2gromacs_2pme_8h}}{}
\item\contentsline{section}{share/template/gromacs/\hyperlink{share_2template_2gromacs_2poscalc_8h}{poscalc.\-h} \\*\-A\-P\-I for structured and optimized calculation of positions }{\pageref{share_2template_2gromacs_2poscalc_8h}}{}
\item\contentsline{section}{share/template/gromacs/\hyperlink{share_2template_2gromacs_2position_8h}{position.\-h} \\*\-A\-P\-I for handling positions }{\pageref{share_2template_2gromacs_2position_8h}}{}
\item\contentsline{section}{share/template/gromacs/\hyperlink{share_2template_2gromacs_2princ_8h}{princ.\-h} }{\pageref{share_2template_2gromacs_2princ_8h}}{}
\item\contentsline{section}{share/template/gromacs/\hyperlink{share_2template_2gromacs_2pull_8h}{pull.\-h} }{\pageref{share_2template_2gromacs_2pull_8h}}{}
\item\contentsline{section}{share/template/gromacs/\hyperlink{share_2template_2gromacs_2pull__rotation_8h}{pull\-\_\-rotation.\-h} }{\pageref{share_2template_2gromacs_2pull__rotation_8h}}{}
\item\contentsline{section}{share/template/gromacs/\hyperlink{share_2template_2gromacs_2qmmm_8h}{qmmm.\-h} }{\pageref{share_2template_2gromacs_2qmmm_8h}}{}
\item\contentsline{section}{share/template/gromacs/\hyperlink{share_2template_2gromacs_2random_8h}{random.\-h} }{\pageref{share_2template_2gromacs_2random_8h}}{}
\item\contentsline{section}{share/template/gromacs/\hyperlink{share_2template_2gromacs_2rbin_8h}{rbin.\-h} }{\pageref{share_2template_2gromacs_2rbin_8h}}{}
\item\contentsline{section}{share/template/gromacs/\hyperlink{share_2template_2gromacs_2readinp_8h}{readinp.\-h} }{\pageref{share_2template_2gromacs_2readinp_8h}}{}
\item\contentsline{section}{share/template/gromacs/\hyperlink{share_2template_2gromacs_2resall_8h}{resall.\-h} }{\pageref{share_2template_2gromacs_2resall_8h}}{}
\item\contentsline{section}{share/template/gromacs/\hyperlink{share_2template_2gromacs_2rmpbc_8h}{rmpbc.\-h} }{\pageref{share_2template_2gromacs_2rmpbc_8h}}{}
\item\contentsline{section}{share/template/gromacs/\hyperlink{share_2template_2gromacs_2selection_8h}{selection.\-h} \\*\-A\-P\-I for handling selection (the {\ttfamily \hyperlink{structgmx__ana__selection__t}{gmx\-\_\-ana\-\_\-selection\-\_\-t}} structure and related functions) }{\pageref{share_2template_2gromacs_2selection_8h}}{}
\item\contentsline{section}{share/template/gromacs/\hyperlink{share_2template_2gromacs_2selmethod_8h}{selmethod.\-h} \\*\-A\-P\-I for handling selection methods }{\pageref{share_2template_2gromacs_2selmethod_8h}}{}
\item\contentsline{section}{share/template/gromacs/\hyperlink{share_2template_2gromacs_2selparam_8h}{selparam.\-h} \\*\-A\-P\-I for handling parameters used in selections }{\pageref{share_2template_2gromacs_2selparam_8h}}{}
\item\contentsline{section}{share/template/gromacs/\hyperlink{share_2template_2gromacs_2selvalue_8h}{selvalue.\-h} \\*\-Declaration of {\ttfamily \hyperlink{structgmx__ana__selvalue__t}{gmx\-\_\-ana\-\_\-selvalue\-\_\-t}} }{\pageref{share_2template_2gromacs_2selvalue_8h}}{}
\item\contentsline{section}{share/template/gromacs/\hyperlink{share_2template_2gromacs_2sfactor_8h}{sfactor.\-h} }{\pageref{share_2template_2gromacs_2sfactor_8h}}{}
\item\contentsline{section}{share/template/gromacs/\hyperlink{share_2template_2gromacs_2shellfc_8h}{shellfc.\-h} }{\pageref{share_2template_2gromacs_2shellfc_8h}}{}
\item\contentsline{section}{share/template/gromacs/\hyperlink{share_2template_2gromacs_2shift_8h}{shift.\-h} }{\pageref{share_2template_2gromacs_2shift_8h}}{}
\item\contentsline{section}{share/template/gromacs/\hyperlink{share_2template_2gromacs_2sighandler_8h}{sighandler.\-h} }{\pageref{share_2template_2gromacs_2sighandler_8h}}{}
\item\contentsline{section}{share/template/gromacs/\hyperlink{share_2template_2gromacs_2sim__util_8h}{sim\-\_\-util.\-h} }{\pageref{share_2template_2gromacs_2sim__util_8h}}{}
\item\contentsline{section}{share/template/gromacs/\hyperlink{share_2template_2gromacs_2smalloc_8h}{smalloc.\-h} }{\pageref{share_2template_2gromacs_2smalloc_8h}}{}
\item\contentsline{section}{share/template/gromacs/\hyperlink{share_2template_2gromacs_2sortwater_8h}{sortwater.\-h} }{\pageref{share_2template_2gromacs_2sortwater_8h}}{}
\item\contentsline{section}{share/template/gromacs/\hyperlink{share_2template_2gromacs_2sparsematrix_8h}{sparsematrix.\-h} }{\pageref{share_2template_2gromacs_2sparsematrix_8h}}{}
\item\contentsline{section}{share/template/gromacs/\hyperlink{share_2template_2gromacs_2split_8h}{split.\-h} }{\pageref{share_2template_2gromacs_2split_8h}}{}
\item\contentsline{section}{share/template/gromacs/\hyperlink{share_2template_2gromacs_2splitter_8h}{splitter.\-h} }{\pageref{share_2template_2gromacs_2splitter_8h}}{}
\item\contentsline{section}{share/template/gromacs/\hyperlink{share_2template_2gromacs_2statutil_8h}{statutil.\-h} }{\pageref{share_2template_2gromacs_2statutil_8h}}{}
\item\contentsline{section}{share/template/gromacs/\hyperlink{share_2template_2gromacs_2strdb_8h}{strdb.\-h} }{\pageref{share_2template_2gromacs_2strdb_8h}}{}
\item\contentsline{section}{share/template/gromacs/\hyperlink{share_2template_2gromacs_2string2_8h}{string2.\-h} \\*\-Generic string handling functions }{\pageref{share_2template_2gromacs_2string2_8h}}{}
\item\contentsline{section}{share/template/gromacs/\hyperlink{share_2template_2gromacs_2symtab_8h}{symtab.\-h} }{\pageref{share_2template_2gromacs_2symtab_8h}}{}
\item\contentsline{section}{share/template/gromacs/\hyperlink{share_2template_2gromacs_2sysstuff_8h}{sysstuff.\-h} }{\pageref{share_2template_2gromacs_2sysstuff_8h}}{}
\item\contentsline{section}{share/template/gromacs/\hyperlink{share_2template_2gromacs_2tables_8h}{tables.\-h} }{\pageref{share_2template_2gromacs_2tables_8h}}{}
\item\contentsline{section}{share/template/gromacs/\hyperlink{share_2template_2gromacs_2tgroup_8h}{tgroup.\-h} }{\pageref{share_2template_2gromacs_2tgroup_8h}}{}
\item\contentsline{section}{share/template/gromacs/\hyperlink{share_2template_2gromacs_2thread__mpi_8h}{thread\-\_\-mpi.\-h} \\*\-Convenience header file for non-\/\-M\-P\-I compatibility }{\pageref{share_2template_2gromacs_2thread__mpi_8h}}{}
\item\contentsline{section}{share/template/gromacs/\hyperlink{share_2template_2gromacs_2tmpi_8h}{tmpi.\-h} \\*\-Convenience header file for \-M\-P\-I compatibility }{\pageref{share_2template_2gromacs_2tmpi_8h}}{}
\item\contentsline{section}{share/template/gromacs/\hyperlink{share_2template_2gromacs_2topsort_8h}{topsort.\-h} }{\pageref{share_2template_2gromacs_2topsort_8h}}{}
\item\contentsline{section}{share/template/gromacs/\hyperlink{share_2template_2gromacs_2toputil_8h}{toputil.\-h} }{\pageref{share_2template_2gromacs_2toputil_8h}}{}
\item\contentsline{section}{share/template/gromacs/\hyperlink{share_2template_2gromacs_2tpxio_8h}{tpxio.\-h} }{\pageref{share_2template_2gromacs_2tpxio_8h}}{}
\item\contentsline{section}{share/template/gromacs/\hyperlink{share_2template_2gromacs_2trajana_8h}{trajana.\-h} \\*\-Main \-A\-P\-I of the trajectory analysis library }{\pageref{share_2template_2gromacs_2trajana_8h}}{}
\item\contentsline{section}{share/template/gromacs/\hyperlink{share_2template_2gromacs_2trnio_8h}{trnio.\-h} }{\pageref{share_2template_2gromacs_2trnio_8h}}{}
\item\contentsline{section}{share/template/gromacs/\hyperlink{share_2template_2gromacs_2txtdump_8h}{txtdump.\-h} }{\pageref{share_2template_2gromacs_2txtdump_8h}}{}
\item\contentsline{section}{share/template/gromacs/\hyperlink{share_2template_2gromacs_2typedefs_8h}{typedefs.\-h} }{\pageref{share_2template_2gromacs_2typedefs_8h}}{}
\item\contentsline{section}{share/template/gromacs/\hyperlink{share_2template_2gromacs_2update_8h}{update.\-h} }{\pageref{share_2template_2gromacs_2update_8h}}{}
\item\contentsline{section}{share/template/gromacs/\hyperlink{share_2template_2gromacs_2vcm_8h}{vcm.\-h} }{\pageref{share_2template_2gromacs_2vcm_8h}}{}
\item\contentsline{section}{share/template/gromacs/\hyperlink{share_2template_2gromacs_2vec_8h}{vec.\-h} }{\pageref{share_2template_2gromacs_2vec_8h}}{}
\item\contentsline{section}{share/template/gromacs/\hyperlink{share_2template_2gromacs_2version_8h}{version.\-h} }{\pageref{share_2template_2gromacs_2version_8h}}{}
\item\contentsline{section}{share/template/gromacs/\hyperlink{share_2template_2gromacs_2viewit_8h}{viewit.\-h} }{\pageref{share_2template_2gromacs_2viewit_8h}}{}
\item\contentsline{section}{share/template/gromacs/\hyperlink{share_2template_2gromacs_2visibility_8h}{visibility.\-h} }{\pageref{share_2template_2gromacs_2visibility_8h}}{}
\item\contentsline{section}{share/template/gromacs/\hyperlink{share_2template_2gromacs_2vsite_8h}{vsite.\-h} }{\pageref{share_2template_2gromacs_2vsite_8h}}{}
\item\contentsline{section}{share/template/gromacs/\hyperlink{share_2template_2gromacs_2warninp_8h}{warninp.\-h} }{\pageref{share_2template_2gromacs_2warninp_8h}}{}
\item\contentsline{section}{share/template/gromacs/\hyperlink{share_2template_2gromacs_2wgms_8h}{wgms.\-h} }{\pageref{share_2template_2gromacs_2wgms_8h}}{}
\item\contentsline{section}{share/template/gromacs/\hyperlink{share_2template_2gromacs_2wman_8h}{wman.\-h} }{\pageref{share_2template_2gromacs_2wman_8h}}{}
\item\contentsline{section}{share/template/gromacs/\hyperlink{share_2template_2gromacs_2writeps_8h}{writeps.\-h} }{\pageref{share_2template_2gromacs_2writeps_8h}}{}
\item\contentsline{section}{share/template/gromacs/\hyperlink{share_2template_2gromacs_2xdrf_8h}{xdrf.\-h} }{\pageref{share_2template_2gromacs_2xdrf_8h}}{}
\item\contentsline{section}{share/template/gromacs/\hyperlink{share_2template_2gromacs_2xtcio_8h}{xtcio.\-h} }{\pageref{share_2template_2gromacs_2xtcio_8h}}{}
\item\contentsline{section}{share/template/gromacs/\hyperlink{share_2template_2gromacs_2xvgr_8h}{xvgr.\-h} }{\pageref{share_2template_2gromacs_2xvgr_8h}}{}
\item\contentsline{section}{share/template/gromacs/thread\-\_\-mpi/\hyperlink{share_2template_2gromacs_2thread__mpi_2atomic_8h}{atomic.\-h} }{\pageref{share_2template_2gromacs_2thread__mpi_2atomic_8h}}{}
\item\contentsline{section}{share/template/gromacs/thread\-\_\-mpi/\hyperlink{share_2template_2gromacs_2thread__mpi_2barrier_8h}{barrier.\-h} }{\pageref{share_2template_2gromacs_2thread__mpi_2barrier_8h}}{}
\item\contentsline{section}{share/template/gromacs/thread\-\_\-mpi/\hyperlink{share_2template_2gromacs_2thread__mpi_2collective_8h}{collective.\-h} \\*\-Collective functions }{\pageref{share_2template_2gromacs_2thread__mpi_2collective_8h}}{}
\item\contentsline{section}{share/template/gromacs/thread\-\_\-mpi/\hyperlink{share_2template_2gromacs_2thread__mpi_2event_8h}{event.\-h} \\*\-Event notification wait and signaling functions }{\pageref{share_2template_2gromacs_2thread__mpi_2event_8h}}{}
\item\contentsline{section}{share/template/gromacs/thread\-\_\-mpi/\hyperlink{share_2template_2gromacs_2thread__mpi_2hwinfo_8h}{hwinfo.\-h} \\*\-C\-P\-U/core/\-H\-T count function }{\pageref{share_2template_2gromacs_2thread__mpi_2hwinfo_8h}}{}
\item\contentsline{section}{share/template/gromacs/thread\-\_\-mpi/\hyperlink{share_2template_2gromacs_2thread__mpi_2list_8h}{list.\-h} \\*\-Lock-\/free list data structures }{\pageref{share_2template_2gromacs_2thread__mpi_2list_8h}}{}
\item\contentsline{section}{share/template/gromacs/thread\-\_\-mpi/\hyperlink{share_2template_2gromacs_2thread__mpi_2lock_8h}{lock.\-h} }{\pageref{share_2template_2gromacs_2thread__mpi_2lock_8h}}{}
\item\contentsline{section}{share/template/gromacs/thread\-\_\-mpi/\hyperlink{share_2template_2gromacs_2thread__mpi_2mpi__bindings_8h}{mpi\-\_\-bindings.\-h} \\*\-M\-P\-I bindings for thread\-\_\-mpi/tmpi.\-h }{\pageref{share_2template_2gromacs_2thread__mpi_2mpi__bindings_8h}}{}
\item\contentsline{section}{share/template/gromacs/thread\-\_\-mpi/\hyperlink{share_2template_2gromacs_2thread__mpi_2mutex_8h}{mutex.\-h} \\*\-Mutex objects with \-C++11 \-A\-P\-I compatibility }{\pageref{share_2template_2gromacs_2thread__mpi_2mutex_8h}}{}
\item\contentsline{section}{share/template/gromacs/thread\-\_\-mpi/\hyperlink{share_2template_2gromacs_2thread__mpi_2numa__malloc_8h}{numa\-\_\-malloc.\-h} \\*\-N\-U\-M\-A aware memory allocators }{\pageref{share_2template_2gromacs_2thread__mpi_2numa__malloc_8h}}{}
\item\contentsline{section}{share/template/gromacs/thread\-\_\-mpi/\hyperlink{share_2template_2gromacs_2thread__mpi_2system__error_8h}{system\-\_\-error.\-h} \\*\-A \-C++11 compatible system\-\_\-error class for reporting exceptions }{\pageref{share_2template_2gromacs_2thread__mpi_2system__error_8h}}{}
\item\contentsline{section}{share/template/gromacs/thread\-\_\-mpi/\hyperlink{share_2template_2gromacs_2thread__mpi_2threads_8h}{threads.\-h} }{\pageref{share_2template_2gromacs_2thread__mpi_2threads_8h}}{}
\item\contentsline{section}{share/template/gromacs/thread\-\_\-mpi/\hyperlink{share_2template_2gromacs_2thread__mpi_2tmpi_8h}{tmpi.\-h} \\*\-Partial implementation of \-M\-P\-I using only threads }{\pageref{share_2template_2gromacs_2thread__mpi_2tmpi_8h}}{}
\item\contentsline{section}{share/template/gromacs/thread\-\_\-mpi/\hyperlink{share_2template_2gromacs_2thread__mpi_2visibility_8h}{visibility.\-h} \\*\-Visibility macros }{\pageref{share_2template_2gromacs_2thread__mpi_2visibility_8h}}{}
\item\contentsline{section}{share/template/gromacs/thread\-\_\-mpi/\hyperlink{share_2template_2gromacs_2thread__mpi_2wait_8h}{wait.\-h} }{\pageref{share_2template_2gromacs_2thread__mpi_2wait_8h}}{}
\item\contentsline{section}{share/template/gromacs/thread\-\_\-mpi/atomic/\hyperlink{share_2template_2gromacs_2thread__mpi_2atomic_2cycles_8h}{cycles.\-h} }{\pageref{share_2template_2gromacs_2thread__mpi_2atomic_2cycles_8h}}{}
\item\contentsline{section}{share/template/gromacs/thread\-\_\-mpi/atomic/\hyperlink{share_2template_2gromacs_2thread__mpi_2atomic_2gcc_8h}{gcc.\-h} }{\pageref{share_2template_2gromacs_2thread__mpi_2atomic_2gcc_8h}}{}
\item\contentsline{section}{share/template/gromacs/thread\-\_\-mpi/atomic/\hyperlink{share_2template_2gromacs_2thread__mpi_2atomic_2gcc__ia64_8h}{gcc\-\_\-ia64.\-h} }{\pageref{share_2template_2gromacs_2thread__mpi_2atomic_2gcc__ia64_8h}}{}
\item\contentsline{section}{share/template/gromacs/thread\-\_\-mpi/atomic/\hyperlink{share_2template_2gromacs_2thread__mpi_2atomic_2gcc__intrinsics_8h}{gcc\-\_\-intrinsics.\-h} }{\pageref{share_2template_2gromacs_2thread__mpi_2atomic_2gcc__intrinsics_8h}}{}
\item\contentsline{section}{share/template/gromacs/thread\-\_\-mpi/atomic/\hyperlink{share_2template_2gromacs_2thread__mpi_2atomic_2gcc__ppc_8h}{gcc\-\_\-ppc.\-h} }{\pageref{share_2template_2gromacs_2thread__mpi_2atomic_2gcc__ppc_8h}}{}
\item\contentsline{section}{share/template/gromacs/thread\-\_\-mpi/atomic/\hyperlink{share_2template_2gromacs_2thread__mpi_2atomic_2gcc__spinlock_8h}{gcc\-\_\-spinlock.\-h} }{\pageref{share_2template_2gromacs_2thread__mpi_2atomic_2gcc__spinlock_8h}}{}
\item\contentsline{section}{share/template/gromacs/thread\-\_\-mpi/atomic/\hyperlink{share_2template_2gromacs_2thread__mpi_2atomic_2gcc__x86_8h}{gcc\-\_\-x86.\-h} }{\pageref{share_2template_2gromacs_2thread__mpi_2atomic_2gcc__x86_8h}}{}
\item\contentsline{section}{share/template/gromacs/thread\-\_\-mpi/atomic/\hyperlink{share_2template_2gromacs_2thread__mpi_2atomic_2msvc_8h}{msvc.\-h} }{\pageref{share_2template_2gromacs_2thread__mpi_2atomic_2msvc_8h}}{}
\item\contentsline{section}{share/template/gromacs/thread\-\_\-mpi/atomic/\hyperlink{share_2template_2gromacs_2thread__mpi_2atomic_2suncc-sparc_8h}{suncc-\/sparc.\-h} }{\pageref{share_2template_2gromacs_2thread__mpi_2atomic_2suncc-sparc_8h}}{}
\item\contentsline{section}{share/template/gromacs/thread\-\_\-mpi/atomic/\hyperlink{share_2template_2gromacs_2thread__mpi_2atomic_2xlc__ppc_8h}{xlc\-\_\-ppc.\-h} }{\pageref{share_2template_2gromacs_2thread__mpi_2atomic_2xlc__ppc_8h}}{}
\item\contentsline{section}{share/template/gromacs/types/\hyperlink{share_2template_2gromacs_2types_2atoms_8h}{atoms.\-h} }{\pageref{share_2template_2gromacs_2types_2atoms_8h}}{}
\item\contentsline{section}{share/template/gromacs/types/\hyperlink{share_2template_2gromacs_2types_2block_8h}{block.\-h} }{\pageref{share_2template_2gromacs_2types_2block_8h}}{}
\item\contentsline{section}{share/template/gromacs/types/\hyperlink{share_2template_2gromacs_2types_2commrec_8h}{commrec.\-h} }{\pageref{share_2template_2gromacs_2types_2commrec_8h}}{}
\item\contentsline{section}{share/template/gromacs/types/\hyperlink{share_2template_2gromacs_2types_2constr_8h}{constr.\-h} }{\pageref{share_2template_2gromacs_2types_2constr_8h}}{}
\item\contentsline{section}{share/template/gromacs/types/\hyperlink{share_2template_2gromacs_2types_2energy_8h}{energy.\-h} }{\pageref{share_2template_2gromacs_2types_2energy_8h}}{}
\item\contentsline{section}{share/template/gromacs/types/\hyperlink{share_2template_2gromacs_2types_2enums_8h}{enums.\-h} }{\pageref{share_2template_2gromacs_2types_2enums_8h}}{}
\item\contentsline{section}{share/template/gromacs/types/\hyperlink{share_2template_2gromacs_2types_2fcdata_8h}{fcdata.\-h} }{\pageref{share_2template_2gromacs_2types_2fcdata_8h}}{}
\item\contentsline{section}{share/template/gromacs/types/\hyperlink{share_2template_2gromacs_2types_2filenm_8h}{filenm.\-h} }{\pageref{share_2template_2gromacs_2types_2filenm_8h}}{}
\item\contentsline{section}{share/template/gromacs/types/\hyperlink{share_2template_2gromacs_2types_2force__flags_8h}{force\-\_\-flags.\-h} }{\pageref{share_2template_2gromacs_2types_2force__flags_8h}}{}
\item\contentsline{section}{share/template/gromacs/types/\hyperlink{share_2template_2gromacs_2types_2forcerec_8h}{forcerec.\-h} }{\pageref{share_2template_2gromacs_2types_2forcerec_8h}}{}
\item\contentsline{section}{share/template/gromacs/types/\hyperlink{share_2template_2gromacs_2types_2genborn_8h}{genborn.\-h} }{\pageref{share_2template_2gromacs_2types_2genborn_8h}}{}
\item\contentsline{section}{share/template/gromacs/types/\hyperlink{share_2template_2gromacs_2types_2globsig_8h}{globsig.\-h} }{\pageref{share_2template_2gromacs_2types_2globsig_8h}}{}
\item\contentsline{section}{share/template/gromacs/types/\hyperlink{share_2template_2gromacs_2types_2graph_8h}{graph.\-h} }{\pageref{share_2template_2gromacs_2types_2graph_8h}}{}
\item\contentsline{section}{share/template/gromacs/types/\hyperlink{share_2template_2gromacs_2types_2group_8h}{group.\-h} }{\pageref{share_2template_2gromacs_2types_2group_8h}}{}
\item\contentsline{section}{share/template/gromacs/types/\hyperlink{share_2template_2gromacs_2types_2hw__info_8h}{hw\-\_\-info.\-h} }{\pageref{share_2template_2gromacs_2types_2hw__info_8h}}{}
\item\contentsline{section}{share/template/gromacs/types/\hyperlink{share_2template_2gromacs_2types_2idef_8h}{idef.\-h} }{\pageref{share_2template_2gromacs_2types_2idef_8h}}{}
\item\contentsline{section}{share/template/gromacs/types/\hyperlink{share_2template_2gromacs_2types_2ifunc_8h}{ifunc.\-h} }{\pageref{share_2template_2gromacs_2types_2ifunc_8h}}{}
\item\contentsline{section}{share/template/gromacs/types/\hyperlink{share_2template_2gromacs_2types_2inputrec_8h}{inputrec.\-h} }{\pageref{share_2template_2gromacs_2types_2inputrec_8h}}{}
\item\contentsline{section}{share/template/gromacs/types/\hyperlink{share_2template_2gromacs_2types_2interaction__const_8h}{interaction\-\_\-const.\-h} }{\pageref{share_2template_2gromacs_2types_2interaction__const_8h}}{}
\item\contentsline{section}{share/template/gromacs/types/\hyperlink{share_2template_2gromacs_2types_2ishift_8h}{ishift.\-h} }{\pageref{share_2template_2gromacs_2types_2ishift_8h}}{}
\item\contentsline{section}{share/template/gromacs/types/\hyperlink{share_2template_2gromacs_2types_2iteratedconstraints_8h}{iteratedconstraints.\-h} }{\pageref{share_2template_2gromacs_2types_2iteratedconstraints_8h}}{}
\item\contentsline{section}{share/template/gromacs/types/\hyperlink{share_2template_2gromacs_2types_2matrix_8h}{matrix.\-h} }{\pageref{share_2template_2gromacs_2types_2matrix_8h}}{}
\item\contentsline{section}{share/template/gromacs/types/\hyperlink{share_2template_2gromacs_2types_2mdatom_8h}{mdatom.\-h} }{\pageref{share_2template_2gromacs_2types_2mdatom_8h}}{}
\item\contentsline{section}{share/template/gromacs/types/\hyperlink{share_2template_2gromacs_2types_2membedt_8h}{membedt.\-h} }{\pageref{share_2template_2gromacs_2types_2membedt_8h}}{}
\item\contentsline{section}{share/template/gromacs/types/\hyperlink{share_2template_2gromacs_2types_2nb__verlet_8h}{nb\-\_\-verlet.\-h} }{\pageref{share_2template_2gromacs_2types_2nb__verlet_8h}}{}
\item\contentsline{section}{share/template/gromacs/types/\hyperlink{share_2template_2gromacs_2types_2nblist_8h}{nblist.\-h} }{\pageref{share_2template_2gromacs_2types_2nblist_8h}}{}
\item\contentsline{section}{share/template/gromacs/types/\hyperlink{share_2template_2gromacs_2types_2nbnxn__cuda__types__ext_8h}{nbnxn\-\_\-cuda\-\_\-types\-\_\-ext.\-h} }{\pageref{share_2template_2gromacs_2types_2nbnxn__cuda__types__ext_8h}}{}
\item\contentsline{section}{share/template/gromacs/types/\hyperlink{share_2template_2gromacs_2types_2nbnxn__pairlist_8h}{nbnxn\-\_\-pairlist.\-h} }{\pageref{share_2template_2gromacs_2types_2nbnxn__pairlist_8h}}{}
\item\contentsline{section}{share/template/gromacs/types/\hyperlink{share_2template_2gromacs_2types_2nlistheuristics_8h}{nlistheuristics.\-h} }{\pageref{share_2template_2gromacs_2types_2nlistheuristics_8h}}{}
\item\contentsline{section}{share/template/gromacs/types/\hyperlink{share_2template_2gromacs_2types_2nrnb_8h}{nrnb.\-h} }{\pageref{share_2template_2gromacs_2types_2nrnb_8h}}{}
\item\contentsline{section}{share/template/gromacs/types/\hyperlink{share_2template_2gromacs_2types_2ns_8h}{ns.\-h} }{\pageref{share_2template_2gromacs_2types_2ns_8h}}{}
\item\contentsline{section}{share/template/gromacs/types/\hyperlink{share_2template_2gromacs_2types_2nsgrid_8h}{nsgrid.\-h} }{\pageref{share_2template_2gromacs_2types_2nsgrid_8h}}{}
\item\contentsline{section}{share/template/gromacs/types/\hyperlink{share_2template_2gromacs_2types_2oenv_8h}{oenv.\-h} }{\pageref{share_2template_2gromacs_2types_2oenv_8h}}{}
\item\contentsline{section}{share/template/gromacs/types/\hyperlink{share_2template_2gromacs_2types_2pbc_8h}{pbc.\-h} }{\pageref{share_2template_2gromacs_2types_2pbc_8h}}{}
\item\contentsline{section}{share/template/gromacs/types/\hyperlink{share_2template_2gromacs_2types_2qmmmrec_8h}{qmmmrec.\-h} }{\pageref{share_2template_2gromacs_2types_2qmmmrec_8h}}{}
\item\contentsline{section}{share/template/gromacs/types/\hyperlink{share_2template_2gromacs_2types_2shellfc_8h}{shellfc.\-h} }{\pageref{share_2template_2gromacs_2types_2shellfc_8h}}{}
\item\contentsline{section}{share/template/gromacs/types/\hyperlink{share_2template_2gromacs_2types_2simple_8h}{simple.\-h} }{\pageref{share_2template_2gromacs_2types_2simple_8h}}{}
\item\contentsline{section}{share/template/gromacs/types/\hyperlink{share_2template_2gromacs_2types_2state_8h}{state.\-h} }{\pageref{share_2template_2gromacs_2types_2state_8h}}{}
\item\contentsline{section}{share/template/gromacs/types/\hyperlink{share_2template_2gromacs_2types_2symtab_8h}{symtab.\-h} }{\pageref{share_2template_2gromacs_2types_2symtab_8h}}{}
\item\contentsline{section}{share/template/gromacs/types/\hyperlink{share_2template_2gromacs_2types_2topology_8h}{topology.\-h} }{\pageref{share_2template_2gromacs_2types_2topology_8h}}{}
\item\contentsline{section}{share/template/gromacs/types/\hyperlink{share_2template_2gromacs_2types_2trx_8h}{trx.\-h} }{\pageref{share_2template_2gromacs_2types_2trx_8h}}{}
\item\contentsline{section}{src/\hyperlink{buildinfo_8h}{buildinfo.\-h} }{\pageref{buildinfo_8h}}{}
\item\contentsline{section}{src/\hyperlink{config_8h}{config.\-h} }{\pageref{config_8h}}{}
\item\contentsline{section}{src/contrib/\hyperlink{anaf_8c}{anaf.\-c} }{\pageref{anaf_8c}}{}
\item\contentsline{section}{src/contrib/\hyperlink{calcfdev_8c}{calcfdev.\-c} }{\pageref{calcfdev_8c}}{}
\item\contentsline{section}{src/contrib/\hyperlink{compnl_8c}{compnl.\-c} }{\pageref{compnl_8c}}{}
\item\contentsline{section}{src/contrib/\hyperlink{copyrgt_8c}{copyrgt.\-c} }{\pageref{copyrgt_8c}}{}
\item\contentsline{section}{src/contrib/\hyperlink{do__multiprot_8c}{do\-\_\-multiprot.\-c} }{\pageref{do__multiprot_8c}}{}
\item\contentsline{section}{src/contrib/\hyperlink{do__shift_8c}{do\-\_\-shift.\-c} }{\pageref{do__shift_8c}}{}
\item\contentsline{section}{src/contrib/\hyperlink{ehanal_8c}{ehanal.\-c} }{\pageref{ehanal_8c}}{}
\item\contentsline{section}{src/contrib/\hyperlink{ehdata_8c}{ehdata.\-c} }{\pageref{ehdata_8c}}{}
\item\contentsline{section}{src/contrib/\hyperlink{ehdata_8h}{ehdata.\-h} }{\pageref{ehdata_8h}}{}
\item\contentsline{section}{src/contrib/\hyperlink{ehole_8c}{ehole.\-c} }{\pageref{ehole_8c}}{}
\item\contentsline{section}{src/contrib/\hyperlink{g__anavel_8c}{g\-\_\-anavel.\-c} }{\pageref{g__anavel_8c}}{}
\item\contentsline{section}{src/contrib/\hyperlink{g__sdf_8c}{g\-\_\-sdf.\-c} }{\pageref{g__sdf_8c}}{}
\item\contentsline{section}{src/contrib/\hyperlink{gen__table_8c}{gen\-\_\-table.\-c} }{\pageref{gen__table_8c}}{}
\item\contentsline{section}{src/contrib/\hyperlink{gmx__sdf_8c}{gmx\-\_\-sdf.\-c} }{\pageref{gmx__sdf_8c}}{}
\item\contentsline{section}{src/contrib/\hyperlink{gmx__stats__test_8c}{gmx\-\_\-stats\-\_\-test.\-c} }{\pageref{gmx__stats__test_8c}}{}
\item\contentsline{section}{src/contrib/\hyperlink{hexamer_8c}{hexamer.\-c} }{\pageref{hexamer_8c}}{}
\item\contentsline{section}{src/contrib/\hyperlink{hrefify_8c}{hrefify.\-c} }{\pageref{hrefify_8c}}{}
\item\contentsline{section}{src/contrib/\hyperlink{md__openmm_8c}{md\-\_\-openmm.\-c} }{\pageref{md__openmm_8c}}{}
\item\contentsline{section}{src/contrib/\hyperlink{md__openmm_8h}{md\-\_\-openmm.\-h} }{\pageref{md__openmm_8h}}{}
\item\contentsline{section}{src/contrib/\hyperlink{mdrun__openmm_8c}{mdrun\-\_\-openmm.\-c} }{\pageref{mdrun__openmm_8c}}{}
\item\contentsline{section}{src/contrib/\hyperlink{mkice_8c}{mkice.\-c} }{\pageref{mkice_8c}}{}
\item\contentsline{section}{src/contrib/\hyperlink{openmm__gpu__utils_8h}{openmm\-\_\-gpu\-\_\-utils.\-h} }{\pageref{openmm__gpu__utils_8h}}{}
\item\contentsline{section}{src/contrib/\hyperlink{openmm__wrapper_8h}{openmm\-\_\-wrapper.\-h} }{\pageref{openmm__wrapper_8h}}{}
\item\contentsline{section}{src/contrib/\hyperlink{optwat_8c}{optwat.\-c} }{\pageref{optwat_8c}}{}
\item\contentsline{section}{src/contrib/\hyperlink{pmetest_8c}{pmetest.\-c} }{\pageref{pmetest_8c}}{}
\item\contentsline{section}{src/contrib/\hyperlink{prfn_8c}{prfn.\-c} }{\pageref{prfn_8c}}{}
\item\contentsline{section}{src/contrib/\hyperlink{contrib_2random_8c}{random.\-c} }{\pageref{contrib_2random_8c}}{}
\item\contentsline{section}{src/contrib/\hyperlink{runner__openmm_8c}{runner\-\_\-openmm.\-c} }{\pageref{runner__openmm_8c}}{}
\item\contentsline{section}{src/contrib/\hyperlink{test_8c}{test.\-c} }{\pageref{test_8c}}{}
\item\contentsline{section}{src/contrib/\hyperlink{test__fatal_8c}{test\-\_\-fatal.\-c} }{\pageref{test__fatal_8c}}{}
\item\contentsline{section}{src/contrib/\hyperlink{testfft_8c}{testfft.\-c} }{\pageref{testfft_8c}}{}
\item\contentsline{section}{src/contrib/\hyperlink{testlr_8c}{testlr.\-c} }{\pageref{testlr_8c}}{}
\item\contentsline{section}{src/contrib/\hyperlink{testtab_8c}{testtab.\-c} }{\pageref{testtab_8c}}{}
\item\contentsline{section}{src/contrib/\hyperlink{timefft_8c}{timefft.\-c} }{\pageref{timefft_8c}}{}
\item\contentsline{section}{src/contrib/\-C\-Make\-Files/\-Compiler\-Id\-C/\hyperlink{src_2contrib_2CMakeFiles_2CompilerIdC_2CMakeCCompilerId_8c}{\-C\-Make\-C\-Compiler\-Id.\-c} }{\pageref{src_2contrib_2CMakeFiles_2CompilerIdC_2CMakeCCompilerId_8c}}{}
\item\contentsline{section}{src/gmxlib/\hyperlink{3dview_8c}{3dview.\-c} }{\pageref{3dview_8c}}{}
\item\contentsline{section}{src/gmxlib/\hyperlink{atomprop_8c}{atomprop.\-c} }{\pageref{atomprop_8c}}{}
\item\contentsline{section}{src/gmxlib/\hyperlink{bondfree_8c}{bondfree.\-c} }{\pageref{bondfree_8c}}{}
\item\contentsline{section}{src/gmxlib/\hyperlink{calcgrid_8c}{calcgrid.\-c} }{\pageref{calcgrid_8c}}{}
\item\contentsline{section}{src/gmxlib/\hyperlink{calch_8c}{calch.\-c} }{\pageref{calch_8c}}{}
\item\contentsline{section}{src/gmxlib/\hyperlink{chargegroup_8c}{chargegroup.\-c} }{\pageref{chargegroup_8c}}{}
\item\contentsline{section}{src/gmxlib/\hyperlink{checkpoint_8c}{checkpoint.\-c} }{\pageref{checkpoint_8c}}{}
\item\contentsline{section}{src/gmxlib/\hyperlink{cinvsqrtdata_8c}{cinvsqrtdata.\-c} }{\pageref{cinvsqrtdata_8c}}{}
\item\contentsline{section}{src/gmxlib/\hyperlink{confio_8c}{confio.\-c} }{\pageref{confio_8c}}{}
\item\contentsline{section}{src/gmxlib/\hyperlink{copyrite_8c}{copyrite.\-c} }{\pageref{copyrite_8c}}{}
\item\contentsline{section}{src/gmxlib/\hyperlink{crecipdata_8c}{crecipdata.\-c} }{\pageref{crecipdata_8c}}{}
\item\contentsline{section}{src/gmxlib/\hyperlink{debugb_8h}{debugb.\-h} }{\pageref{debugb_8h}}{}
\item\contentsline{section}{src/gmxlib/\hyperlink{disre_8c}{disre.\-c} }{\pageref{disre_8c}}{}
\item\contentsline{section}{src/gmxlib/\hyperlink{dlb_8h}{dlb.\-h} }{\pageref{dlb_8h}}{}
\item\contentsline{section}{src/gmxlib/\hyperlink{do__fit_8c}{do\-\_\-fit.\-c} }{\pageref{do__fit_8c}}{}
\item\contentsline{section}{src/gmxlib/\hyperlink{enxio_8c}{enxio.\-c} }{\pageref{enxio_8c}}{}
\item\contentsline{section}{src/gmxlib/\hyperlink{ewald__util_8c}{ewald\-\_\-util.\-c} }{\pageref{ewald__util_8c}}{}
\item\contentsline{section}{src/gmxlib/\hyperlink{ffscanf_8c}{ffscanf.\-c} }{\pageref{ffscanf_8c}}{}
\item\contentsline{section}{src/gmxlib/\hyperlink{filenm_8c}{filenm.\-c} }{\pageref{filenm_8c}}{}
\item\contentsline{section}{src/gmxlib/\hyperlink{futil_8c}{futil.\-c} }{\pageref{futil_8c}}{}
\item\contentsline{section}{src/gmxlib/\hyperlink{gbutil_8c}{gbutil.\-c} }{\pageref{gbutil_8c}}{}
\item\contentsline{section}{src/gmxlib/\hyperlink{gmx__arpack_8c}{gmx\-\_\-arpack.\-c} }{\pageref{gmx__arpack_8c}}{}
\item\contentsline{section}{src/gmxlib/\hyperlink{gmx__cpuid_8c}{gmx\-\_\-cpuid.\-c} }{\pageref{gmx__cpuid_8c}}{}
\item\contentsline{section}{src/gmxlib/\hyperlink{gmx__cyclecounter_8c}{gmx\-\_\-cyclecounter.\-c} }{\pageref{gmx__cyclecounter_8c}}{}
\item\contentsline{section}{src/gmxlib/\hyperlink{gmx__detect__hardware_8c}{gmx\-\_\-detect\-\_\-hardware.\-c} }{\pageref{gmx__detect__hardware_8c}}{}
\item\contentsline{section}{src/gmxlib/\hyperlink{gmx__fatal_8c}{gmx\-\_\-fatal.\-c} }{\pageref{gmx__fatal_8c}}{}
\item\contentsline{section}{src/gmxlib/\hyperlink{gmx__matrix_8c}{gmx\-\_\-matrix.\-c} }{\pageref{gmx__matrix_8c}}{}
\item\contentsline{section}{src/gmxlib/\hyperlink{gmx__omp_8c}{gmx\-\_\-omp.\-c} }{\pageref{gmx__omp_8c}}{}
\item\contentsline{section}{src/gmxlib/\hyperlink{gmx__omp__nthreads_8c}{gmx\-\_\-omp\-\_\-nthreads.\-c} }{\pageref{gmx__omp__nthreads_8c}}{}
\item\contentsline{section}{src/gmxlib/\hyperlink{gmx__random_8c}{gmx\-\_\-random.\-c} }{\pageref{gmx__random_8c}}{}
\item\contentsline{section}{src/gmxlib/\hyperlink{gmx__random__gausstable_8h}{gmx\-\_\-random\-\_\-gausstable.\-h} }{\pageref{gmx__random__gausstable_8h}}{}
\item\contentsline{section}{src/gmxlib/\hyperlink{gmx__sort_8c}{gmx\-\_\-sort.\-c} }{\pageref{gmx__sort_8c}}{}
\item\contentsline{section}{src/gmxlib/\hyperlink{gmx__system__xdr_8c}{gmx\-\_\-system\-\_\-xdr.\-c} }{\pageref{gmx__system__xdr_8c}}{}
\item\contentsline{section}{src/gmxlib/\hyperlink{gmx__thread__affinity_8c}{gmx\-\_\-thread\-\_\-affinity.\-c} }{\pageref{gmx__thread__affinity_8c}}{}
\item\contentsline{section}{src/gmxlib/\hyperlink{gmxcpp_8c}{gmxcpp.\-c} }{\pageref{gmxcpp_8c}}{}
\item\contentsline{section}{src/gmxlib/\hyperlink{gmxfio_8c}{gmxfio.\-c} }{\pageref{gmxfio_8c}}{}
\item\contentsline{section}{src/gmxlib/\hyperlink{gmxfio__asc_8c}{gmxfio\-\_\-asc.\-c} }{\pageref{gmxfio__asc_8c}}{}
\item\contentsline{section}{src/gmxlib/\hyperlink{gmxfio__bin_8c}{gmxfio\-\_\-bin.\-c} }{\pageref{gmxfio__bin_8c}}{}
\item\contentsline{section}{src/gmxlib/\hyperlink{gmxfio__int_8h}{gmxfio\-\_\-int.\-h} }{\pageref{gmxfio__int_8h}}{}
\item\contentsline{section}{src/gmxlib/\hyperlink{gmxfio__rw_8c}{gmxfio\-\_\-rw.\-c} }{\pageref{gmxfio__rw_8c}}{}
\item\contentsline{section}{src/gmxlib/\hyperlink{gmxfio__xdr_8c}{gmxfio\-\_\-xdr.\-c} }{\pageref{gmxfio__xdr_8c}}{}
\item\contentsline{section}{src/gmxlib/\hyperlink{ifunc_8c}{ifunc.\-c} }{\pageref{ifunc_8c}}{}
\item\contentsline{section}{src/gmxlib/\hyperlink{index_8c}{index.\-c} }{\pageref{index_8c}}{}
\item\contentsline{section}{src/gmxlib/\hyperlink{inputrec_8c}{inputrec.\-c} }{\pageref{inputrec_8c}}{}
\item\contentsline{section}{src/gmxlib/\hyperlink{invblock_8c}{invblock.\-c} }{\pageref{invblock_8c}}{}
\item\contentsline{section}{src/gmxlib/\hyperlink{invsqrt__test_8c}{invsqrt\-\_\-test.\-c} }{\pageref{invsqrt__test_8c}}{}
\item\contentsline{section}{src/gmxlib/\hyperlink{libxdrf_8c}{libxdrf.\-c} }{\pageref{libxdrf_8c}}{}
\item\contentsline{section}{src/gmxlib/\hyperlink{macros_8c}{macros.\-c} }{\pageref{macros_8c}}{}
\item\contentsline{section}{src/gmxlib/\hyperlink{gmxlib_2main_8c}{main.\-c} }{\pageref{gmxlib_2main_8c}}{}
\item\contentsline{section}{src/gmxlib/\hyperlink{maths_8c}{maths.\-c} }{\pageref{maths_8c}}{}
\item\contentsline{section}{src/gmxlib/\hyperlink{matio_8c}{matio.\-c} }{\pageref{matio_8c}}{}
\item\contentsline{section}{src/gmxlib/\hyperlink{md5_8c}{md5.\-c} }{\pageref{md5_8c}}{}
\item\contentsline{section}{src/gmxlib/\hyperlink{md__logging_8c}{md\-\_\-logging.\-c} }{\pageref{md__logging_8c}}{}
\item\contentsline{section}{src/gmxlib/\hyperlink{minvert_8h}{minvert.\-h} }{\pageref{minvert_8h}}{}
\item\contentsline{section}{src/gmxlib/\hyperlink{molfile__plugin_8h}{molfile\-\_\-plugin.\-h} }{\pageref{molfile__plugin_8h}}{}
\item\contentsline{section}{src/gmxlib/\hyperlink{mshift_8c}{mshift.\-c} }{\pageref{mshift_8c}}{}
\item\contentsline{section}{src/gmxlib/\hyperlink{mtop__util_8c}{mtop\-\_\-util.\-c} }{\pageref{mtop__util_8c}}{}
\item\contentsline{section}{src/gmxlib/\hyperlink{mtxio_8c}{mtxio.\-c} }{\pageref{mtxio_8c}}{}
\item\contentsline{section}{src/gmxlib/\hyperlink{mvdata_8c}{mvdata.\-c} }{\pageref{mvdata_8c}}{}
\item\contentsline{section}{src/gmxlib/\hyperlink{names_8c}{names.\-c} }{\pageref{names_8c}}{}
\item\contentsline{section}{src/gmxlib/\hyperlink{network_8c}{network.\-c} }{\pageref{network_8c}}{}
\item\contentsline{section}{src/gmxlib/\hyperlink{nrama_8c}{nrama.\-c} }{\pageref{nrama_8c}}{}
\item\contentsline{section}{src/gmxlib/\hyperlink{nrjac_8c}{nrjac.\-c} }{\pageref{nrjac_8c}}{}
\item\contentsline{section}{src/gmxlib/\hyperlink{nrnb_8c}{nrnb.\-c} }{\pageref{nrnb_8c}}{}
\item\contentsline{section}{src/gmxlib/\hyperlink{oenv_8c}{oenv.\-c} }{\pageref{oenv_8c}}{}
\item\contentsline{section}{src/gmxlib/\hyperlink{orires_8c}{orires.\-c} }{\pageref{orires_8c}}{}
\item\contentsline{section}{src/gmxlib/\hyperlink{pargs_8c}{pargs.\-c} }{\pageref{pargs_8c}}{}
\item\contentsline{section}{src/gmxlib/\hyperlink{pbc_8c}{pbc.\-c} }{\pageref{pbc_8c}}{}
\item\contentsline{section}{src/gmxlib/\hyperlink{pdbio_8c}{pdbio.\-c} }{\pageref{pdbio_8c}}{}
\item\contentsline{section}{src/gmxlib/\hyperlink{princ_8c}{princ.\-c} }{\pageref{princ_8c}}{}
\item\contentsline{section}{src/gmxlib/\hyperlink{rando_8c}{rando.\-c} }{\pageref{rando_8c}}{}
\item\contentsline{section}{src/gmxlib/\hyperlink{gmxlib_2random_8c}{random.\-c} }{\pageref{gmxlib_2random_8c}}{}
\item\contentsline{section}{src/gmxlib/\hyperlink{rbin_8c}{rbin.\-c} }{\pageref{rbin_8c}}{}
\item\contentsline{section}{src/gmxlib/\hyperlink{readinp_8c}{readinp.\-c} }{\pageref{readinp_8c}}{}
\item\contentsline{section}{src/gmxlib/\hyperlink{replace_8c}{replace.\-c} }{\pageref{replace_8c}}{}
\item\contentsline{section}{src/gmxlib/\hyperlink{replace_8h}{replace.\-h} }{\pageref{replace_8h}}{}
\item\contentsline{section}{src/gmxlib/\hyperlink{rmpbc_8c}{rmpbc.\-c} }{\pageref{rmpbc_8c}}{}
\item\contentsline{section}{src/gmxlib/\hyperlink{sfactor_8c}{sfactor.\-c} }{\pageref{sfactor_8c}}{}
\item\contentsline{section}{src/gmxlib/\hyperlink{shift__util_8c}{shift\-\_\-util.\-c} }{\pageref{shift__util_8c}}{}
\item\contentsline{section}{src/gmxlib/\hyperlink{sighandler_8c}{sighandler.\-c} }{\pageref{sighandler_8c}}{}
\item\contentsline{section}{src/gmxlib/\hyperlink{smalloc_8c}{smalloc.\-c} }{\pageref{smalloc_8c}}{}
\item\contentsline{section}{src/gmxlib/\hyperlink{sortwater_8c}{sortwater.\-c} }{\pageref{sortwater_8c}}{}
\item\contentsline{section}{src/gmxlib/\hyperlink{sparsematrix_8c}{sparsematrix.\-c} }{\pageref{sparsematrix_8c}}{}
\item\contentsline{section}{src/gmxlib/\hyperlink{splitter_8c}{splitter.\-c} }{\pageref{splitter_8c}}{}
\item\contentsline{section}{src/gmxlib/\hyperlink{statutil_8c}{statutil.\-c} }{\pageref{statutil_8c}}{}
\item\contentsline{section}{src/gmxlib/\hyperlink{strdb_8c}{strdb.\-c} }{\pageref{strdb_8c}}{}
\item\contentsline{section}{src/gmxlib/\hyperlink{string2_8c}{string2.\-c} }{\pageref{string2_8c}}{}
\item\contentsline{section}{src/gmxlib/\hyperlink{symtab_8c}{symtab.\-c} }{\pageref{symtab_8c}}{}
\item\contentsline{section}{src/gmxlib/\hyperlink{tcontrol_8c}{tcontrol.\-c} }{\pageref{tcontrol_8c}}{}
\item\contentsline{section}{src/gmxlib/\hyperlink{topsort_8c}{topsort.\-c} }{\pageref{topsort_8c}}{}
\item\contentsline{section}{src/gmxlib/\hyperlink{tpxio_8c}{tpxio.\-c} }{\pageref{tpxio_8c}}{}
\item\contentsline{section}{src/gmxlib/\hyperlink{trnio_8c}{trnio.\-c} }{\pageref{trnio_8c}}{}
\item\contentsline{section}{src/gmxlib/\hyperlink{trxio_8c}{trxio.\-c} }{\pageref{trxio_8c}}{}
\item\contentsline{section}{src/gmxlib/\hyperlink{txtdump_8c}{txtdump.\-c} }{\pageref{txtdump_8c}}{}
\item\contentsline{section}{src/gmxlib/\hyperlink{typedefs_8c}{typedefs.\-c} }{\pageref{typedefs_8c}}{}
\item\contentsline{section}{src/gmxlib/\hyperlink{version_8c}{version.\-c} }{\pageref{version_8c}}{}
\item\contentsline{section}{src/gmxlib/\hyperlink{src_2gmxlib_2version_8h}{version.\-h} }{\pageref{src_2gmxlib_2version_8h}}{}
\item\contentsline{section}{src/gmxlib/\hyperlink{viewit_8c}{viewit.\-c} }{\pageref{viewit_8c}}{}
\item\contentsline{section}{src/gmxlib/\hyperlink{vmddlopen_8c}{vmddlopen.\-c} }{\pageref{vmddlopen_8c}}{}
\item\contentsline{section}{src/gmxlib/\hyperlink{vmddlopen_8h}{vmddlopen.\-h} }{\pageref{vmddlopen_8h}}{}
\item\contentsline{section}{src/gmxlib/\hyperlink{vmdio_8c}{vmdio.\-c} }{\pageref{vmdio_8c}}{}
\item\contentsline{section}{src/gmxlib/\hyperlink{vmdio_8h}{vmdio.\-h} }{\pageref{vmdio_8h}}{}
\item\contentsline{section}{src/gmxlib/\hyperlink{vmdplugin_8h}{vmdplugin.\-h} }{\pageref{vmdplugin_8h}}{}
\item\contentsline{section}{src/gmxlib/\hyperlink{warninp_8c}{warninp.\-c} }{\pageref{warninp_8c}}{}
\item\contentsline{section}{src/gmxlib/\hyperlink{wgms_8c}{wgms.\-c} }{\pageref{wgms_8c}}{}
\item\contentsline{section}{src/gmxlib/\hyperlink{wman_8c}{wman.\-c} }{\pageref{wman_8c}}{}
\item\contentsline{section}{src/gmxlib/\hyperlink{writeps_8c}{writeps.\-c} }{\pageref{writeps_8c}}{}
\item\contentsline{section}{src/gmxlib/\hyperlink{xdrd_8c}{xdrd.\-c} }{\pageref{xdrd_8c}}{}
\item\contentsline{section}{src/gmxlib/\hyperlink{xtcio_8c}{xtcio.\-c} }{\pageref{xtcio_8c}}{}
\item\contentsline{section}{src/gmxlib/\hyperlink{xvgr_8c}{xvgr.\-c} }{\pageref{xvgr_8c}}{}
\item\contentsline{section}{src/gmxlib/gmx\-\_\-blas/\hyperlink{dasum_8c}{dasum.\-c} }{\pageref{dasum_8c}}{}
\item\contentsline{section}{src/gmxlib/gmx\-\_\-blas/\hyperlink{daxpy_8c}{daxpy.\-c} }{\pageref{daxpy_8c}}{}
\item\contentsline{section}{src/gmxlib/gmx\-\_\-blas/\hyperlink{dcopy_8c}{dcopy.\-c} }{\pageref{dcopy_8c}}{}
\item\contentsline{section}{src/gmxlib/gmx\-\_\-blas/\hyperlink{ddot_8c}{ddot.\-c} }{\pageref{ddot_8c}}{}
\item\contentsline{section}{src/gmxlib/gmx\-\_\-blas/\hyperlink{dgemm_8c}{dgemm.\-c} }{\pageref{dgemm_8c}}{}
\item\contentsline{section}{src/gmxlib/gmx\-\_\-blas/\hyperlink{dgemv_8c}{dgemv.\-c} }{\pageref{dgemv_8c}}{}
\item\contentsline{section}{src/gmxlib/gmx\-\_\-blas/\hyperlink{dger_8c}{dger.\-c} }{\pageref{dger_8c}}{}
\item\contentsline{section}{src/gmxlib/gmx\-\_\-blas/\hyperlink{dnrm2_8c}{dnrm2.\-c} }{\pageref{dnrm2_8c}}{}
\item\contentsline{section}{src/gmxlib/gmx\-\_\-blas/\hyperlink{drot_8c}{drot.\-c} }{\pageref{drot_8c}}{}
\item\contentsline{section}{src/gmxlib/gmx\-\_\-blas/\hyperlink{dscal_8c}{dscal.\-c} }{\pageref{dscal_8c}}{}
\item\contentsline{section}{src/gmxlib/gmx\-\_\-blas/\hyperlink{dswap_8c}{dswap.\-c} }{\pageref{dswap_8c}}{}
\item\contentsline{section}{src/gmxlib/gmx\-\_\-blas/\hyperlink{dsymv_8c}{dsymv.\-c} }{\pageref{dsymv_8c}}{}
\item\contentsline{section}{src/gmxlib/gmx\-\_\-blas/\hyperlink{dsyr2_8c}{dsyr2.\-c} }{\pageref{dsyr2_8c}}{}
\item\contentsline{section}{src/gmxlib/gmx\-\_\-blas/\hyperlink{dsyr2k_8c}{dsyr2k.\-c} }{\pageref{dsyr2k_8c}}{}
\item\contentsline{section}{src/gmxlib/gmx\-\_\-blas/\hyperlink{dtrmm_8c}{dtrmm.\-c} }{\pageref{dtrmm_8c}}{}
\item\contentsline{section}{src/gmxlib/gmx\-\_\-blas/\hyperlink{dtrmv_8c}{dtrmv.\-c} }{\pageref{dtrmv_8c}}{}
\item\contentsline{section}{src/gmxlib/gmx\-\_\-blas/\hyperlink{dtrsm_8c}{dtrsm.\-c} }{\pageref{dtrsm_8c}}{}
\item\contentsline{section}{src/gmxlib/gmx\-\_\-blas/\hyperlink{idamax_8c}{idamax.\-c} }{\pageref{idamax_8c}}{}
\item\contentsline{section}{src/gmxlib/gmx\-\_\-blas/\hyperlink{isamax_8c}{isamax.\-c} }{\pageref{isamax_8c}}{}
\item\contentsline{section}{src/gmxlib/gmx\-\_\-blas/\hyperlink{sasum_8c}{sasum.\-c} }{\pageref{sasum_8c}}{}
\item\contentsline{section}{src/gmxlib/gmx\-\_\-blas/\hyperlink{saxpy_8c}{saxpy.\-c} }{\pageref{saxpy_8c}}{}
\item\contentsline{section}{src/gmxlib/gmx\-\_\-blas/\hyperlink{scopy_8c}{scopy.\-c} }{\pageref{scopy_8c}}{}
\item\contentsline{section}{src/gmxlib/gmx\-\_\-blas/\hyperlink{sdot_8c}{sdot.\-c} }{\pageref{sdot_8c}}{}
\item\contentsline{section}{src/gmxlib/gmx\-\_\-blas/\hyperlink{sgemm_8c}{sgemm.\-c} }{\pageref{sgemm_8c}}{}
\item\contentsline{section}{src/gmxlib/gmx\-\_\-blas/\hyperlink{sgemv_8c}{sgemv.\-c} }{\pageref{sgemv_8c}}{}
\item\contentsline{section}{src/gmxlib/gmx\-\_\-blas/\hyperlink{sger_8c}{sger.\-c} }{\pageref{sger_8c}}{}
\item\contentsline{section}{src/gmxlib/gmx\-\_\-blas/\hyperlink{snrm2_8c}{snrm2.\-c} }{\pageref{snrm2_8c}}{}
\item\contentsline{section}{src/gmxlib/gmx\-\_\-blas/\hyperlink{srot_8c}{srot.\-c} }{\pageref{srot_8c}}{}
\item\contentsline{section}{src/gmxlib/gmx\-\_\-blas/\hyperlink{sscal_8c}{sscal.\-c} }{\pageref{sscal_8c}}{}
\item\contentsline{section}{src/gmxlib/gmx\-\_\-blas/\hyperlink{sswap_8c}{sswap.\-c} }{\pageref{sswap_8c}}{}
\item\contentsline{section}{src/gmxlib/gmx\-\_\-blas/\hyperlink{ssymv_8c}{ssymv.\-c} }{\pageref{ssymv_8c}}{}
\item\contentsline{section}{src/gmxlib/gmx\-\_\-blas/\hyperlink{ssyr2_8c}{ssyr2.\-c} }{\pageref{ssyr2_8c}}{}
\item\contentsline{section}{src/gmxlib/gmx\-\_\-blas/\hyperlink{ssyr2k_8c}{ssyr2k.\-c} }{\pageref{ssyr2k_8c}}{}
\item\contentsline{section}{src/gmxlib/gmx\-\_\-blas/\hyperlink{strmm_8c}{strmm.\-c} }{\pageref{strmm_8c}}{}
\item\contentsline{section}{src/gmxlib/gmx\-\_\-blas/\hyperlink{strmv_8c}{strmv.\-c} }{\pageref{strmv_8c}}{}
\item\contentsline{section}{src/gmxlib/gmx\-\_\-blas/\hyperlink{strsm_8c}{strsm.\-c} }{\pageref{strsm_8c}}{}
\item\contentsline{section}{src/gmxlib/gmx\-\_\-lapack/\hyperlink{dbdsdc_8c}{dbdsdc.\-c} }{\pageref{dbdsdc_8c}}{}
\item\contentsline{section}{src/gmxlib/gmx\-\_\-lapack/\hyperlink{dbdsqr_8c}{dbdsqr.\-c} }{\pageref{dbdsqr_8c}}{}
\item\contentsline{section}{src/gmxlib/gmx\-\_\-lapack/\hyperlink{dgebd2_8c}{dgebd2.\-c} }{\pageref{dgebd2_8c}}{}
\item\contentsline{section}{src/gmxlib/gmx\-\_\-lapack/\hyperlink{dgebrd_8c}{dgebrd.\-c} }{\pageref{dgebrd_8c}}{}
\item\contentsline{section}{src/gmxlib/gmx\-\_\-lapack/\hyperlink{dgelq2_8c}{dgelq2.\-c} }{\pageref{dgelq2_8c}}{}
\item\contentsline{section}{src/gmxlib/gmx\-\_\-lapack/\hyperlink{dgelqf_8c}{dgelqf.\-c} }{\pageref{dgelqf_8c}}{}
\item\contentsline{section}{src/gmxlib/gmx\-\_\-lapack/\hyperlink{dgeqr2_8c}{dgeqr2.\-c} }{\pageref{dgeqr2_8c}}{}
\item\contentsline{section}{src/gmxlib/gmx\-\_\-lapack/\hyperlink{dgeqrf_8c}{dgeqrf.\-c} }{\pageref{dgeqrf_8c}}{}
\item\contentsline{section}{src/gmxlib/gmx\-\_\-lapack/\hyperlink{dgesdd_8c}{dgesdd.\-c} }{\pageref{dgesdd_8c}}{}
\item\contentsline{section}{src/gmxlib/gmx\-\_\-lapack/\hyperlink{dgetf2_8c}{dgetf2.\-c} }{\pageref{dgetf2_8c}}{}
\item\contentsline{section}{src/gmxlib/gmx\-\_\-lapack/\hyperlink{dgetrf_8c}{dgetrf.\-c} }{\pageref{dgetrf_8c}}{}
\item\contentsline{section}{src/gmxlib/gmx\-\_\-lapack/\hyperlink{dgetri_8c}{dgetri.\-c} }{\pageref{dgetri_8c}}{}
\item\contentsline{section}{src/gmxlib/gmx\-\_\-lapack/\hyperlink{dgetrs_8c}{dgetrs.\-c} }{\pageref{dgetrs_8c}}{}
\item\contentsline{section}{src/gmxlib/gmx\-\_\-lapack/\hyperlink{dlabrd_8c}{dlabrd.\-c} }{\pageref{dlabrd_8c}}{}
\item\contentsline{section}{src/gmxlib/gmx\-\_\-lapack/\hyperlink{dlacpy_8c}{dlacpy.\-c} }{\pageref{dlacpy_8c}}{}
\item\contentsline{section}{src/gmxlib/gmx\-\_\-lapack/\hyperlink{dlae2_8c}{dlae2.\-c} }{\pageref{dlae2_8c}}{}
\item\contentsline{section}{src/gmxlib/gmx\-\_\-lapack/\hyperlink{dlaebz_8c}{dlaebz.\-c} }{\pageref{dlaebz_8c}}{}
\item\contentsline{section}{src/gmxlib/gmx\-\_\-lapack/\hyperlink{dlaed6_8c}{dlaed6.\-c} }{\pageref{dlaed6_8c}}{}
\item\contentsline{section}{src/gmxlib/gmx\-\_\-lapack/\hyperlink{dlaev2_8c}{dlaev2.\-c} }{\pageref{dlaev2_8c}}{}
\item\contentsline{section}{src/gmxlib/gmx\-\_\-lapack/\hyperlink{dlagtf_8c}{dlagtf.\-c} }{\pageref{dlagtf_8c}}{}
\item\contentsline{section}{src/gmxlib/gmx\-\_\-lapack/\hyperlink{dlagts_8c}{dlagts.\-c} }{\pageref{dlagts_8c}}{}
\item\contentsline{section}{src/gmxlib/gmx\-\_\-lapack/\hyperlink{dlamrg_8c}{dlamrg.\-c} }{\pageref{dlamrg_8c}}{}
\item\contentsline{section}{src/gmxlib/gmx\-\_\-lapack/\hyperlink{dlange_8c}{dlange.\-c} }{\pageref{dlange_8c}}{}
\item\contentsline{section}{src/gmxlib/gmx\-\_\-lapack/\hyperlink{dlanst_8c}{dlanst.\-c} }{\pageref{dlanst_8c}}{}
\item\contentsline{section}{src/gmxlib/gmx\-\_\-lapack/\hyperlink{dlansy_8c}{dlansy.\-c} }{\pageref{dlansy_8c}}{}
\item\contentsline{section}{src/gmxlib/gmx\-\_\-lapack/\hyperlink{dlapy2_8c}{dlapy2.\-c} }{\pageref{dlapy2_8c}}{}
\item\contentsline{section}{src/gmxlib/gmx\-\_\-lapack/\hyperlink{dlar1vx_8c}{dlar1vx.\-c} }{\pageref{dlar1vx_8c}}{}
\item\contentsline{section}{src/gmxlib/gmx\-\_\-lapack/\hyperlink{dlarf_8c}{dlarf.\-c} }{\pageref{dlarf_8c}}{}
\item\contentsline{section}{src/gmxlib/gmx\-\_\-lapack/\hyperlink{dlarfb_8c}{dlarfb.\-c} }{\pageref{dlarfb_8c}}{}
\item\contentsline{section}{src/gmxlib/gmx\-\_\-lapack/\hyperlink{dlarfg_8c}{dlarfg.\-c} }{\pageref{dlarfg_8c}}{}
\item\contentsline{section}{src/gmxlib/gmx\-\_\-lapack/\hyperlink{dlarft_8c}{dlarft.\-c} }{\pageref{dlarft_8c}}{}
\item\contentsline{section}{src/gmxlib/gmx\-\_\-lapack/\hyperlink{dlarnv_8c}{dlarnv.\-c} }{\pageref{dlarnv_8c}}{}
\item\contentsline{section}{src/gmxlib/gmx\-\_\-lapack/\hyperlink{dlarrbx_8c}{dlarrbx.\-c} }{\pageref{dlarrbx_8c}}{}
\item\contentsline{section}{src/gmxlib/gmx\-\_\-lapack/\hyperlink{dlarrex_8c}{dlarrex.\-c} }{\pageref{dlarrex_8c}}{}
\item\contentsline{section}{src/gmxlib/gmx\-\_\-lapack/\hyperlink{dlarrfx_8c}{dlarrfx.\-c} }{\pageref{dlarrfx_8c}}{}
\item\contentsline{section}{src/gmxlib/gmx\-\_\-lapack/\hyperlink{dlarrvx_8c}{dlarrvx.\-c} }{\pageref{dlarrvx_8c}}{}
\item\contentsline{section}{src/gmxlib/gmx\-\_\-lapack/\hyperlink{dlartg_8c}{dlartg.\-c} }{\pageref{dlartg_8c}}{}
\item\contentsline{section}{src/gmxlib/gmx\-\_\-lapack/\hyperlink{dlaruv_8c}{dlaruv.\-c} }{\pageref{dlaruv_8c}}{}
\item\contentsline{section}{src/gmxlib/gmx\-\_\-lapack/\hyperlink{dlas2_8c}{dlas2.\-c} }{\pageref{dlas2_8c}}{}
\item\contentsline{section}{src/gmxlib/gmx\-\_\-lapack/\hyperlink{dlascl_8c}{dlascl.\-c} }{\pageref{dlascl_8c}}{}
\item\contentsline{section}{src/gmxlib/gmx\-\_\-lapack/\hyperlink{dlasd0_8c}{dlasd0.\-c} }{\pageref{dlasd0_8c}}{}
\item\contentsline{section}{src/gmxlib/gmx\-\_\-lapack/\hyperlink{dlasd1_8c}{dlasd1.\-c} }{\pageref{dlasd1_8c}}{}
\item\contentsline{section}{src/gmxlib/gmx\-\_\-lapack/\hyperlink{dlasd2_8c}{dlasd2.\-c} }{\pageref{dlasd2_8c}}{}
\item\contentsline{section}{src/gmxlib/gmx\-\_\-lapack/\hyperlink{dlasd3_8c}{dlasd3.\-c} }{\pageref{dlasd3_8c}}{}
\item\contentsline{section}{src/gmxlib/gmx\-\_\-lapack/\hyperlink{dlasd4_8c}{dlasd4.\-c} }{\pageref{dlasd4_8c}}{}
\item\contentsline{section}{src/gmxlib/gmx\-\_\-lapack/\hyperlink{dlasd5_8c}{dlasd5.\-c} }{\pageref{dlasd5_8c}}{}
\item\contentsline{section}{src/gmxlib/gmx\-\_\-lapack/\hyperlink{dlasd6_8c}{dlasd6.\-c} }{\pageref{dlasd6_8c}}{}
\item\contentsline{section}{src/gmxlib/gmx\-\_\-lapack/\hyperlink{dlasd7_8c}{dlasd7.\-c} }{\pageref{dlasd7_8c}}{}
\item\contentsline{section}{src/gmxlib/gmx\-\_\-lapack/\hyperlink{dlasd8_8c}{dlasd8.\-c} }{\pageref{dlasd8_8c}}{}
\item\contentsline{section}{src/gmxlib/gmx\-\_\-lapack/\hyperlink{dlasda_8c}{dlasda.\-c} }{\pageref{dlasda_8c}}{}
\item\contentsline{section}{src/gmxlib/gmx\-\_\-lapack/\hyperlink{dlasdq_8c}{dlasdq.\-c} }{\pageref{dlasdq_8c}}{}
\item\contentsline{section}{src/gmxlib/gmx\-\_\-lapack/\hyperlink{dlasdt_8c}{dlasdt.\-c} }{\pageref{dlasdt_8c}}{}
\item\contentsline{section}{src/gmxlib/gmx\-\_\-lapack/\hyperlink{dlaset_8c}{dlaset.\-c} }{\pageref{dlaset_8c}}{}
\item\contentsline{section}{src/gmxlib/gmx\-\_\-lapack/\hyperlink{dlasq1_8c}{dlasq1.\-c} }{\pageref{dlasq1_8c}}{}
\item\contentsline{section}{src/gmxlib/gmx\-\_\-lapack/\hyperlink{dlasq2_8c}{dlasq2.\-c} }{\pageref{dlasq2_8c}}{}
\item\contentsline{section}{src/gmxlib/gmx\-\_\-lapack/\hyperlink{dlasq3_8c}{dlasq3.\-c} }{\pageref{dlasq3_8c}}{}
\item\contentsline{section}{src/gmxlib/gmx\-\_\-lapack/\hyperlink{dlasq4_8c}{dlasq4.\-c} }{\pageref{dlasq4_8c}}{}
\item\contentsline{section}{src/gmxlib/gmx\-\_\-lapack/\hyperlink{dlasq5_8c}{dlasq5.\-c} }{\pageref{dlasq5_8c}}{}
\item\contentsline{section}{src/gmxlib/gmx\-\_\-lapack/\hyperlink{dlasq6_8c}{dlasq6.\-c} }{\pageref{dlasq6_8c}}{}
\item\contentsline{section}{src/gmxlib/gmx\-\_\-lapack/\hyperlink{dlasr_8c}{dlasr.\-c} }{\pageref{dlasr_8c}}{}
\item\contentsline{section}{src/gmxlib/gmx\-\_\-lapack/\hyperlink{dlasrt_8c}{dlasrt.\-c} }{\pageref{dlasrt_8c}}{}
\item\contentsline{section}{src/gmxlib/gmx\-\_\-lapack/\hyperlink{dlasrt2_8c}{dlasrt2.\-c} }{\pageref{dlasrt2_8c}}{}
\item\contentsline{section}{src/gmxlib/gmx\-\_\-lapack/\hyperlink{dlassq_8c}{dlassq.\-c} }{\pageref{dlassq_8c}}{}
\item\contentsline{section}{src/gmxlib/gmx\-\_\-lapack/\hyperlink{dlasv2_8c}{dlasv2.\-c} }{\pageref{dlasv2_8c}}{}
\item\contentsline{section}{src/gmxlib/gmx\-\_\-lapack/\hyperlink{dlaswp_8c}{dlaswp.\-c} }{\pageref{dlaswp_8c}}{}
\item\contentsline{section}{src/gmxlib/gmx\-\_\-lapack/\hyperlink{dlatrd_8c}{dlatrd.\-c} }{\pageref{dlatrd_8c}}{}
\item\contentsline{section}{src/gmxlib/gmx\-\_\-lapack/\hyperlink{dorg2r_8c}{dorg2r.\-c} }{\pageref{dorg2r_8c}}{}
\item\contentsline{section}{src/gmxlib/gmx\-\_\-lapack/\hyperlink{dorgbr_8c}{dorgbr.\-c} }{\pageref{dorgbr_8c}}{}
\item\contentsline{section}{src/gmxlib/gmx\-\_\-lapack/\hyperlink{dorgl2_8c}{dorgl2.\-c} }{\pageref{dorgl2_8c}}{}
\item\contentsline{section}{src/gmxlib/gmx\-\_\-lapack/\hyperlink{dorglq_8c}{dorglq.\-c} }{\pageref{dorglq_8c}}{}
\item\contentsline{section}{src/gmxlib/gmx\-\_\-lapack/\hyperlink{dorgqr_8c}{dorgqr.\-c} }{\pageref{dorgqr_8c}}{}
\item\contentsline{section}{src/gmxlib/gmx\-\_\-lapack/\hyperlink{dorm2l_8c}{dorm2l.\-c} }{\pageref{dorm2l_8c}}{}
\item\contentsline{section}{src/gmxlib/gmx\-\_\-lapack/\hyperlink{dorm2r_8c}{dorm2r.\-c} }{\pageref{dorm2r_8c}}{}
\item\contentsline{section}{src/gmxlib/gmx\-\_\-lapack/\hyperlink{dormbr_8c}{dormbr.\-c} }{\pageref{dormbr_8c}}{}
\item\contentsline{section}{src/gmxlib/gmx\-\_\-lapack/\hyperlink{dorml2_8c}{dorml2.\-c} }{\pageref{dorml2_8c}}{}
\item\contentsline{section}{src/gmxlib/gmx\-\_\-lapack/\hyperlink{dormlq_8c}{dormlq.\-c} }{\pageref{dormlq_8c}}{}
\item\contentsline{section}{src/gmxlib/gmx\-\_\-lapack/\hyperlink{dormql_8c}{dormql.\-c} }{\pageref{dormql_8c}}{}
\item\contentsline{section}{src/gmxlib/gmx\-\_\-lapack/\hyperlink{dormqr_8c}{dormqr.\-c} }{\pageref{dormqr_8c}}{}
\item\contentsline{section}{src/gmxlib/gmx\-\_\-lapack/\hyperlink{dormtr_8c}{dormtr.\-c} }{\pageref{dormtr_8c}}{}
\item\contentsline{section}{src/gmxlib/gmx\-\_\-lapack/\hyperlink{dstebz_8c}{dstebz.\-c} }{\pageref{dstebz_8c}}{}
\item\contentsline{section}{src/gmxlib/gmx\-\_\-lapack/\hyperlink{dstegr_8c}{dstegr.\-c} }{\pageref{dstegr_8c}}{}
\item\contentsline{section}{src/gmxlib/gmx\-\_\-lapack/\hyperlink{dstein_8c}{dstein.\-c} }{\pageref{dstein_8c}}{}
\item\contentsline{section}{src/gmxlib/gmx\-\_\-lapack/\hyperlink{dsteqr_8c}{dsteqr.\-c} }{\pageref{dsteqr_8c}}{}
\item\contentsline{section}{src/gmxlib/gmx\-\_\-lapack/\hyperlink{dsterf_8c}{dsterf.\-c} }{\pageref{dsterf_8c}}{}
\item\contentsline{section}{src/gmxlib/gmx\-\_\-lapack/\hyperlink{dstevr_8c}{dstevr.\-c} }{\pageref{dstevr_8c}}{}
\item\contentsline{section}{src/gmxlib/gmx\-\_\-lapack/\hyperlink{dsyevr_8c}{dsyevr.\-c} }{\pageref{dsyevr_8c}}{}
\item\contentsline{section}{src/gmxlib/gmx\-\_\-lapack/\hyperlink{dsytd2_8c}{dsytd2.\-c} }{\pageref{dsytd2_8c}}{}
\item\contentsline{section}{src/gmxlib/gmx\-\_\-lapack/\hyperlink{dsytrd_8c}{dsytrd.\-c} }{\pageref{dsytrd_8c}}{}
\item\contentsline{section}{src/gmxlib/gmx\-\_\-lapack/\hyperlink{dtrti2_8c}{dtrti2.\-c} }{\pageref{dtrti2_8c}}{}
\item\contentsline{section}{src/gmxlib/gmx\-\_\-lapack/\hyperlink{dtrtri_8c}{dtrtri.\-c} }{\pageref{dtrtri_8c}}{}
\item\contentsline{section}{src/gmxlib/gmx\-\_\-lapack/\hyperlink{ilasrt2_8c}{ilasrt2.\-c} }{\pageref{ilasrt2_8c}}{}
\item\contentsline{section}{src/gmxlib/gmx\-\_\-lapack/\hyperlink{lapack__limits_8h}{lapack\-\_\-limits.\-h} }{\pageref{lapack__limits_8h}}{}
\item\contentsline{section}{src/gmxlib/gmx\-\_\-lapack/\hyperlink{sbdsdc_8c}{sbdsdc.\-c} }{\pageref{sbdsdc_8c}}{}
\item\contentsline{section}{src/gmxlib/gmx\-\_\-lapack/\hyperlink{sbdsqr_8c}{sbdsqr.\-c} }{\pageref{sbdsqr_8c}}{}
\item\contentsline{section}{src/gmxlib/gmx\-\_\-lapack/\hyperlink{sgebd2_8c}{sgebd2.\-c} }{\pageref{sgebd2_8c}}{}
\item\contentsline{section}{src/gmxlib/gmx\-\_\-lapack/\hyperlink{sgebrd_8c}{sgebrd.\-c} }{\pageref{sgebrd_8c}}{}
\item\contentsline{section}{src/gmxlib/gmx\-\_\-lapack/\hyperlink{sgelq2_8c}{sgelq2.\-c} }{\pageref{sgelq2_8c}}{}
\item\contentsline{section}{src/gmxlib/gmx\-\_\-lapack/\hyperlink{sgelqf_8c}{sgelqf.\-c} }{\pageref{sgelqf_8c}}{}
\item\contentsline{section}{src/gmxlib/gmx\-\_\-lapack/\hyperlink{sgeqr2_8c}{sgeqr2.\-c} }{\pageref{sgeqr2_8c}}{}
\item\contentsline{section}{src/gmxlib/gmx\-\_\-lapack/\hyperlink{sgeqrf_8c}{sgeqrf.\-c} }{\pageref{sgeqrf_8c}}{}
\item\contentsline{section}{src/gmxlib/gmx\-\_\-lapack/\hyperlink{sgesdd_8c}{sgesdd.\-c} }{\pageref{sgesdd_8c}}{}
\item\contentsline{section}{src/gmxlib/gmx\-\_\-lapack/\hyperlink{sgetf2_8c}{sgetf2.\-c} }{\pageref{sgetf2_8c}}{}
\item\contentsline{section}{src/gmxlib/gmx\-\_\-lapack/\hyperlink{sgetrf_8c}{sgetrf.\-c} }{\pageref{sgetrf_8c}}{}
\item\contentsline{section}{src/gmxlib/gmx\-\_\-lapack/\hyperlink{sgetri_8c}{sgetri.\-c} }{\pageref{sgetri_8c}}{}
\item\contentsline{section}{src/gmxlib/gmx\-\_\-lapack/\hyperlink{sgetrs_8c}{sgetrs.\-c} }{\pageref{sgetrs_8c}}{}
\item\contentsline{section}{src/gmxlib/gmx\-\_\-lapack/\hyperlink{slabrd_8c}{slabrd.\-c} }{\pageref{slabrd_8c}}{}
\item\contentsline{section}{src/gmxlib/gmx\-\_\-lapack/\hyperlink{slacpy_8c}{slacpy.\-c} }{\pageref{slacpy_8c}}{}
\item\contentsline{section}{src/gmxlib/gmx\-\_\-lapack/\hyperlink{slae2_8c}{slae2.\-c} }{\pageref{slae2_8c}}{}
\item\contentsline{section}{src/gmxlib/gmx\-\_\-lapack/\hyperlink{slaebz_8c}{slaebz.\-c} }{\pageref{slaebz_8c}}{}
\item\contentsline{section}{src/gmxlib/gmx\-\_\-lapack/\hyperlink{slaed6_8c}{slaed6.\-c} }{\pageref{slaed6_8c}}{}
\item\contentsline{section}{src/gmxlib/gmx\-\_\-lapack/\hyperlink{slaev2_8c}{slaev2.\-c} }{\pageref{slaev2_8c}}{}
\item\contentsline{section}{src/gmxlib/gmx\-\_\-lapack/\hyperlink{slagtf_8c}{slagtf.\-c} }{\pageref{slagtf_8c}}{}
\item\contentsline{section}{src/gmxlib/gmx\-\_\-lapack/\hyperlink{slagts_8c}{slagts.\-c} }{\pageref{slagts_8c}}{}
\item\contentsline{section}{src/gmxlib/gmx\-\_\-lapack/\hyperlink{slamrg_8c}{slamrg.\-c} }{\pageref{slamrg_8c}}{}
\item\contentsline{section}{src/gmxlib/gmx\-\_\-lapack/\hyperlink{slange_8c}{slange.\-c} }{\pageref{slange_8c}}{}
\item\contentsline{section}{src/gmxlib/gmx\-\_\-lapack/\hyperlink{slanst_8c}{slanst.\-c} }{\pageref{slanst_8c}}{}
\item\contentsline{section}{src/gmxlib/gmx\-\_\-lapack/\hyperlink{slansy_8c}{slansy.\-c} }{\pageref{slansy_8c}}{}
\item\contentsline{section}{src/gmxlib/gmx\-\_\-lapack/\hyperlink{slapy2_8c}{slapy2.\-c} }{\pageref{slapy2_8c}}{}
\item\contentsline{section}{src/gmxlib/gmx\-\_\-lapack/\hyperlink{slar1vx_8c}{slar1vx.\-c} }{\pageref{slar1vx_8c}}{}
\item\contentsline{section}{src/gmxlib/gmx\-\_\-lapack/\hyperlink{slarf_8c}{slarf.\-c} }{\pageref{slarf_8c}}{}
\item\contentsline{section}{src/gmxlib/gmx\-\_\-lapack/\hyperlink{slarfb_8c}{slarfb.\-c} }{\pageref{slarfb_8c}}{}
\item\contentsline{section}{src/gmxlib/gmx\-\_\-lapack/\hyperlink{slarfg_8c}{slarfg.\-c} }{\pageref{slarfg_8c}}{}
\item\contentsline{section}{src/gmxlib/gmx\-\_\-lapack/\hyperlink{slarft_8c}{slarft.\-c} }{\pageref{slarft_8c}}{}
\item\contentsline{section}{src/gmxlib/gmx\-\_\-lapack/\hyperlink{slarnv_8c}{slarnv.\-c} }{\pageref{slarnv_8c}}{}
\item\contentsline{section}{src/gmxlib/gmx\-\_\-lapack/\hyperlink{slarrbx_8c}{slarrbx.\-c} }{\pageref{slarrbx_8c}}{}
\item\contentsline{section}{src/gmxlib/gmx\-\_\-lapack/\hyperlink{slarrex_8c}{slarrex.\-c} }{\pageref{slarrex_8c}}{}
\item\contentsline{section}{src/gmxlib/gmx\-\_\-lapack/\hyperlink{slarrfx_8c}{slarrfx.\-c} }{\pageref{slarrfx_8c}}{}
\item\contentsline{section}{src/gmxlib/gmx\-\_\-lapack/\hyperlink{slarrvx_8c}{slarrvx.\-c} }{\pageref{slarrvx_8c}}{}
\item\contentsline{section}{src/gmxlib/gmx\-\_\-lapack/\hyperlink{slartg_8c}{slartg.\-c} }{\pageref{slartg_8c}}{}
\item\contentsline{section}{src/gmxlib/gmx\-\_\-lapack/\hyperlink{slaruv_8c}{slaruv.\-c} }{\pageref{slaruv_8c}}{}
\item\contentsline{section}{src/gmxlib/gmx\-\_\-lapack/\hyperlink{slas2_8c}{slas2.\-c} }{\pageref{slas2_8c}}{}
\item\contentsline{section}{src/gmxlib/gmx\-\_\-lapack/\hyperlink{slascl_8c}{slascl.\-c} }{\pageref{slascl_8c}}{}
\item\contentsline{section}{src/gmxlib/gmx\-\_\-lapack/\hyperlink{slasd0_8c}{slasd0.\-c} }{\pageref{slasd0_8c}}{}
\item\contentsline{section}{src/gmxlib/gmx\-\_\-lapack/\hyperlink{slasd1_8c}{slasd1.\-c} }{\pageref{slasd1_8c}}{}
\item\contentsline{section}{src/gmxlib/gmx\-\_\-lapack/\hyperlink{slasd2_8c}{slasd2.\-c} }{\pageref{slasd2_8c}}{}
\item\contentsline{section}{src/gmxlib/gmx\-\_\-lapack/\hyperlink{slasd3_8c}{slasd3.\-c} }{\pageref{slasd3_8c}}{}
\item\contentsline{section}{src/gmxlib/gmx\-\_\-lapack/\hyperlink{slasd4_8c}{slasd4.\-c} }{\pageref{slasd4_8c}}{}
\item\contentsline{section}{src/gmxlib/gmx\-\_\-lapack/\hyperlink{slasd5_8c}{slasd5.\-c} }{\pageref{slasd5_8c}}{}
\item\contentsline{section}{src/gmxlib/gmx\-\_\-lapack/\hyperlink{slasd6_8c}{slasd6.\-c} }{\pageref{slasd6_8c}}{}
\item\contentsline{section}{src/gmxlib/gmx\-\_\-lapack/\hyperlink{slasd7_8c}{slasd7.\-c} }{\pageref{slasd7_8c}}{}
\item\contentsline{section}{src/gmxlib/gmx\-\_\-lapack/\hyperlink{slasd8_8c}{slasd8.\-c} }{\pageref{slasd8_8c}}{}
\item\contentsline{section}{src/gmxlib/gmx\-\_\-lapack/\hyperlink{slasda_8c}{slasda.\-c} }{\pageref{slasda_8c}}{}
\item\contentsline{section}{src/gmxlib/gmx\-\_\-lapack/\hyperlink{slasdq_8c}{slasdq.\-c} }{\pageref{slasdq_8c}}{}
\item\contentsline{section}{src/gmxlib/gmx\-\_\-lapack/\hyperlink{slasdt_8c}{slasdt.\-c} }{\pageref{slasdt_8c}}{}
\item\contentsline{section}{src/gmxlib/gmx\-\_\-lapack/\hyperlink{slaset_8c}{slaset.\-c} }{\pageref{slaset_8c}}{}
\item\contentsline{section}{src/gmxlib/gmx\-\_\-lapack/\hyperlink{slasq1_8c}{slasq1.\-c} }{\pageref{slasq1_8c}}{}
\item\contentsline{section}{src/gmxlib/gmx\-\_\-lapack/\hyperlink{slasq2_8c}{slasq2.\-c} }{\pageref{slasq2_8c}}{}
\item\contentsline{section}{src/gmxlib/gmx\-\_\-lapack/\hyperlink{slasq3_8c}{slasq3.\-c} }{\pageref{slasq3_8c}}{}
\item\contentsline{section}{src/gmxlib/gmx\-\_\-lapack/\hyperlink{slasq4_8c}{slasq4.\-c} }{\pageref{slasq4_8c}}{}
\item\contentsline{section}{src/gmxlib/gmx\-\_\-lapack/\hyperlink{slasq5_8c}{slasq5.\-c} }{\pageref{slasq5_8c}}{}
\item\contentsline{section}{src/gmxlib/gmx\-\_\-lapack/\hyperlink{slasq6_8c}{slasq6.\-c} }{\pageref{slasq6_8c}}{}
\item\contentsline{section}{src/gmxlib/gmx\-\_\-lapack/\hyperlink{slasr_8c}{slasr.\-c} }{\pageref{slasr_8c}}{}
\item\contentsline{section}{src/gmxlib/gmx\-\_\-lapack/\hyperlink{slasrt_8c}{slasrt.\-c} }{\pageref{slasrt_8c}}{}
\item\contentsline{section}{src/gmxlib/gmx\-\_\-lapack/\hyperlink{slasrt2_8c}{slasrt2.\-c} }{\pageref{slasrt2_8c}}{}
\item\contentsline{section}{src/gmxlib/gmx\-\_\-lapack/\hyperlink{slassq_8c}{slassq.\-c} }{\pageref{slassq_8c}}{}
\item\contentsline{section}{src/gmxlib/gmx\-\_\-lapack/\hyperlink{slasv2_8c}{slasv2.\-c} }{\pageref{slasv2_8c}}{}
\item\contentsline{section}{src/gmxlib/gmx\-\_\-lapack/\hyperlink{slaswp_8c}{slaswp.\-c} }{\pageref{slaswp_8c}}{}
\item\contentsline{section}{src/gmxlib/gmx\-\_\-lapack/\hyperlink{slatrd_8c}{slatrd.\-c} }{\pageref{slatrd_8c}}{}
\item\contentsline{section}{src/gmxlib/gmx\-\_\-lapack/\hyperlink{sorg2r_8c}{sorg2r.\-c} }{\pageref{sorg2r_8c}}{}
\item\contentsline{section}{src/gmxlib/gmx\-\_\-lapack/\hyperlink{sorgbr_8c}{sorgbr.\-c} }{\pageref{sorgbr_8c}}{}
\item\contentsline{section}{src/gmxlib/gmx\-\_\-lapack/\hyperlink{sorgl2_8c}{sorgl2.\-c} }{\pageref{sorgl2_8c}}{}
\item\contentsline{section}{src/gmxlib/gmx\-\_\-lapack/\hyperlink{sorglq_8c}{sorglq.\-c} }{\pageref{sorglq_8c}}{}
\item\contentsline{section}{src/gmxlib/gmx\-\_\-lapack/\hyperlink{sorgqr_8c}{sorgqr.\-c} }{\pageref{sorgqr_8c}}{}
\item\contentsline{section}{src/gmxlib/gmx\-\_\-lapack/\hyperlink{sorm2l_8c}{sorm2l.\-c} }{\pageref{sorm2l_8c}}{}
\item\contentsline{section}{src/gmxlib/gmx\-\_\-lapack/\hyperlink{sorm2r_8c}{sorm2r.\-c} }{\pageref{sorm2r_8c}}{}
\item\contentsline{section}{src/gmxlib/gmx\-\_\-lapack/\hyperlink{sormbr_8c}{sormbr.\-c} }{\pageref{sormbr_8c}}{}
\item\contentsline{section}{src/gmxlib/gmx\-\_\-lapack/\hyperlink{sorml2_8c}{sorml2.\-c} }{\pageref{sorml2_8c}}{}
\item\contentsline{section}{src/gmxlib/gmx\-\_\-lapack/\hyperlink{sormlq_8c}{sormlq.\-c} }{\pageref{sormlq_8c}}{}
\item\contentsline{section}{src/gmxlib/gmx\-\_\-lapack/\hyperlink{sormql_8c}{sormql.\-c} }{\pageref{sormql_8c}}{}
\item\contentsline{section}{src/gmxlib/gmx\-\_\-lapack/\hyperlink{sormqr_8c}{sormqr.\-c} }{\pageref{sormqr_8c}}{}
\item\contentsline{section}{src/gmxlib/gmx\-\_\-lapack/\hyperlink{sormtr_8c}{sormtr.\-c} }{\pageref{sormtr_8c}}{}
\item\contentsline{section}{src/gmxlib/gmx\-\_\-lapack/\hyperlink{sstebz_8c}{sstebz.\-c} }{\pageref{sstebz_8c}}{}
\item\contentsline{section}{src/gmxlib/gmx\-\_\-lapack/\hyperlink{sstegr_8c}{sstegr.\-c} }{\pageref{sstegr_8c}}{}
\item\contentsline{section}{src/gmxlib/gmx\-\_\-lapack/\hyperlink{sstein_8c}{sstein.\-c} }{\pageref{sstein_8c}}{}
\item\contentsline{section}{src/gmxlib/gmx\-\_\-lapack/\hyperlink{ssteqr_8c}{ssteqr.\-c} }{\pageref{ssteqr_8c}}{}
\item\contentsline{section}{src/gmxlib/gmx\-\_\-lapack/\hyperlink{ssterf_8c}{ssterf.\-c} }{\pageref{ssterf_8c}}{}
\item\contentsline{section}{src/gmxlib/gmx\-\_\-lapack/\hyperlink{sstevr_8c}{sstevr.\-c} }{\pageref{sstevr_8c}}{}
\item\contentsline{section}{src/gmxlib/gmx\-\_\-lapack/\hyperlink{ssyevr_8c}{ssyevr.\-c} }{\pageref{ssyevr_8c}}{}
\item\contentsline{section}{src/gmxlib/gmx\-\_\-lapack/\hyperlink{ssytd2_8c}{ssytd2.\-c} }{\pageref{ssytd2_8c}}{}
\item\contentsline{section}{src/gmxlib/gmx\-\_\-lapack/\hyperlink{ssytrd_8c}{ssytrd.\-c} }{\pageref{ssytrd_8c}}{}
\item\contentsline{section}{src/gmxlib/gmx\-\_\-lapack/\hyperlink{strti2_8c}{strti2.\-c} }{\pageref{strti2_8c}}{}
\item\contentsline{section}{src/gmxlib/gmx\-\_\-lapack/\hyperlink{strtri_8c}{strtri.\-c} }{\pageref{strtri_8c}}{}
\item\contentsline{section}{src/gmxlib/gpu\-\_\-utils/\hyperlink{memtestG80__core_8h}{memtest\-G80\-\_\-core.\-h} }{\pageref{memtestG80__core_8h}}{}
\item\contentsline{section}{src/gmxlib/nonbonded/\hyperlink{nb__free__energy_8c}{nb\-\_\-free\-\_\-energy.\-c} }{\pageref{nb__free__energy_8c}}{}
\item\contentsline{section}{src/gmxlib/nonbonded/\hyperlink{nb__free__energy_8h}{nb\-\_\-free\-\_\-energy.\-h} }{\pageref{nb__free__energy_8h}}{}
\item\contentsline{section}{src/gmxlib/nonbonded/\hyperlink{nb__generic_8c}{nb\-\_\-generic.\-c} }{\pageref{nb__generic_8c}}{}
\item\contentsline{section}{src/gmxlib/nonbonded/\hyperlink{nb__generic_8h}{nb\-\_\-generic.\-h} }{\pageref{nb__generic_8h}}{}
\item\contentsline{section}{src/gmxlib/nonbonded/\hyperlink{nb__generic__adress_8c}{nb\-\_\-generic\-\_\-adress.\-c} }{\pageref{nb__generic__adress_8c}}{}
\item\contentsline{section}{src/gmxlib/nonbonded/\hyperlink{nb__generic__adress_8h}{nb\-\_\-generic\-\_\-adress.\-h} }{\pageref{nb__generic__adress_8h}}{}
\item\contentsline{section}{src/gmxlib/nonbonded/\hyperlink{nb__generic__cg_8c}{nb\-\_\-generic\-\_\-cg.\-c} }{\pageref{nb__generic__cg_8c}}{}
\item\contentsline{section}{src/gmxlib/nonbonded/\hyperlink{nb__generic__cg_8h}{nb\-\_\-generic\-\_\-cg.\-h} }{\pageref{nb__generic__cg_8h}}{}
\item\contentsline{section}{src/gmxlib/nonbonded/\hyperlink{nb__kernel_8c}{nb\-\_\-kernel.\-c} }{\pageref{nb__kernel_8c}}{}
\item\contentsline{section}{src/gmxlib/nonbonded/\hyperlink{nb__kernel_8h}{nb\-\_\-kernel.\-h} }{\pageref{nb__kernel_8h}}{}
\item\contentsline{section}{src/gmxlib/nonbonded/\hyperlink{nonbonded_8c}{nonbonded.\-c} }{\pageref{nonbonded_8c}}{}
\item\contentsline{section}{src/gmxlib/nonbonded/nb\-\_\-kernel\-\_\-avx\-\_\-128\-\_\-fma\-\_\-double/\hyperlink{kernelutil__x86__avx__128__fma__double_8h}{kernelutil\-\_\-x86\-\_\-avx\-\_\-128\-\_\-fma\-\_\-double.\-h} }{\pageref{kernelutil__x86__avx__128__fma__double_8h}}{}
\item\contentsline{section}{src/gmxlib/nonbonded/nb\-\_\-kernel\-\_\-avx\-\_\-128\-\_\-fma\-\_\-double/\hyperlink{nb__kernel__avx__128__fma__double_8c}{nb\-\_\-kernel\-\_\-avx\-\_\-128\-\_\-fma\-\_\-double.\-c} }{\pageref{nb__kernel__avx__128__fma__double_8c}}{}
\item\contentsline{section}{src/gmxlib/nonbonded/nb\-\_\-kernel\-\_\-avx\-\_\-128\-\_\-fma\-\_\-double/\hyperlink{nb__kernel__avx__128__fma__double_8h}{nb\-\_\-kernel\-\_\-avx\-\_\-128\-\_\-fma\-\_\-double.\-h} }{\pageref{nb__kernel__avx__128__fma__double_8h}}{}
\item\contentsline{section}{src/gmxlib/nonbonded/nb\-\_\-kernel\-\_\-avx\-\_\-128\-\_\-fma\-\_\-double/\hyperlink{nb__kernel__ElecCoul__VdwCSTab__GeomP1P1__avx__128__fma__double_8c}{nb\-\_\-kernel\-\_\-\-Elec\-Coul\-\_\-\-Vdw\-C\-S\-Tab\-\_\-\-Geom\-P1\-P1\-\_\-avx\-\_\-128\-\_\-fma\-\_\-double.\-c} }{\pageref{nb__kernel__ElecCoul__VdwCSTab__GeomP1P1__avx__128__fma__double_8c}}{}
\item\contentsline{section}{src/gmxlib/nonbonded/nb\-\_\-kernel\-\_\-avx\-\_\-128\-\_\-fma\-\_\-double/\hyperlink{nb__kernel__ElecCoul__VdwCSTab__GeomW3P1__avx__128__fma__double_8c}{nb\-\_\-kernel\-\_\-\-Elec\-Coul\-\_\-\-Vdw\-C\-S\-Tab\-\_\-\-Geom\-W3\-P1\-\_\-avx\-\_\-128\-\_\-fma\-\_\-double.\-c} }{\pageref{nb__kernel__ElecCoul__VdwCSTab__GeomW3P1__avx__128__fma__double_8c}}{}
\item\contentsline{section}{src/gmxlib/nonbonded/nb\-\_\-kernel\-\_\-avx\-\_\-128\-\_\-fma\-\_\-double/\hyperlink{nb__kernel__ElecCoul__VdwCSTab__GeomW3W3__avx__128__fma__double_8c}{nb\-\_\-kernel\-\_\-\-Elec\-Coul\-\_\-\-Vdw\-C\-S\-Tab\-\_\-\-Geom\-W3\-W3\-\_\-avx\-\_\-128\-\_\-fma\-\_\-double.\-c} }{\pageref{nb__kernel__ElecCoul__VdwCSTab__GeomW3W3__avx__128__fma__double_8c}}{}
\item\contentsline{section}{src/gmxlib/nonbonded/nb\-\_\-kernel\-\_\-avx\-\_\-128\-\_\-fma\-\_\-double/\hyperlink{nb__kernel__ElecCoul__VdwCSTab__GeomW4P1__avx__128__fma__double_8c}{nb\-\_\-kernel\-\_\-\-Elec\-Coul\-\_\-\-Vdw\-C\-S\-Tab\-\_\-\-Geom\-W4\-P1\-\_\-avx\-\_\-128\-\_\-fma\-\_\-double.\-c} }{\pageref{nb__kernel__ElecCoul__VdwCSTab__GeomW4P1__avx__128__fma__double_8c}}{}
\item\contentsline{section}{src/gmxlib/nonbonded/nb\-\_\-kernel\-\_\-avx\-\_\-128\-\_\-fma\-\_\-double/\hyperlink{nb__kernel__ElecCoul__VdwCSTab__GeomW4W4__avx__128__fma__double_8c}{nb\-\_\-kernel\-\_\-\-Elec\-Coul\-\_\-\-Vdw\-C\-S\-Tab\-\_\-\-Geom\-W4\-W4\-\_\-avx\-\_\-128\-\_\-fma\-\_\-double.\-c} }{\pageref{nb__kernel__ElecCoul__VdwCSTab__GeomW4W4__avx__128__fma__double_8c}}{}
\item\contentsline{section}{src/gmxlib/nonbonded/nb\-\_\-kernel\-\_\-avx\-\_\-128\-\_\-fma\-\_\-double/\hyperlink{nb__kernel__ElecCoul__VdwLJ__GeomP1P1__avx__128__fma__double_8c}{nb\-\_\-kernel\-\_\-\-Elec\-Coul\-\_\-\-Vdw\-L\-J\-\_\-\-Geom\-P1\-P1\-\_\-avx\-\_\-128\-\_\-fma\-\_\-double.\-c} }{\pageref{nb__kernel__ElecCoul__VdwLJ__GeomP1P1__avx__128__fma__double_8c}}{}
\item\contentsline{section}{src/gmxlib/nonbonded/nb\-\_\-kernel\-\_\-avx\-\_\-128\-\_\-fma\-\_\-double/\hyperlink{nb__kernel__ElecCoul__VdwLJ__GeomW3P1__avx__128__fma__double_8c}{nb\-\_\-kernel\-\_\-\-Elec\-Coul\-\_\-\-Vdw\-L\-J\-\_\-\-Geom\-W3\-P1\-\_\-avx\-\_\-128\-\_\-fma\-\_\-double.\-c} }{\pageref{nb__kernel__ElecCoul__VdwLJ__GeomW3P1__avx__128__fma__double_8c}}{}
\item\contentsline{section}{src/gmxlib/nonbonded/nb\-\_\-kernel\-\_\-avx\-\_\-128\-\_\-fma\-\_\-double/\hyperlink{nb__kernel__ElecCoul__VdwLJ__GeomW3W3__avx__128__fma__double_8c}{nb\-\_\-kernel\-\_\-\-Elec\-Coul\-\_\-\-Vdw\-L\-J\-\_\-\-Geom\-W3\-W3\-\_\-avx\-\_\-128\-\_\-fma\-\_\-double.\-c} }{\pageref{nb__kernel__ElecCoul__VdwLJ__GeomW3W3__avx__128__fma__double_8c}}{}
\item\contentsline{section}{src/gmxlib/nonbonded/nb\-\_\-kernel\-\_\-avx\-\_\-128\-\_\-fma\-\_\-double/\hyperlink{nb__kernel__ElecCoul__VdwLJ__GeomW4P1__avx__128__fma__double_8c}{nb\-\_\-kernel\-\_\-\-Elec\-Coul\-\_\-\-Vdw\-L\-J\-\_\-\-Geom\-W4\-P1\-\_\-avx\-\_\-128\-\_\-fma\-\_\-double.\-c} }{\pageref{nb__kernel__ElecCoul__VdwLJ__GeomW4P1__avx__128__fma__double_8c}}{}
\item\contentsline{section}{src/gmxlib/nonbonded/nb\-\_\-kernel\-\_\-avx\-\_\-128\-\_\-fma\-\_\-double/\hyperlink{nb__kernel__ElecCoul__VdwLJ__GeomW4W4__avx__128__fma__double_8c}{nb\-\_\-kernel\-\_\-\-Elec\-Coul\-\_\-\-Vdw\-L\-J\-\_\-\-Geom\-W4\-W4\-\_\-avx\-\_\-128\-\_\-fma\-\_\-double.\-c} }{\pageref{nb__kernel__ElecCoul__VdwLJ__GeomW4W4__avx__128__fma__double_8c}}{}
\item\contentsline{section}{src/gmxlib/nonbonded/nb\-\_\-kernel\-\_\-avx\-\_\-128\-\_\-fma\-\_\-double/\hyperlink{nb__kernel__ElecCoul__VdwNone__GeomP1P1__avx__128__fma__double_8c}{nb\-\_\-kernel\-\_\-\-Elec\-Coul\-\_\-\-Vdw\-None\-\_\-\-Geom\-P1\-P1\-\_\-avx\-\_\-128\-\_\-fma\-\_\-double.\-c} }{\pageref{nb__kernel__ElecCoul__VdwNone__GeomP1P1__avx__128__fma__double_8c}}{}
\item\contentsline{section}{src/gmxlib/nonbonded/nb\-\_\-kernel\-\_\-avx\-\_\-128\-\_\-fma\-\_\-double/\hyperlink{nb__kernel__ElecCoul__VdwNone__GeomW3P1__avx__128__fma__double_8c}{nb\-\_\-kernel\-\_\-\-Elec\-Coul\-\_\-\-Vdw\-None\-\_\-\-Geom\-W3\-P1\-\_\-avx\-\_\-128\-\_\-fma\-\_\-double.\-c} }{\pageref{nb__kernel__ElecCoul__VdwNone__GeomW3P1__avx__128__fma__double_8c}}{}
\item\contentsline{section}{src/gmxlib/nonbonded/nb\-\_\-kernel\-\_\-avx\-\_\-128\-\_\-fma\-\_\-double/\hyperlink{nb__kernel__ElecCoul__VdwNone__GeomW3W3__avx__128__fma__double_8c}{nb\-\_\-kernel\-\_\-\-Elec\-Coul\-\_\-\-Vdw\-None\-\_\-\-Geom\-W3\-W3\-\_\-avx\-\_\-128\-\_\-fma\-\_\-double.\-c} }{\pageref{nb__kernel__ElecCoul__VdwNone__GeomW3W3__avx__128__fma__double_8c}}{}
\item\contentsline{section}{src/gmxlib/nonbonded/nb\-\_\-kernel\-\_\-avx\-\_\-128\-\_\-fma\-\_\-double/\hyperlink{nb__kernel__ElecCoul__VdwNone__GeomW4P1__avx__128__fma__double_8c}{nb\-\_\-kernel\-\_\-\-Elec\-Coul\-\_\-\-Vdw\-None\-\_\-\-Geom\-W4\-P1\-\_\-avx\-\_\-128\-\_\-fma\-\_\-double.\-c} }{\pageref{nb__kernel__ElecCoul__VdwNone__GeomW4P1__avx__128__fma__double_8c}}{}
\item\contentsline{section}{src/gmxlib/nonbonded/nb\-\_\-kernel\-\_\-avx\-\_\-128\-\_\-fma\-\_\-double/\hyperlink{nb__kernel__ElecCoul__VdwNone__GeomW4W4__avx__128__fma__double_8c}{nb\-\_\-kernel\-\_\-\-Elec\-Coul\-\_\-\-Vdw\-None\-\_\-\-Geom\-W4\-W4\-\_\-avx\-\_\-128\-\_\-fma\-\_\-double.\-c} }{\pageref{nb__kernel__ElecCoul__VdwNone__GeomW4W4__avx__128__fma__double_8c}}{}
\item\contentsline{section}{src/gmxlib/nonbonded/nb\-\_\-kernel\-\_\-avx\-\_\-128\-\_\-fma\-\_\-double/\hyperlink{nb__kernel__ElecCSTab__VdwCSTab__GeomP1P1__avx__128__fma__double_8c}{nb\-\_\-kernel\-\_\-\-Elec\-C\-S\-Tab\-\_\-\-Vdw\-C\-S\-Tab\-\_\-\-Geom\-P1\-P1\-\_\-avx\-\_\-128\-\_\-fma\-\_\-double.\-c} }{\pageref{nb__kernel__ElecCSTab__VdwCSTab__GeomP1P1__avx__128__fma__double_8c}}{}
\item\contentsline{section}{src/gmxlib/nonbonded/nb\-\_\-kernel\-\_\-avx\-\_\-128\-\_\-fma\-\_\-double/\hyperlink{nb__kernel__ElecCSTab__VdwCSTab__GeomW3P1__avx__128__fma__double_8c}{nb\-\_\-kernel\-\_\-\-Elec\-C\-S\-Tab\-\_\-\-Vdw\-C\-S\-Tab\-\_\-\-Geom\-W3\-P1\-\_\-avx\-\_\-128\-\_\-fma\-\_\-double.\-c} }{\pageref{nb__kernel__ElecCSTab__VdwCSTab__GeomW3P1__avx__128__fma__double_8c}}{}
\item\contentsline{section}{src/gmxlib/nonbonded/nb\-\_\-kernel\-\_\-avx\-\_\-128\-\_\-fma\-\_\-double/\hyperlink{nb__kernel__ElecCSTab__VdwCSTab__GeomW3W3__avx__128__fma__double_8c}{nb\-\_\-kernel\-\_\-\-Elec\-C\-S\-Tab\-\_\-\-Vdw\-C\-S\-Tab\-\_\-\-Geom\-W3\-W3\-\_\-avx\-\_\-128\-\_\-fma\-\_\-double.\-c} }{\pageref{nb__kernel__ElecCSTab__VdwCSTab__GeomW3W3__avx__128__fma__double_8c}}{}
\item\contentsline{section}{src/gmxlib/nonbonded/nb\-\_\-kernel\-\_\-avx\-\_\-128\-\_\-fma\-\_\-double/\hyperlink{nb__kernel__ElecCSTab__VdwCSTab__GeomW4P1__avx__128__fma__double_8c}{nb\-\_\-kernel\-\_\-\-Elec\-C\-S\-Tab\-\_\-\-Vdw\-C\-S\-Tab\-\_\-\-Geom\-W4\-P1\-\_\-avx\-\_\-128\-\_\-fma\-\_\-double.\-c} }{\pageref{nb__kernel__ElecCSTab__VdwCSTab__GeomW4P1__avx__128__fma__double_8c}}{}
\item\contentsline{section}{src/gmxlib/nonbonded/nb\-\_\-kernel\-\_\-avx\-\_\-128\-\_\-fma\-\_\-double/\hyperlink{nb__kernel__ElecCSTab__VdwCSTab__GeomW4W4__avx__128__fma__double_8c}{nb\-\_\-kernel\-\_\-\-Elec\-C\-S\-Tab\-\_\-\-Vdw\-C\-S\-Tab\-\_\-\-Geom\-W4\-W4\-\_\-avx\-\_\-128\-\_\-fma\-\_\-double.\-c} }{\pageref{nb__kernel__ElecCSTab__VdwCSTab__GeomW4W4__avx__128__fma__double_8c}}{}
\item\contentsline{section}{src/gmxlib/nonbonded/nb\-\_\-kernel\-\_\-avx\-\_\-128\-\_\-fma\-\_\-double/\hyperlink{nb__kernel__ElecCSTab__VdwLJ__GeomP1P1__avx__128__fma__double_8c}{nb\-\_\-kernel\-\_\-\-Elec\-C\-S\-Tab\-\_\-\-Vdw\-L\-J\-\_\-\-Geom\-P1\-P1\-\_\-avx\-\_\-128\-\_\-fma\-\_\-double.\-c} }{\pageref{nb__kernel__ElecCSTab__VdwLJ__GeomP1P1__avx__128__fma__double_8c}}{}
\item\contentsline{section}{src/gmxlib/nonbonded/nb\-\_\-kernel\-\_\-avx\-\_\-128\-\_\-fma\-\_\-double/\hyperlink{nb__kernel__ElecCSTab__VdwLJ__GeomW3P1__avx__128__fma__double_8c}{nb\-\_\-kernel\-\_\-\-Elec\-C\-S\-Tab\-\_\-\-Vdw\-L\-J\-\_\-\-Geom\-W3\-P1\-\_\-avx\-\_\-128\-\_\-fma\-\_\-double.\-c} }{\pageref{nb__kernel__ElecCSTab__VdwLJ__GeomW3P1__avx__128__fma__double_8c}}{}
\item\contentsline{section}{src/gmxlib/nonbonded/nb\-\_\-kernel\-\_\-avx\-\_\-128\-\_\-fma\-\_\-double/\hyperlink{nb__kernel__ElecCSTab__VdwLJ__GeomW3W3__avx__128__fma__double_8c}{nb\-\_\-kernel\-\_\-\-Elec\-C\-S\-Tab\-\_\-\-Vdw\-L\-J\-\_\-\-Geom\-W3\-W3\-\_\-avx\-\_\-128\-\_\-fma\-\_\-double.\-c} }{\pageref{nb__kernel__ElecCSTab__VdwLJ__GeomW3W3__avx__128__fma__double_8c}}{}
\item\contentsline{section}{src/gmxlib/nonbonded/nb\-\_\-kernel\-\_\-avx\-\_\-128\-\_\-fma\-\_\-double/\hyperlink{nb__kernel__ElecCSTab__VdwLJ__GeomW4P1__avx__128__fma__double_8c}{nb\-\_\-kernel\-\_\-\-Elec\-C\-S\-Tab\-\_\-\-Vdw\-L\-J\-\_\-\-Geom\-W4\-P1\-\_\-avx\-\_\-128\-\_\-fma\-\_\-double.\-c} }{\pageref{nb__kernel__ElecCSTab__VdwLJ__GeomW4P1__avx__128__fma__double_8c}}{}
\item\contentsline{section}{src/gmxlib/nonbonded/nb\-\_\-kernel\-\_\-avx\-\_\-128\-\_\-fma\-\_\-double/\hyperlink{nb__kernel__ElecCSTab__VdwLJ__GeomW4W4__avx__128__fma__double_8c}{nb\-\_\-kernel\-\_\-\-Elec\-C\-S\-Tab\-\_\-\-Vdw\-L\-J\-\_\-\-Geom\-W4\-W4\-\_\-avx\-\_\-128\-\_\-fma\-\_\-double.\-c} }{\pageref{nb__kernel__ElecCSTab__VdwLJ__GeomW4W4__avx__128__fma__double_8c}}{}
\item\contentsline{section}{src/gmxlib/nonbonded/nb\-\_\-kernel\-\_\-avx\-\_\-128\-\_\-fma\-\_\-double/\hyperlink{nb__kernel__ElecCSTab__VdwNone__GeomP1P1__avx__128__fma__double_8c}{nb\-\_\-kernel\-\_\-\-Elec\-C\-S\-Tab\-\_\-\-Vdw\-None\-\_\-\-Geom\-P1\-P1\-\_\-avx\-\_\-128\-\_\-fma\-\_\-double.\-c} }{\pageref{nb__kernel__ElecCSTab__VdwNone__GeomP1P1__avx__128__fma__double_8c}}{}
\item\contentsline{section}{src/gmxlib/nonbonded/nb\-\_\-kernel\-\_\-avx\-\_\-128\-\_\-fma\-\_\-double/\hyperlink{nb__kernel__ElecCSTab__VdwNone__GeomW3P1__avx__128__fma__double_8c}{nb\-\_\-kernel\-\_\-\-Elec\-C\-S\-Tab\-\_\-\-Vdw\-None\-\_\-\-Geom\-W3\-P1\-\_\-avx\-\_\-128\-\_\-fma\-\_\-double.\-c} }{\pageref{nb__kernel__ElecCSTab__VdwNone__GeomW3P1__avx__128__fma__double_8c}}{}
\item\contentsline{section}{src/gmxlib/nonbonded/nb\-\_\-kernel\-\_\-avx\-\_\-128\-\_\-fma\-\_\-double/\hyperlink{nb__kernel__ElecCSTab__VdwNone__GeomW3W3__avx__128__fma__double_8c}{nb\-\_\-kernel\-\_\-\-Elec\-C\-S\-Tab\-\_\-\-Vdw\-None\-\_\-\-Geom\-W3\-W3\-\_\-avx\-\_\-128\-\_\-fma\-\_\-double.\-c} }{\pageref{nb__kernel__ElecCSTab__VdwNone__GeomW3W3__avx__128__fma__double_8c}}{}
\item\contentsline{section}{src/gmxlib/nonbonded/nb\-\_\-kernel\-\_\-avx\-\_\-128\-\_\-fma\-\_\-double/\hyperlink{nb__kernel__ElecCSTab__VdwNone__GeomW4P1__avx__128__fma__double_8c}{nb\-\_\-kernel\-\_\-\-Elec\-C\-S\-Tab\-\_\-\-Vdw\-None\-\_\-\-Geom\-W4\-P1\-\_\-avx\-\_\-128\-\_\-fma\-\_\-double.\-c} }{\pageref{nb__kernel__ElecCSTab__VdwNone__GeomW4P1__avx__128__fma__double_8c}}{}
\item\contentsline{section}{src/gmxlib/nonbonded/nb\-\_\-kernel\-\_\-avx\-\_\-128\-\_\-fma\-\_\-double/\hyperlink{nb__kernel__ElecCSTab__VdwNone__GeomW4W4__avx__128__fma__double_8c}{nb\-\_\-kernel\-\_\-\-Elec\-C\-S\-Tab\-\_\-\-Vdw\-None\-\_\-\-Geom\-W4\-W4\-\_\-avx\-\_\-128\-\_\-fma\-\_\-double.\-c} }{\pageref{nb__kernel__ElecCSTab__VdwNone__GeomW4W4__avx__128__fma__double_8c}}{}
\item\contentsline{section}{src/gmxlib/nonbonded/nb\-\_\-kernel\-\_\-avx\-\_\-128\-\_\-fma\-\_\-double/\hyperlink{nb__kernel__ElecEw__VdwCSTab__GeomP1P1__avx__128__fma__double_8c}{nb\-\_\-kernel\-\_\-\-Elec\-Ew\-\_\-\-Vdw\-C\-S\-Tab\-\_\-\-Geom\-P1\-P1\-\_\-avx\-\_\-128\-\_\-fma\-\_\-double.\-c} }{\pageref{nb__kernel__ElecEw__VdwCSTab__GeomP1P1__avx__128__fma__double_8c}}{}
\item\contentsline{section}{src/gmxlib/nonbonded/nb\-\_\-kernel\-\_\-avx\-\_\-128\-\_\-fma\-\_\-double/\hyperlink{nb__kernel__ElecEw__VdwCSTab__GeomW3P1__avx__128__fma__double_8c}{nb\-\_\-kernel\-\_\-\-Elec\-Ew\-\_\-\-Vdw\-C\-S\-Tab\-\_\-\-Geom\-W3\-P1\-\_\-avx\-\_\-128\-\_\-fma\-\_\-double.\-c} }{\pageref{nb__kernel__ElecEw__VdwCSTab__GeomW3P1__avx__128__fma__double_8c}}{}
\item\contentsline{section}{src/gmxlib/nonbonded/nb\-\_\-kernel\-\_\-avx\-\_\-128\-\_\-fma\-\_\-double/\hyperlink{nb__kernel__ElecEw__VdwCSTab__GeomW3W3__avx__128__fma__double_8c}{nb\-\_\-kernel\-\_\-\-Elec\-Ew\-\_\-\-Vdw\-C\-S\-Tab\-\_\-\-Geom\-W3\-W3\-\_\-avx\-\_\-128\-\_\-fma\-\_\-double.\-c} }{\pageref{nb__kernel__ElecEw__VdwCSTab__GeomW3W3__avx__128__fma__double_8c}}{}
\item\contentsline{section}{src/gmxlib/nonbonded/nb\-\_\-kernel\-\_\-avx\-\_\-128\-\_\-fma\-\_\-double/\hyperlink{nb__kernel__ElecEw__VdwCSTab__GeomW4P1__avx__128__fma__double_8c}{nb\-\_\-kernel\-\_\-\-Elec\-Ew\-\_\-\-Vdw\-C\-S\-Tab\-\_\-\-Geom\-W4\-P1\-\_\-avx\-\_\-128\-\_\-fma\-\_\-double.\-c} }{\pageref{nb__kernel__ElecEw__VdwCSTab__GeomW4P1__avx__128__fma__double_8c}}{}
\item\contentsline{section}{src/gmxlib/nonbonded/nb\-\_\-kernel\-\_\-avx\-\_\-128\-\_\-fma\-\_\-double/\hyperlink{nb__kernel__ElecEw__VdwCSTab__GeomW4W4__avx__128__fma__double_8c}{nb\-\_\-kernel\-\_\-\-Elec\-Ew\-\_\-\-Vdw\-C\-S\-Tab\-\_\-\-Geom\-W4\-W4\-\_\-avx\-\_\-128\-\_\-fma\-\_\-double.\-c} }{\pageref{nb__kernel__ElecEw__VdwCSTab__GeomW4W4__avx__128__fma__double_8c}}{}
\item\contentsline{section}{src/gmxlib/nonbonded/nb\-\_\-kernel\-\_\-avx\-\_\-128\-\_\-fma\-\_\-double/\hyperlink{nb__kernel__ElecEw__VdwLJ__GeomP1P1__avx__128__fma__double_8c}{nb\-\_\-kernel\-\_\-\-Elec\-Ew\-\_\-\-Vdw\-L\-J\-\_\-\-Geom\-P1\-P1\-\_\-avx\-\_\-128\-\_\-fma\-\_\-double.\-c} }{\pageref{nb__kernel__ElecEw__VdwLJ__GeomP1P1__avx__128__fma__double_8c}}{}
\item\contentsline{section}{src/gmxlib/nonbonded/nb\-\_\-kernel\-\_\-avx\-\_\-128\-\_\-fma\-\_\-double/\hyperlink{nb__kernel__ElecEw__VdwLJ__GeomW3P1__avx__128__fma__double_8c}{nb\-\_\-kernel\-\_\-\-Elec\-Ew\-\_\-\-Vdw\-L\-J\-\_\-\-Geom\-W3\-P1\-\_\-avx\-\_\-128\-\_\-fma\-\_\-double.\-c} }{\pageref{nb__kernel__ElecEw__VdwLJ__GeomW3P1__avx__128__fma__double_8c}}{}
\item\contentsline{section}{src/gmxlib/nonbonded/nb\-\_\-kernel\-\_\-avx\-\_\-128\-\_\-fma\-\_\-double/\hyperlink{nb__kernel__ElecEw__VdwLJ__GeomW3W3__avx__128__fma__double_8c}{nb\-\_\-kernel\-\_\-\-Elec\-Ew\-\_\-\-Vdw\-L\-J\-\_\-\-Geom\-W3\-W3\-\_\-avx\-\_\-128\-\_\-fma\-\_\-double.\-c} }{\pageref{nb__kernel__ElecEw__VdwLJ__GeomW3W3__avx__128__fma__double_8c}}{}
\item\contentsline{section}{src/gmxlib/nonbonded/nb\-\_\-kernel\-\_\-avx\-\_\-128\-\_\-fma\-\_\-double/\hyperlink{nb__kernel__ElecEw__VdwLJ__GeomW4P1__avx__128__fma__double_8c}{nb\-\_\-kernel\-\_\-\-Elec\-Ew\-\_\-\-Vdw\-L\-J\-\_\-\-Geom\-W4\-P1\-\_\-avx\-\_\-128\-\_\-fma\-\_\-double.\-c} }{\pageref{nb__kernel__ElecEw__VdwLJ__GeomW4P1__avx__128__fma__double_8c}}{}
\item\contentsline{section}{src/gmxlib/nonbonded/nb\-\_\-kernel\-\_\-avx\-\_\-128\-\_\-fma\-\_\-double/\hyperlink{nb__kernel__ElecEw__VdwLJ__GeomW4W4__avx__128__fma__double_8c}{nb\-\_\-kernel\-\_\-\-Elec\-Ew\-\_\-\-Vdw\-L\-J\-\_\-\-Geom\-W4\-W4\-\_\-avx\-\_\-128\-\_\-fma\-\_\-double.\-c} }{\pageref{nb__kernel__ElecEw__VdwLJ__GeomW4W4__avx__128__fma__double_8c}}{}
\item\contentsline{section}{src/gmxlib/nonbonded/nb\-\_\-kernel\-\_\-avx\-\_\-128\-\_\-fma\-\_\-double/\hyperlink{nb__kernel__ElecEw__VdwNone__GeomP1P1__avx__128__fma__double_8c}{nb\-\_\-kernel\-\_\-\-Elec\-Ew\-\_\-\-Vdw\-None\-\_\-\-Geom\-P1\-P1\-\_\-avx\-\_\-128\-\_\-fma\-\_\-double.\-c} }{\pageref{nb__kernel__ElecEw__VdwNone__GeomP1P1__avx__128__fma__double_8c}}{}
\item\contentsline{section}{src/gmxlib/nonbonded/nb\-\_\-kernel\-\_\-avx\-\_\-128\-\_\-fma\-\_\-double/\hyperlink{nb__kernel__ElecEw__VdwNone__GeomW3P1__avx__128__fma__double_8c}{nb\-\_\-kernel\-\_\-\-Elec\-Ew\-\_\-\-Vdw\-None\-\_\-\-Geom\-W3\-P1\-\_\-avx\-\_\-128\-\_\-fma\-\_\-double.\-c} }{\pageref{nb__kernel__ElecEw__VdwNone__GeomW3P1__avx__128__fma__double_8c}}{}
\item\contentsline{section}{src/gmxlib/nonbonded/nb\-\_\-kernel\-\_\-avx\-\_\-128\-\_\-fma\-\_\-double/\hyperlink{nb__kernel__ElecEw__VdwNone__GeomW3W3__avx__128__fma__double_8c}{nb\-\_\-kernel\-\_\-\-Elec\-Ew\-\_\-\-Vdw\-None\-\_\-\-Geom\-W3\-W3\-\_\-avx\-\_\-128\-\_\-fma\-\_\-double.\-c} }{\pageref{nb__kernel__ElecEw__VdwNone__GeomW3W3__avx__128__fma__double_8c}}{}
\item\contentsline{section}{src/gmxlib/nonbonded/nb\-\_\-kernel\-\_\-avx\-\_\-128\-\_\-fma\-\_\-double/\hyperlink{nb__kernel__ElecEw__VdwNone__GeomW4P1__avx__128__fma__double_8c}{nb\-\_\-kernel\-\_\-\-Elec\-Ew\-\_\-\-Vdw\-None\-\_\-\-Geom\-W4\-P1\-\_\-avx\-\_\-128\-\_\-fma\-\_\-double.\-c} }{\pageref{nb__kernel__ElecEw__VdwNone__GeomW4P1__avx__128__fma__double_8c}}{}
\item\contentsline{section}{src/gmxlib/nonbonded/nb\-\_\-kernel\-\_\-avx\-\_\-128\-\_\-fma\-\_\-double/\hyperlink{nb__kernel__ElecEw__VdwNone__GeomW4W4__avx__128__fma__double_8c}{nb\-\_\-kernel\-\_\-\-Elec\-Ew\-\_\-\-Vdw\-None\-\_\-\-Geom\-W4\-W4\-\_\-avx\-\_\-128\-\_\-fma\-\_\-double.\-c} }{\pageref{nb__kernel__ElecEw__VdwNone__GeomW4W4__avx__128__fma__double_8c}}{}
\item\contentsline{section}{src/gmxlib/nonbonded/nb\-\_\-kernel\-\_\-avx\-\_\-128\-\_\-fma\-\_\-double/\hyperlink{nb__kernel__ElecEwSh__VdwLJSh__GeomP1P1__avx__128__fma__double_8c}{nb\-\_\-kernel\-\_\-\-Elec\-Ew\-Sh\-\_\-\-Vdw\-L\-J\-Sh\-\_\-\-Geom\-P1\-P1\-\_\-avx\-\_\-128\-\_\-fma\-\_\-double.\-c} }{\pageref{nb__kernel__ElecEwSh__VdwLJSh__GeomP1P1__avx__128__fma__double_8c}}{}
\item\contentsline{section}{src/gmxlib/nonbonded/nb\-\_\-kernel\-\_\-avx\-\_\-128\-\_\-fma\-\_\-double/\hyperlink{nb__kernel__ElecEwSh__VdwLJSh__GeomW3P1__avx__128__fma__double_8c}{nb\-\_\-kernel\-\_\-\-Elec\-Ew\-Sh\-\_\-\-Vdw\-L\-J\-Sh\-\_\-\-Geom\-W3\-P1\-\_\-avx\-\_\-128\-\_\-fma\-\_\-double.\-c} }{\pageref{nb__kernel__ElecEwSh__VdwLJSh__GeomW3P1__avx__128__fma__double_8c}}{}
\item\contentsline{section}{src/gmxlib/nonbonded/nb\-\_\-kernel\-\_\-avx\-\_\-128\-\_\-fma\-\_\-double/\hyperlink{nb__kernel__ElecEwSh__VdwLJSh__GeomW3W3__avx__128__fma__double_8c}{nb\-\_\-kernel\-\_\-\-Elec\-Ew\-Sh\-\_\-\-Vdw\-L\-J\-Sh\-\_\-\-Geom\-W3\-W3\-\_\-avx\-\_\-128\-\_\-fma\-\_\-double.\-c} }{\pageref{nb__kernel__ElecEwSh__VdwLJSh__GeomW3W3__avx__128__fma__double_8c}}{}
\item\contentsline{section}{src/gmxlib/nonbonded/nb\-\_\-kernel\-\_\-avx\-\_\-128\-\_\-fma\-\_\-double/\hyperlink{nb__kernel__ElecEwSh__VdwLJSh__GeomW4P1__avx__128__fma__double_8c}{nb\-\_\-kernel\-\_\-\-Elec\-Ew\-Sh\-\_\-\-Vdw\-L\-J\-Sh\-\_\-\-Geom\-W4\-P1\-\_\-avx\-\_\-128\-\_\-fma\-\_\-double.\-c} }{\pageref{nb__kernel__ElecEwSh__VdwLJSh__GeomW4P1__avx__128__fma__double_8c}}{}
\item\contentsline{section}{src/gmxlib/nonbonded/nb\-\_\-kernel\-\_\-avx\-\_\-128\-\_\-fma\-\_\-double/\hyperlink{nb__kernel__ElecEwSh__VdwLJSh__GeomW4W4__avx__128__fma__double_8c}{nb\-\_\-kernel\-\_\-\-Elec\-Ew\-Sh\-\_\-\-Vdw\-L\-J\-Sh\-\_\-\-Geom\-W4\-W4\-\_\-avx\-\_\-128\-\_\-fma\-\_\-double.\-c} }{\pageref{nb__kernel__ElecEwSh__VdwLJSh__GeomW4W4__avx__128__fma__double_8c}}{}
\item\contentsline{section}{src/gmxlib/nonbonded/nb\-\_\-kernel\-\_\-avx\-\_\-128\-\_\-fma\-\_\-double/\hyperlink{nb__kernel__ElecEwSh__VdwNone__GeomP1P1__avx__128__fma__double_8c}{nb\-\_\-kernel\-\_\-\-Elec\-Ew\-Sh\-\_\-\-Vdw\-None\-\_\-\-Geom\-P1\-P1\-\_\-avx\-\_\-128\-\_\-fma\-\_\-double.\-c} }{\pageref{nb__kernel__ElecEwSh__VdwNone__GeomP1P1__avx__128__fma__double_8c}}{}
\item\contentsline{section}{src/gmxlib/nonbonded/nb\-\_\-kernel\-\_\-avx\-\_\-128\-\_\-fma\-\_\-double/\hyperlink{nb__kernel__ElecEwSh__VdwNone__GeomW3P1__avx__128__fma__double_8c}{nb\-\_\-kernel\-\_\-\-Elec\-Ew\-Sh\-\_\-\-Vdw\-None\-\_\-\-Geom\-W3\-P1\-\_\-avx\-\_\-128\-\_\-fma\-\_\-double.\-c} }{\pageref{nb__kernel__ElecEwSh__VdwNone__GeomW3P1__avx__128__fma__double_8c}}{}
\item\contentsline{section}{src/gmxlib/nonbonded/nb\-\_\-kernel\-\_\-avx\-\_\-128\-\_\-fma\-\_\-double/\hyperlink{nb__kernel__ElecEwSh__VdwNone__GeomW3W3__avx__128__fma__double_8c}{nb\-\_\-kernel\-\_\-\-Elec\-Ew\-Sh\-\_\-\-Vdw\-None\-\_\-\-Geom\-W3\-W3\-\_\-avx\-\_\-128\-\_\-fma\-\_\-double.\-c} }{\pageref{nb__kernel__ElecEwSh__VdwNone__GeomW3W3__avx__128__fma__double_8c}}{}
\item\contentsline{section}{src/gmxlib/nonbonded/nb\-\_\-kernel\-\_\-avx\-\_\-128\-\_\-fma\-\_\-double/\hyperlink{nb__kernel__ElecEwSh__VdwNone__GeomW4P1__avx__128__fma__double_8c}{nb\-\_\-kernel\-\_\-\-Elec\-Ew\-Sh\-\_\-\-Vdw\-None\-\_\-\-Geom\-W4\-P1\-\_\-avx\-\_\-128\-\_\-fma\-\_\-double.\-c} }{\pageref{nb__kernel__ElecEwSh__VdwNone__GeomW4P1__avx__128__fma__double_8c}}{}
\item\contentsline{section}{src/gmxlib/nonbonded/nb\-\_\-kernel\-\_\-avx\-\_\-128\-\_\-fma\-\_\-double/\hyperlink{nb__kernel__ElecEwSh__VdwNone__GeomW4W4__avx__128__fma__double_8c}{nb\-\_\-kernel\-\_\-\-Elec\-Ew\-Sh\-\_\-\-Vdw\-None\-\_\-\-Geom\-W4\-W4\-\_\-avx\-\_\-128\-\_\-fma\-\_\-double.\-c} }{\pageref{nb__kernel__ElecEwSh__VdwNone__GeomW4W4__avx__128__fma__double_8c}}{}
\item\contentsline{section}{src/gmxlib/nonbonded/nb\-\_\-kernel\-\_\-avx\-\_\-128\-\_\-fma\-\_\-double/\hyperlink{nb__kernel__ElecEwSw__VdwLJSw__GeomP1P1__avx__128__fma__double_8c}{nb\-\_\-kernel\-\_\-\-Elec\-Ew\-Sw\-\_\-\-Vdw\-L\-J\-Sw\-\_\-\-Geom\-P1\-P1\-\_\-avx\-\_\-128\-\_\-fma\-\_\-double.\-c} }{\pageref{nb__kernel__ElecEwSw__VdwLJSw__GeomP1P1__avx__128__fma__double_8c}}{}
\item\contentsline{section}{src/gmxlib/nonbonded/nb\-\_\-kernel\-\_\-avx\-\_\-128\-\_\-fma\-\_\-double/\hyperlink{nb__kernel__ElecEwSw__VdwLJSw__GeomW3P1__avx__128__fma__double_8c}{nb\-\_\-kernel\-\_\-\-Elec\-Ew\-Sw\-\_\-\-Vdw\-L\-J\-Sw\-\_\-\-Geom\-W3\-P1\-\_\-avx\-\_\-128\-\_\-fma\-\_\-double.\-c} }{\pageref{nb__kernel__ElecEwSw__VdwLJSw__GeomW3P1__avx__128__fma__double_8c}}{}
\item\contentsline{section}{src/gmxlib/nonbonded/nb\-\_\-kernel\-\_\-avx\-\_\-128\-\_\-fma\-\_\-double/\hyperlink{nb__kernel__ElecEwSw__VdwLJSw__GeomW3W3__avx__128__fma__double_8c}{nb\-\_\-kernel\-\_\-\-Elec\-Ew\-Sw\-\_\-\-Vdw\-L\-J\-Sw\-\_\-\-Geom\-W3\-W3\-\_\-avx\-\_\-128\-\_\-fma\-\_\-double.\-c} }{\pageref{nb__kernel__ElecEwSw__VdwLJSw__GeomW3W3__avx__128__fma__double_8c}}{}
\item\contentsline{section}{src/gmxlib/nonbonded/nb\-\_\-kernel\-\_\-avx\-\_\-128\-\_\-fma\-\_\-double/\hyperlink{nb__kernel__ElecEwSw__VdwLJSw__GeomW4P1__avx__128__fma__double_8c}{nb\-\_\-kernel\-\_\-\-Elec\-Ew\-Sw\-\_\-\-Vdw\-L\-J\-Sw\-\_\-\-Geom\-W4\-P1\-\_\-avx\-\_\-128\-\_\-fma\-\_\-double.\-c} }{\pageref{nb__kernel__ElecEwSw__VdwLJSw__GeomW4P1__avx__128__fma__double_8c}}{}
\item\contentsline{section}{src/gmxlib/nonbonded/nb\-\_\-kernel\-\_\-avx\-\_\-128\-\_\-fma\-\_\-double/\hyperlink{nb__kernel__ElecEwSw__VdwLJSw__GeomW4W4__avx__128__fma__double_8c}{nb\-\_\-kernel\-\_\-\-Elec\-Ew\-Sw\-\_\-\-Vdw\-L\-J\-Sw\-\_\-\-Geom\-W4\-W4\-\_\-avx\-\_\-128\-\_\-fma\-\_\-double.\-c} }{\pageref{nb__kernel__ElecEwSw__VdwLJSw__GeomW4W4__avx__128__fma__double_8c}}{}
\item\contentsline{section}{src/gmxlib/nonbonded/nb\-\_\-kernel\-\_\-avx\-\_\-128\-\_\-fma\-\_\-double/\hyperlink{nb__kernel__ElecEwSw__VdwNone__GeomP1P1__avx__128__fma__double_8c}{nb\-\_\-kernel\-\_\-\-Elec\-Ew\-Sw\-\_\-\-Vdw\-None\-\_\-\-Geom\-P1\-P1\-\_\-avx\-\_\-128\-\_\-fma\-\_\-double.\-c} }{\pageref{nb__kernel__ElecEwSw__VdwNone__GeomP1P1__avx__128__fma__double_8c}}{}
\item\contentsline{section}{src/gmxlib/nonbonded/nb\-\_\-kernel\-\_\-avx\-\_\-128\-\_\-fma\-\_\-double/\hyperlink{nb__kernel__ElecEwSw__VdwNone__GeomW3P1__avx__128__fma__double_8c}{nb\-\_\-kernel\-\_\-\-Elec\-Ew\-Sw\-\_\-\-Vdw\-None\-\_\-\-Geom\-W3\-P1\-\_\-avx\-\_\-128\-\_\-fma\-\_\-double.\-c} }{\pageref{nb__kernel__ElecEwSw__VdwNone__GeomW3P1__avx__128__fma__double_8c}}{}
\item\contentsline{section}{src/gmxlib/nonbonded/nb\-\_\-kernel\-\_\-avx\-\_\-128\-\_\-fma\-\_\-double/\hyperlink{nb__kernel__ElecEwSw__VdwNone__GeomW3W3__avx__128__fma__double_8c}{nb\-\_\-kernel\-\_\-\-Elec\-Ew\-Sw\-\_\-\-Vdw\-None\-\_\-\-Geom\-W3\-W3\-\_\-avx\-\_\-128\-\_\-fma\-\_\-double.\-c} }{\pageref{nb__kernel__ElecEwSw__VdwNone__GeomW3W3__avx__128__fma__double_8c}}{}
\item\contentsline{section}{src/gmxlib/nonbonded/nb\-\_\-kernel\-\_\-avx\-\_\-128\-\_\-fma\-\_\-double/\hyperlink{nb__kernel__ElecEwSw__VdwNone__GeomW4P1__avx__128__fma__double_8c}{nb\-\_\-kernel\-\_\-\-Elec\-Ew\-Sw\-\_\-\-Vdw\-None\-\_\-\-Geom\-W4\-P1\-\_\-avx\-\_\-128\-\_\-fma\-\_\-double.\-c} }{\pageref{nb__kernel__ElecEwSw__VdwNone__GeomW4P1__avx__128__fma__double_8c}}{}
\item\contentsline{section}{src/gmxlib/nonbonded/nb\-\_\-kernel\-\_\-avx\-\_\-128\-\_\-fma\-\_\-double/\hyperlink{nb__kernel__ElecEwSw__VdwNone__GeomW4W4__avx__128__fma__double_8c}{nb\-\_\-kernel\-\_\-\-Elec\-Ew\-Sw\-\_\-\-Vdw\-None\-\_\-\-Geom\-W4\-W4\-\_\-avx\-\_\-128\-\_\-fma\-\_\-double.\-c} }{\pageref{nb__kernel__ElecEwSw__VdwNone__GeomW4W4__avx__128__fma__double_8c}}{}
\item\contentsline{section}{src/gmxlib/nonbonded/nb\-\_\-kernel\-\_\-avx\-\_\-128\-\_\-fma\-\_\-double/\hyperlink{nb__kernel__ElecGB__VdwCSTab__GeomP1P1__avx__128__fma__double_8c}{nb\-\_\-kernel\-\_\-\-Elec\-G\-B\-\_\-\-Vdw\-C\-S\-Tab\-\_\-\-Geom\-P1\-P1\-\_\-avx\-\_\-128\-\_\-fma\-\_\-double.\-c} }{\pageref{nb__kernel__ElecGB__VdwCSTab__GeomP1P1__avx__128__fma__double_8c}}{}
\item\contentsline{section}{src/gmxlib/nonbonded/nb\-\_\-kernel\-\_\-avx\-\_\-128\-\_\-fma\-\_\-double/\hyperlink{nb__kernel__ElecGB__VdwLJ__GeomP1P1__avx__128__fma__double_8c}{nb\-\_\-kernel\-\_\-\-Elec\-G\-B\-\_\-\-Vdw\-L\-J\-\_\-\-Geom\-P1\-P1\-\_\-avx\-\_\-128\-\_\-fma\-\_\-double.\-c} }{\pageref{nb__kernel__ElecGB__VdwLJ__GeomP1P1__avx__128__fma__double_8c}}{}
\item\contentsline{section}{src/gmxlib/nonbonded/nb\-\_\-kernel\-\_\-avx\-\_\-128\-\_\-fma\-\_\-double/\hyperlink{nb__kernel__ElecGB__VdwNone__GeomP1P1__avx__128__fma__double_8c}{nb\-\_\-kernel\-\_\-\-Elec\-G\-B\-\_\-\-Vdw\-None\-\_\-\-Geom\-P1\-P1\-\_\-avx\-\_\-128\-\_\-fma\-\_\-double.\-c} }{\pageref{nb__kernel__ElecGB__VdwNone__GeomP1P1__avx__128__fma__double_8c}}{}
\item\contentsline{section}{src/gmxlib/nonbonded/nb\-\_\-kernel\-\_\-avx\-\_\-128\-\_\-fma\-\_\-double/\hyperlink{nb__kernel__ElecNone__VdwCSTab__GeomP1P1__avx__128__fma__double_8c}{nb\-\_\-kernel\-\_\-\-Elec\-None\-\_\-\-Vdw\-C\-S\-Tab\-\_\-\-Geom\-P1\-P1\-\_\-avx\-\_\-128\-\_\-fma\-\_\-double.\-c} }{\pageref{nb__kernel__ElecNone__VdwCSTab__GeomP1P1__avx__128__fma__double_8c}}{}
\item\contentsline{section}{src/gmxlib/nonbonded/nb\-\_\-kernel\-\_\-avx\-\_\-128\-\_\-fma\-\_\-double/\hyperlink{nb__kernel__ElecNone__VdwLJ__GeomP1P1__avx__128__fma__double_8c}{nb\-\_\-kernel\-\_\-\-Elec\-None\-\_\-\-Vdw\-L\-J\-\_\-\-Geom\-P1\-P1\-\_\-avx\-\_\-128\-\_\-fma\-\_\-double.\-c} }{\pageref{nb__kernel__ElecNone__VdwLJ__GeomP1P1__avx__128__fma__double_8c}}{}
\item\contentsline{section}{src/gmxlib/nonbonded/nb\-\_\-kernel\-\_\-avx\-\_\-128\-\_\-fma\-\_\-double/\hyperlink{nb__kernel__ElecNone__VdwLJSh__GeomP1P1__avx__128__fma__double_8c}{nb\-\_\-kernel\-\_\-\-Elec\-None\-\_\-\-Vdw\-L\-J\-Sh\-\_\-\-Geom\-P1\-P1\-\_\-avx\-\_\-128\-\_\-fma\-\_\-double.\-c} }{\pageref{nb__kernel__ElecNone__VdwLJSh__GeomP1P1__avx__128__fma__double_8c}}{}
\item\contentsline{section}{src/gmxlib/nonbonded/nb\-\_\-kernel\-\_\-avx\-\_\-128\-\_\-fma\-\_\-double/\hyperlink{nb__kernel__ElecNone__VdwLJSw__GeomP1P1__avx__128__fma__double_8c}{nb\-\_\-kernel\-\_\-\-Elec\-None\-\_\-\-Vdw\-L\-J\-Sw\-\_\-\-Geom\-P1\-P1\-\_\-avx\-\_\-128\-\_\-fma\-\_\-double.\-c} }{\pageref{nb__kernel__ElecNone__VdwLJSw__GeomP1P1__avx__128__fma__double_8c}}{}
\item\contentsline{section}{src/gmxlib/nonbonded/nb\-\_\-kernel\-\_\-avx\-\_\-128\-\_\-fma\-\_\-double/\hyperlink{nb__kernel__ElecRF__VdwCSTab__GeomP1P1__avx__128__fma__double_8c}{nb\-\_\-kernel\-\_\-\-Elec\-R\-F\-\_\-\-Vdw\-C\-S\-Tab\-\_\-\-Geom\-P1\-P1\-\_\-avx\-\_\-128\-\_\-fma\-\_\-double.\-c} }{\pageref{nb__kernel__ElecRF__VdwCSTab__GeomP1P1__avx__128__fma__double_8c}}{}
\item\contentsline{section}{src/gmxlib/nonbonded/nb\-\_\-kernel\-\_\-avx\-\_\-128\-\_\-fma\-\_\-double/\hyperlink{nb__kernel__ElecRF__VdwCSTab__GeomW3P1__avx__128__fma__double_8c}{nb\-\_\-kernel\-\_\-\-Elec\-R\-F\-\_\-\-Vdw\-C\-S\-Tab\-\_\-\-Geom\-W3\-P1\-\_\-avx\-\_\-128\-\_\-fma\-\_\-double.\-c} }{\pageref{nb__kernel__ElecRF__VdwCSTab__GeomW3P1__avx__128__fma__double_8c}}{}
\item\contentsline{section}{src/gmxlib/nonbonded/nb\-\_\-kernel\-\_\-avx\-\_\-128\-\_\-fma\-\_\-double/\hyperlink{nb__kernel__ElecRF__VdwCSTab__GeomW3W3__avx__128__fma__double_8c}{nb\-\_\-kernel\-\_\-\-Elec\-R\-F\-\_\-\-Vdw\-C\-S\-Tab\-\_\-\-Geom\-W3\-W3\-\_\-avx\-\_\-128\-\_\-fma\-\_\-double.\-c} }{\pageref{nb__kernel__ElecRF__VdwCSTab__GeomW3W3__avx__128__fma__double_8c}}{}
\item\contentsline{section}{src/gmxlib/nonbonded/nb\-\_\-kernel\-\_\-avx\-\_\-128\-\_\-fma\-\_\-double/\hyperlink{nb__kernel__ElecRF__VdwCSTab__GeomW4P1__avx__128__fma__double_8c}{nb\-\_\-kernel\-\_\-\-Elec\-R\-F\-\_\-\-Vdw\-C\-S\-Tab\-\_\-\-Geom\-W4\-P1\-\_\-avx\-\_\-128\-\_\-fma\-\_\-double.\-c} }{\pageref{nb__kernel__ElecRF__VdwCSTab__GeomW4P1__avx__128__fma__double_8c}}{}
\item\contentsline{section}{src/gmxlib/nonbonded/nb\-\_\-kernel\-\_\-avx\-\_\-128\-\_\-fma\-\_\-double/\hyperlink{nb__kernel__ElecRF__VdwCSTab__GeomW4W4__avx__128__fma__double_8c}{nb\-\_\-kernel\-\_\-\-Elec\-R\-F\-\_\-\-Vdw\-C\-S\-Tab\-\_\-\-Geom\-W4\-W4\-\_\-avx\-\_\-128\-\_\-fma\-\_\-double.\-c} }{\pageref{nb__kernel__ElecRF__VdwCSTab__GeomW4W4__avx__128__fma__double_8c}}{}
\item\contentsline{section}{src/gmxlib/nonbonded/nb\-\_\-kernel\-\_\-avx\-\_\-128\-\_\-fma\-\_\-double/\hyperlink{nb__kernel__ElecRF__VdwLJ__GeomP1P1__avx__128__fma__double_8c}{nb\-\_\-kernel\-\_\-\-Elec\-R\-F\-\_\-\-Vdw\-L\-J\-\_\-\-Geom\-P1\-P1\-\_\-avx\-\_\-128\-\_\-fma\-\_\-double.\-c} }{\pageref{nb__kernel__ElecRF__VdwLJ__GeomP1P1__avx__128__fma__double_8c}}{}
\item\contentsline{section}{src/gmxlib/nonbonded/nb\-\_\-kernel\-\_\-avx\-\_\-128\-\_\-fma\-\_\-double/\hyperlink{nb__kernel__ElecRF__VdwLJ__GeomW3P1__avx__128__fma__double_8c}{nb\-\_\-kernel\-\_\-\-Elec\-R\-F\-\_\-\-Vdw\-L\-J\-\_\-\-Geom\-W3\-P1\-\_\-avx\-\_\-128\-\_\-fma\-\_\-double.\-c} }{\pageref{nb__kernel__ElecRF__VdwLJ__GeomW3P1__avx__128__fma__double_8c}}{}
\item\contentsline{section}{src/gmxlib/nonbonded/nb\-\_\-kernel\-\_\-avx\-\_\-128\-\_\-fma\-\_\-double/\hyperlink{nb__kernel__ElecRF__VdwLJ__GeomW3W3__avx__128__fma__double_8c}{nb\-\_\-kernel\-\_\-\-Elec\-R\-F\-\_\-\-Vdw\-L\-J\-\_\-\-Geom\-W3\-W3\-\_\-avx\-\_\-128\-\_\-fma\-\_\-double.\-c} }{\pageref{nb__kernel__ElecRF__VdwLJ__GeomW3W3__avx__128__fma__double_8c}}{}
\item\contentsline{section}{src/gmxlib/nonbonded/nb\-\_\-kernel\-\_\-avx\-\_\-128\-\_\-fma\-\_\-double/\hyperlink{nb__kernel__ElecRF__VdwLJ__GeomW4P1__avx__128__fma__double_8c}{nb\-\_\-kernel\-\_\-\-Elec\-R\-F\-\_\-\-Vdw\-L\-J\-\_\-\-Geom\-W4\-P1\-\_\-avx\-\_\-128\-\_\-fma\-\_\-double.\-c} }{\pageref{nb__kernel__ElecRF__VdwLJ__GeomW4P1__avx__128__fma__double_8c}}{}
\item\contentsline{section}{src/gmxlib/nonbonded/nb\-\_\-kernel\-\_\-avx\-\_\-128\-\_\-fma\-\_\-double/\hyperlink{nb__kernel__ElecRF__VdwLJ__GeomW4W4__avx__128__fma__double_8c}{nb\-\_\-kernel\-\_\-\-Elec\-R\-F\-\_\-\-Vdw\-L\-J\-\_\-\-Geom\-W4\-W4\-\_\-avx\-\_\-128\-\_\-fma\-\_\-double.\-c} }{\pageref{nb__kernel__ElecRF__VdwLJ__GeomW4W4__avx__128__fma__double_8c}}{}
\item\contentsline{section}{src/gmxlib/nonbonded/nb\-\_\-kernel\-\_\-avx\-\_\-128\-\_\-fma\-\_\-double/\hyperlink{nb__kernel__ElecRF__VdwNone__GeomP1P1__avx__128__fma__double_8c}{nb\-\_\-kernel\-\_\-\-Elec\-R\-F\-\_\-\-Vdw\-None\-\_\-\-Geom\-P1\-P1\-\_\-avx\-\_\-128\-\_\-fma\-\_\-double.\-c} }{\pageref{nb__kernel__ElecRF__VdwNone__GeomP1P1__avx__128__fma__double_8c}}{}
\item\contentsline{section}{src/gmxlib/nonbonded/nb\-\_\-kernel\-\_\-avx\-\_\-128\-\_\-fma\-\_\-double/\hyperlink{nb__kernel__ElecRF__VdwNone__GeomW3P1__avx__128__fma__double_8c}{nb\-\_\-kernel\-\_\-\-Elec\-R\-F\-\_\-\-Vdw\-None\-\_\-\-Geom\-W3\-P1\-\_\-avx\-\_\-128\-\_\-fma\-\_\-double.\-c} }{\pageref{nb__kernel__ElecRF__VdwNone__GeomW3P1__avx__128__fma__double_8c}}{}
\item\contentsline{section}{src/gmxlib/nonbonded/nb\-\_\-kernel\-\_\-avx\-\_\-128\-\_\-fma\-\_\-double/\hyperlink{nb__kernel__ElecRF__VdwNone__GeomW3W3__avx__128__fma__double_8c}{nb\-\_\-kernel\-\_\-\-Elec\-R\-F\-\_\-\-Vdw\-None\-\_\-\-Geom\-W3\-W3\-\_\-avx\-\_\-128\-\_\-fma\-\_\-double.\-c} }{\pageref{nb__kernel__ElecRF__VdwNone__GeomW3W3__avx__128__fma__double_8c}}{}
\item\contentsline{section}{src/gmxlib/nonbonded/nb\-\_\-kernel\-\_\-avx\-\_\-128\-\_\-fma\-\_\-double/\hyperlink{nb__kernel__ElecRF__VdwNone__GeomW4P1__avx__128__fma__double_8c}{nb\-\_\-kernel\-\_\-\-Elec\-R\-F\-\_\-\-Vdw\-None\-\_\-\-Geom\-W4\-P1\-\_\-avx\-\_\-128\-\_\-fma\-\_\-double.\-c} }{\pageref{nb__kernel__ElecRF__VdwNone__GeomW4P1__avx__128__fma__double_8c}}{}
\item\contentsline{section}{src/gmxlib/nonbonded/nb\-\_\-kernel\-\_\-avx\-\_\-128\-\_\-fma\-\_\-double/\hyperlink{nb__kernel__ElecRF__VdwNone__GeomW4W4__avx__128__fma__double_8c}{nb\-\_\-kernel\-\_\-\-Elec\-R\-F\-\_\-\-Vdw\-None\-\_\-\-Geom\-W4\-W4\-\_\-avx\-\_\-128\-\_\-fma\-\_\-double.\-c} }{\pageref{nb__kernel__ElecRF__VdwNone__GeomW4W4__avx__128__fma__double_8c}}{}
\item\contentsline{section}{src/gmxlib/nonbonded/nb\-\_\-kernel\-\_\-avx\-\_\-128\-\_\-fma\-\_\-double/\hyperlink{nb__kernel__ElecRFCut__VdwCSTab__GeomP1P1__avx__128__fma__double_8c}{nb\-\_\-kernel\-\_\-\-Elec\-R\-F\-Cut\-\_\-\-Vdw\-C\-S\-Tab\-\_\-\-Geom\-P1\-P1\-\_\-avx\-\_\-128\-\_\-fma\-\_\-double.\-c} }{\pageref{nb__kernel__ElecRFCut__VdwCSTab__GeomP1P1__avx__128__fma__double_8c}}{}
\item\contentsline{section}{src/gmxlib/nonbonded/nb\-\_\-kernel\-\_\-avx\-\_\-128\-\_\-fma\-\_\-double/\hyperlink{nb__kernel__ElecRFCut__VdwCSTab__GeomW3P1__avx__128__fma__double_8c}{nb\-\_\-kernel\-\_\-\-Elec\-R\-F\-Cut\-\_\-\-Vdw\-C\-S\-Tab\-\_\-\-Geom\-W3\-P1\-\_\-avx\-\_\-128\-\_\-fma\-\_\-double.\-c} }{\pageref{nb__kernel__ElecRFCut__VdwCSTab__GeomW3P1__avx__128__fma__double_8c}}{}
\item\contentsline{section}{src/gmxlib/nonbonded/nb\-\_\-kernel\-\_\-avx\-\_\-128\-\_\-fma\-\_\-double/\hyperlink{nb__kernel__ElecRFCut__VdwCSTab__GeomW3W3__avx__128__fma__double_8c}{nb\-\_\-kernel\-\_\-\-Elec\-R\-F\-Cut\-\_\-\-Vdw\-C\-S\-Tab\-\_\-\-Geom\-W3\-W3\-\_\-avx\-\_\-128\-\_\-fma\-\_\-double.\-c} }{\pageref{nb__kernel__ElecRFCut__VdwCSTab__GeomW3W3__avx__128__fma__double_8c}}{}
\item\contentsline{section}{src/gmxlib/nonbonded/nb\-\_\-kernel\-\_\-avx\-\_\-128\-\_\-fma\-\_\-double/\hyperlink{nb__kernel__ElecRFCut__VdwCSTab__GeomW4P1__avx__128__fma__double_8c}{nb\-\_\-kernel\-\_\-\-Elec\-R\-F\-Cut\-\_\-\-Vdw\-C\-S\-Tab\-\_\-\-Geom\-W4\-P1\-\_\-avx\-\_\-128\-\_\-fma\-\_\-double.\-c} }{\pageref{nb__kernel__ElecRFCut__VdwCSTab__GeomW4P1__avx__128__fma__double_8c}}{}
\item\contentsline{section}{src/gmxlib/nonbonded/nb\-\_\-kernel\-\_\-avx\-\_\-128\-\_\-fma\-\_\-double/\hyperlink{nb__kernel__ElecRFCut__VdwCSTab__GeomW4W4__avx__128__fma__double_8c}{nb\-\_\-kernel\-\_\-\-Elec\-R\-F\-Cut\-\_\-\-Vdw\-C\-S\-Tab\-\_\-\-Geom\-W4\-W4\-\_\-avx\-\_\-128\-\_\-fma\-\_\-double.\-c} }{\pageref{nb__kernel__ElecRFCut__VdwCSTab__GeomW4W4__avx__128__fma__double_8c}}{}
\item\contentsline{section}{src/gmxlib/nonbonded/nb\-\_\-kernel\-\_\-avx\-\_\-128\-\_\-fma\-\_\-double/\hyperlink{nb__kernel__ElecRFCut__VdwLJSh__GeomP1P1__avx__128__fma__double_8c}{nb\-\_\-kernel\-\_\-\-Elec\-R\-F\-Cut\-\_\-\-Vdw\-L\-J\-Sh\-\_\-\-Geom\-P1\-P1\-\_\-avx\-\_\-128\-\_\-fma\-\_\-double.\-c} }{\pageref{nb__kernel__ElecRFCut__VdwLJSh__GeomP1P1__avx__128__fma__double_8c}}{}
\item\contentsline{section}{src/gmxlib/nonbonded/nb\-\_\-kernel\-\_\-avx\-\_\-128\-\_\-fma\-\_\-double/\hyperlink{nb__kernel__ElecRFCut__VdwLJSh__GeomW3P1__avx__128__fma__double_8c}{nb\-\_\-kernel\-\_\-\-Elec\-R\-F\-Cut\-\_\-\-Vdw\-L\-J\-Sh\-\_\-\-Geom\-W3\-P1\-\_\-avx\-\_\-128\-\_\-fma\-\_\-double.\-c} }{\pageref{nb__kernel__ElecRFCut__VdwLJSh__GeomW3P1__avx__128__fma__double_8c}}{}
\item\contentsline{section}{src/gmxlib/nonbonded/nb\-\_\-kernel\-\_\-avx\-\_\-128\-\_\-fma\-\_\-double/\hyperlink{nb__kernel__ElecRFCut__VdwLJSh__GeomW3W3__avx__128__fma__double_8c}{nb\-\_\-kernel\-\_\-\-Elec\-R\-F\-Cut\-\_\-\-Vdw\-L\-J\-Sh\-\_\-\-Geom\-W3\-W3\-\_\-avx\-\_\-128\-\_\-fma\-\_\-double.\-c} }{\pageref{nb__kernel__ElecRFCut__VdwLJSh__GeomW3W3__avx__128__fma__double_8c}}{}
\item\contentsline{section}{src/gmxlib/nonbonded/nb\-\_\-kernel\-\_\-avx\-\_\-128\-\_\-fma\-\_\-double/\hyperlink{nb__kernel__ElecRFCut__VdwLJSh__GeomW4P1__avx__128__fma__double_8c}{nb\-\_\-kernel\-\_\-\-Elec\-R\-F\-Cut\-\_\-\-Vdw\-L\-J\-Sh\-\_\-\-Geom\-W4\-P1\-\_\-avx\-\_\-128\-\_\-fma\-\_\-double.\-c} }{\pageref{nb__kernel__ElecRFCut__VdwLJSh__GeomW4P1__avx__128__fma__double_8c}}{}
\item\contentsline{section}{src/gmxlib/nonbonded/nb\-\_\-kernel\-\_\-avx\-\_\-128\-\_\-fma\-\_\-double/\hyperlink{nb__kernel__ElecRFCut__VdwLJSh__GeomW4W4__avx__128__fma__double_8c}{nb\-\_\-kernel\-\_\-\-Elec\-R\-F\-Cut\-\_\-\-Vdw\-L\-J\-Sh\-\_\-\-Geom\-W4\-W4\-\_\-avx\-\_\-128\-\_\-fma\-\_\-double.\-c} }{\pageref{nb__kernel__ElecRFCut__VdwLJSh__GeomW4W4__avx__128__fma__double_8c}}{}
\item\contentsline{section}{src/gmxlib/nonbonded/nb\-\_\-kernel\-\_\-avx\-\_\-128\-\_\-fma\-\_\-double/\hyperlink{nb__kernel__ElecRFCut__VdwLJSw__GeomP1P1__avx__128__fma__double_8c}{nb\-\_\-kernel\-\_\-\-Elec\-R\-F\-Cut\-\_\-\-Vdw\-L\-J\-Sw\-\_\-\-Geom\-P1\-P1\-\_\-avx\-\_\-128\-\_\-fma\-\_\-double.\-c} }{\pageref{nb__kernel__ElecRFCut__VdwLJSw__GeomP1P1__avx__128__fma__double_8c}}{}
\item\contentsline{section}{src/gmxlib/nonbonded/nb\-\_\-kernel\-\_\-avx\-\_\-128\-\_\-fma\-\_\-double/\hyperlink{nb__kernel__ElecRFCut__VdwLJSw__GeomW3P1__avx__128__fma__double_8c}{nb\-\_\-kernel\-\_\-\-Elec\-R\-F\-Cut\-\_\-\-Vdw\-L\-J\-Sw\-\_\-\-Geom\-W3\-P1\-\_\-avx\-\_\-128\-\_\-fma\-\_\-double.\-c} }{\pageref{nb__kernel__ElecRFCut__VdwLJSw__GeomW3P1__avx__128__fma__double_8c}}{}
\item\contentsline{section}{src/gmxlib/nonbonded/nb\-\_\-kernel\-\_\-avx\-\_\-128\-\_\-fma\-\_\-double/\hyperlink{nb__kernel__ElecRFCut__VdwLJSw__GeomW3W3__avx__128__fma__double_8c}{nb\-\_\-kernel\-\_\-\-Elec\-R\-F\-Cut\-\_\-\-Vdw\-L\-J\-Sw\-\_\-\-Geom\-W3\-W3\-\_\-avx\-\_\-128\-\_\-fma\-\_\-double.\-c} }{\pageref{nb__kernel__ElecRFCut__VdwLJSw__GeomW3W3__avx__128__fma__double_8c}}{}
\item\contentsline{section}{src/gmxlib/nonbonded/nb\-\_\-kernel\-\_\-avx\-\_\-128\-\_\-fma\-\_\-double/\hyperlink{nb__kernel__ElecRFCut__VdwLJSw__GeomW4P1__avx__128__fma__double_8c}{nb\-\_\-kernel\-\_\-\-Elec\-R\-F\-Cut\-\_\-\-Vdw\-L\-J\-Sw\-\_\-\-Geom\-W4\-P1\-\_\-avx\-\_\-128\-\_\-fma\-\_\-double.\-c} }{\pageref{nb__kernel__ElecRFCut__VdwLJSw__GeomW4P1__avx__128__fma__double_8c}}{}
\item\contentsline{section}{src/gmxlib/nonbonded/nb\-\_\-kernel\-\_\-avx\-\_\-128\-\_\-fma\-\_\-double/\hyperlink{nb__kernel__ElecRFCut__VdwLJSw__GeomW4W4__avx__128__fma__double_8c}{nb\-\_\-kernel\-\_\-\-Elec\-R\-F\-Cut\-\_\-\-Vdw\-L\-J\-Sw\-\_\-\-Geom\-W4\-W4\-\_\-avx\-\_\-128\-\_\-fma\-\_\-double.\-c} }{\pageref{nb__kernel__ElecRFCut__VdwLJSw__GeomW4W4__avx__128__fma__double_8c}}{}
\item\contentsline{section}{src/gmxlib/nonbonded/nb\-\_\-kernel\-\_\-avx\-\_\-128\-\_\-fma\-\_\-double/\hyperlink{nb__kernel__ElecRFCut__VdwNone__GeomP1P1__avx__128__fma__double_8c}{nb\-\_\-kernel\-\_\-\-Elec\-R\-F\-Cut\-\_\-\-Vdw\-None\-\_\-\-Geom\-P1\-P1\-\_\-avx\-\_\-128\-\_\-fma\-\_\-double.\-c} }{\pageref{nb__kernel__ElecRFCut__VdwNone__GeomP1P1__avx__128__fma__double_8c}}{}
\item\contentsline{section}{src/gmxlib/nonbonded/nb\-\_\-kernel\-\_\-avx\-\_\-128\-\_\-fma\-\_\-double/\hyperlink{nb__kernel__ElecRFCut__VdwNone__GeomW3P1__avx__128__fma__double_8c}{nb\-\_\-kernel\-\_\-\-Elec\-R\-F\-Cut\-\_\-\-Vdw\-None\-\_\-\-Geom\-W3\-P1\-\_\-avx\-\_\-128\-\_\-fma\-\_\-double.\-c} }{\pageref{nb__kernel__ElecRFCut__VdwNone__GeomW3P1__avx__128__fma__double_8c}}{}
\item\contentsline{section}{src/gmxlib/nonbonded/nb\-\_\-kernel\-\_\-avx\-\_\-128\-\_\-fma\-\_\-double/\hyperlink{nb__kernel__ElecRFCut__VdwNone__GeomW3W3__avx__128__fma__double_8c}{nb\-\_\-kernel\-\_\-\-Elec\-R\-F\-Cut\-\_\-\-Vdw\-None\-\_\-\-Geom\-W3\-W3\-\_\-avx\-\_\-128\-\_\-fma\-\_\-double.\-c} }{\pageref{nb__kernel__ElecRFCut__VdwNone__GeomW3W3__avx__128__fma__double_8c}}{}
\item\contentsline{section}{src/gmxlib/nonbonded/nb\-\_\-kernel\-\_\-avx\-\_\-128\-\_\-fma\-\_\-double/\hyperlink{nb__kernel__ElecRFCut__VdwNone__GeomW4P1__avx__128__fma__double_8c}{nb\-\_\-kernel\-\_\-\-Elec\-R\-F\-Cut\-\_\-\-Vdw\-None\-\_\-\-Geom\-W4\-P1\-\_\-avx\-\_\-128\-\_\-fma\-\_\-double.\-c} }{\pageref{nb__kernel__ElecRFCut__VdwNone__GeomW4P1__avx__128__fma__double_8c}}{}
\item\contentsline{section}{src/gmxlib/nonbonded/nb\-\_\-kernel\-\_\-avx\-\_\-128\-\_\-fma\-\_\-double/\hyperlink{nb__kernel__ElecRFCut__VdwNone__GeomW4W4__avx__128__fma__double_8c}{nb\-\_\-kernel\-\_\-\-Elec\-R\-F\-Cut\-\_\-\-Vdw\-None\-\_\-\-Geom\-W4\-W4\-\_\-avx\-\_\-128\-\_\-fma\-\_\-double.\-c} }{\pageref{nb__kernel__ElecRFCut__VdwNone__GeomW4W4__avx__128__fma__double_8c}}{}
\item\contentsline{section}{src/gmxlib/nonbonded/nb\-\_\-kernel\-\_\-avx\-\_\-128\-\_\-fma\-\_\-single/\hyperlink{kernelutil__x86__avx__128__fma__single_8h}{kernelutil\-\_\-x86\-\_\-avx\-\_\-128\-\_\-fma\-\_\-single.\-h} }{\pageref{kernelutil__x86__avx__128__fma__single_8h}}{}
\item\contentsline{section}{src/gmxlib/nonbonded/nb\-\_\-kernel\-\_\-avx\-\_\-128\-\_\-fma\-\_\-single/\hyperlink{nb__kernel__avx__128__fma__single_8c}{nb\-\_\-kernel\-\_\-avx\-\_\-128\-\_\-fma\-\_\-single.\-c} }{\pageref{nb__kernel__avx__128__fma__single_8c}}{}
\item\contentsline{section}{src/gmxlib/nonbonded/nb\-\_\-kernel\-\_\-avx\-\_\-128\-\_\-fma\-\_\-single/\hyperlink{nb__kernel__avx__128__fma__single_8h}{nb\-\_\-kernel\-\_\-avx\-\_\-128\-\_\-fma\-\_\-single.\-h} }{\pageref{nb__kernel__avx__128__fma__single_8h}}{}
\item\contentsline{section}{src/gmxlib/nonbonded/nb\-\_\-kernel\-\_\-avx\-\_\-128\-\_\-fma\-\_\-single/\hyperlink{nb__kernel__ElecCoul__VdwCSTab__GeomP1P1__avx__128__fma__single_8c}{nb\-\_\-kernel\-\_\-\-Elec\-Coul\-\_\-\-Vdw\-C\-S\-Tab\-\_\-\-Geom\-P1\-P1\-\_\-avx\-\_\-128\-\_\-fma\-\_\-single.\-c} }{\pageref{nb__kernel__ElecCoul__VdwCSTab__GeomP1P1__avx__128__fma__single_8c}}{}
\item\contentsline{section}{src/gmxlib/nonbonded/nb\-\_\-kernel\-\_\-avx\-\_\-128\-\_\-fma\-\_\-single/\hyperlink{nb__kernel__ElecCoul__VdwCSTab__GeomW3P1__avx__128__fma__single_8c}{nb\-\_\-kernel\-\_\-\-Elec\-Coul\-\_\-\-Vdw\-C\-S\-Tab\-\_\-\-Geom\-W3\-P1\-\_\-avx\-\_\-128\-\_\-fma\-\_\-single.\-c} }{\pageref{nb__kernel__ElecCoul__VdwCSTab__GeomW3P1__avx__128__fma__single_8c}}{}
\item\contentsline{section}{src/gmxlib/nonbonded/nb\-\_\-kernel\-\_\-avx\-\_\-128\-\_\-fma\-\_\-single/\hyperlink{nb__kernel__ElecCoul__VdwCSTab__GeomW3W3__avx__128__fma__single_8c}{nb\-\_\-kernel\-\_\-\-Elec\-Coul\-\_\-\-Vdw\-C\-S\-Tab\-\_\-\-Geom\-W3\-W3\-\_\-avx\-\_\-128\-\_\-fma\-\_\-single.\-c} }{\pageref{nb__kernel__ElecCoul__VdwCSTab__GeomW3W3__avx__128__fma__single_8c}}{}
\item\contentsline{section}{src/gmxlib/nonbonded/nb\-\_\-kernel\-\_\-avx\-\_\-128\-\_\-fma\-\_\-single/\hyperlink{nb__kernel__ElecCoul__VdwCSTab__GeomW4P1__avx__128__fma__single_8c}{nb\-\_\-kernel\-\_\-\-Elec\-Coul\-\_\-\-Vdw\-C\-S\-Tab\-\_\-\-Geom\-W4\-P1\-\_\-avx\-\_\-128\-\_\-fma\-\_\-single.\-c} }{\pageref{nb__kernel__ElecCoul__VdwCSTab__GeomW4P1__avx__128__fma__single_8c}}{}
\item\contentsline{section}{src/gmxlib/nonbonded/nb\-\_\-kernel\-\_\-avx\-\_\-128\-\_\-fma\-\_\-single/\hyperlink{nb__kernel__ElecCoul__VdwCSTab__GeomW4W4__avx__128__fma__single_8c}{nb\-\_\-kernel\-\_\-\-Elec\-Coul\-\_\-\-Vdw\-C\-S\-Tab\-\_\-\-Geom\-W4\-W4\-\_\-avx\-\_\-128\-\_\-fma\-\_\-single.\-c} }{\pageref{nb__kernel__ElecCoul__VdwCSTab__GeomW4W4__avx__128__fma__single_8c}}{}
\item\contentsline{section}{src/gmxlib/nonbonded/nb\-\_\-kernel\-\_\-avx\-\_\-128\-\_\-fma\-\_\-single/\hyperlink{nb__kernel__ElecCoul__VdwLJ__GeomP1P1__avx__128__fma__single_8c}{nb\-\_\-kernel\-\_\-\-Elec\-Coul\-\_\-\-Vdw\-L\-J\-\_\-\-Geom\-P1\-P1\-\_\-avx\-\_\-128\-\_\-fma\-\_\-single.\-c} }{\pageref{nb__kernel__ElecCoul__VdwLJ__GeomP1P1__avx__128__fma__single_8c}}{}
\item\contentsline{section}{src/gmxlib/nonbonded/nb\-\_\-kernel\-\_\-avx\-\_\-128\-\_\-fma\-\_\-single/\hyperlink{nb__kernel__ElecCoul__VdwLJ__GeomW3P1__avx__128__fma__single_8c}{nb\-\_\-kernel\-\_\-\-Elec\-Coul\-\_\-\-Vdw\-L\-J\-\_\-\-Geom\-W3\-P1\-\_\-avx\-\_\-128\-\_\-fma\-\_\-single.\-c} }{\pageref{nb__kernel__ElecCoul__VdwLJ__GeomW3P1__avx__128__fma__single_8c}}{}
\item\contentsline{section}{src/gmxlib/nonbonded/nb\-\_\-kernel\-\_\-avx\-\_\-128\-\_\-fma\-\_\-single/\hyperlink{nb__kernel__ElecCoul__VdwLJ__GeomW3W3__avx__128__fma__single_8c}{nb\-\_\-kernel\-\_\-\-Elec\-Coul\-\_\-\-Vdw\-L\-J\-\_\-\-Geom\-W3\-W3\-\_\-avx\-\_\-128\-\_\-fma\-\_\-single.\-c} }{\pageref{nb__kernel__ElecCoul__VdwLJ__GeomW3W3__avx__128__fma__single_8c}}{}
\item\contentsline{section}{src/gmxlib/nonbonded/nb\-\_\-kernel\-\_\-avx\-\_\-128\-\_\-fma\-\_\-single/\hyperlink{nb__kernel__ElecCoul__VdwLJ__GeomW4P1__avx__128__fma__single_8c}{nb\-\_\-kernel\-\_\-\-Elec\-Coul\-\_\-\-Vdw\-L\-J\-\_\-\-Geom\-W4\-P1\-\_\-avx\-\_\-128\-\_\-fma\-\_\-single.\-c} }{\pageref{nb__kernel__ElecCoul__VdwLJ__GeomW4P1__avx__128__fma__single_8c}}{}
\item\contentsline{section}{src/gmxlib/nonbonded/nb\-\_\-kernel\-\_\-avx\-\_\-128\-\_\-fma\-\_\-single/\hyperlink{nb__kernel__ElecCoul__VdwLJ__GeomW4W4__avx__128__fma__single_8c}{nb\-\_\-kernel\-\_\-\-Elec\-Coul\-\_\-\-Vdw\-L\-J\-\_\-\-Geom\-W4\-W4\-\_\-avx\-\_\-128\-\_\-fma\-\_\-single.\-c} }{\pageref{nb__kernel__ElecCoul__VdwLJ__GeomW4W4__avx__128__fma__single_8c}}{}
\item\contentsline{section}{src/gmxlib/nonbonded/nb\-\_\-kernel\-\_\-avx\-\_\-128\-\_\-fma\-\_\-single/\hyperlink{nb__kernel__ElecCoul__VdwNone__GeomP1P1__avx__128__fma__single_8c}{nb\-\_\-kernel\-\_\-\-Elec\-Coul\-\_\-\-Vdw\-None\-\_\-\-Geom\-P1\-P1\-\_\-avx\-\_\-128\-\_\-fma\-\_\-single.\-c} }{\pageref{nb__kernel__ElecCoul__VdwNone__GeomP1P1__avx__128__fma__single_8c}}{}
\item\contentsline{section}{src/gmxlib/nonbonded/nb\-\_\-kernel\-\_\-avx\-\_\-128\-\_\-fma\-\_\-single/\hyperlink{nb__kernel__ElecCoul__VdwNone__GeomW3P1__avx__128__fma__single_8c}{nb\-\_\-kernel\-\_\-\-Elec\-Coul\-\_\-\-Vdw\-None\-\_\-\-Geom\-W3\-P1\-\_\-avx\-\_\-128\-\_\-fma\-\_\-single.\-c} }{\pageref{nb__kernel__ElecCoul__VdwNone__GeomW3P1__avx__128__fma__single_8c}}{}
\item\contentsline{section}{src/gmxlib/nonbonded/nb\-\_\-kernel\-\_\-avx\-\_\-128\-\_\-fma\-\_\-single/\hyperlink{nb__kernel__ElecCoul__VdwNone__GeomW3W3__avx__128__fma__single_8c}{nb\-\_\-kernel\-\_\-\-Elec\-Coul\-\_\-\-Vdw\-None\-\_\-\-Geom\-W3\-W3\-\_\-avx\-\_\-128\-\_\-fma\-\_\-single.\-c} }{\pageref{nb__kernel__ElecCoul__VdwNone__GeomW3W3__avx__128__fma__single_8c}}{}
\item\contentsline{section}{src/gmxlib/nonbonded/nb\-\_\-kernel\-\_\-avx\-\_\-128\-\_\-fma\-\_\-single/\hyperlink{nb__kernel__ElecCoul__VdwNone__GeomW4P1__avx__128__fma__single_8c}{nb\-\_\-kernel\-\_\-\-Elec\-Coul\-\_\-\-Vdw\-None\-\_\-\-Geom\-W4\-P1\-\_\-avx\-\_\-128\-\_\-fma\-\_\-single.\-c} }{\pageref{nb__kernel__ElecCoul__VdwNone__GeomW4P1__avx__128__fma__single_8c}}{}
\item\contentsline{section}{src/gmxlib/nonbonded/nb\-\_\-kernel\-\_\-avx\-\_\-128\-\_\-fma\-\_\-single/\hyperlink{nb__kernel__ElecCoul__VdwNone__GeomW4W4__avx__128__fma__single_8c}{nb\-\_\-kernel\-\_\-\-Elec\-Coul\-\_\-\-Vdw\-None\-\_\-\-Geom\-W4\-W4\-\_\-avx\-\_\-128\-\_\-fma\-\_\-single.\-c} }{\pageref{nb__kernel__ElecCoul__VdwNone__GeomW4W4__avx__128__fma__single_8c}}{}
\item\contentsline{section}{src/gmxlib/nonbonded/nb\-\_\-kernel\-\_\-avx\-\_\-128\-\_\-fma\-\_\-single/\hyperlink{nb__kernel__ElecCSTab__VdwCSTab__GeomP1P1__avx__128__fma__single_8c}{nb\-\_\-kernel\-\_\-\-Elec\-C\-S\-Tab\-\_\-\-Vdw\-C\-S\-Tab\-\_\-\-Geom\-P1\-P1\-\_\-avx\-\_\-128\-\_\-fma\-\_\-single.\-c} }{\pageref{nb__kernel__ElecCSTab__VdwCSTab__GeomP1P1__avx__128__fma__single_8c}}{}
\item\contentsline{section}{src/gmxlib/nonbonded/nb\-\_\-kernel\-\_\-avx\-\_\-128\-\_\-fma\-\_\-single/\hyperlink{nb__kernel__ElecCSTab__VdwCSTab__GeomW3P1__avx__128__fma__single_8c}{nb\-\_\-kernel\-\_\-\-Elec\-C\-S\-Tab\-\_\-\-Vdw\-C\-S\-Tab\-\_\-\-Geom\-W3\-P1\-\_\-avx\-\_\-128\-\_\-fma\-\_\-single.\-c} }{\pageref{nb__kernel__ElecCSTab__VdwCSTab__GeomW3P1__avx__128__fma__single_8c}}{}
\item\contentsline{section}{src/gmxlib/nonbonded/nb\-\_\-kernel\-\_\-avx\-\_\-128\-\_\-fma\-\_\-single/\hyperlink{nb__kernel__ElecCSTab__VdwCSTab__GeomW3W3__avx__128__fma__single_8c}{nb\-\_\-kernel\-\_\-\-Elec\-C\-S\-Tab\-\_\-\-Vdw\-C\-S\-Tab\-\_\-\-Geom\-W3\-W3\-\_\-avx\-\_\-128\-\_\-fma\-\_\-single.\-c} }{\pageref{nb__kernel__ElecCSTab__VdwCSTab__GeomW3W3__avx__128__fma__single_8c}}{}
\item\contentsline{section}{src/gmxlib/nonbonded/nb\-\_\-kernel\-\_\-avx\-\_\-128\-\_\-fma\-\_\-single/\hyperlink{nb__kernel__ElecCSTab__VdwCSTab__GeomW4P1__avx__128__fma__single_8c}{nb\-\_\-kernel\-\_\-\-Elec\-C\-S\-Tab\-\_\-\-Vdw\-C\-S\-Tab\-\_\-\-Geom\-W4\-P1\-\_\-avx\-\_\-128\-\_\-fma\-\_\-single.\-c} }{\pageref{nb__kernel__ElecCSTab__VdwCSTab__GeomW4P1__avx__128__fma__single_8c}}{}
\item\contentsline{section}{src/gmxlib/nonbonded/nb\-\_\-kernel\-\_\-avx\-\_\-128\-\_\-fma\-\_\-single/\hyperlink{nb__kernel__ElecCSTab__VdwCSTab__GeomW4W4__avx__128__fma__single_8c}{nb\-\_\-kernel\-\_\-\-Elec\-C\-S\-Tab\-\_\-\-Vdw\-C\-S\-Tab\-\_\-\-Geom\-W4\-W4\-\_\-avx\-\_\-128\-\_\-fma\-\_\-single.\-c} }{\pageref{nb__kernel__ElecCSTab__VdwCSTab__GeomW4W4__avx__128__fma__single_8c}}{}
\item\contentsline{section}{src/gmxlib/nonbonded/nb\-\_\-kernel\-\_\-avx\-\_\-128\-\_\-fma\-\_\-single/\hyperlink{nb__kernel__ElecCSTab__VdwLJ__GeomP1P1__avx__128__fma__single_8c}{nb\-\_\-kernel\-\_\-\-Elec\-C\-S\-Tab\-\_\-\-Vdw\-L\-J\-\_\-\-Geom\-P1\-P1\-\_\-avx\-\_\-128\-\_\-fma\-\_\-single.\-c} }{\pageref{nb__kernel__ElecCSTab__VdwLJ__GeomP1P1__avx__128__fma__single_8c}}{}
\item\contentsline{section}{src/gmxlib/nonbonded/nb\-\_\-kernel\-\_\-avx\-\_\-128\-\_\-fma\-\_\-single/\hyperlink{nb__kernel__ElecCSTab__VdwLJ__GeomW3P1__avx__128__fma__single_8c}{nb\-\_\-kernel\-\_\-\-Elec\-C\-S\-Tab\-\_\-\-Vdw\-L\-J\-\_\-\-Geom\-W3\-P1\-\_\-avx\-\_\-128\-\_\-fma\-\_\-single.\-c} }{\pageref{nb__kernel__ElecCSTab__VdwLJ__GeomW3P1__avx__128__fma__single_8c}}{}
\item\contentsline{section}{src/gmxlib/nonbonded/nb\-\_\-kernel\-\_\-avx\-\_\-128\-\_\-fma\-\_\-single/\hyperlink{nb__kernel__ElecCSTab__VdwLJ__GeomW3W3__avx__128__fma__single_8c}{nb\-\_\-kernel\-\_\-\-Elec\-C\-S\-Tab\-\_\-\-Vdw\-L\-J\-\_\-\-Geom\-W3\-W3\-\_\-avx\-\_\-128\-\_\-fma\-\_\-single.\-c} }{\pageref{nb__kernel__ElecCSTab__VdwLJ__GeomW3W3__avx__128__fma__single_8c}}{}
\item\contentsline{section}{src/gmxlib/nonbonded/nb\-\_\-kernel\-\_\-avx\-\_\-128\-\_\-fma\-\_\-single/\hyperlink{nb__kernel__ElecCSTab__VdwLJ__GeomW4P1__avx__128__fma__single_8c}{nb\-\_\-kernel\-\_\-\-Elec\-C\-S\-Tab\-\_\-\-Vdw\-L\-J\-\_\-\-Geom\-W4\-P1\-\_\-avx\-\_\-128\-\_\-fma\-\_\-single.\-c} }{\pageref{nb__kernel__ElecCSTab__VdwLJ__GeomW4P1__avx__128__fma__single_8c}}{}
\item\contentsline{section}{src/gmxlib/nonbonded/nb\-\_\-kernel\-\_\-avx\-\_\-128\-\_\-fma\-\_\-single/\hyperlink{nb__kernel__ElecCSTab__VdwLJ__GeomW4W4__avx__128__fma__single_8c}{nb\-\_\-kernel\-\_\-\-Elec\-C\-S\-Tab\-\_\-\-Vdw\-L\-J\-\_\-\-Geom\-W4\-W4\-\_\-avx\-\_\-128\-\_\-fma\-\_\-single.\-c} }{\pageref{nb__kernel__ElecCSTab__VdwLJ__GeomW4W4__avx__128__fma__single_8c}}{}
\item\contentsline{section}{src/gmxlib/nonbonded/nb\-\_\-kernel\-\_\-avx\-\_\-128\-\_\-fma\-\_\-single/\hyperlink{nb__kernel__ElecCSTab__VdwNone__GeomP1P1__avx__128__fma__single_8c}{nb\-\_\-kernel\-\_\-\-Elec\-C\-S\-Tab\-\_\-\-Vdw\-None\-\_\-\-Geom\-P1\-P1\-\_\-avx\-\_\-128\-\_\-fma\-\_\-single.\-c} }{\pageref{nb__kernel__ElecCSTab__VdwNone__GeomP1P1__avx__128__fma__single_8c}}{}
\item\contentsline{section}{src/gmxlib/nonbonded/nb\-\_\-kernel\-\_\-avx\-\_\-128\-\_\-fma\-\_\-single/\hyperlink{nb__kernel__ElecCSTab__VdwNone__GeomW3P1__avx__128__fma__single_8c}{nb\-\_\-kernel\-\_\-\-Elec\-C\-S\-Tab\-\_\-\-Vdw\-None\-\_\-\-Geom\-W3\-P1\-\_\-avx\-\_\-128\-\_\-fma\-\_\-single.\-c} }{\pageref{nb__kernel__ElecCSTab__VdwNone__GeomW3P1__avx__128__fma__single_8c}}{}
\item\contentsline{section}{src/gmxlib/nonbonded/nb\-\_\-kernel\-\_\-avx\-\_\-128\-\_\-fma\-\_\-single/\hyperlink{nb__kernel__ElecCSTab__VdwNone__GeomW3W3__avx__128__fma__single_8c}{nb\-\_\-kernel\-\_\-\-Elec\-C\-S\-Tab\-\_\-\-Vdw\-None\-\_\-\-Geom\-W3\-W3\-\_\-avx\-\_\-128\-\_\-fma\-\_\-single.\-c} }{\pageref{nb__kernel__ElecCSTab__VdwNone__GeomW3W3__avx__128__fma__single_8c}}{}
\item\contentsline{section}{src/gmxlib/nonbonded/nb\-\_\-kernel\-\_\-avx\-\_\-128\-\_\-fma\-\_\-single/\hyperlink{nb__kernel__ElecCSTab__VdwNone__GeomW4P1__avx__128__fma__single_8c}{nb\-\_\-kernel\-\_\-\-Elec\-C\-S\-Tab\-\_\-\-Vdw\-None\-\_\-\-Geom\-W4\-P1\-\_\-avx\-\_\-128\-\_\-fma\-\_\-single.\-c} }{\pageref{nb__kernel__ElecCSTab__VdwNone__GeomW4P1__avx__128__fma__single_8c}}{}
\item\contentsline{section}{src/gmxlib/nonbonded/nb\-\_\-kernel\-\_\-avx\-\_\-128\-\_\-fma\-\_\-single/\hyperlink{nb__kernel__ElecCSTab__VdwNone__GeomW4W4__avx__128__fma__single_8c}{nb\-\_\-kernel\-\_\-\-Elec\-C\-S\-Tab\-\_\-\-Vdw\-None\-\_\-\-Geom\-W4\-W4\-\_\-avx\-\_\-128\-\_\-fma\-\_\-single.\-c} }{\pageref{nb__kernel__ElecCSTab__VdwNone__GeomW4W4__avx__128__fma__single_8c}}{}
\item\contentsline{section}{src/gmxlib/nonbonded/nb\-\_\-kernel\-\_\-avx\-\_\-128\-\_\-fma\-\_\-single/\hyperlink{nb__kernel__ElecEw__VdwCSTab__GeomP1P1__avx__128__fma__single_8c}{nb\-\_\-kernel\-\_\-\-Elec\-Ew\-\_\-\-Vdw\-C\-S\-Tab\-\_\-\-Geom\-P1\-P1\-\_\-avx\-\_\-128\-\_\-fma\-\_\-single.\-c} }{\pageref{nb__kernel__ElecEw__VdwCSTab__GeomP1P1__avx__128__fma__single_8c}}{}
\item\contentsline{section}{src/gmxlib/nonbonded/nb\-\_\-kernel\-\_\-avx\-\_\-128\-\_\-fma\-\_\-single/\hyperlink{nb__kernel__ElecEw__VdwCSTab__GeomW3P1__avx__128__fma__single_8c}{nb\-\_\-kernel\-\_\-\-Elec\-Ew\-\_\-\-Vdw\-C\-S\-Tab\-\_\-\-Geom\-W3\-P1\-\_\-avx\-\_\-128\-\_\-fma\-\_\-single.\-c} }{\pageref{nb__kernel__ElecEw__VdwCSTab__GeomW3P1__avx__128__fma__single_8c}}{}
\item\contentsline{section}{src/gmxlib/nonbonded/nb\-\_\-kernel\-\_\-avx\-\_\-128\-\_\-fma\-\_\-single/\hyperlink{nb__kernel__ElecEw__VdwCSTab__GeomW3W3__avx__128__fma__single_8c}{nb\-\_\-kernel\-\_\-\-Elec\-Ew\-\_\-\-Vdw\-C\-S\-Tab\-\_\-\-Geom\-W3\-W3\-\_\-avx\-\_\-128\-\_\-fma\-\_\-single.\-c} }{\pageref{nb__kernel__ElecEw__VdwCSTab__GeomW3W3__avx__128__fma__single_8c}}{}
\item\contentsline{section}{src/gmxlib/nonbonded/nb\-\_\-kernel\-\_\-avx\-\_\-128\-\_\-fma\-\_\-single/\hyperlink{nb__kernel__ElecEw__VdwCSTab__GeomW4P1__avx__128__fma__single_8c}{nb\-\_\-kernel\-\_\-\-Elec\-Ew\-\_\-\-Vdw\-C\-S\-Tab\-\_\-\-Geom\-W4\-P1\-\_\-avx\-\_\-128\-\_\-fma\-\_\-single.\-c} }{\pageref{nb__kernel__ElecEw__VdwCSTab__GeomW4P1__avx__128__fma__single_8c}}{}
\item\contentsline{section}{src/gmxlib/nonbonded/nb\-\_\-kernel\-\_\-avx\-\_\-128\-\_\-fma\-\_\-single/\hyperlink{nb__kernel__ElecEw__VdwCSTab__GeomW4W4__avx__128__fma__single_8c}{nb\-\_\-kernel\-\_\-\-Elec\-Ew\-\_\-\-Vdw\-C\-S\-Tab\-\_\-\-Geom\-W4\-W4\-\_\-avx\-\_\-128\-\_\-fma\-\_\-single.\-c} }{\pageref{nb__kernel__ElecEw__VdwCSTab__GeomW4W4__avx__128__fma__single_8c}}{}
\item\contentsline{section}{src/gmxlib/nonbonded/nb\-\_\-kernel\-\_\-avx\-\_\-128\-\_\-fma\-\_\-single/\hyperlink{nb__kernel__ElecEw__VdwLJ__GeomP1P1__avx__128__fma__single_8c}{nb\-\_\-kernel\-\_\-\-Elec\-Ew\-\_\-\-Vdw\-L\-J\-\_\-\-Geom\-P1\-P1\-\_\-avx\-\_\-128\-\_\-fma\-\_\-single.\-c} }{\pageref{nb__kernel__ElecEw__VdwLJ__GeomP1P1__avx__128__fma__single_8c}}{}
\item\contentsline{section}{src/gmxlib/nonbonded/nb\-\_\-kernel\-\_\-avx\-\_\-128\-\_\-fma\-\_\-single/\hyperlink{nb__kernel__ElecEw__VdwLJ__GeomW3P1__avx__128__fma__single_8c}{nb\-\_\-kernel\-\_\-\-Elec\-Ew\-\_\-\-Vdw\-L\-J\-\_\-\-Geom\-W3\-P1\-\_\-avx\-\_\-128\-\_\-fma\-\_\-single.\-c} }{\pageref{nb__kernel__ElecEw__VdwLJ__GeomW3P1__avx__128__fma__single_8c}}{}
\item\contentsline{section}{src/gmxlib/nonbonded/nb\-\_\-kernel\-\_\-avx\-\_\-128\-\_\-fma\-\_\-single/\hyperlink{nb__kernel__ElecEw__VdwLJ__GeomW3W3__avx__128__fma__single_8c}{nb\-\_\-kernel\-\_\-\-Elec\-Ew\-\_\-\-Vdw\-L\-J\-\_\-\-Geom\-W3\-W3\-\_\-avx\-\_\-128\-\_\-fma\-\_\-single.\-c} }{\pageref{nb__kernel__ElecEw__VdwLJ__GeomW3W3__avx__128__fma__single_8c}}{}
\item\contentsline{section}{src/gmxlib/nonbonded/nb\-\_\-kernel\-\_\-avx\-\_\-128\-\_\-fma\-\_\-single/\hyperlink{nb__kernel__ElecEw__VdwLJ__GeomW4P1__avx__128__fma__single_8c}{nb\-\_\-kernel\-\_\-\-Elec\-Ew\-\_\-\-Vdw\-L\-J\-\_\-\-Geom\-W4\-P1\-\_\-avx\-\_\-128\-\_\-fma\-\_\-single.\-c} }{\pageref{nb__kernel__ElecEw__VdwLJ__GeomW4P1__avx__128__fma__single_8c}}{}
\item\contentsline{section}{src/gmxlib/nonbonded/nb\-\_\-kernel\-\_\-avx\-\_\-128\-\_\-fma\-\_\-single/\hyperlink{nb__kernel__ElecEw__VdwLJ__GeomW4W4__avx__128__fma__single_8c}{nb\-\_\-kernel\-\_\-\-Elec\-Ew\-\_\-\-Vdw\-L\-J\-\_\-\-Geom\-W4\-W4\-\_\-avx\-\_\-128\-\_\-fma\-\_\-single.\-c} }{\pageref{nb__kernel__ElecEw__VdwLJ__GeomW4W4__avx__128__fma__single_8c}}{}
\item\contentsline{section}{src/gmxlib/nonbonded/nb\-\_\-kernel\-\_\-avx\-\_\-128\-\_\-fma\-\_\-single/\hyperlink{nb__kernel__ElecEw__VdwNone__GeomP1P1__avx__128__fma__single_8c}{nb\-\_\-kernel\-\_\-\-Elec\-Ew\-\_\-\-Vdw\-None\-\_\-\-Geom\-P1\-P1\-\_\-avx\-\_\-128\-\_\-fma\-\_\-single.\-c} }{\pageref{nb__kernel__ElecEw__VdwNone__GeomP1P1__avx__128__fma__single_8c}}{}
\item\contentsline{section}{src/gmxlib/nonbonded/nb\-\_\-kernel\-\_\-avx\-\_\-128\-\_\-fma\-\_\-single/\hyperlink{nb__kernel__ElecEw__VdwNone__GeomW3P1__avx__128__fma__single_8c}{nb\-\_\-kernel\-\_\-\-Elec\-Ew\-\_\-\-Vdw\-None\-\_\-\-Geom\-W3\-P1\-\_\-avx\-\_\-128\-\_\-fma\-\_\-single.\-c} }{\pageref{nb__kernel__ElecEw__VdwNone__GeomW3P1__avx__128__fma__single_8c}}{}
\item\contentsline{section}{src/gmxlib/nonbonded/nb\-\_\-kernel\-\_\-avx\-\_\-128\-\_\-fma\-\_\-single/\hyperlink{nb__kernel__ElecEw__VdwNone__GeomW3W3__avx__128__fma__single_8c}{nb\-\_\-kernel\-\_\-\-Elec\-Ew\-\_\-\-Vdw\-None\-\_\-\-Geom\-W3\-W3\-\_\-avx\-\_\-128\-\_\-fma\-\_\-single.\-c} }{\pageref{nb__kernel__ElecEw__VdwNone__GeomW3W3__avx__128__fma__single_8c}}{}
\item\contentsline{section}{src/gmxlib/nonbonded/nb\-\_\-kernel\-\_\-avx\-\_\-128\-\_\-fma\-\_\-single/\hyperlink{nb__kernel__ElecEw__VdwNone__GeomW4P1__avx__128__fma__single_8c}{nb\-\_\-kernel\-\_\-\-Elec\-Ew\-\_\-\-Vdw\-None\-\_\-\-Geom\-W4\-P1\-\_\-avx\-\_\-128\-\_\-fma\-\_\-single.\-c} }{\pageref{nb__kernel__ElecEw__VdwNone__GeomW4P1__avx__128__fma__single_8c}}{}
\item\contentsline{section}{src/gmxlib/nonbonded/nb\-\_\-kernel\-\_\-avx\-\_\-128\-\_\-fma\-\_\-single/\hyperlink{nb__kernel__ElecEw__VdwNone__GeomW4W4__avx__128__fma__single_8c}{nb\-\_\-kernel\-\_\-\-Elec\-Ew\-\_\-\-Vdw\-None\-\_\-\-Geom\-W4\-W4\-\_\-avx\-\_\-128\-\_\-fma\-\_\-single.\-c} }{\pageref{nb__kernel__ElecEw__VdwNone__GeomW4W4__avx__128__fma__single_8c}}{}
\item\contentsline{section}{src/gmxlib/nonbonded/nb\-\_\-kernel\-\_\-avx\-\_\-128\-\_\-fma\-\_\-single/\hyperlink{nb__kernel__ElecEwSh__VdwLJSh__GeomP1P1__avx__128__fma__single_8c}{nb\-\_\-kernel\-\_\-\-Elec\-Ew\-Sh\-\_\-\-Vdw\-L\-J\-Sh\-\_\-\-Geom\-P1\-P1\-\_\-avx\-\_\-128\-\_\-fma\-\_\-single.\-c} }{\pageref{nb__kernel__ElecEwSh__VdwLJSh__GeomP1P1__avx__128__fma__single_8c}}{}
\item\contentsline{section}{src/gmxlib/nonbonded/nb\-\_\-kernel\-\_\-avx\-\_\-128\-\_\-fma\-\_\-single/\hyperlink{nb__kernel__ElecEwSh__VdwLJSh__GeomW3P1__avx__128__fma__single_8c}{nb\-\_\-kernel\-\_\-\-Elec\-Ew\-Sh\-\_\-\-Vdw\-L\-J\-Sh\-\_\-\-Geom\-W3\-P1\-\_\-avx\-\_\-128\-\_\-fma\-\_\-single.\-c} }{\pageref{nb__kernel__ElecEwSh__VdwLJSh__GeomW3P1__avx__128__fma__single_8c}}{}
\item\contentsline{section}{src/gmxlib/nonbonded/nb\-\_\-kernel\-\_\-avx\-\_\-128\-\_\-fma\-\_\-single/\hyperlink{nb__kernel__ElecEwSh__VdwLJSh__GeomW3W3__avx__128__fma__single_8c}{nb\-\_\-kernel\-\_\-\-Elec\-Ew\-Sh\-\_\-\-Vdw\-L\-J\-Sh\-\_\-\-Geom\-W3\-W3\-\_\-avx\-\_\-128\-\_\-fma\-\_\-single.\-c} }{\pageref{nb__kernel__ElecEwSh__VdwLJSh__GeomW3W3__avx__128__fma__single_8c}}{}
\item\contentsline{section}{src/gmxlib/nonbonded/nb\-\_\-kernel\-\_\-avx\-\_\-128\-\_\-fma\-\_\-single/\hyperlink{nb__kernel__ElecEwSh__VdwLJSh__GeomW4P1__avx__128__fma__single_8c}{nb\-\_\-kernel\-\_\-\-Elec\-Ew\-Sh\-\_\-\-Vdw\-L\-J\-Sh\-\_\-\-Geom\-W4\-P1\-\_\-avx\-\_\-128\-\_\-fma\-\_\-single.\-c} }{\pageref{nb__kernel__ElecEwSh__VdwLJSh__GeomW4P1__avx__128__fma__single_8c}}{}
\item\contentsline{section}{src/gmxlib/nonbonded/nb\-\_\-kernel\-\_\-avx\-\_\-128\-\_\-fma\-\_\-single/\hyperlink{nb__kernel__ElecEwSh__VdwLJSh__GeomW4W4__avx__128__fma__single_8c}{nb\-\_\-kernel\-\_\-\-Elec\-Ew\-Sh\-\_\-\-Vdw\-L\-J\-Sh\-\_\-\-Geom\-W4\-W4\-\_\-avx\-\_\-128\-\_\-fma\-\_\-single.\-c} }{\pageref{nb__kernel__ElecEwSh__VdwLJSh__GeomW4W4__avx__128__fma__single_8c}}{}
\item\contentsline{section}{src/gmxlib/nonbonded/nb\-\_\-kernel\-\_\-avx\-\_\-128\-\_\-fma\-\_\-single/\hyperlink{nb__kernel__ElecEwSh__VdwNone__GeomP1P1__avx__128__fma__single_8c}{nb\-\_\-kernel\-\_\-\-Elec\-Ew\-Sh\-\_\-\-Vdw\-None\-\_\-\-Geom\-P1\-P1\-\_\-avx\-\_\-128\-\_\-fma\-\_\-single.\-c} }{\pageref{nb__kernel__ElecEwSh__VdwNone__GeomP1P1__avx__128__fma__single_8c}}{}
\item\contentsline{section}{src/gmxlib/nonbonded/nb\-\_\-kernel\-\_\-avx\-\_\-128\-\_\-fma\-\_\-single/\hyperlink{nb__kernel__ElecEwSh__VdwNone__GeomW3P1__avx__128__fma__single_8c}{nb\-\_\-kernel\-\_\-\-Elec\-Ew\-Sh\-\_\-\-Vdw\-None\-\_\-\-Geom\-W3\-P1\-\_\-avx\-\_\-128\-\_\-fma\-\_\-single.\-c} }{\pageref{nb__kernel__ElecEwSh__VdwNone__GeomW3P1__avx__128__fma__single_8c}}{}
\item\contentsline{section}{src/gmxlib/nonbonded/nb\-\_\-kernel\-\_\-avx\-\_\-128\-\_\-fma\-\_\-single/\hyperlink{nb__kernel__ElecEwSh__VdwNone__GeomW3W3__avx__128__fma__single_8c}{nb\-\_\-kernel\-\_\-\-Elec\-Ew\-Sh\-\_\-\-Vdw\-None\-\_\-\-Geom\-W3\-W3\-\_\-avx\-\_\-128\-\_\-fma\-\_\-single.\-c} }{\pageref{nb__kernel__ElecEwSh__VdwNone__GeomW3W3__avx__128__fma__single_8c}}{}
\item\contentsline{section}{src/gmxlib/nonbonded/nb\-\_\-kernel\-\_\-avx\-\_\-128\-\_\-fma\-\_\-single/\hyperlink{nb__kernel__ElecEwSh__VdwNone__GeomW4P1__avx__128__fma__single_8c}{nb\-\_\-kernel\-\_\-\-Elec\-Ew\-Sh\-\_\-\-Vdw\-None\-\_\-\-Geom\-W4\-P1\-\_\-avx\-\_\-128\-\_\-fma\-\_\-single.\-c} }{\pageref{nb__kernel__ElecEwSh__VdwNone__GeomW4P1__avx__128__fma__single_8c}}{}
\item\contentsline{section}{src/gmxlib/nonbonded/nb\-\_\-kernel\-\_\-avx\-\_\-128\-\_\-fma\-\_\-single/\hyperlink{nb__kernel__ElecEwSh__VdwNone__GeomW4W4__avx__128__fma__single_8c}{nb\-\_\-kernel\-\_\-\-Elec\-Ew\-Sh\-\_\-\-Vdw\-None\-\_\-\-Geom\-W4\-W4\-\_\-avx\-\_\-128\-\_\-fma\-\_\-single.\-c} }{\pageref{nb__kernel__ElecEwSh__VdwNone__GeomW4W4__avx__128__fma__single_8c}}{}
\item\contentsline{section}{src/gmxlib/nonbonded/nb\-\_\-kernel\-\_\-avx\-\_\-128\-\_\-fma\-\_\-single/\hyperlink{nb__kernel__ElecEwSw__VdwLJSw__GeomP1P1__avx__128__fma__single_8c}{nb\-\_\-kernel\-\_\-\-Elec\-Ew\-Sw\-\_\-\-Vdw\-L\-J\-Sw\-\_\-\-Geom\-P1\-P1\-\_\-avx\-\_\-128\-\_\-fma\-\_\-single.\-c} }{\pageref{nb__kernel__ElecEwSw__VdwLJSw__GeomP1P1__avx__128__fma__single_8c}}{}
\item\contentsline{section}{src/gmxlib/nonbonded/nb\-\_\-kernel\-\_\-avx\-\_\-128\-\_\-fma\-\_\-single/\hyperlink{nb__kernel__ElecEwSw__VdwLJSw__GeomW3P1__avx__128__fma__single_8c}{nb\-\_\-kernel\-\_\-\-Elec\-Ew\-Sw\-\_\-\-Vdw\-L\-J\-Sw\-\_\-\-Geom\-W3\-P1\-\_\-avx\-\_\-128\-\_\-fma\-\_\-single.\-c} }{\pageref{nb__kernel__ElecEwSw__VdwLJSw__GeomW3P1__avx__128__fma__single_8c}}{}
\item\contentsline{section}{src/gmxlib/nonbonded/nb\-\_\-kernel\-\_\-avx\-\_\-128\-\_\-fma\-\_\-single/\hyperlink{nb__kernel__ElecEwSw__VdwLJSw__GeomW3W3__avx__128__fma__single_8c}{nb\-\_\-kernel\-\_\-\-Elec\-Ew\-Sw\-\_\-\-Vdw\-L\-J\-Sw\-\_\-\-Geom\-W3\-W3\-\_\-avx\-\_\-128\-\_\-fma\-\_\-single.\-c} }{\pageref{nb__kernel__ElecEwSw__VdwLJSw__GeomW3W3__avx__128__fma__single_8c}}{}
\item\contentsline{section}{src/gmxlib/nonbonded/nb\-\_\-kernel\-\_\-avx\-\_\-128\-\_\-fma\-\_\-single/\hyperlink{nb__kernel__ElecEwSw__VdwLJSw__GeomW4P1__avx__128__fma__single_8c}{nb\-\_\-kernel\-\_\-\-Elec\-Ew\-Sw\-\_\-\-Vdw\-L\-J\-Sw\-\_\-\-Geom\-W4\-P1\-\_\-avx\-\_\-128\-\_\-fma\-\_\-single.\-c} }{\pageref{nb__kernel__ElecEwSw__VdwLJSw__GeomW4P1__avx__128__fma__single_8c}}{}
\item\contentsline{section}{src/gmxlib/nonbonded/nb\-\_\-kernel\-\_\-avx\-\_\-128\-\_\-fma\-\_\-single/\hyperlink{nb__kernel__ElecEwSw__VdwLJSw__GeomW4W4__avx__128__fma__single_8c}{nb\-\_\-kernel\-\_\-\-Elec\-Ew\-Sw\-\_\-\-Vdw\-L\-J\-Sw\-\_\-\-Geom\-W4\-W4\-\_\-avx\-\_\-128\-\_\-fma\-\_\-single.\-c} }{\pageref{nb__kernel__ElecEwSw__VdwLJSw__GeomW4W4__avx__128__fma__single_8c}}{}
\item\contentsline{section}{src/gmxlib/nonbonded/nb\-\_\-kernel\-\_\-avx\-\_\-128\-\_\-fma\-\_\-single/\hyperlink{nb__kernel__ElecEwSw__VdwNone__GeomP1P1__avx__128__fma__single_8c}{nb\-\_\-kernel\-\_\-\-Elec\-Ew\-Sw\-\_\-\-Vdw\-None\-\_\-\-Geom\-P1\-P1\-\_\-avx\-\_\-128\-\_\-fma\-\_\-single.\-c} }{\pageref{nb__kernel__ElecEwSw__VdwNone__GeomP1P1__avx__128__fma__single_8c}}{}
\item\contentsline{section}{src/gmxlib/nonbonded/nb\-\_\-kernel\-\_\-avx\-\_\-128\-\_\-fma\-\_\-single/\hyperlink{nb__kernel__ElecEwSw__VdwNone__GeomW3P1__avx__128__fma__single_8c}{nb\-\_\-kernel\-\_\-\-Elec\-Ew\-Sw\-\_\-\-Vdw\-None\-\_\-\-Geom\-W3\-P1\-\_\-avx\-\_\-128\-\_\-fma\-\_\-single.\-c} }{\pageref{nb__kernel__ElecEwSw__VdwNone__GeomW3P1__avx__128__fma__single_8c}}{}
\item\contentsline{section}{src/gmxlib/nonbonded/nb\-\_\-kernel\-\_\-avx\-\_\-128\-\_\-fma\-\_\-single/\hyperlink{nb__kernel__ElecEwSw__VdwNone__GeomW3W3__avx__128__fma__single_8c}{nb\-\_\-kernel\-\_\-\-Elec\-Ew\-Sw\-\_\-\-Vdw\-None\-\_\-\-Geom\-W3\-W3\-\_\-avx\-\_\-128\-\_\-fma\-\_\-single.\-c} }{\pageref{nb__kernel__ElecEwSw__VdwNone__GeomW3W3__avx__128__fma__single_8c}}{}
\item\contentsline{section}{src/gmxlib/nonbonded/nb\-\_\-kernel\-\_\-avx\-\_\-128\-\_\-fma\-\_\-single/\hyperlink{nb__kernel__ElecEwSw__VdwNone__GeomW4P1__avx__128__fma__single_8c}{nb\-\_\-kernel\-\_\-\-Elec\-Ew\-Sw\-\_\-\-Vdw\-None\-\_\-\-Geom\-W4\-P1\-\_\-avx\-\_\-128\-\_\-fma\-\_\-single.\-c} }{\pageref{nb__kernel__ElecEwSw__VdwNone__GeomW4P1__avx__128__fma__single_8c}}{}
\item\contentsline{section}{src/gmxlib/nonbonded/nb\-\_\-kernel\-\_\-avx\-\_\-128\-\_\-fma\-\_\-single/\hyperlink{nb__kernel__ElecEwSw__VdwNone__GeomW4W4__avx__128__fma__single_8c}{nb\-\_\-kernel\-\_\-\-Elec\-Ew\-Sw\-\_\-\-Vdw\-None\-\_\-\-Geom\-W4\-W4\-\_\-avx\-\_\-128\-\_\-fma\-\_\-single.\-c} }{\pageref{nb__kernel__ElecEwSw__VdwNone__GeomW4W4__avx__128__fma__single_8c}}{}
\item\contentsline{section}{src/gmxlib/nonbonded/nb\-\_\-kernel\-\_\-avx\-\_\-128\-\_\-fma\-\_\-single/\hyperlink{nb__kernel__ElecGB__VdwCSTab__GeomP1P1__avx__128__fma__single_8c}{nb\-\_\-kernel\-\_\-\-Elec\-G\-B\-\_\-\-Vdw\-C\-S\-Tab\-\_\-\-Geom\-P1\-P1\-\_\-avx\-\_\-128\-\_\-fma\-\_\-single.\-c} }{\pageref{nb__kernel__ElecGB__VdwCSTab__GeomP1P1__avx__128__fma__single_8c}}{}
\item\contentsline{section}{src/gmxlib/nonbonded/nb\-\_\-kernel\-\_\-avx\-\_\-128\-\_\-fma\-\_\-single/\hyperlink{nb__kernel__ElecGB__VdwLJ__GeomP1P1__avx__128__fma__single_8c}{nb\-\_\-kernel\-\_\-\-Elec\-G\-B\-\_\-\-Vdw\-L\-J\-\_\-\-Geom\-P1\-P1\-\_\-avx\-\_\-128\-\_\-fma\-\_\-single.\-c} }{\pageref{nb__kernel__ElecGB__VdwLJ__GeomP1P1__avx__128__fma__single_8c}}{}
\item\contentsline{section}{src/gmxlib/nonbonded/nb\-\_\-kernel\-\_\-avx\-\_\-128\-\_\-fma\-\_\-single/\hyperlink{nb__kernel__ElecGB__VdwNone__GeomP1P1__avx__128__fma__single_8c}{nb\-\_\-kernel\-\_\-\-Elec\-G\-B\-\_\-\-Vdw\-None\-\_\-\-Geom\-P1\-P1\-\_\-avx\-\_\-128\-\_\-fma\-\_\-single.\-c} }{\pageref{nb__kernel__ElecGB__VdwNone__GeomP1P1__avx__128__fma__single_8c}}{}
\item\contentsline{section}{src/gmxlib/nonbonded/nb\-\_\-kernel\-\_\-avx\-\_\-128\-\_\-fma\-\_\-single/\hyperlink{nb__kernel__ElecNone__VdwCSTab__GeomP1P1__avx__128__fma__single_8c}{nb\-\_\-kernel\-\_\-\-Elec\-None\-\_\-\-Vdw\-C\-S\-Tab\-\_\-\-Geom\-P1\-P1\-\_\-avx\-\_\-128\-\_\-fma\-\_\-single.\-c} }{\pageref{nb__kernel__ElecNone__VdwCSTab__GeomP1P1__avx__128__fma__single_8c}}{}
\item\contentsline{section}{src/gmxlib/nonbonded/nb\-\_\-kernel\-\_\-avx\-\_\-128\-\_\-fma\-\_\-single/\hyperlink{nb__kernel__ElecNone__VdwLJ__GeomP1P1__avx__128__fma__single_8c}{nb\-\_\-kernel\-\_\-\-Elec\-None\-\_\-\-Vdw\-L\-J\-\_\-\-Geom\-P1\-P1\-\_\-avx\-\_\-128\-\_\-fma\-\_\-single.\-c} }{\pageref{nb__kernel__ElecNone__VdwLJ__GeomP1P1__avx__128__fma__single_8c}}{}
\item\contentsline{section}{src/gmxlib/nonbonded/nb\-\_\-kernel\-\_\-avx\-\_\-128\-\_\-fma\-\_\-single/\hyperlink{nb__kernel__ElecNone__VdwLJSh__GeomP1P1__avx__128__fma__single_8c}{nb\-\_\-kernel\-\_\-\-Elec\-None\-\_\-\-Vdw\-L\-J\-Sh\-\_\-\-Geom\-P1\-P1\-\_\-avx\-\_\-128\-\_\-fma\-\_\-single.\-c} }{\pageref{nb__kernel__ElecNone__VdwLJSh__GeomP1P1__avx__128__fma__single_8c}}{}
\item\contentsline{section}{src/gmxlib/nonbonded/nb\-\_\-kernel\-\_\-avx\-\_\-128\-\_\-fma\-\_\-single/\hyperlink{nb__kernel__ElecNone__VdwLJSw__GeomP1P1__avx__128__fma__single_8c}{nb\-\_\-kernel\-\_\-\-Elec\-None\-\_\-\-Vdw\-L\-J\-Sw\-\_\-\-Geom\-P1\-P1\-\_\-avx\-\_\-128\-\_\-fma\-\_\-single.\-c} }{\pageref{nb__kernel__ElecNone__VdwLJSw__GeomP1P1__avx__128__fma__single_8c}}{}
\item\contentsline{section}{src/gmxlib/nonbonded/nb\-\_\-kernel\-\_\-avx\-\_\-128\-\_\-fma\-\_\-single/\hyperlink{nb__kernel__ElecRF__VdwCSTab__GeomP1P1__avx__128__fma__single_8c}{nb\-\_\-kernel\-\_\-\-Elec\-R\-F\-\_\-\-Vdw\-C\-S\-Tab\-\_\-\-Geom\-P1\-P1\-\_\-avx\-\_\-128\-\_\-fma\-\_\-single.\-c} }{\pageref{nb__kernel__ElecRF__VdwCSTab__GeomP1P1__avx__128__fma__single_8c}}{}
\item\contentsline{section}{src/gmxlib/nonbonded/nb\-\_\-kernel\-\_\-avx\-\_\-128\-\_\-fma\-\_\-single/\hyperlink{nb__kernel__ElecRF__VdwCSTab__GeomW3P1__avx__128__fma__single_8c}{nb\-\_\-kernel\-\_\-\-Elec\-R\-F\-\_\-\-Vdw\-C\-S\-Tab\-\_\-\-Geom\-W3\-P1\-\_\-avx\-\_\-128\-\_\-fma\-\_\-single.\-c} }{\pageref{nb__kernel__ElecRF__VdwCSTab__GeomW3P1__avx__128__fma__single_8c}}{}
\item\contentsline{section}{src/gmxlib/nonbonded/nb\-\_\-kernel\-\_\-avx\-\_\-128\-\_\-fma\-\_\-single/\hyperlink{nb__kernel__ElecRF__VdwCSTab__GeomW3W3__avx__128__fma__single_8c}{nb\-\_\-kernel\-\_\-\-Elec\-R\-F\-\_\-\-Vdw\-C\-S\-Tab\-\_\-\-Geom\-W3\-W3\-\_\-avx\-\_\-128\-\_\-fma\-\_\-single.\-c} }{\pageref{nb__kernel__ElecRF__VdwCSTab__GeomW3W3__avx__128__fma__single_8c}}{}
\item\contentsline{section}{src/gmxlib/nonbonded/nb\-\_\-kernel\-\_\-avx\-\_\-128\-\_\-fma\-\_\-single/\hyperlink{nb__kernel__ElecRF__VdwCSTab__GeomW4P1__avx__128__fma__single_8c}{nb\-\_\-kernel\-\_\-\-Elec\-R\-F\-\_\-\-Vdw\-C\-S\-Tab\-\_\-\-Geom\-W4\-P1\-\_\-avx\-\_\-128\-\_\-fma\-\_\-single.\-c} }{\pageref{nb__kernel__ElecRF__VdwCSTab__GeomW4P1__avx__128__fma__single_8c}}{}
\item\contentsline{section}{src/gmxlib/nonbonded/nb\-\_\-kernel\-\_\-avx\-\_\-128\-\_\-fma\-\_\-single/\hyperlink{nb__kernel__ElecRF__VdwCSTab__GeomW4W4__avx__128__fma__single_8c}{nb\-\_\-kernel\-\_\-\-Elec\-R\-F\-\_\-\-Vdw\-C\-S\-Tab\-\_\-\-Geom\-W4\-W4\-\_\-avx\-\_\-128\-\_\-fma\-\_\-single.\-c} }{\pageref{nb__kernel__ElecRF__VdwCSTab__GeomW4W4__avx__128__fma__single_8c}}{}
\item\contentsline{section}{src/gmxlib/nonbonded/nb\-\_\-kernel\-\_\-avx\-\_\-128\-\_\-fma\-\_\-single/\hyperlink{nb__kernel__ElecRF__VdwLJ__GeomP1P1__avx__128__fma__single_8c}{nb\-\_\-kernel\-\_\-\-Elec\-R\-F\-\_\-\-Vdw\-L\-J\-\_\-\-Geom\-P1\-P1\-\_\-avx\-\_\-128\-\_\-fma\-\_\-single.\-c} }{\pageref{nb__kernel__ElecRF__VdwLJ__GeomP1P1__avx__128__fma__single_8c}}{}
\item\contentsline{section}{src/gmxlib/nonbonded/nb\-\_\-kernel\-\_\-avx\-\_\-128\-\_\-fma\-\_\-single/\hyperlink{nb__kernel__ElecRF__VdwLJ__GeomW3P1__avx__128__fma__single_8c}{nb\-\_\-kernel\-\_\-\-Elec\-R\-F\-\_\-\-Vdw\-L\-J\-\_\-\-Geom\-W3\-P1\-\_\-avx\-\_\-128\-\_\-fma\-\_\-single.\-c} }{\pageref{nb__kernel__ElecRF__VdwLJ__GeomW3P1__avx__128__fma__single_8c}}{}
\item\contentsline{section}{src/gmxlib/nonbonded/nb\-\_\-kernel\-\_\-avx\-\_\-128\-\_\-fma\-\_\-single/\hyperlink{nb__kernel__ElecRF__VdwLJ__GeomW3W3__avx__128__fma__single_8c}{nb\-\_\-kernel\-\_\-\-Elec\-R\-F\-\_\-\-Vdw\-L\-J\-\_\-\-Geom\-W3\-W3\-\_\-avx\-\_\-128\-\_\-fma\-\_\-single.\-c} }{\pageref{nb__kernel__ElecRF__VdwLJ__GeomW3W3__avx__128__fma__single_8c}}{}
\item\contentsline{section}{src/gmxlib/nonbonded/nb\-\_\-kernel\-\_\-avx\-\_\-128\-\_\-fma\-\_\-single/\hyperlink{nb__kernel__ElecRF__VdwLJ__GeomW4P1__avx__128__fma__single_8c}{nb\-\_\-kernel\-\_\-\-Elec\-R\-F\-\_\-\-Vdw\-L\-J\-\_\-\-Geom\-W4\-P1\-\_\-avx\-\_\-128\-\_\-fma\-\_\-single.\-c} }{\pageref{nb__kernel__ElecRF__VdwLJ__GeomW4P1__avx__128__fma__single_8c}}{}
\item\contentsline{section}{src/gmxlib/nonbonded/nb\-\_\-kernel\-\_\-avx\-\_\-128\-\_\-fma\-\_\-single/\hyperlink{nb__kernel__ElecRF__VdwLJ__GeomW4W4__avx__128__fma__single_8c}{nb\-\_\-kernel\-\_\-\-Elec\-R\-F\-\_\-\-Vdw\-L\-J\-\_\-\-Geom\-W4\-W4\-\_\-avx\-\_\-128\-\_\-fma\-\_\-single.\-c} }{\pageref{nb__kernel__ElecRF__VdwLJ__GeomW4W4__avx__128__fma__single_8c}}{}
\item\contentsline{section}{src/gmxlib/nonbonded/nb\-\_\-kernel\-\_\-avx\-\_\-128\-\_\-fma\-\_\-single/\hyperlink{nb__kernel__ElecRF__VdwNone__GeomP1P1__avx__128__fma__single_8c}{nb\-\_\-kernel\-\_\-\-Elec\-R\-F\-\_\-\-Vdw\-None\-\_\-\-Geom\-P1\-P1\-\_\-avx\-\_\-128\-\_\-fma\-\_\-single.\-c} }{\pageref{nb__kernel__ElecRF__VdwNone__GeomP1P1__avx__128__fma__single_8c}}{}
\item\contentsline{section}{src/gmxlib/nonbonded/nb\-\_\-kernel\-\_\-avx\-\_\-128\-\_\-fma\-\_\-single/\hyperlink{nb__kernel__ElecRF__VdwNone__GeomW3P1__avx__128__fma__single_8c}{nb\-\_\-kernel\-\_\-\-Elec\-R\-F\-\_\-\-Vdw\-None\-\_\-\-Geom\-W3\-P1\-\_\-avx\-\_\-128\-\_\-fma\-\_\-single.\-c} }{\pageref{nb__kernel__ElecRF__VdwNone__GeomW3P1__avx__128__fma__single_8c}}{}
\item\contentsline{section}{src/gmxlib/nonbonded/nb\-\_\-kernel\-\_\-avx\-\_\-128\-\_\-fma\-\_\-single/\hyperlink{nb__kernel__ElecRF__VdwNone__GeomW3W3__avx__128__fma__single_8c}{nb\-\_\-kernel\-\_\-\-Elec\-R\-F\-\_\-\-Vdw\-None\-\_\-\-Geom\-W3\-W3\-\_\-avx\-\_\-128\-\_\-fma\-\_\-single.\-c} }{\pageref{nb__kernel__ElecRF__VdwNone__GeomW3W3__avx__128__fma__single_8c}}{}
\item\contentsline{section}{src/gmxlib/nonbonded/nb\-\_\-kernel\-\_\-avx\-\_\-128\-\_\-fma\-\_\-single/\hyperlink{nb__kernel__ElecRF__VdwNone__GeomW4P1__avx__128__fma__single_8c}{nb\-\_\-kernel\-\_\-\-Elec\-R\-F\-\_\-\-Vdw\-None\-\_\-\-Geom\-W4\-P1\-\_\-avx\-\_\-128\-\_\-fma\-\_\-single.\-c} }{\pageref{nb__kernel__ElecRF__VdwNone__GeomW4P1__avx__128__fma__single_8c}}{}
\item\contentsline{section}{src/gmxlib/nonbonded/nb\-\_\-kernel\-\_\-avx\-\_\-128\-\_\-fma\-\_\-single/\hyperlink{nb__kernel__ElecRF__VdwNone__GeomW4W4__avx__128__fma__single_8c}{nb\-\_\-kernel\-\_\-\-Elec\-R\-F\-\_\-\-Vdw\-None\-\_\-\-Geom\-W4\-W4\-\_\-avx\-\_\-128\-\_\-fma\-\_\-single.\-c} }{\pageref{nb__kernel__ElecRF__VdwNone__GeomW4W4__avx__128__fma__single_8c}}{}
\item\contentsline{section}{src/gmxlib/nonbonded/nb\-\_\-kernel\-\_\-avx\-\_\-128\-\_\-fma\-\_\-single/\hyperlink{nb__kernel__ElecRFCut__VdwCSTab__GeomP1P1__avx__128__fma__single_8c}{nb\-\_\-kernel\-\_\-\-Elec\-R\-F\-Cut\-\_\-\-Vdw\-C\-S\-Tab\-\_\-\-Geom\-P1\-P1\-\_\-avx\-\_\-128\-\_\-fma\-\_\-single.\-c} }{\pageref{nb__kernel__ElecRFCut__VdwCSTab__GeomP1P1__avx__128__fma__single_8c}}{}
\item\contentsline{section}{src/gmxlib/nonbonded/nb\-\_\-kernel\-\_\-avx\-\_\-128\-\_\-fma\-\_\-single/\hyperlink{nb__kernel__ElecRFCut__VdwCSTab__GeomW3P1__avx__128__fma__single_8c}{nb\-\_\-kernel\-\_\-\-Elec\-R\-F\-Cut\-\_\-\-Vdw\-C\-S\-Tab\-\_\-\-Geom\-W3\-P1\-\_\-avx\-\_\-128\-\_\-fma\-\_\-single.\-c} }{\pageref{nb__kernel__ElecRFCut__VdwCSTab__GeomW3P1__avx__128__fma__single_8c}}{}
\item\contentsline{section}{src/gmxlib/nonbonded/nb\-\_\-kernel\-\_\-avx\-\_\-128\-\_\-fma\-\_\-single/\hyperlink{nb__kernel__ElecRFCut__VdwCSTab__GeomW3W3__avx__128__fma__single_8c}{nb\-\_\-kernel\-\_\-\-Elec\-R\-F\-Cut\-\_\-\-Vdw\-C\-S\-Tab\-\_\-\-Geom\-W3\-W3\-\_\-avx\-\_\-128\-\_\-fma\-\_\-single.\-c} }{\pageref{nb__kernel__ElecRFCut__VdwCSTab__GeomW3W3__avx__128__fma__single_8c}}{}
\item\contentsline{section}{src/gmxlib/nonbonded/nb\-\_\-kernel\-\_\-avx\-\_\-128\-\_\-fma\-\_\-single/\hyperlink{nb__kernel__ElecRFCut__VdwCSTab__GeomW4P1__avx__128__fma__single_8c}{nb\-\_\-kernel\-\_\-\-Elec\-R\-F\-Cut\-\_\-\-Vdw\-C\-S\-Tab\-\_\-\-Geom\-W4\-P1\-\_\-avx\-\_\-128\-\_\-fma\-\_\-single.\-c} }{\pageref{nb__kernel__ElecRFCut__VdwCSTab__GeomW4P1__avx__128__fma__single_8c}}{}
\item\contentsline{section}{src/gmxlib/nonbonded/nb\-\_\-kernel\-\_\-avx\-\_\-128\-\_\-fma\-\_\-single/\hyperlink{nb__kernel__ElecRFCut__VdwCSTab__GeomW4W4__avx__128__fma__single_8c}{nb\-\_\-kernel\-\_\-\-Elec\-R\-F\-Cut\-\_\-\-Vdw\-C\-S\-Tab\-\_\-\-Geom\-W4\-W4\-\_\-avx\-\_\-128\-\_\-fma\-\_\-single.\-c} }{\pageref{nb__kernel__ElecRFCut__VdwCSTab__GeomW4W4__avx__128__fma__single_8c}}{}
\item\contentsline{section}{src/gmxlib/nonbonded/nb\-\_\-kernel\-\_\-avx\-\_\-128\-\_\-fma\-\_\-single/\hyperlink{nb__kernel__ElecRFCut__VdwLJSh__GeomP1P1__avx__128__fma__single_8c}{nb\-\_\-kernel\-\_\-\-Elec\-R\-F\-Cut\-\_\-\-Vdw\-L\-J\-Sh\-\_\-\-Geom\-P1\-P1\-\_\-avx\-\_\-128\-\_\-fma\-\_\-single.\-c} }{\pageref{nb__kernel__ElecRFCut__VdwLJSh__GeomP1P1__avx__128__fma__single_8c}}{}
\item\contentsline{section}{src/gmxlib/nonbonded/nb\-\_\-kernel\-\_\-avx\-\_\-128\-\_\-fma\-\_\-single/\hyperlink{nb__kernel__ElecRFCut__VdwLJSh__GeomW3P1__avx__128__fma__single_8c}{nb\-\_\-kernel\-\_\-\-Elec\-R\-F\-Cut\-\_\-\-Vdw\-L\-J\-Sh\-\_\-\-Geom\-W3\-P1\-\_\-avx\-\_\-128\-\_\-fma\-\_\-single.\-c} }{\pageref{nb__kernel__ElecRFCut__VdwLJSh__GeomW3P1__avx__128__fma__single_8c}}{}
\item\contentsline{section}{src/gmxlib/nonbonded/nb\-\_\-kernel\-\_\-avx\-\_\-128\-\_\-fma\-\_\-single/\hyperlink{nb__kernel__ElecRFCut__VdwLJSh__GeomW3W3__avx__128__fma__single_8c}{nb\-\_\-kernel\-\_\-\-Elec\-R\-F\-Cut\-\_\-\-Vdw\-L\-J\-Sh\-\_\-\-Geom\-W3\-W3\-\_\-avx\-\_\-128\-\_\-fma\-\_\-single.\-c} }{\pageref{nb__kernel__ElecRFCut__VdwLJSh__GeomW3W3__avx__128__fma__single_8c}}{}
\item\contentsline{section}{src/gmxlib/nonbonded/nb\-\_\-kernel\-\_\-avx\-\_\-128\-\_\-fma\-\_\-single/\hyperlink{nb__kernel__ElecRFCut__VdwLJSh__GeomW4P1__avx__128__fma__single_8c}{nb\-\_\-kernel\-\_\-\-Elec\-R\-F\-Cut\-\_\-\-Vdw\-L\-J\-Sh\-\_\-\-Geom\-W4\-P1\-\_\-avx\-\_\-128\-\_\-fma\-\_\-single.\-c} }{\pageref{nb__kernel__ElecRFCut__VdwLJSh__GeomW4P1__avx__128__fma__single_8c}}{}
\item\contentsline{section}{src/gmxlib/nonbonded/nb\-\_\-kernel\-\_\-avx\-\_\-128\-\_\-fma\-\_\-single/\hyperlink{nb__kernel__ElecRFCut__VdwLJSh__GeomW4W4__avx__128__fma__single_8c}{nb\-\_\-kernel\-\_\-\-Elec\-R\-F\-Cut\-\_\-\-Vdw\-L\-J\-Sh\-\_\-\-Geom\-W4\-W4\-\_\-avx\-\_\-128\-\_\-fma\-\_\-single.\-c} }{\pageref{nb__kernel__ElecRFCut__VdwLJSh__GeomW4W4__avx__128__fma__single_8c}}{}
\item\contentsline{section}{src/gmxlib/nonbonded/nb\-\_\-kernel\-\_\-avx\-\_\-128\-\_\-fma\-\_\-single/\hyperlink{nb__kernel__ElecRFCut__VdwLJSw__GeomP1P1__avx__128__fma__single_8c}{nb\-\_\-kernel\-\_\-\-Elec\-R\-F\-Cut\-\_\-\-Vdw\-L\-J\-Sw\-\_\-\-Geom\-P1\-P1\-\_\-avx\-\_\-128\-\_\-fma\-\_\-single.\-c} }{\pageref{nb__kernel__ElecRFCut__VdwLJSw__GeomP1P1__avx__128__fma__single_8c}}{}
\item\contentsline{section}{src/gmxlib/nonbonded/nb\-\_\-kernel\-\_\-avx\-\_\-128\-\_\-fma\-\_\-single/\hyperlink{nb__kernel__ElecRFCut__VdwLJSw__GeomW3P1__avx__128__fma__single_8c}{nb\-\_\-kernel\-\_\-\-Elec\-R\-F\-Cut\-\_\-\-Vdw\-L\-J\-Sw\-\_\-\-Geom\-W3\-P1\-\_\-avx\-\_\-128\-\_\-fma\-\_\-single.\-c} }{\pageref{nb__kernel__ElecRFCut__VdwLJSw__GeomW3P1__avx__128__fma__single_8c}}{}
\item\contentsline{section}{src/gmxlib/nonbonded/nb\-\_\-kernel\-\_\-avx\-\_\-128\-\_\-fma\-\_\-single/\hyperlink{nb__kernel__ElecRFCut__VdwLJSw__GeomW3W3__avx__128__fma__single_8c}{nb\-\_\-kernel\-\_\-\-Elec\-R\-F\-Cut\-\_\-\-Vdw\-L\-J\-Sw\-\_\-\-Geom\-W3\-W3\-\_\-avx\-\_\-128\-\_\-fma\-\_\-single.\-c} }{\pageref{nb__kernel__ElecRFCut__VdwLJSw__GeomW3W3__avx__128__fma__single_8c}}{}
\item\contentsline{section}{src/gmxlib/nonbonded/nb\-\_\-kernel\-\_\-avx\-\_\-128\-\_\-fma\-\_\-single/\hyperlink{nb__kernel__ElecRFCut__VdwLJSw__GeomW4P1__avx__128__fma__single_8c}{nb\-\_\-kernel\-\_\-\-Elec\-R\-F\-Cut\-\_\-\-Vdw\-L\-J\-Sw\-\_\-\-Geom\-W4\-P1\-\_\-avx\-\_\-128\-\_\-fma\-\_\-single.\-c} }{\pageref{nb__kernel__ElecRFCut__VdwLJSw__GeomW4P1__avx__128__fma__single_8c}}{}
\item\contentsline{section}{src/gmxlib/nonbonded/nb\-\_\-kernel\-\_\-avx\-\_\-128\-\_\-fma\-\_\-single/\hyperlink{nb__kernel__ElecRFCut__VdwLJSw__GeomW4W4__avx__128__fma__single_8c}{nb\-\_\-kernel\-\_\-\-Elec\-R\-F\-Cut\-\_\-\-Vdw\-L\-J\-Sw\-\_\-\-Geom\-W4\-W4\-\_\-avx\-\_\-128\-\_\-fma\-\_\-single.\-c} }{\pageref{nb__kernel__ElecRFCut__VdwLJSw__GeomW4W4__avx__128__fma__single_8c}}{}
\item\contentsline{section}{src/gmxlib/nonbonded/nb\-\_\-kernel\-\_\-avx\-\_\-128\-\_\-fma\-\_\-single/\hyperlink{nb__kernel__ElecRFCut__VdwNone__GeomP1P1__avx__128__fma__single_8c}{nb\-\_\-kernel\-\_\-\-Elec\-R\-F\-Cut\-\_\-\-Vdw\-None\-\_\-\-Geom\-P1\-P1\-\_\-avx\-\_\-128\-\_\-fma\-\_\-single.\-c} }{\pageref{nb__kernel__ElecRFCut__VdwNone__GeomP1P1__avx__128__fma__single_8c}}{}
\item\contentsline{section}{src/gmxlib/nonbonded/nb\-\_\-kernel\-\_\-avx\-\_\-128\-\_\-fma\-\_\-single/\hyperlink{nb__kernel__ElecRFCut__VdwNone__GeomW3P1__avx__128__fma__single_8c}{nb\-\_\-kernel\-\_\-\-Elec\-R\-F\-Cut\-\_\-\-Vdw\-None\-\_\-\-Geom\-W3\-P1\-\_\-avx\-\_\-128\-\_\-fma\-\_\-single.\-c} }{\pageref{nb__kernel__ElecRFCut__VdwNone__GeomW3P1__avx__128__fma__single_8c}}{}
\item\contentsline{section}{src/gmxlib/nonbonded/nb\-\_\-kernel\-\_\-avx\-\_\-128\-\_\-fma\-\_\-single/\hyperlink{nb__kernel__ElecRFCut__VdwNone__GeomW3W3__avx__128__fma__single_8c}{nb\-\_\-kernel\-\_\-\-Elec\-R\-F\-Cut\-\_\-\-Vdw\-None\-\_\-\-Geom\-W3\-W3\-\_\-avx\-\_\-128\-\_\-fma\-\_\-single.\-c} }{\pageref{nb__kernel__ElecRFCut__VdwNone__GeomW3W3__avx__128__fma__single_8c}}{}
\item\contentsline{section}{src/gmxlib/nonbonded/nb\-\_\-kernel\-\_\-avx\-\_\-128\-\_\-fma\-\_\-single/\hyperlink{nb__kernel__ElecRFCut__VdwNone__GeomW4P1__avx__128__fma__single_8c}{nb\-\_\-kernel\-\_\-\-Elec\-R\-F\-Cut\-\_\-\-Vdw\-None\-\_\-\-Geom\-W4\-P1\-\_\-avx\-\_\-128\-\_\-fma\-\_\-single.\-c} }{\pageref{nb__kernel__ElecRFCut__VdwNone__GeomW4P1__avx__128__fma__single_8c}}{}
\item\contentsline{section}{src/gmxlib/nonbonded/nb\-\_\-kernel\-\_\-avx\-\_\-128\-\_\-fma\-\_\-single/\hyperlink{nb__kernel__ElecRFCut__VdwNone__GeomW4W4__avx__128__fma__single_8c}{nb\-\_\-kernel\-\_\-\-Elec\-R\-F\-Cut\-\_\-\-Vdw\-None\-\_\-\-Geom\-W4\-W4\-\_\-avx\-\_\-128\-\_\-fma\-\_\-single.\-c} }{\pageref{nb__kernel__ElecRFCut__VdwNone__GeomW4W4__avx__128__fma__single_8c}}{}
\item\contentsline{section}{src/gmxlib/nonbonded/nb\-\_\-kernel\-\_\-avx\-\_\-256\-\_\-double/\hyperlink{kernelutil__x86__avx__256__double_8h}{kernelutil\-\_\-x86\-\_\-avx\-\_\-256\-\_\-double.\-h} }{\pageref{kernelutil__x86__avx__256__double_8h}}{}
\item\contentsline{section}{src/gmxlib/nonbonded/nb\-\_\-kernel\-\_\-avx\-\_\-256\-\_\-double/\hyperlink{nb__kernel__avx__256__double_8c}{nb\-\_\-kernel\-\_\-avx\-\_\-256\-\_\-double.\-c} }{\pageref{nb__kernel__avx__256__double_8c}}{}
\item\contentsline{section}{src/gmxlib/nonbonded/nb\-\_\-kernel\-\_\-avx\-\_\-256\-\_\-double/\hyperlink{nb__kernel__avx__256__double_8h}{nb\-\_\-kernel\-\_\-avx\-\_\-256\-\_\-double.\-h} }{\pageref{nb__kernel__avx__256__double_8h}}{}
\item\contentsline{section}{src/gmxlib/nonbonded/nb\-\_\-kernel\-\_\-avx\-\_\-256\-\_\-double/\hyperlink{nb__kernel__ElecCoul__VdwCSTab__GeomP1P1__avx__256__double_8c}{nb\-\_\-kernel\-\_\-\-Elec\-Coul\-\_\-\-Vdw\-C\-S\-Tab\-\_\-\-Geom\-P1\-P1\-\_\-avx\-\_\-256\-\_\-double.\-c} }{\pageref{nb__kernel__ElecCoul__VdwCSTab__GeomP1P1__avx__256__double_8c}}{}
\item\contentsline{section}{src/gmxlib/nonbonded/nb\-\_\-kernel\-\_\-avx\-\_\-256\-\_\-double/\hyperlink{nb__kernel__ElecCoul__VdwCSTab__GeomW3P1__avx__256__double_8c}{nb\-\_\-kernel\-\_\-\-Elec\-Coul\-\_\-\-Vdw\-C\-S\-Tab\-\_\-\-Geom\-W3\-P1\-\_\-avx\-\_\-256\-\_\-double.\-c} }{\pageref{nb__kernel__ElecCoul__VdwCSTab__GeomW3P1__avx__256__double_8c}}{}
\item\contentsline{section}{src/gmxlib/nonbonded/nb\-\_\-kernel\-\_\-avx\-\_\-256\-\_\-double/\hyperlink{nb__kernel__ElecCoul__VdwCSTab__GeomW3W3__avx__256__double_8c}{nb\-\_\-kernel\-\_\-\-Elec\-Coul\-\_\-\-Vdw\-C\-S\-Tab\-\_\-\-Geom\-W3\-W3\-\_\-avx\-\_\-256\-\_\-double.\-c} }{\pageref{nb__kernel__ElecCoul__VdwCSTab__GeomW3W3__avx__256__double_8c}}{}
\item\contentsline{section}{src/gmxlib/nonbonded/nb\-\_\-kernel\-\_\-avx\-\_\-256\-\_\-double/\hyperlink{nb__kernel__ElecCoul__VdwCSTab__GeomW4P1__avx__256__double_8c}{nb\-\_\-kernel\-\_\-\-Elec\-Coul\-\_\-\-Vdw\-C\-S\-Tab\-\_\-\-Geom\-W4\-P1\-\_\-avx\-\_\-256\-\_\-double.\-c} }{\pageref{nb__kernel__ElecCoul__VdwCSTab__GeomW4P1__avx__256__double_8c}}{}
\item\contentsline{section}{src/gmxlib/nonbonded/nb\-\_\-kernel\-\_\-avx\-\_\-256\-\_\-double/\hyperlink{nb__kernel__ElecCoul__VdwCSTab__GeomW4W4__avx__256__double_8c}{nb\-\_\-kernel\-\_\-\-Elec\-Coul\-\_\-\-Vdw\-C\-S\-Tab\-\_\-\-Geom\-W4\-W4\-\_\-avx\-\_\-256\-\_\-double.\-c} }{\pageref{nb__kernel__ElecCoul__VdwCSTab__GeomW4W4__avx__256__double_8c}}{}
\item\contentsline{section}{src/gmxlib/nonbonded/nb\-\_\-kernel\-\_\-avx\-\_\-256\-\_\-double/\hyperlink{nb__kernel__ElecCoul__VdwLJ__GeomP1P1__avx__256__double_8c}{nb\-\_\-kernel\-\_\-\-Elec\-Coul\-\_\-\-Vdw\-L\-J\-\_\-\-Geom\-P1\-P1\-\_\-avx\-\_\-256\-\_\-double.\-c} }{\pageref{nb__kernel__ElecCoul__VdwLJ__GeomP1P1__avx__256__double_8c}}{}
\item\contentsline{section}{src/gmxlib/nonbonded/nb\-\_\-kernel\-\_\-avx\-\_\-256\-\_\-double/\hyperlink{nb__kernel__ElecCoul__VdwLJ__GeomW3P1__avx__256__double_8c}{nb\-\_\-kernel\-\_\-\-Elec\-Coul\-\_\-\-Vdw\-L\-J\-\_\-\-Geom\-W3\-P1\-\_\-avx\-\_\-256\-\_\-double.\-c} }{\pageref{nb__kernel__ElecCoul__VdwLJ__GeomW3P1__avx__256__double_8c}}{}
\item\contentsline{section}{src/gmxlib/nonbonded/nb\-\_\-kernel\-\_\-avx\-\_\-256\-\_\-double/\hyperlink{nb__kernel__ElecCoul__VdwLJ__GeomW3W3__avx__256__double_8c}{nb\-\_\-kernel\-\_\-\-Elec\-Coul\-\_\-\-Vdw\-L\-J\-\_\-\-Geom\-W3\-W3\-\_\-avx\-\_\-256\-\_\-double.\-c} }{\pageref{nb__kernel__ElecCoul__VdwLJ__GeomW3W3__avx__256__double_8c}}{}
\item\contentsline{section}{src/gmxlib/nonbonded/nb\-\_\-kernel\-\_\-avx\-\_\-256\-\_\-double/\hyperlink{nb__kernel__ElecCoul__VdwLJ__GeomW4P1__avx__256__double_8c}{nb\-\_\-kernel\-\_\-\-Elec\-Coul\-\_\-\-Vdw\-L\-J\-\_\-\-Geom\-W4\-P1\-\_\-avx\-\_\-256\-\_\-double.\-c} }{\pageref{nb__kernel__ElecCoul__VdwLJ__GeomW4P1__avx__256__double_8c}}{}
\item\contentsline{section}{src/gmxlib/nonbonded/nb\-\_\-kernel\-\_\-avx\-\_\-256\-\_\-double/\hyperlink{nb__kernel__ElecCoul__VdwLJ__GeomW4W4__avx__256__double_8c}{nb\-\_\-kernel\-\_\-\-Elec\-Coul\-\_\-\-Vdw\-L\-J\-\_\-\-Geom\-W4\-W4\-\_\-avx\-\_\-256\-\_\-double.\-c} }{\pageref{nb__kernel__ElecCoul__VdwLJ__GeomW4W4__avx__256__double_8c}}{}
\item\contentsline{section}{src/gmxlib/nonbonded/nb\-\_\-kernel\-\_\-avx\-\_\-256\-\_\-double/\hyperlink{nb__kernel__ElecCoul__VdwNone__GeomP1P1__avx__256__double_8c}{nb\-\_\-kernel\-\_\-\-Elec\-Coul\-\_\-\-Vdw\-None\-\_\-\-Geom\-P1\-P1\-\_\-avx\-\_\-256\-\_\-double.\-c} }{\pageref{nb__kernel__ElecCoul__VdwNone__GeomP1P1__avx__256__double_8c}}{}
\item\contentsline{section}{src/gmxlib/nonbonded/nb\-\_\-kernel\-\_\-avx\-\_\-256\-\_\-double/\hyperlink{nb__kernel__ElecCoul__VdwNone__GeomW3P1__avx__256__double_8c}{nb\-\_\-kernel\-\_\-\-Elec\-Coul\-\_\-\-Vdw\-None\-\_\-\-Geom\-W3\-P1\-\_\-avx\-\_\-256\-\_\-double.\-c} }{\pageref{nb__kernel__ElecCoul__VdwNone__GeomW3P1__avx__256__double_8c}}{}
\item\contentsline{section}{src/gmxlib/nonbonded/nb\-\_\-kernel\-\_\-avx\-\_\-256\-\_\-double/\hyperlink{nb__kernel__ElecCoul__VdwNone__GeomW3W3__avx__256__double_8c}{nb\-\_\-kernel\-\_\-\-Elec\-Coul\-\_\-\-Vdw\-None\-\_\-\-Geom\-W3\-W3\-\_\-avx\-\_\-256\-\_\-double.\-c} }{\pageref{nb__kernel__ElecCoul__VdwNone__GeomW3W3__avx__256__double_8c}}{}
\item\contentsline{section}{src/gmxlib/nonbonded/nb\-\_\-kernel\-\_\-avx\-\_\-256\-\_\-double/\hyperlink{nb__kernel__ElecCoul__VdwNone__GeomW4P1__avx__256__double_8c}{nb\-\_\-kernel\-\_\-\-Elec\-Coul\-\_\-\-Vdw\-None\-\_\-\-Geom\-W4\-P1\-\_\-avx\-\_\-256\-\_\-double.\-c} }{\pageref{nb__kernel__ElecCoul__VdwNone__GeomW4P1__avx__256__double_8c}}{}
\item\contentsline{section}{src/gmxlib/nonbonded/nb\-\_\-kernel\-\_\-avx\-\_\-256\-\_\-double/\hyperlink{nb__kernel__ElecCoul__VdwNone__GeomW4W4__avx__256__double_8c}{nb\-\_\-kernel\-\_\-\-Elec\-Coul\-\_\-\-Vdw\-None\-\_\-\-Geom\-W4\-W4\-\_\-avx\-\_\-256\-\_\-double.\-c} }{\pageref{nb__kernel__ElecCoul__VdwNone__GeomW4W4__avx__256__double_8c}}{}
\item\contentsline{section}{src/gmxlib/nonbonded/nb\-\_\-kernel\-\_\-avx\-\_\-256\-\_\-double/\hyperlink{nb__kernel__ElecCSTab__VdwCSTab__GeomP1P1__avx__256__double_8c}{nb\-\_\-kernel\-\_\-\-Elec\-C\-S\-Tab\-\_\-\-Vdw\-C\-S\-Tab\-\_\-\-Geom\-P1\-P1\-\_\-avx\-\_\-256\-\_\-double.\-c} }{\pageref{nb__kernel__ElecCSTab__VdwCSTab__GeomP1P1__avx__256__double_8c}}{}
\item\contentsline{section}{src/gmxlib/nonbonded/nb\-\_\-kernel\-\_\-avx\-\_\-256\-\_\-double/\hyperlink{nb__kernel__ElecCSTab__VdwCSTab__GeomW3P1__avx__256__double_8c}{nb\-\_\-kernel\-\_\-\-Elec\-C\-S\-Tab\-\_\-\-Vdw\-C\-S\-Tab\-\_\-\-Geom\-W3\-P1\-\_\-avx\-\_\-256\-\_\-double.\-c} }{\pageref{nb__kernel__ElecCSTab__VdwCSTab__GeomW3P1__avx__256__double_8c}}{}
\item\contentsline{section}{src/gmxlib/nonbonded/nb\-\_\-kernel\-\_\-avx\-\_\-256\-\_\-double/\hyperlink{nb__kernel__ElecCSTab__VdwCSTab__GeomW3W3__avx__256__double_8c}{nb\-\_\-kernel\-\_\-\-Elec\-C\-S\-Tab\-\_\-\-Vdw\-C\-S\-Tab\-\_\-\-Geom\-W3\-W3\-\_\-avx\-\_\-256\-\_\-double.\-c} }{\pageref{nb__kernel__ElecCSTab__VdwCSTab__GeomW3W3__avx__256__double_8c}}{}
\item\contentsline{section}{src/gmxlib/nonbonded/nb\-\_\-kernel\-\_\-avx\-\_\-256\-\_\-double/\hyperlink{nb__kernel__ElecCSTab__VdwCSTab__GeomW4P1__avx__256__double_8c}{nb\-\_\-kernel\-\_\-\-Elec\-C\-S\-Tab\-\_\-\-Vdw\-C\-S\-Tab\-\_\-\-Geom\-W4\-P1\-\_\-avx\-\_\-256\-\_\-double.\-c} }{\pageref{nb__kernel__ElecCSTab__VdwCSTab__GeomW4P1__avx__256__double_8c}}{}
\item\contentsline{section}{src/gmxlib/nonbonded/nb\-\_\-kernel\-\_\-avx\-\_\-256\-\_\-double/\hyperlink{nb__kernel__ElecCSTab__VdwCSTab__GeomW4W4__avx__256__double_8c}{nb\-\_\-kernel\-\_\-\-Elec\-C\-S\-Tab\-\_\-\-Vdw\-C\-S\-Tab\-\_\-\-Geom\-W4\-W4\-\_\-avx\-\_\-256\-\_\-double.\-c} }{\pageref{nb__kernel__ElecCSTab__VdwCSTab__GeomW4W4__avx__256__double_8c}}{}
\item\contentsline{section}{src/gmxlib/nonbonded/nb\-\_\-kernel\-\_\-avx\-\_\-256\-\_\-double/\hyperlink{nb__kernel__ElecCSTab__VdwLJ__GeomP1P1__avx__256__double_8c}{nb\-\_\-kernel\-\_\-\-Elec\-C\-S\-Tab\-\_\-\-Vdw\-L\-J\-\_\-\-Geom\-P1\-P1\-\_\-avx\-\_\-256\-\_\-double.\-c} }{\pageref{nb__kernel__ElecCSTab__VdwLJ__GeomP1P1__avx__256__double_8c}}{}
\item\contentsline{section}{src/gmxlib/nonbonded/nb\-\_\-kernel\-\_\-avx\-\_\-256\-\_\-double/\hyperlink{nb__kernel__ElecCSTab__VdwLJ__GeomW3P1__avx__256__double_8c}{nb\-\_\-kernel\-\_\-\-Elec\-C\-S\-Tab\-\_\-\-Vdw\-L\-J\-\_\-\-Geom\-W3\-P1\-\_\-avx\-\_\-256\-\_\-double.\-c} }{\pageref{nb__kernel__ElecCSTab__VdwLJ__GeomW3P1__avx__256__double_8c}}{}
\item\contentsline{section}{src/gmxlib/nonbonded/nb\-\_\-kernel\-\_\-avx\-\_\-256\-\_\-double/\hyperlink{nb__kernel__ElecCSTab__VdwLJ__GeomW3W3__avx__256__double_8c}{nb\-\_\-kernel\-\_\-\-Elec\-C\-S\-Tab\-\_\-\-Vdw\-L\-J\-\_\-\-Geom\-W3\-W3\-\_\-avx\-\_\-256\-\_\-double.\-c} }{\pageref{nb__kernel__ElecCSTab__VdwLJ__GeomW3W3__avx__256__double_8c}}{}
\item\contentsline{section}{src/gmxlib/nonbonded/nb\-\_\-kernel\-\_\-avx\-\_\-256\-\_\-double/\hyperlink{nb__kernel__ElecCSTab__VdwLJ__GeomW4P1__avx__256__double_8c}{nb\-\_\-kernel\-\_\-\-Elec\-C\-S\-Tab\-\_\-\-Vdw\-L\-J\-\_\-\-Geom\-W4\-P1\-\_\-avx\-\_\-256\-\_\-double.\-c} }{\pageref{nb__kernel__ElecCSTab__VdwLJ__GeomW4P1__avx__256__double_8c}}{}
\item\contentsline{section}{src/gmxlib/nonbonded/nb\-\_\-kernel\-\_\-avx\-\_\-256\-\_\-double/\hyperlink{nb__kernel__ElecCSTab__VdwLJ__GeomW4W4__avx__256__double_8c}{nb\-\_\-kernel\-\_\-\-Elec\-C\-S\-Tab\-\_\-\-Vdw\-L\-J\-\_\-\-Geom\-W4\-W4\-\_\-avx\-\_\-256\-\_\-double.\-c} }{\pageref{nb__kernel__ElecCSTab__VdwLJ__GeomW4W4__avx__256__double_8c}}{}
\item\contentsline{section}{src/gmxlib/nonbonded/nb\-\_\-kernel\-\_\-avx\-\_\-256\-\_\-double/\hyperlink{nb__kernel__ElecCSTab__VdwNone__GeomP1P1__avx__256__double_8c}{nb\-\_\-kernel\-\_\-\-Elec\-C\-S\-Tab\-\_\-\-Vdw\-None\-\_\-\-Geom\-P1\-P1\-\_\-avx\-\_\-256\-\_\-double.\-c} }{\pageref{nb__kernel__ElecCSTab__VdwNone__GeomP1P1__avx__256__double_8c}}{}
\item\contentsline{section}{src/gmxlib/nonbonded/nb\-\_\-kernel\-\_\-avx\-\_\-256\-\_\-double/\hyperlink{nb__kernel__ElecCSTab__VdwNone__GeomW3P1__avx__256__double_8c}{nb\-\_\-kernel\-\_\-\-Elec\-C\-S\-Tab\-\_\-\-Vdw\-None\-\_\-\-Geom\-W3\-P1\-\_\-avx\-\_\-256\-\_\-double.\-c} }{\pageref{nb__kernel__ElecCSTab__VdwNone__GeomW3P1__avx__256__double_8c}}{}
\item\contentsline{section}{src/gmxlib/nonbonded/nb\-\_\-kernel\-\_\-avx\-\_\-256\-\_\-double/\hyperlink{nb__kernel__ElecCSTab__VdwNone__GeomW3W3__avx__256__double_8c}{nb\-\_\-kernel\-\_\-\-Elec\-C\-S\-Tab\-\_\-\-Vdw\-None\-\_\-\-Geom\-W3\-W3\-\_\-avx\-\_\-256\-\_\-double.\-c} }{\pageref{nb__kernel__ElecCSTab__VdwNone__GeomW3W3__avx__256__double_8c}}{}
\item\contentsline{section}{src/gmxlib/nonbonded/nb\-\_\-kernel\-\_\-avx\-\_\-256\-\_\-double/\hyperlink{nb__kernel__ElecCSTab__VdwNone__GeomW4P1__avx__256__double_8c}{nb\-\_\-kernel\-\_\-\-Elec\-C\-S\-Tab\-\_\-\-Vdw\-None\-\_\-\-Geom\-W4\-P1\-\_\-avx\-\_\-256\-\_\-double.\-c} }{\pageref{nb__kernel__ElecCSTab__VdwNone__GeomW4P1__avx__256__double_8c}}{}
\item\contentsline{section}{src/gmxlib/nonbonded/nb\-\_\-kernel\-\_\-avx\-\_\-256\-\_\-double/\hyperlink{nb__kernel__ElecCSTab__VdwNone__GeomW4W4__avx__256__double_8c}{nb\-\_\-kernel\-\_\-\-Elec\-C\-S\-Tab\-\_\-\-Vdw\-None\-\_\-\-Geom\-W4\-W4\-\_\-avx\-\_\-256\-\_\-double.\-c} }{\pageref{nb__kernel__ElecCSTab__VdwNone__GeomW4W4__avx__256__double_8c}}{}
\item\contentsline{section}{src/gmxlib/nonbonded/nb\-\_\-kernel\-\_\-avx\-\_\-256\-\_\-double/\hyperlink{nb__kernel__ElecEw__VdwCSTab__GeomP1P1__avx__256__double_8c}{nb\-\_\-kernel\-\_\-\-Elec\-Ew\-\_\-\-Vdw\-C\-S\-Tab\-\_\-\-Geom\-P1\-P1\-\_\-avx\-\_\-256\-\_\-double.\-c} }{\pageref{nb__kernel__ElecEw__VdwCSTab__GeomP1P1__avx__256__double_8c}}{}
\item\contentsline{section}{src/gmxlib/nonbonded/nb\-\_\-kernel\-\_\-avx\-\_\-256\-\_\-double/\hyperlink{nb__kernel__ElecEw__VdwCSTab__GeomW3P1__avx__256__double_8c}{nb\-\_\-kernel\-\_\-\-Elec\-Ew\-\_\-\-Vdw\-C\-S\-Tab\-\_\-\-Geom\-W3\-P1\-\_\-avx\-\_\-256\-\_\-double.\-c} }{\pageref{nb__kernel__ElecEw__VdwCSTab__GeomW3P1__avx__256__double_8c}}{}
\item\contentsline{section}{src/gmxlib/nonbonded/nb\-\_\-kernel\-\_\-avx\-\_\-256\-\_\-double/\hyperlink{nb__kernel__ElecEw__VdwCSTab__GeomW3W3__avx__256__double_8c}{nb\-\_\-kernel\-\_\-\-Elec\-Ew\-\_\-\-Vdw\-C\-S\-Tab\-\_\-\-Geom\-W3\-W3\-\_\-avx\-\_\-256\-\_\-double.\-c} }{\pageref{nb__kernel__ElecEw__VdwCSTab__GeomW3W3__avx__256__double_8c}}{}
\item\contentsline{section}{src/gmxlib/nonbonded/nb\-\_\-kernel\-\_\-avx\-\_\-256\-\_\-double/\hyperlink{nb__kernel__ElecEw__VdwCSTab__GeomW4P1__avx__256__double_8c}{nb\-\_\-kernel\-\_\-\-Elec\-Ew\-\_\-\-Vdw\-C\-S\-Tab\-\_\-\-Geom\-W4\-P1\-\_\-avx\-\_\-256\-\_\-double.\-c} }{\pageref{nb__kernel__ElecEw__VdwCSTab__GeomW4P1__avx__256__double_8c}}{}
\item\contentsline{section}{src/gmxlib/nonbonded/nb\-\_\-kernel\-\_\-avx\-\_\-256\-\_\-double/\hyperlink{nb__kernel__ElecEw__VdwCSTab__GeomW4W4__avx__256__double_8c}{nb\-\_\-kernel\-\_\-\-Elec\-Ew\-\_\-\-Vdw\-C\-S\-Tab\-\_\-\-Geom\-W4\-W4\-\_\-avx\-\_\-256\-\_\-double.\-c} }{\pageref{nb__kernel__ElecEw__VdwCSTab__GeomW4W4__avx__256__double_8c}}{}
\item\contentsline{section}{src/gmxlib/nonbonded/nb\-\_\-kernel\-\_\-avx\-\_\-256\-\_\-double/\hyperlink{nb__kernel__ElecEw__VdwLJ__GeomP1P1__avx__256__double_8c}{nb\-\_\-kernel\-\_\-\-Elec\-Ew\-\_\-\-Vdw\-L\-J\-\_\-\-Geom\-P1\-P1\-\_\-avx\-\_\-256\-\_\-double.\-c} }{\pageref{nb__kernel__ElecEw__VdwLJ__GeomP1P1__avx__256__double_8c}}{}
\item\contentsline{section}{src/gmxlib/nonbonded/nb\-\_\-kernel\-\_\-avx\-\_\-256\-\_\-double/\hyperlink{nb__kernel__ElecEw__VdwLJ__GeomW3P1__avx__256__double_8c}{nb\-\_\-kernel\-\_\-\-Elec\-Ew\-\_\-\-Vdw\-L\-J\-\_\-\-Geom\-W3\-P1\-\_\-avx\-\_\-256\-\_\-double.\-c} }{\pageref{nb__kernel__ElecEw__VdwLJ__GeomW3P1__avx__256__double_8c}}{}
\item\contentsline{section}{src/gmxlib/nonbonded/nb\-\_\-kernel\-\_\-avx\-\_\-256\-\_\-double/\hyperlink{nb__kernel__ElecEw__VdwLJ__GeomW3W3__avx__256__double_8c}{nb\-\_\-kernel\-\_\-\-Elec\-Ew\-\_\-\-Vdw\-L\-J\-\_\-\-Geom\-W3\-W3\-\_\-avx\-\_\-256\-\_\-double.\-c} }{\pageref{nb__kernel__ElecEw__VdwLJ__GeomW3W3__avx__256__double_8c}}{}
\item\contentsline{section}{src/gmxlib/nonbonded/nb\-\_\-kernel\-\_\-avx\-\_\-256\-\_\-double/\hyperlink{nb__kernel__ElecEw__VdwLJ__GeomW4P1__avx__256__double_8c}{nb\-\_\-kernel\-\_\-\-Elec\-Ew\-\_\-\-Vdw\-L\-J\-\_\-\-Geom\-W4\-P1\-\_\-avx\-\_\-256\-\_\-double.\-c} }{\pageref{nb__kernel__ElecEw__VdwLJ__GeomW4P1__avx__256__double_8c}}{}
\item\contentsline{section}{src/gmxlib/nonbonded/nb\-\_\-kernel\-\_\-avx\-\_\-256\-\_\-double/\hyperlink{nb__kernel__ElecEw__VdwLJ__GeomW4W4__avx__256__double_8c}{nb\-\_\-kernel\-\_\-\-Elec\-Ew\-\_\-\-Vdw\-L\-J\-\_\-\-Geom\-W4\-W4\-\_\-avx\-\_\-256\-\_\-double.\-c} }{\pageref{nb__kernel__ElecEw__VdwLJ__GeomW4W4__avx__256__double_8c}}{}
\item\contentsline{section}{src/gmxlib/nonbonded/nb\-\_\-kernel\-\_\-avx\-\_\-256\-\_\-double/\hyperlink{nb__kernel__ElecEw__VdwNone__GeomP1P1__avx__256__double_8c}{nb\-\_\-kernel\-\_\-\-Elec\-Ew\-\_\-\-Vdw\-None\-\_\-\-Geom\-P1\-P1\-\_\-avx\-\_\-256\-\_\-double.\-c} }{\pageref{nb__kernel__ElecEw__VdwNone__GeomP1P1__avx__256__double_8c}}{}
\item\contentsline{section}{src/gmxlib/nonbonded/nb\-\_\-kernel\-\_\-avx\-\_\-256\-\_\-double/\hyperlink{nb__kernel__ElecEw__VdwNone__GeomW3P1__avx__256__double_8c}{nb\-\_\-kernel\-\_\-\-Elec\-Ew\-\_\-\-Vdw\-None\-\_\-\-Geom\-W3\-P1\-\_\-avx\-\_\-256\-\_\-double.\-c} }{\pageref{nb__kernel__ElecEw__VdwNone__GeomW3P1__avx__256__double_8c}}{}
\item\contentsline{section}{src/gmxlib/nonbonded/nb\-\_\-kernel\-\_\-avx\-\_\-256\-\_\-double/\hyperlink{nb__kernel__ElecEw__VdwNone__GeomW3W3__avx__256__double_8c}{nb\-\_\-kernel\-\_\-\-Elec\-Ew\-\_\-\-Vdw\-None\-\_\-\-Geom\-W3\-W3\-\_\-avx\-\_\-256\-\_\-double.\-c} }{\pageref{nb__kernel__ElecEw__VdwNone__GeomW3W3__avx__256__double_8c}}{}
\item\contentsline{section}{src/gmxlib/nonbonded/nb\-\_\-kernel\-\_\-avx\-\_\-256\-\_\-double/\hyperlink{nb__kernel__ElecEw__VdwNone__GeomW4P1__avx__256__double_8c}{nb\-\_\-kernel\-\_\-\-Elec\-Ew\-\_\-\-Vdw\-None\-\_\-\-Geom\-W4\-P1\-\_\-avx\-\_\-256\-\_\-double.\-c} }{\pageref{nb__kernel__ElecEw__VdwNone__GeomW4P1__avx__256__double_8c}}{}
\item\contentsline{section}{src/gmxlib/nonbonded/nb\-\_\-kernel\-\_\-avx\-\_\-256\-\_\-double/\hyperlink{nb__kernel__ElecEw__VdwNone__GeomW4W4__avx__256__double_8c}{nb\-\_\-kernel\-\_\-\-Elec\-Ew\-\_\-\-Vdw\-None\-\_\-\-Geom\-W4\-W4\-\_\-avx\-\_\-256\-\_\-double.\-c} }{\pageref{nb__kernel__ElecEw__VdwNone__GeomW4W4__avx__256__double_8c}}{}
\item\contentsline{section}{src/gmxlib/nonbonded/nb\-\_\-kernel\-\_\-avx\-\_\-256\-\_\-double/\hyperlink{nb__kernel__ElecEwSh__VdwLJSh__GeomP1P1__avx__256__double_8c}{nb\-\_\-kernel\-\_\-\-Elec\-Ew\-Sh\-\_\-\-Vdw\-L\-J\-Sh\-\_\-\-Geom\-P1\-P1\-\_\-avx\-\_\-256\-\_\-double.\-c} }{\pageref{nb__kernel__ElecEwSh__VdwLJSh__GeomP1P1__avx__256__double_8c}}{}
\item\contentsline{section}{src/gmxlib/nonbonded/nb\-\_\-kernel\-\_\-avx\-\_\-256\-\_\-double/\hyperlink{nb__kernel__ElecEwSh__VdwLJSh__GeomW3P1__avx__256__double_8c}{nb\-\_\-kernel\-\_\-\-Elec\-Ew\-Sh\-\_\-\-Vdw\-L\-J\-Sh\-\_\-\-Geom\-W3\-P1\-\_\-avx\-\_\-256\-\_\-double.\-c} }{\pageref{nb__kernel__ElecEwSh__VdwLJSh__GeomW3P1__avx__256__double_8c}}{}
\item\contentsline{section}{src/gmxlib/nonbonded/nb\-\_\-kernel\-\_\-avx\-\_\-256\-\_\-double/\hyperlink{nb__kernel__ElecEwSh__VdwLJSh__GeomW3W3__avx__256__double_8c}{nb\-\_\-kernel\-\_\-\-Elec\-Ew\-Sh\-\_\-\-Vdw\-L\-J\-Sh\-\_\-\-Geom\-W3\-W3\-\_\-avx\-\_\-256\-\_\-double.\-c} }{\pageref{nb__kernel__ElecEwSh__VdwLJSh__GeomW3W3__avx__256__double_8c}}{}
\item\contentsline{section}{src/gmxlib/nonbonded/nb\-\_\-kernel\-\_\-avx\-\_\-256\-\_\-double/\hyperlink{nb__kernel__ElecEwSh__VdwLJSh__GeomW4P1__avx__256__double_8c}{nb\-\_\-kernel\-\_\-\-Elec\-Ew\-Sh\-\_\-\-Vdw\-L\-J\-Sh\-\_\-\-Geom\-W4\-P1\-\_\-avx\-\_\-256\-\_\-double.\-c} }{\pageref{nb__kernel__ElecEwSh__VdwLJSh__GeomW4P1__avx__256__double_8c}}{}
\item\contentsline{section}{src/gmxlib/nonbonded/nb\-\_\-kernel\-\_\-avx\-\_\-256\-\_\-double/\hyperlink{nb__kernel__ElecEwSh__VdwLJSh__GeomW4W4__avx__256__double_8c}{nb\-\_\-kernel\-\_\-\-Elec\-Ew\-Sh\-\_\-\-Vdw\-L\-J\-Sh\-\_\-\-Geom\-W4\-W4\-\_\-avx\-\_\-256\-\_\-double.\-c} }{\pageref{nb__kernel__ElecEwSh__VdwLJSh__GeomW4W4__avx__256__double_8c}}{}
\item\contentsline{section}{src/gmxlib/nonbonded/nb\-\_\-kernel\-\_\-avx\-\_\-256\-\_\-double/\hyperlink{nb__kernel__ElecEwSh__VdwNone__GeomP1P1__avx__256__double_8c}{nb\-\_\-kernel\-\_\-\-Elec\-Ew\-Sh\-\_\-\-Vdw\-None\-\_\-\-Geom\-P1\-P1\-\_\-avx\-\_\-256\-\_\-double.\-c} }{\pageref{nb__kernel__ElecEwSh__VdwNone__GeomP1P1__avx__256__double_8c}}{}
\item\contentsline{section}{src/gmxlib/nonbonded/nb\-\_\-kernel\-\_\-avx\-\_\-256\-\_\-double/\hyperlink{nb__kernel__ElecEwSh__VdwNone__GeomW3P1__avx__256__double_8c}{nb\-\_\-kernel\-\_\-\-Elec\-Ew\-Sh\-\_\-\-Vdw\-None\-\_\-\-Geom\-W3\-P1\-\_\-avx\-\_\-256\-\_\-double.\-c} }{\pageref{nb__kernel__ElecEwSh__VdwNone__GeomW3P1__avx__256__double_8c}}{}
\item\contentsline{section}{src/gmxlib/nonbonded/nb\-\_\-kernel\-\_\-avx\-\_\-256\-\_\-double/\hyperlink{nb__kernel__ElecEwSh__VdwNone__GeomW3W3__avx__256__double_8c}{nb\-\_\-kernel\-\_\-\-Elec\-Ew\-Sh\-\_\-\-Vdw\-None\-\_\-\-Geom\-W3\-W3\-\_\-avx\-\_\-256\-\_\-double.\-c} }{\pageref{nb__kernel__ElecEwSh__VdwNone__GeomW3W3__avx__256__double_8c}}{}
\item\contentsline{section}{src/gmxlib/nonbonded/nb\-\_\-kernel\-\_\-avx\-\_\-256\-\_\-double/\hyperlink{nb__kernel__ElecEwSh__VdwNone__GeomW4P1__avx__256__double_8c}{nb\-\_\-kernel\-\_\-\-Elec\-Ew\-Sh\-\_\-\-Vdw\-None\-\_\-\-Geom\-W4\-P1\-\_\-avx\-\_\-256\-\_\-double.\-c} }{\pageref{nb__kernel__ElecEwSh__VdwNone__GeomW4P1__avx__256__double_8c}}{}
\item\contentsline{section}{src/gmxlib/nonbonded/nb\-\_\-kernel\-\_\-avx\-\_\-256\-\_\-double/\hyperlink{nb__kernel__ElecEwSh__VdwNone__GeomW4W4__avx__256__double_8c}{nb\-\_\-kernel\-\_\-\-Elec\-Ew\-Sh\-\_\-\-Vdw\-None\-\_\-\-Geom\-W4\-W4\-\_\-avx\-\_\-256\-\_\-double.\-c} }{\pageref{nb__kernel__ElecEwSh__VdwNone__GeomW4W4__avx__256__double_8c}}{}
\item\contentsline{section}{src/gmxlib/nonbonded/nb\-\_\-kernel\-\_\-avx\-\_\-256\-\_\-double/\hyperlink{nb__kernel__ElecEwSw__VdwLJSw__GeomP1P1__avx__256__double_8c}{nb\-\_\-kernel\-\_\-\-Elec\-Ew\-Sw\-\_\-\-Vdw\-L\-J\-Sw\-\_\-\-Geom\-P1\-P1\-\_\-avx\-\_\-256\-\_\-double.\-c} }{\pageref{nb__kernel__ElecEwSw__VdwLJSw__GeomP1P1__avx__256__double_8c}}{}
\item\contentsline{section}{src/gmxlib/nonbonded/nb\-\_\-kernel\-\_\-avx\-\_\-256\-\_\-double/\hyperlink{nb__kernel__ElecEwSw__VdwLJSw__GeomW3P1__avx__256__double_8c}{nb\-\_\-kernel\-\_\-\-Elec\-Ew\-Sw\-\_\-\-Vdw\-L\-J\-Sw\-\_\-\-Geom\-W3\-P1\-\_\-avx\-\_\-256\-\_\-double.\-c} }{\pageref{nb__kernel__ElecEwSw__VdwLJSw__GeomW3P1__avx__256__double_8c}}{}
\item\contentsline{section}{src/gmxlib/nonbonded/nb\-\_\-kernel\-\_\-avx\-\_\-256\-\_\-double/\hyperlink{nb__kernel__ElecEwSw__VdwLJSw__GeomW3W3__avx__256__double_8c}{nb\-\_\-kernel\-\_\-\-Elec\-Ew\-Sw\-\_\-\-Vdw\-L\-J\-Sw\-\_\-\-Geom\-W3\-W3\-\_\-avx\-\_\-256\-\_\-double.\-c} }{\pageref{nb__kernel__ElecEwSw__VdwLJSw__GeomW3W3__avx__256__double_8c}}{}
\item\contentsline{section}{src/gmxlib/nonbonded/nb\-\_\-kernel\-\_\-avx\-\_\-256\-\_\-double/\hyperlink{nb__kernel__ElecEwSw__VdwLJSw__GeomW4P1__avx__256__double_8c}{nb\-\_\-kernel\-\_\-\-Elec\-Ew\-Sw\-\_\-\-Vdw\-L\-J\-Sw\-\_\-\-Geom\-W4\-P1\-\_\-avx\-\_\-256\-\_\-double.\-c} }{\pageref{nb__kernel__ElecEwSw__VdwLJSw__GeomW4P1__avx__256__double_8c}}{}
\item\contentsline{section}{src/gmxlib/nonbonded/nb\-\_\-kernel\-\_\-avx\-\_\-256\-\_\-double/\hyperlink{nb__kernel__ElecEwSw__VdwLJSw__GeomW4W4__avx__256__double_8c}{nb\-\_\-kernel\-\_\-\-Elec\-Ew\-Sw\-\_\-\-Vdw\-L\-J\-Sw\-\_\-\-Geom\-W4\-W4\-\_\-avx\-\_\-256\-\_\-double.\-c} }{\pageref{nb__kernel__ElecEwSw__VdwLJSw__GeomW4W4__avx__256__double_8c}}{}
\item\contentsline{section}{src/gmxlib/nonbonded/nb\-\_\-kernel\-\_\-avx\-\_\-256\-\_\-double/\hyperlink{nb__kernel__ElecEwSw__VdwNone__GeomP1P1__avx__256__double_8c}{nb\-\_\-kernel\-\_\-\-Elec\-Ew\-Sw\-\_\-\-Vdw\-None\-\_\-\-Geom\-P1\-P1\-\_\-avx\-\_\-256\-\_\-double.\-c} }{\pageref{nb__kernel__ElecEwSw__VdwNone__GeomP1P1__avx__256__double_8c}}{}
\item\contentsline{section}{src/gmxlib/nonbonded/nb\-\_\-kernel\-\_\-avx\-\_\-256\-\_\-double/\hyperlink{nb__kernel__ElecEwSw__VdwNone__GeomW3P1__avx__256__double_8c}{nb\-\_\-kernel\-\_\-\-Elec\-Ew\-Sw\-\_\-\-Vdw\-None\-\_\-\-Geom\-W3\-P1\-\_\-avx\-\_\-256\-\_\-double.\-c} }{\pageref{nb__kernel__ElecEwSw__VdwNone__GeomW3P1__avx__256__double_8c}}{}
\item\contentsline{section}{src/gmxlib/nonbonded/nb\-\_\-kernel\-\_\-avx\-\_\-256\-\_\-double/\hyperlink{nb__kernel__ElecEwSw__VdwNone__GeomW3W3__avx__256__double_8c}{nb\-\_\-kernel\-\_\-\-Elec\-Ew\-Sw\-\_\-\-Vdw\-None\-\_\-\-Geom\-W3\-W3\-\_\-avx\-\_\-256\-\_\-double.\-c} }{\pageref{nb__kernel__ElecEwSw__VdwNone__GeomW3W3__avx__256__double_8c}}{}
\item\contentsline{section}{src/gmxlib/nonbonded/nb\-\_\-kernel\-\_\-avx\-\_\-256\-\_\-double/\hyperlink{nb__kernel__ElecEwSw__VdwNone__GeomW4P1__avx__256__double_8c}{nb\-\_\-kernel\-\_\-\-Elec\-Ew\-Sw\-\_\-\-Vdw\-None\-\_\-\-Geom\-W4\-P1\-\_\-avx\-\_\-256\-\_\-double.\-c} }{\pageref{nb__kernel__ElecEwSw__VdwNone__GeomW4P1__avx__256__double_8c}}{}
\item\contentsline{section}{src/gmxlib/nonbonded/nb\-\_\-kernel\-\_\-avx\-\_\-256\-\_\-double/\hyperlink{nb__kernel__ElecEwSw__VdwNone__GeomW4W4__avx__256__double_8c}{nb\-\_\-kernel\-\_\-\-Elec\-Ew\-Sw\-\_\-\-Vdw\-None\-\_\-\-Geom\-W4\-W4\-\_\-avx\-\_\-256\-\_\-double.\-c} }{\pageref{nb__kernel__ElecEwSw__VdwNone__GeomW4W4__avx__256__double_8c}}{}
\item\contentsline{section}{src/gmxlib/nonbonded/nb\-\_\-kernel\-\_\-avx\-\_\-256\-\_\-double/\hyperlink{nb__kernel__ElecGB__VdwCSTab__GeomP1P1__avx__256__double_8c}{nb\-\_\-kernel\-\_\-\-Elec\-G\-B\-\_\-\-Vdw\-C\-S\-Tab\-\_\-\-Geom\-P1\-P1\-\_\-avx\-\_\-256\-\_\-double.\-c} }{\pageref{nb__kernel__ElecGB__VdwCSTab__GeomP1P1__avx__256__double_8c}}{}
\item\contentsline{section}{src/gmxlib/nonbonded/nb\-\_\-kernel\-\_\-avx\-\_\-256\-\_\-double/\hyperlink{nb__kernel__ElecGB__VdwLJ__GeomP1P1__avx__256__double_8c}{nb\-\_\-kernel\-\_\-\-Elec\-G\-B\-\_\-\-Vdw\-L\-J\-\_\-\-Geom\-P1\-P1\-\_\-avx\-\_\-256\-\_\-double.\-c} }{\pageref{nb__kernel__ElecGB__VdwLJ__GeomP1P1__avx__256__double_8c}}{}
\item\contentsline{section}{src/gmxlib/nonbonded/nb\-\_\-kernel\-\_\-avx\-\_\-256\-\_\-double/\hyperlink{nb__kernel__ElecGB__VdwNone__GeomP1P1__avx__256__double_8c}{nb\-\_\-kernel\-\_\-\-Elec\-G\-B\-\_\-\-Vdw\-None\-\_\-\-Geom\-P1\-P1\-\_\-avx\-\_\-256\-\_\-double.\-c} }{\pageref{nb__kernel__ElecGB__VdwNone__GeomP1P1__avx__256__double_8c}}{}
\item\contentsline{section}{src/gmxlib/nonbonded/nb\-\_\-kernel\-\_\-avx\-\_\-256\-\_\-double/\hyperlink{nb__kernel__ElecNone__VdwCSTab__GeomP1P1__avx__256__double_8c}{nb\-\_\-kernel\-\_\-\-Elec\-None\-\_\-\-Vdw\-C\-S\-Tab\-\_\-\-Geom\-P1\-P1\-\_\-avx\-\_\-256\-\_\-double.\-c} }{\pageref{nb__kernel__ElecNone__VdwCSTab__GeomP1P1__avx__256__double_8c}}{}
\item\contentsline{section}{src/gmxlib/nonbonded/nb\-\_\-kernel\-\_\-avx\-\_\-256\-\_\-double/\hyperlink{nb__kernel__ElecNone__VdwLJ__GeomP1P1__avx__256__double_8c}{nb\-\_\-kernel\-\_\-\-Elec\-None\-\_\-\-Vdw\-L\-J\-\_\-\-Geom\-P1\-P1\-\_\-avx\-\_\-256\-\_\-double.\-c} }{\pageref{nb__kernel__ElecNone__VdwLJ__GeomP1P1__avx__256__double_8c}}{}
\item\contentsline{section}{src/gmxlib/nonbonded/nb\-\_\-kernel\-\_\-avx\-\_\-256\-\_\-double/\hyperlink{nb__kernel__ElecNone__VdwLJSh__GeomP1P1__avx__256__double_8c}{nb\-\_\-kernel\-\_\-\-Elec\-None\-\_\-\-Vdw\-L\-J\-Sh\-\_\-\-Geom\-P1\-P1\-\_\-avx\-\_\-256\-\_\-double.\-c} }{\pageref{nb__kernel__ElecNone__VdwLJSh__GeomP1P1__avx__256__double_8c}}{}
\item\contentsline{section}{src/gmxlib/nonbonded/nb\-\_\-kernel\-\_\-avx\-\_\-256\-\_\-double/\hyperlink{nb__kernel__ElecNone__VdwLJSw__GeomP1P1__avx__256__double_8c}{nb\-\_\-kernel\-\_\-\-Elec\-None\-\_\-\-Vdw\-L\-J\-Sw\-\_\-\-Geom\-P1\-P1\-\_\-avx\-\_\-256\-\_\-double.\-c} }{\pageref{nb__kernel__ElecNone__VdwLJSw__GeomP1P1__avx__256__double_8c}}{}
\item\contentsline{section}{src/gmxlib/nonbonded/nb\-\_\-kernel\-\_\-avx\-\_\-256\-\_\-double/\hyperlink{nb__kernel__ElecRF__VdwCSTab__GeomP1P1__avx__256__double_8c}{nb\-\_\-kernel\-\_\-\-Elec\-R\-F\-\_\-\-Vdw\-C\-S\-Tab\-\_\-\-Geom\-P1\-P1\-\_\-avx\-\_\-256\-\_\-double.\-c} }{\pageref{nb__kernel__ElecRF__VdwCSTab__GeomP1P1__avx__256__double_8c}}{}
\item\contentsline{section}{src/gmxlib/nonbonded/nb\-\_\-kernel\-\_\-avx\-\_\-256\-\_\-double/\hyperlink{nb__kernel__ElecRF__VdwCSTab__GeomW3P1__avx__256__double_8c}{nb\-\_\-kernel\-\_\-\-Elec\-R\-F\-\_\-\-Vdw\-C\-S\-Tab\-\_\-\-Geom\-W3\-P1\-\_\-avx\-\_\-256\-\_\-double.\-c} }{\pageref{nb__kernel__ElecRF__VdwCSTab__GeomW3P1__avx__256__double_8c}}{}
\item\contentsline{section}{src/gmxlib/nonbonded/nb\-\_\-kernel\-\_\-avx\-\_\-256\-\_\-double/\hyperlink{nb__kernel__ElecRF__VdwCSTab__GeomW3W3__avx__256__double_8c}{nb\-\_\-kernel\-\_\-\-Elec\-R\-F\-\_\-\-Vdw\-C\-S\-Tab\-\_\-\-Geom\-W3\-W3\-\_\-avx\-\_\-256\-\_\-double.\-c} }{\pageref{nb__kernel__ElecRF__VdwCSTab__GeomW3W3__avx__256__double_8c}}{}
\item\contentsline{section}{src/gmxlib/nonbonded/nb\-\_\-kernel\-\_\-avx\-\_\-256\-\_\-double/\hyperlink{nb__kernel__ElecRF__VdwCSTab__GeomW4P1__avx__256__double_8c}{nb\-\_\-kernel\-\_\-\-Elec\-R\-F\-\_\-\-Vdw\-C\-S\-Tab\-\_\-\-Geom\-W4\-P1\-\_\-avx\-\_\-256\-\_\-double.\-c} }{\pageref{nb__kernel__ElecRF__VdwCSTab__GeomW4P1__avx__256__double_8c}}{}
\item\contentsline{section}{src/gmxlib/nonbonded/nb\-\_\-kernel\-\_\-avx\-\_\-256\-\_\-double/\hyperlink{nb__kernel__ElecRF__VdwCSTab__GeomW4W4__avx__256__double_8c}{nb\-\_\-kernel\-\_\-\-Elec\-R\-F\-\_\-\-Vdw\-C\-S\-Tab\-\_\-\-Geom\-W4\-W4\-\_\-avx\-\_\-256\-\_\-double.\-c} }{\pageref{nb__kernel__ElecRF__VdwCSTab__GeomW4W4__avx__256__double_8c}}{}
\item\contentsline{section}{src/gmxlib/nonbonded/nb\-\_\-kernel\-\_\-avx\-\_\-256\-\_\-double/\hyperlink{nb__kernel__ElecRF__VdwLJ__GeomP1P1__avx__256__double_8c}{nb\-\_\-kernel\-\_\-\-Elec\-R\-F\-\_\-\-Vdw\-L\-J\-\_\-\-Geom\-P1\-P1\-\_\-avx\-\_\-256\-\_\-double.\-c} }{\pageref{nb__kernel__ElecRF__VdwLJ__GeomP1P1__avx__256__double_8c}}{}
\item\contentsline{section}{src/gmxlib/nonbonded/nb\-\_\-kernel\-\_\-avx\-\_\-256\-\_\-double/\hyperlink{nb__kernel__ElecRF__VdwLJ__GeomW3P1__avx__256__double_8c}{nb\-\_\-kernel\-\_\-\-Elec\-R\-F\-\_\-\-Vdw\-L\-J\-\_\-\-Geom\-W3\-P1\-\_\-avx\-\_\-256\-\_\-double.\-c} }{\pageref{nb__kernel__ElecRF__VdwLJ__GeomW3P1__avx__256__double_8c}}{}
\item\contentsline{section}{src/gmxlib/nonbonded/nb\-\_\-kernel\-\_\-avx\-\_\-256\-\_\-double/\hyperlink{nb__kernel__ElecRF__VdwLJ__GeomW3W3__avx__256__double_8c}{nb\-\_\-kernel\-\_\-\-Elec\-R\-F\-\_\-\-Vdw\-L\-J\-\_\-\-Geom\-W3\-W3\-\_\-avx\-\_\-256\-\_\-double.\-c} }{\pageref{nb__kernel__ElecRF__VdwLJ__GeomW3W3__avx__256__double_8c}}{}
\item\contentsline{section}{src/gmxlib/nonbonded/nb\-\_\-kernel\-\_\-avx\-\_\-256\-\_\-double/\hyperlink{nb__kernel__ElecRF__VdwLJ__GeomW4P1__avx__256__double_8c}{nb\-\_\-kernel\-\_\-\-Elec\-R\-F\-\_\-\-Vdw\-L\-J\-\_\-\-Geom\-W4\-P1\-\_\-avx\-\_\-256\-\_\-double.\-c} }{\pageref{nb__kernel__ElecRF__VdwLJ__GeomW4P1__avx__256__double_8c}}{}
\item\contentsline{section}{src/gmxlib/nonbonded/nb\-\_\-kernel\-\_\-avx\-\_\-256\-\_\-double/\hyperlink{nb__kernel__ElecRF__VdwLJ__GeomW4W4__avx__256__double_8c}{nb\-\_\-kernel\-\_\-\-Elec\-R\-F\-\_\-\-Vdw\-L\-J\-\_\-\-Geom\-W4\-W4\-\_\-avx\-\_\-256\-\_\-double.\-c} }{\pageref{nb__kernel__ElecRF__VdwLJ__GeomW4W4__avx__256__double_8c}}{}
\item\contentsline{section}{src/gmxlib/nonbonded/nb\-\_\-kernel\-\_\-avx\-\_\-256\-\_\-double/\hyperlink{nb__kernel__ElecRF__VdwNone__GeomP1P1__avx__256__double_8c}{nb\-\_\-kernel\-\_\-\-Elec\-R\-F\-\_\-\-Vdw\-None\-\_\-\-Geom\-P1\-P1\-\_\-avx\-\_\-256\-\_\-double.\-c} }{\pageref{nb__kernel__ElecRF__VdwNone__GeomP1P1__avx__256__double_8c}}{}
\item\contentsline{section}{src/gmxlib/nonbonded/nb\-\_\-kernel\-\_\-avx\-\_\-256\-\_\-double/\hyperlink{nb__kernel__ElecRF__VdwNone__GeomW3P1__avx__256__double_8c}{nb\-\_\-kernel\-\_\-\-Elec\-R\-F\-\_\-\-Vdw\-None\-\_\-\-Geom\-W3\-P1\-\_\-avx\-\_\-256\-\_\-double.\-c} }{\pageref{nb__kernel__ElecRF__VdwNone__GeomW3P1__avx__256__double_8c}}{}
\item\contentsline{section}{src/gmxlib/nonbonded/nb\-\_\-kernel\-\_\-avx\-\_\-256\-\_\-double/\hyperlink{nb__kernel__ElecRF__VdwNone__GeomW3W3__avx__256__double_8c}{nb\-\_\-kernel\-\_\-\-Elec\-R\-F\-\_\-\-Vdw\-None\-\_\-\-Geom\-W3\-W3\-\_\-avx\-\_\-256\-\_\-double.\-c} }{\pageref{nb__kernel__ElecRF__VdwNone__GeomW3W3__avx__256__double_8c}}{}
\item\contentsline{section}{src/gmxlib/nonbonded/nb\-\_\-kernel\-\_\-avx\-\_\-256\-\_\-double/\hyperlink{nb__kernel__ElecRF__VdwNone__GeomW4P1__avx__256__double_8c}{nb\-\_\-kernel\-\_\-\-Elec\-R\-F\-\_\-\-Vdw\-None\-\_\-\-Geom\-W4\-P1\-\_\-avx\-\_\-256\-\_\-double.\-c} }{\pageref{nb__kernel__ElecRF__VdwNone__GeomW4P1__avx__256__double_8c}}{}
\item\contentsline{section}{src/gmxlib/nonbonded/nb\-\_\-kernel\-\_\-avx\-\_\-256\-\_\-double/\hyperlink{nb__kernel__ElecRF__VdwNone__GeomW4W4__avx__256__double_8c}{nb\-\_\-kernel\-\_\-\-Elec\-R\-F\-\_\-\-Vdw\-None\-\_\-\-Geom\-W4\-W4\-\_\-avx\-\_\-256\-\_\-double.\-c} }{\pageref{nb__kernel__ElecRF__VdwNone__GeomW4W4__avx__256__double_8c}}{}
\item\contentsline{section}{src/gmxlib/nonbonded/nb\-\_\-kernel\-\_\-avx\-\_\-256\-\_\-double/\hyperlink{nb__kernel__ElecRFCut__VdwCSTab__GeomP1P1__avx__256__double_8c}{nb\-\_\-kernel\-\_\-\-Elec\-R\-F\-Cut\-\_\-\-Vdw\-C\-S\-Tab\-\_\-\-Geom\-P1\-P1\-\_\-avx\-\_\-256\-\_\-double.\-c} }{\pageref{nb__kernel__ElecRFCut__VdwCSTab__GeomP1P1__avx__256__double_8c}}{}
\item\contentsline{section}{src/gmxlib/nonbonded/nb\-\_\-kernel\-\_\-avx\-\_\-256\-\_\-double/\hyperlink{nb__kernel__ElecRFCut__VdwCSTab__GeomW3P1__avx__256__double_8c}{nb\-\_\-kernel\-\_\-\-Elec\-R\-F\-Cut\-\_\-\-Vdw\-C\-S\-Tab\-\_\-\-Geom\-W3\-P1\-\_\-avx\-\_\-256\-\_\-double.\-c} }{\pageref{nb__kernel__ElecRFCut__VdwCSTab__GeomW3P1__avx__256__double_8c}}{}
\item\contentsline{section}{src/gmxlib/nonbonded/nb\-\_\-kernel\-\_\-avx\-\_\-256\-\_\-double/\hyperlink{nb__kernel__ElecRFCut__VdwCSTab__GeomW3W3__avx__256__double_8c}{nb\-\_\-kernel\-\_\-\-Elec\-R\-F\-Cut\-\_\-\-Vdw\-C\-S\-Tab\-\_\-\-Geom\-W3\-W3\-\_\-avx\-\_\-256\-\_\-double.\-c} }{\pageref{nb__kernel__ElecRFCut__VdwCSTab__GeomW3W3__avx__256__double_8c}}{}
\item\contentsline{section}{src/gmxlib/nonbonded/nb\-\_\-kernel\-\_\-avx\-\_\-256\-\_\-double/\hyperlink{nb__kernel__ElecRFCut__VdwCSTab__GeomW4P1__avx__256__double_8c}{nb\-\_\-kernel\-\_\-\-Elec\-R\-F\-Cut\-\_\-\-Vdw\-C\-S\-Tab\-\_\-\-Geom\-W4\-P1\-\_\-avx\-\_\-256\-\_\-double.\-c} }{\pageref{nb__kernel__ElecRFCut__VdwCSTab__GeomW4P1__avx__256__double_8c}}{}
\item\contentsline{section}{src/gmxlib/nonbonded/nb\-\_\-kernel\-\_\-avx\-\_\-256\-\_\-double/\hyperlink{nb__kernel__ElecRFCut__VdwCSTab__GeomW4W4__avx__256__double_8c}{nb\-\_\-kernel\-\_\-\-Elec\-R\-F\-Cut\-\_\-\-Vdw\-C\-S\-Tab\-\_\-\-Geom\-W4\-W4\-\_\-avx\-\_\-256\-\_\-double.\-c} }{\pageref{nb__kernel__ElecRFCut__VdwCSTab__GeomW4W4__avx__256__double_8c}}{}
\item\contentsline{section}{src/gmxlib/nonbonded/nb\-\_\-kernel\-\_\-avx\-\_\-256\-\_\-double/\hyperlink{nb__kernel__ElecRFCut__VdwLJSh__GeomP1P1__avx__256__double_8c}{nb\-\_\-kernel\-\_\-\-Elec\-R\-F\-Cut\-\_\-\-Vdw\-L\-J\-Sh\-\_\-\-Geom\-P1\-P1\-\_\-avx\-\_\-256\-\_\-double.\-c} }{\pageref{nb__kernel__ElecRFCut__VdwLJSh__GeomP1P1__avx__256__double_8c}}{}
\item\contentsline{section}{src/gmxlib/nonbonded/nb\-\_\-kernel\-\_\-avx\-\_\-256\-\_\-double/\hyperlink{nb__kernel__ElecRFCut__VdwLJSh__GeomW3P1__avx__256__double_8c}{nb\-\_\-kernel\-\_\-\-Elec\-R\-F\-Cut\-\_\-\-Vdw\-L\-J\-Sh\-\_\-\-Geom\-W3\-P1\-\_\-avx\-\_\-256\-\_\-double.\-c} }{\pageref{nb__kernel__ElecRFCut__VdwLJSh__GeomW3P1__avx__256__double_8c}}{}
\item\contentsline{section}{src/gmxlib/nonbonded/nb\-\_\-kernel\-\_\-avx\-\_\-256\-\_\-double/\hyperlink{nb__kernel__ElecRFCut__VdwLJSh__GeomW3W3__avx__256__double_8c}{nb\-\_\-kernel\-\_\-\-Elec\-R\-F\-Cut\-\_\-\-Vdw\-L\-J\-Sh\-\_\-\-Geom\-W3\-W3\-\_\-avx\-\_\-256\-\_\-double.\-c} }{\pageref{nb__kernel__ElecRFCut__VdwLJSh__GeomW3W3__avx__256__double_8c}}{}
\item\contentsline{section}{src/gmxlib/nonbonded/nb\-\_\-kernel\-\_\-avx\-\_\-256\-\_\-double/\hyperlink{nb__kernel__ElecRFCut__VdwLJSh__GeomW4P1__avx__256__double_8c}{nb\-\_\-kernel\-\_\-\-Elec\-R\-F\-Cut\-\_\-\-Vdw\-L\-J\-Sh\-\_\-\-Geom\-W4\-P1\-\_\-avx\-\_\-256\-\_\-double.\-c} }{\pageref{nb__kernel__ElecRFCut__VdwLJSh__GeomW4P1__avx__256__double_8c}}{}
\item\contentsline{section}{src/gmxlib/nonbonded/nb\-\_\-kernel\-\_\-avx\-\_\-256\-\_\-double/\hyperlink{nb__kernel__ElecRFCut__VdwLJSh__GeomW4W4__avx__256__double_8c}{nb\-\_\-kernel\-\_\-\-Elec\-R\-F\-Cut\-\_\-\-Vdw\-L\-J\-Sh\-\_\-\-Geom\-W4\-W4\-\_\-avx\-\_\-256\-\_\-double.\-c} }{\pageref{nb__kernel__ElecRFCut__VdwLJSh__GeomW4W4__avx__256__double_8c}}{}
\item\contentsline{section}{src/gmxlib/nonbonded/nb\-\_\-kernel\-\_\-avx\-\_\-256\-\_\-double/\hyperlink{nb__kernel__ElecRFCut__VdwLJSw__GeomP1P1__avx__256__double_8c}{nb\-\_\-kernel\-\_\-\-Elec\-R\-F\-Cut\-\_\-\-Vdw\-L\-J\-Sw\-\_\-\-Geom\-P1\-P1\-\_\-avx\-\_\-256\-\_\-double.\-c} }{\pageref{nb__kernel__ElecRFCut__VdwLJSw__GeomP1P1__avx__256__double_8c}}{}
\item\contentsline{section}{src/gmxlib/nonbonded/nb\-\_\-kernel\-\_\-avx\-\_\-256\-\_\-double/\hyperlink{nb__kernel__ElecRFCut__VdwLJSw__GeomW3P1__avx__256__double_8c}{nb\-\_\-kernel\-\_\-\-Elec\-R\-F\-Cut\-\_\-\-Vdw\-L\-J\-Sw\-\_\-\-Geom\-W3\-P1\-\_\-avx\-\_\-256\-\_\-double.\-c} }{\pageref{nb__kernel__ElecRFCut__VdwLJSw__GeomW3P1__avx__256__double_8c}}{}
\item\contentsline{section}{src/gmxlib/nonbonded/nb\-\_\-kernel\-\_\-avx\-\_\-256\-\_\-double/\hyperlink{nb__kernel__ElecRFCut__VdwLJSw__GeomW3W3__avx__256__double_8c}{nb\-\_\-kernel\-\_\-\-Elec\-R\-F\-Cut\-\_\-\-Vdw\-L\-J\-Sw\-\_\-\-Geom\-W3\-W3\-\_\-avx\-\_\-256\-\_\-double.\-c} }{\pageref{nb__kernel__ElecRFCut__VdwLJSw__GeomW3W3__avx__256__double_8c}}{}
\item\contentsline{section}{src/gmxlib/nonbonded/nb\-\_\-kernel\-\_\-avx\-\_\-256\-\_\-double/\hyperlink{nb__kernel__ElecRFCut__VdwLJSw__GeomW4P1__avx__256__double_8c}{nb\-\_\-kernel\-\_\-\-Elec\-R\-F\-Cut\-\_\-\-Vdw\-L\-J\-Sw\-\_\-\-Geom\-W4\-P1\-\_\-avx\-\_\-256\-\_\-double.\-c} }{\pageref{nb__kernel__ElecRFCut__VdwLJSw__GeomW4P1__avx__256__double_8c}}{}
\item\contentsline{section}{src/gmxlib/nonbonded/nb\-\_\-kernel\-\_\-avx\-\_\-256\-\_\-double/\hyperlink{nb__kernel__ElecRFCut__VdwLJSw__GeomW4W4__avx__256__double_8c}{nb\-\_\-kernel\-\_\-\-Elec\-R\-F\-Cut\-\_\-\-Vdw\-L\-J\-Sw\-\_\-\-Geom\-W4\-W4\-\_\-avx\-\_\-256\-\_\-double.\-c} }{\pageref{nb__kernel__ElecRFCut__VdwLJSw__GeomW4W4__avx__256__double_8c}}{}
\item\contentsline{section}{src/gmxlib/nonbonded/nb\-\_\-kernel\-\_\-avx\-\_\-256\-\_\-double/\hyperlink{nb__kernel__ElecRFCut__VdwNone__GeomP1P1__avx__256__double_8c}{nb\-\_\-kernel\-\_\-\-Elec\-R\-F\-Cut\-\_\-\-Vdw\-None\-\_\-\-Geom\-P1\-P1\-\_\-avx\-\_\-256\-\_\-double.\-c} }{\pageref{nb__kernel__ElecRFCut__VdwNone__GeomP1P1__avx__256__double_8c}}{}
\item\contentsline{section}{src/gmxlib/nonbonded/nb\-\_\-kernel\-\_\-avx\-\_\-256\-\_\-double/\hyperlink{nb__kernel__ElecRFCut__VdwNone__GeomW3P1__avx__256__double_8c}{nb\-\_\-kernel\-\_\-\-Elec\-R\-F\-Cut\-\_\-\-Vdw\-None\-\_\-\-Geom\-W3\-P1\-\_\-avx\-\_\-256\-\_\-double.\-c} }{\pageref{nb__kernel__ElecRFCut__VdwNone__GeomW3P1__avx__256__double_8c}}{}
\item\contentsline{section}{src/gmxlib/nonbonded/nb\-\_\-kernel\-\_\-avx\-\_\-256\-\_\-double/\hyperlink{nb__kernel__ElecRFCut__VdwNone__GeomW3W3__avx__256__double_8c}{nb\-\_\-kernel\-\_\-\-Elec\-R\-F\-Cut\-\_\-\-Vdw\-None\-\_\-\-Geom\-W3\-W3\-\_\-avx\-\_\-256\-\_\-double.\-c} }{\pageref{nb__kernel__ElecRFCut__VdwNone__GeomW3W3__avx__256__double_8c}}{}
\item\contentsline{section}{src/gmxlib/nonbonded/nb\-\_\-kernel\-\_\-avx\-\_\-256\-\_\-double/\hyperlink{nb__kernel__ElecRFCut__VdwNone__GeomW4P1__avx__256__double_8c}{nb\-\_\-kernel\-\_\-\-Elec\-R\-F\-Cut\-\_\-\-Vdw\-None\-\_\-\-Geom\-W4\-P1\-\_\-avx\-\_\-256\-\_\-double.\-c} }{\pageref{nb__kernel__ElecRFCut__VdwNone__GeomW4P1__avx__256__double_8c}}{}
\item\contentsline{section}{src/gmxlib/nonbonded/nb\-\_\-kernel\-\_\-avx\-\_\-256\-\_\-double/\hyperlink{nb__kernel__ElecRFCut__VdwNone__GeomW4W4__avx__256__double_8c}{nb\-\_\-kernel\-\_\-\-Elec\-R\-F\-Cut\-\_\-\-Vdw\-None\-\_\-\-Geom\-W4\-W4\-\_\-avx\-\_\-256\-\_\-double.\-c} }{\pageref{nb__kernel__ElecRFCut__VdwNone__GeomW4W4__avx__256__double_8c}}{}
\item\contentsline{section}{src/gmxlib/nonbonded/nb\-\_\-kernel\-\_\-avx\-\_\-256\-\_\-single/\hyperlink{kernelutil__x86__avx__256__single_8h}{kernelutil\-\_\-x86\-\_\-avx\-\_\-256\-\_\-single.\-h} }{\pageref{kernelutil__x86__avx__256__single_8h}}{}
\item\contentsline{section}{src/gmxlib/nonbonded/nb\-\_\-kernel\-\_\-avx\-\_\-256\-\_\-single/\hyperlink{nb__kernel__avx__256__single_8c}{nb\-\_\-kernel\-\_\-avx\-\_\-256\-\_\-single.\-c} }{\pageref{nb__kernel__avx__256__single_8c}}{}
\item\contentsline{section}{src/gmxlib/nonbonded/nb\-\_\-kernel\-\_\-avx\-\_\-256\-\_\-single/\hyperlink{nb__kernel__avx__256__single_8h}{nb\-\_\-kernel\-\_\-avx\-\_\-256\-\_\-single.\-h} }{\pageref{nb__kernel__avx__256__single_8h}}{}
\item\contentsline{section}{src/gmxlib/nonbonded/nb\-\_\-kernel\-\_\-avx\-\_\-256\-\_\-single/\hyperlink{nb__kernel__ElecCoul__VdwCSTab__GeomP1P1__avx__256__single_8c}{nb\-\_\-kernel\-\_\-\-Elec\-Coul\-\_\-\-Vdw\-C\-S\-Tab\-\_\-\-Geom\-P1\-P1\-\_\-avx\-\_\-256\-\_\-single.\-c} }{\pageref{nb__kernel__ElecCoul__VdwCSTab__GeomP1P1__avx__256__single_8c}}{}
\item\contentsline{section}{src/gmxlib/nonbonded/nb\-\_\-kernel\-\_\-avx\-\_\-256\-\_\-single/\hyperlink{nb__kernel__ElecCoul__VdwCSTab__GeomW3P1__avx__256__single_8c}{nb\-\_\-kernel\-\_\-\-Elec\-Coul\-\_\-\-Vdw\-C\-S\-Tab\-\_\-\-Geom\-W3\-P1\-\_\-avx\-\_\-256\-\_\-single.\-c} }{\pageref{nb__kernel__ElecCoul__VdwCSTab__GeomW3P1__avx__256__single_8c}}{}
\item\contentsline{section}{src/gmxlib/nonbonded/nb\-\_\-kernel\-\_\-avx\-\_\-256\-\_\-single/\hyperlink{nb__kernel__ElecCoul__VdwCSTab__GeomW3W3__avx__256__single_8c}{nb\-\_\-kernel\-\_\-\-Elec\-Coul\-\_\-\-Vdw\-C\-S\-Tab\-\_\-\-Geom\-W3\-W3\-\_\-avx\-\_\-256\-\_\-single.\-c} }{\pageref{nb__kernel__ElecCoul__VdwCSTab__GeomW3W3__avx__256__single_8c}}{}
\item\contentsline{section}{src/gmxlib/nonbonded/nb\-\_\-kernel\-\_\-avx\-\_\-256\-\_\-single/\hyperlink{nb__kernel__ElecCoul__VdwCSTab__GeomW4P1__avx__256__single_8c}{nb\-\_\-kernel\-\_\-\-Elec\-Coul\-\_\-\-Vdw\-C\-S\-Tab\-\_\-\-Geom\-W4\-P1\-\_\-avx\-\_\-256\-\_\-single.\-c} }{\pageref{nb__kernel__ElecCoul__VdwCSTab__GeomW4P1__avx__256__single_8c}}{}
\item\contentsline{section}{src/gmxlib/nonbonded/nb\-\_\-kernel\-\_\-avx\-\_\-256\-\_\-single/\hyperlink{nb__kernel__ElecCoul__VdwCSTab__GeomW4W4__avx__256__single_8c}{nb\-\_\-kernel\-\_\-\-Elec\-Coul\-\_\-\-Vdw\-C\-S\-Tab\-\_\-\-Geom\-W4\-W4\-\_\-avx\-\_\-256\-\_\-single.\-c} }{\pageref{nb__kernel__ElecCoul__VdwCSTab__GeomW4W4__avx__256__single_8c}}{}
\item\contentsline{section}{src/gmxlib/nonbonded/nb\-\_\-kernel\-\_\-avx\-\_\-256\-\_\-single/\hyperlink{nb__kernel__ElecCoul__VdwLJ__GeomP1P1__avx__256__single_8c}{nb\-\_\-kernel\-\_\-\-Elec\-Coul\-\_\-\-Vdw\-L\-J\-\_\-\-Geom\-P1\-P1\-\_\-avx\-\_\-256\-\_\-single.\-c} }{\pageref{nb__kernel__ElecCoul__VdwLJ__GeomP1P1__avx__256__single_8c}}{}
\item\contentsline{section}{src/gmxlib/nonbonded/nb\-\_\-kernel\-\_\-avx\-\_\-256\-\_\-single/\hyperlink{nb__kernel__ElecCoul__VdwLJ__GeomW3P1__avx__256__single_8c}{nb\-\_\-kernel\-\_\-\-Elec\-Coul\-\_\-\-Vdw\-L\-J\-\_\-\-Geom\-W3\-P1\-\_\-avx\-\_\-256\-\_\-single.\-c} }{\pageref{nb__kernel__ElecCoul__VdwLJ__GeomW3P1__avx__256__single_8c}}{}
\item\contentsline{section}{src/gmxlib/nonbonded/nb\-\_\-kernel\-\_\-avx\-\_\-256\-\_\-single/\hyperlink{nb__kernel__ElecCoul__VdwLJ__GeomW3W3__avx__256__single_8c}{nb\-\_\-kernel\-\_\-\-Elec\-Coul\-\_\-\-Vdw\-L\-J\-\_\-\-Geom\-W3\-W3\-\_\-avx\-\_\-256\-\_\-single.\-c} }{\pageref{nb__kernel__ElecCoul__VdwLJ__GeomW3W3__avx__256__single_8c}}{}
\item\contentsline{section}{src/gmxlib/nonbonded/nb\-\_\-kernel\-\_\-avx\-\_\-256\-\_\-single/\hyperlink{nb__kernel__ElecCoul__VdwLJ__GeomW4P1__avx__256__single_8c}{nb\-\_\-kernel\-\_\-\-Elec\-Coul\-\_\-\-Vdw\-L\-J\-\_\-\-Geom\-W4\-P1\-\_\-avx\-\_\-256\-\_\-single.\-c} }{\pageref{nb__kernel__ElecCoul__VdwLJ__GeomW4P1__avx__256__single_8c}}{}
\item\contentsline{section}{src/gmxlib/nonbonded/nb\-\_\-kernel\-\_\-avx\-\_\-256\-\_\-single/\hyperlink{nb__kernel__ElecCoul__VdwLJ__GeomW4W4__avx__256__single_8c}{nb\-\_\-kernel\-\_\-\-Elec\-Coul\-\_\-\-Vdw\-L\-J\-\_\-\-Geom\-W4\-W4\-\_\-avx\-\_\-256\-\_\-single.\-c} }{\pageref{nb__kernel__ElecCoul__VdwLJ__GeomW4W4__avx__256__single_8c}}{}
\item\contentsline{section}{src/gmxlib/nonbonded/nb\-\_\-kernel\-\_\-avx\-\_\-256\-\_\-single/\hyperlink{nb__kernel__ElecCoul__VdwNone__GeomP1P1__avx__256__single_8c}{nb\-\_\-kernel\-\_\-\-Elec\-Coul\-\_\-\-Vdw\-None\-\_\-\-Geom\-P1\-P1\-\_\-avx\-\_\-256\-\_\-single.\-c} }{\pageref{nb__kernel__ElecCoul__VdwNone__GeomP1P1__avx__256__single_8c}}{}
\item\contentsline{section}{src/gmxlib/nonbonded/nb\-\_\-kernel\-\_\-avx\-\_\-256\-\_\-single/\hyperlink{nb__kernel__ElecCoul__VdwNone__GeomW3P1__avx__256__single_8c}{nb\-\_\-kernel\-\_\-\-Elec\-Coul\-\_\-\-Vdw\-None\-\_\-\-Geom\-W3\-P1\-\_\-avx\-\_\-256\-\_\-single.\-c} }{\pageref{nb__kernel__ElecCoul__VdwNone__GeomW3P1__avx__256__single_8c}}{}
\item\contentsline{section}{src/gmxlib/nonbonded/nb\-\_\-kernel\-\_\-avx\-\_\-256\-\_\-single/\hyperlink{nb__kernel__ElecCoul__VdwNone__GeomW3W3__avx__256__single_8c}{nb\-\_\-kernel\-\_\-\-Elec\-Coul\-\_\-\-Vdw\-None\-\_\-\-Geom\-W3\-W3\-\_\-avx\-\_\-256\-\_\-single.\-c} }{\pageref{nb__kernel__ElecCoul__VdwNone__GeomW3W3__avx__256__single_8c}}{}
\item\contentsline{section}{src/gmxlib/nonbonded/nb\-\_\-kernel\-\_\-avx\-\_\-256\-\_\-single/\hyperlink{nb__kernel__ElecCoul__VdwNone__GeomW4P1__avx__256__single_8c}{nb\-\_\-kernel\-\_\-\-Elec\-Coul\-\_\-\-Vdw\-None\-\_\-\-Geom\-W4\-P1\-\_\-avx\-\_\-256\-\_\-single.\-c} }{\pageref{nb__kernel__ElecCoul__VdwNone__GeomW4P1__avx__256__single_8c}}{}
\item\contentsline{section}{src/gmxlib/nonbonded/nb\-\_\-kernel\-\_\-avx\-\_\-256\-\_\-single/\hyperlink{nb__kernel__ElecCoul__VdwNone__GeomW4W4__avx__256__single_8c}{nb\-\_\-kernel\-\_\-\-Elec\-Coul\-\_\-\-Vdw\-None\-\_\-\-Geom\-W4\-W4\-\_\-avx\-\_\-256\-\_\-single.\-c} }{\pageref{nb__kernel__ElecCoul__VdwNone__GeomW4W4__avx__256__single_8c}}{}
\item\contentsline{section}{src/gmxlib/nonbonded/nb\-\_\-kernel\-\_\-avx\-\_\-256\-\_\-single/\hyperlink{nb__kernel__ElecCSTab__VdwCSTab__GeomP1P1__avx__256__single_8c}{nb\-\_\-kernel\-\_\-\-Elec\-C\-S\-Tab\-\_\-\-Vdw\-C\-S\-Tab\-\_\-\-Geom\-P1\-P1\-\_\-avx\-\_\-256\-\_\-single.\-c} }{\pageref{nb__kernel__ElecCSTab__VdwCSTab__GeomP1P1__avx__256__single_8c}}{}
\item\contentsline{section}{src/gmxlib/nonbonded/nb\-\_\-kernel\-\_\-avx\-\_\-256\-\_\-single/\hyperlink{nb__kernel__ElecCSTab__VdwCSTab__GeomW3P1__avx__256__single_8c}{nb\-\_\-kernel\-\_\-\-Elec\-C\-S\-Tab\-\_\-\-Vdw\-C\-S\-Tab\-\_\-\-Geom\-W3\-P1\-\_\-avx\-\_\-256\-\_\-single.\-c} }{\pageref{nb__kernel__ElecCSTab__VdwCSTab__GeomW3P1__avx__256__single_8c}}{}
\item\contentsline{section}{src/gmxlib/nonbonded/nb\-\_\-kernel\-\_\-avx\-\_\-256\-\_\-single/\hyperlink{nb__kernel__ElecCSTab__VdwCSTab__GeomW3W3__avx__256__single_8c}{nb\-\_\-kernel\-\_\-\-Elec\-C\-S\-Tab\-\_\-\-Vdw\-C\-S\-Tab\-\_\-\-Geom\-W3\-W3\-\_\-avx\-\_\-256\-\_\-single.\-c} }{\pageref{nb__kernel__ElecCSTab__VdwCSTab__GeomW3W3__avx__256__single_8c}}{}
\item\contentsline{section}{src/gmxlib/nonbonded/nb\-\_\-kernel\-\_\-avx\-\_\-256\-\_\-single/\hyperlink{nb__kernel__ElecCSTab__VdwCSTab__GeomW4P1__avx__256__single_8c}{nb\-\_\-kernel\-\_\-\-Elec\-C\-S\-Tab\-\_\-\-Vdw\-C\-S\-Tab\-\_\-\-Geom\-W4\-P1\-\_\-avx\-\_\-256\-\_\-single.\-c} }{\pageref{nb__kernel__ElecCSTab__VdwCSTab__GeomW4P1__avx__256__single_8c}}{}
\item\contentsline{section}{src/gmxlib/nonbonded/nb\-\_\-kernel\-\_\-avx\-\_\-256\-\_\-single/\hyperlink{nb__kernel__ElecCSTab__VdwCSTab__GeomW4W4__avx__256__single_8c}{nb\-\_\-kernel\-\_\-\-Elec\-C\-S\-Tab\-\_\-\-Vdw\-C\-S\-Tab\-\_\-\-Geom\-W4\-W4\-\_\-avx\-\_\-256\-\_\-single.\-c} }{\pageref{nb__kernel__ElecCSTab__VdwCSTab__GeomW4W4__avx__256__single_8c}}{}
\item\contentsline{section}{src/gmxlib/nonbonded/nb\-\_\-kernel\-\_\-avx\-\_\-256\-\_\-single/\hyperlink{nb__kernel__ElecCSTab__VdwLJ__GeomP1P1__avx__256__single_8c}{nb\-\_\-kernel\-\_\-\-Elec\-C\-S\-Tab\-\_\-\-Vdw\-L\-J\-\_\-\-Geom\-P1\-P1\-\_\-avx\-\_\-256\-\_\-single.\-c} }{\pageref{nb__kernel__ElecCSTab__VdwLJ__GeomP1P1__avx__256__single_8c}}{}
\item\contentsline{section}{src/gmxlib/nonbonded/nb\-\_\-kernel\-\_\-avx\-\_\-256\-\_\-single/\hyperlink{nb__kernel__ElecCSTab__VdwLJ__GeomW3P1__avx__256__single_8c}{nb\-\_\-kernel\-\_\-\-Elec\-C\-S\-Tab\-\_\-\-Vdw\-L\-J\-\_\-\-Geom\-W3\-P1\-\_\-avx\-\_\-256\-\_\-single.\-c} }{\pageref{nb__kernel__ElecCSTab__VdwLJ__GeomW3P1__avx__256__single_8c}}{}
\item\contentsline{section}{src/gmxlib/nonbonded/nb\-\_\-kernel\-\_\-avx\-\_\-256\-\_\-single/\hyperlink{nb__kernel__ElecCSTab__VdwLJ__GeomW3W3__avx__256__single_8c}{nb\-\_\-kernel\-\_\-\-Elec\-C\-S\-Tab\-\_\-\-Vdw\-L\-J\-\_\-\-Geom\-W3\-W3\-\_\-avx\-\_\-256\-\_\-single.\-c} }{\pageref{nb__kernel__ElecCSTab__VdwLJ__GeomW3W3__avx__256__single_8c}}{}
\item\contentsline{section}{src/gmxlib/nonbonded/nb\-\_\-kernel\-\_\-avx\-\_\-256\-\_\-single/\hyperlink{nb__kernel__ElecCSTab__VdwLJ__GeomW4P1__avx__256__single_8c}{nb\-\_\-kernel\-\_\-\-Elec\-C\-S\-Tab\-\_\-\-Vdw\-L\-J\-\_\-\-Geom\-W4\-P1\-\_\-avx\-\_\-256\-\_\-single.\-c} }{\pageref{nb__kernel__ElecCSTab__VdwLJ__GeomW4P1__avx__256__single_8c}}{}
\item\contentsline{section}{src/gmxlib/nonbonded/nb\-\_\-kernel\-\_\-avx\-\_\-256\-\_\-single/\hyperlink{nb__kernel__ElecCSTab__VdwLJ__GeomW4W4__avx__256__single_8c}{nb\-\_\-kernel\-\_\-\-Elec\-C\-S\-Tab\-\_\-\-Vdw\-L\-J\-\_\-\-Geom\-W4\-W4\-\_\-avx\-\_\-256\-\_\-single.\-c} }{\pageref{nb__kernel__ElecCSTab__VdwLJ__GeomW4W4__avx__256__single_8c}}{}
\item\contentsline{section}{src/gmxlib/nonbonded/nb\-\_\-kernel\-\_\-avx\-\_\-256\-\_\-single/\hyperlink{nb__kernel__ElecCSTab__VdwNone__GeomP1P1__avx__256__single_8c}{nb\-\_\-kernel\-\_\-\-Elec\-C\-S\-Tab\-\_\-\-Vdw\-None\-\_\-\-Geom\-P1\-P1\-\_\-avx\-\_\-256\-\_\-single.\-c} }{\pageref{nb__kernel__ElecCSTab__VdwNone__GeomP1P1__avx__256__single_8c}}{}
\item\contentsline{section}{src/gmxlib/nonbonded/nb\-\_\-kernel\-\_\-avx\-\_\-256\-\_\-single/\hyperlink{nb__kernel__ElecCSTab__VdwNone__GeomW3P1__avx__256__single_8c}{nb\-\_\-kernel\-\_\-\-Elec\-C\-S\-Tab\-\_\-\-Vdw\-None\-\_\-\-Geom\-W3\-P1\-\_\-avx\-\_\-256\-\_\-single.\-c} }{\pageref{nb__kernel__ElecCSTab__VdwNone__GeomW3P1__avx__256__single_8c}}{}
\item\contentsline{section}{src/gmxlib/nonbonded/nb\-\_\-kernel\-\_\-avx\-\_\-256\-\_\-single/\hyperlink{nb__kernel__ElecCSTab__VdwNone__GeomW3W3__avx__256__single_8c}{nb\-\_\-kernel\-\_\-\-Elec\-C\-S\-Tab\-\_\-\-Vdw\-None\-\_\-\-Geom\-W3\-W3\-\_\-avx\-\_\-256\-\_\-single.\-c} }{\pageref{nb__kernel__ElecCSTab__VdwNone__GeomW3W3__avx__256__single_8c}}{}
\item\contentsline{section}{src/gmxlib/nonbonded/nb\-\_\-kernel\-\_\-avx\-\_\-256\-\_\-single/\hyperlink{nb__kernel__ElecCSTab__VdwNone__GeomW4P1__avx__256__single_8c}{nb\-\_\-kernel\-\_\-\-Elec\-C\-S\-Tab\-\_\-\-Vdw\-None\-\_\-\-Geom\-W4\-P1\-\_\-avx\-\_\-256\-\_\-single.\-c} }{\pageref{nb__kernel__ElecCSTab__VdwNone__GeomW4P1__avx__256__single_8c}}{}
\item\contentsline{section}{src/gmxlib/nonbonded/nb\-\_\-kernel\-\_\-avx\-\_\-256\-\_\-single/\hyperlink{nb__kernel__ElecCSTab__VdwNone__GeomW4W4__avx__256__single_8c}{nb\-\_\-kernel\-\_\-\-Elec\-C\-S\-Tab\-\_\-\-Vdw\-None\-\_\-\-Geom\-W4\-W4\-\_\-avx\-\_\-256\-\_\-single.\-c} }{\pageref{nb__kernel__ElecCSTab__VdwNone__GeomW4W4__avx__256__single_8c}}{}
\item\contentsline{section}{src/gmxlib/nonbonded/nb\-\_\-kernel\-\_\-avx\-\_\-256\-\_\-single/\hyperlink{nb__kernel__ElecEw__VdwCSTab__GeomP1P1__avx__256__single_8c}{nb\-\_\-kernel\-\_\-\-Elec\-Ew\-\_\-\-Vdw\-C\-S\-Tab\-\_\-\-Geom\-P1\-P1\-\_\-avx\-\_\-256\-\_\-single.\-c} }{\pageref{nb__kernel__ElecEw__VdwCSTab__GeomP1P1__avx__256__single_8c}}{}
\item\contentsline{section}{src/gmxlib/nonbonded/nb\-\_\-kernel\-\_\-avx\-\_\-256\-\_\-single/\hyperlink{nb__kernel__ElecEw__VdwCSTab__GeomW3P1__avx__256__single_8c}{nb\-\_\-kernel\-\_\-\-Elec\-Ew\-\_\-\-Vdw\-C\-S\-Tab\-\_\-\-Geom\-W3\-P1\-\_\-avx\-\_\-256\-\_\-single.\-c} }{\pageref{nb__kernel__ElecEw__VdwCSTab__GeomW3P1__avx__256__single_8c}}{}
\item\contentsline{section}{src/gmxlib/nonbonded/nb\-\_\-kernel\-\_\-avx\-\_\-256\-\_\-single/\hyperlink{nb__kernel__ElecEw__VdwCSTab__GeomW3W3__avx__256__single_8c}{nb\-\_\-kernel\-\_\-\-Elec\-Ew\-\_\-\-Vdw\-C\-S\-Tab\-\_\-\-Geom\-W3\-W3\-\_\-avx\-\_\-256\-\_\-single.\-c} }{\pageref{nb__kernel__ElecEw__VdwCSTab__GeomW3W3__avx__256__single_8c}}{}
\item\contentsline{section}{src/gmxlib/nonbonded/nb\-\_\-kernel\-\_\-avx\-\_\-256\-\_\-single/\hyperlink{nb__kernel__ElecEw__VdwCSTab__GeomW4P1__avx__256__single_8c}{nb\-\_\-kernel\-\_\-\-Elec\-Ew\-\_\-\-Vdw\-C\-S\-Tab\-\_\-\-Geom\-W4\-P1\-\_\-avx\-\_\-256\-\_\-single.\-c} }{\pageref{nb__kernel__ElecEw__VdwCSTab__GeomW4P1__avx__256__single_8c}}{}
\item\contentsline{section}{src/gmxlib/nonbonded/nb\-\_\-kernel\-\_\-avx\-\_\-256\-\_\-single/\hyperlink{nb__kernel__ElecEw__VdwCSTab__GeomW4W4__avx__256__single_8c}{nb\-\_\-kernel\-\_\-\-Elec\-Ew\-\_\-\-Vdw\-C\-S\-Tab\-\_\-\-Geom\-W4\-W4\-\_\-avx\-\_\-256\-\_\-single.\-c} }{\pageref{nb__kernel__ElecEw__VdwCSTab__GeomW4W4__avx__256__single_8c}}{}
\item\contentsline{section}{src/gmxlib/nonbonded/nb\-\_\-kernel\-\_\-avx\-\_\-256\-\_\-single/\hyperlink{nb__kernel__ElecEw__VdwLJ__GeomP1P1__avx__256__single_8c}{nb\-\_\-kernel\-\_\-\-Elec\-Ew\-\_\-\-Vdw\-L\-J\-\_\-\-Geom\-P1\-P1\-\_\-avx\-\_\-256\-\_\-single.\-c} }{\pageref{nb__kernel__ElecEw__VdwLJ__GeomP1P1__avx__256__single_8c}}{}
\item\contentsline{section}{src/gmxlib/nonbonded/nb\-\_\-kernel\-\_\-avx\-\_\-256\-\_\-single/\hyperlink{nb__kernel__ElecEw__VdwLJ__GeomW3P1__avx__256__single_8c}{nb\-\_\-kernel\-\_\-\-Elec\-Ew\-\_\-\-Vdw\-L\-J\-\_\-\-Geom\-W3\-P1\-\_\-avx\-\_\-256\-\_\-single.\-c} }{\pageref{nb__kernel__ElecEw__VdwLJ__GeomW3P1__avx__256__single_8c}}{}
\item\contentsline{section}{src/gmxlib/nonbonded/nb\-\_\-kernel\-\_\-avx\-\_\-256\-\_\-single/\hyperlink{nb__kernel__ElecEw__VdwLJ__GeomW3W3__avx__256__single_8c}{nb\-\_\-kernel\-\_\-\-Elec\-Ew\-\_\-\-Vdw\-L\-J\-\_\-\-Geom\-W3\-W3\-\_\-avx\-\_\-256\-\_\-single.\-c} }{\pageref{nb__kernel__ElecEw__VdwLJ__GeomW3W3__avx__256__single_8c}}{}
\item\contentsline{section}{src/gmxlib/nonbonded/nb\-\_\-kernel\-\_\-avx\-\_\-256\-\_\-single/\hyperlink{nb__kernel__ElecEw__VdwLJ__GeomW4P1__avx__256__single_8c}{nb\-\_\-kernel\-\_\-\-Elec\-Ew\-\_\-\-Vdw\-L\-J\-\_\-\-Geom\-W4\-P1\-\_\-avx\-\_\-256\-\_\-single.\-c} }{\pageref{nb__kernel__ElecEw__VdwLJ__GeomW4P1__avx__256__single_8c}}{}
\item\contentsline{section}{src/gmxlib/nonbonded/nb\-\_\-kernel\-\_\-avx\-\_\-256\-\_\-single/\hyperlink{nb__kernel__ElecEw__VdwLJ__GeomW4W4__avx__256__single_8c}{nb\-\_\-kernel\-\_\-\-Elec\-Ew\-\_\-\-Vdw\-L\-J\-\_\-\-Geom\-W4\-W4\-\_\-avx\-\_\-256\-\_\-single.\-c} }{\pageref{nb__kernel__ElecEw__VdwLJ__GeomW4W4__avx__256__single_8c}}{}
\item\contentsline{section}{src/gmxlib/nonbonded/nb\-\_\-kernel\-\_\-avx\-\_\-256\-\_\-single/\hyperlink{nb__kernel__ElecEw__VdwNone__GeomP1P1__avx__256__single_8c}{nb\-\_\-kernel\-\_\-\-Elec\-Ew\-\_\-\-Vdw\-None\-\_\-\-Geom\-P1\-P1\-\_\-avx\-\_\-256\-\_\-single.\-c} }{\pageref{nb__kernel__ElecEw__VdwNone__GeomP1P1__avx__256__single_8c}}{}
\item\contentsline{section}{src/gmxlib/nonbonded/nb\-\_\-kernel\-\_\-avx\-\_\-256\-\_\-single/\hyperlink{nb__kernel__ElecEw__VdwNone__GeomW3P1__avx__256__single_8c}{nb\-\_\-kernel\-\_\-\-Elec\-Ew\-\_\-\-Vdw\-None\-\_\-\-Geom\-W3\-P1\-\_\-avx\-\_\-256\-\_\-single.\-c} }{\pageref{nb__kernel__ElecEw__VdwNone__GeomW3P1__avx__256__single_8c}}{}
\item\contentsline{section}{src/gmxlib/nonbonded/nb\-\_\-kernel\-\_\-avx\-\_\-256\-\_\-single/\hyperlink{nb__kernel__ElecEw__VdwNone__GeomW3W3__avx__256__single_8c}{nb\-\_\-kernel\-\_\-\-Elec\-Ew\-\_\-\-Vdw\-None\-\_\-\-Geom\-W3\-W3\-\_\-avx\-\_\-256\-\_\-single.\-c} }{\pageref{nb__kernel__ElecEw__VdwNone__GeomW3W3__avx__256__single_8c}}{}
\item\contentsline{section}{src/gmxlib/nonbonded/nb\-\_\-kernel\-\_\-avx\-\_\-256\-\_\-single/\hyperlink{nb__kernel__ElecEw__VdwNone__GeomW4P1__avx__256__single_8c}{nb\-\_\-kernel\-\_\-\-Elec\-Ew\-\_\-\-Vdw\-None\-\_\-\-Geom\-W4\-P1\-\_\-avx\-\_\-256\-\_\-single.\-c} }{\pageref{nb__kernel__ElecEw__VdwNone__GeomW4P1__avx__256__single_8c}}{}
\item\contentsline{section}{src/gmxlib/nonbonded/nb\-\_\-kernel\-\_\-avx\-\_\-256\-\_\-single/\hyperlink{nb__kernel__ElecEw__VdwNone__GeomW4W4__avx__256__single_8c}{nb\-\_\-kernel\-\_\-\-Elec\-Ew\-\_\-\-Vdw\-None\-\_\-\-Geom\-W4\-W4\-\_\-avx\-\_\-256\-\_\-single.\-c} }{\pageref{nb__kernel__ElecEw__VdwNone__GeomW4W4__avx__256__single_8c}}{}
\item\contentsline{section}{src/gmxlib/nonbonded/nb\-\_\-kernel\-\_\-avx\-\_\-256\-\_\-single/\hyperlink{nb__kernel__ElecEwSh__VdwLJSh__GeomP1P1__avx__256__single_8c}{nb\-\_\-kernel\-\_\-\-Elec\-Ew\-Sh\-\_\-\-Vdw\-L\-J\-Sh\-\_\-\-Geom\-P1\-P1\-\_\-avx\-\_\-256\-\_\-single.\-c} }{\pageref{nb__kernel__ElecEwSh__VdwLJSh__GeomP1P1__avx__256__single_8c}}{}
\item\contentsline{section}{src/gmxlib/nonbonded/nb\-\_\-kernel\-\_\-avx\-\_\-256\-\_\-single/\hyperlink{nb__kernel__ElecEwSh__VdwLJSh__GeomW3P1__avx__256__single_8c}{nb\-\_\-kernel\-\_\-\-Elec\-Ew\-Sh\-\_\-\-Vdw\-L\-J\-Sh\-\_\-\-Geom\-W3\-P1\-\_\-avx\-\_\-256\-\_\-single.\-c} }{\pageref{nb__kernel__ElecEwSh__VdwLJSh__GeomW3P1__avx__256__single_8c}}{}
\item\contentsline{section}{src/gmxlib/nonbonded/nb\-\_\-kernel\-\_\-avx\-\_\-256\-\_\-single/\hyperlink{nb__kernel__ElecEwSh__VdwLJSh__GeomW3W3__avx__256__single_8c}{nb\-\_\-kernel\-\_\-\-Elec\-Ew\-Sh\-\_\-\-Vdw\-L\-J\-Sh\-\_\-\-Geom\-W3\-W3\-\_\-avx\-\_\-256\-\_\-single.\-c} }{\pageref{nb__kernel__ElecEwSh__VdwLJSh__GeomW3W3__avx__256__single_8c}}{}
\item\contentsline{section}{src/gmxlib/nonbonded/nb\-\_\-kernel\-\_\-avx\-\_\-256\-\_\-single/\hyperlink{nb__kernel__ElecEwSh__VdwLJSh__GeomW4P1__avx__256__single_8c}{nb\-\_\-kernel\-\_\-\-Elec\-Ew\-Sh\-\_\-\-Vdw\-L\-J\-Sh\-\_\-\-Geom\-W4\-P1\-\_\-avx\-\_\-256\-\_\-single.\-c} }{\pageref{nb__kernel__ElecEwSh__VdwLJSh__GeomW4P1__avx__256__single_8c}}{}
\item\contentsline{section}{src/gmxlib/nonbonded/nb\-\_\-kernel\-\_\-avx\-\_\-256\-\_\-single/\hyperlink{nb__kernel__ElecEwSh__VdwLJSh__GeomW4W4__avx__256__single_8c}{nb\-\_\-kernel\-\_\-\-Elec\-Ew\-Sh\-\_\-\-Vdw\-L\-J\-Sh\-\_\-\-Geom\-W4\-W4\-\_\-avx\-\_\-256\-\_\-single.\-c} }{\pageref{nb__kernel__ElecEwSh__VdwLJSh__GeomW4W4__avx__256__single_8c}}{}
\item\contentsline{section}{src/gmxlib/nonbonded/nb\-\_\-kernel\-\_\-avx\-\_\-256\-\_\-single/\hyperlink{nb__kernel__ElecEwSh__VdwNone__GeomP1P1__avx__256__single_8c}{nb\-\_\-kernel\-\_\-\-Elec\-Ew\-Sh\-\_\-\-Vdw\-None\-\_\-\-Geom\-P1\-P1\-\_\-avx\-\_\-256\-\_\-single.\-c} }{\pageref{nb__kernel__ElecEwSh__VdwNone__GeomP1P1__avx__256__single_8c}}{}
\item\contentsline{section}{src/gmxlib/nonbonded/nb\-\_\-kernel\-\_\-avx\-\_\-256\-\_\-single/\hyperlink{nb__kernel__ElecEwSh__VdwNone__GeomW3P1__avx__256__single_8c}{nb\-\_\-kernel\-\_\-\-Elec\-Ew\-Sh\-\_\-\-Vdw\-None\-\_\-\-Geom\-W3\-P1\-\_\-avx\-\_\-256\-\_\-single.\-c} }{\pageref{nb__kernel__ElecEwSh__VdwNone__GeomW3P1__avx__256__single_8c}}{}
\item\contentsline{section}{src/gmxlib/nonbonded/nb\-\_\-kernel\-\_\-avx\-\_\-256\-\_\-single/\hyperlink{nb__kernel__ElecEwSh__VdwNone__GeomW3W3__avx__256__single_8c}{nb\-\_\-kernel\-\_\-\-Elec\-Ew\-Sh\-\_\-\-Vdw\-None\-\_\-\-Geom\-W3\-W3\-\_\-avx\-\_\-256\-\_\-single.\-c} }{\pageref{nb__kernel__ElecEwSh__VdwNone__GeomW3W3__avx__256__single_8c}}{}
\item\contentsline{section}{src/gmxlib/nonbonded/nb\-\_\-kernel\-\_\-avx\-\_\-256\-\_\-single/\hyperlink{nb__kernel__ElecEwSh__VdwNone__GeomW4P1__avx__256__single_8c}{nb\-\_\-kernel\-\_\-\-Elec\-Ew\-Sh\-\_\-\-Vdw\-None\-\_\-\-Geom\-W4\-P1\-\_\-avx\-\_\-256\-\_\-single.\-c} }{\pageref{nb__kernel__ElecEwSh__VdwNone__GeomW4P1__avx__256__single_8c}}{}
\item\contentsline{section}{src/gmxlib/nonbonded/nb\-\_\-kernel\-\_\-avx\-\_\-256\-\_\-single/\hyperlink{nb__kernel__ElecEwSh__VdwNone__GeomW4W4__avx__256__single_8c}{nb\-\_\-kernel\-\_\-\-Elec\-Ew\-Sh\-\_\-\-Vdw\-None\-\_\-\-Geom\-W4\-W4\-\_\-avx\-\_\-256\-\_\-single.\-c} }{\pageref{nb__kernel__ElecEwSh__VdwNone__GeomW4W4__avx__256__single_8c}}{}
\item\contentsline{section}{src/gmxlib/nonbonded/nb\-\_\-kernel\-\_\-avx\-\_\-256\-\_\-single/\hyperlink{nb__kernel__ElecEwSw__VdwLJSw__GeomP1P1__avx__256__single_8c}{nb\-\_\-kernel\-\_\-\-Elec\-Ew\-Sw\-\_\-\-Vdw\-L\-J\-Sw\-\_\-\-Geom\-P1\-P1\-\_\-avx\-\_\-256\-\_\-single.\-c} }{\pageref{nb__kernel__ElecEwSw__VdwLJSw__GeomP1P1__avx__256__single_8c}}{}
\item\contentsline{section}{src/gmxlib/nonbonded/nb\-\_\-kernel\-\_\-avx\-\_\-256\-\_\-single/\hyperlink{nb__kernel__ElecEwSw__VdwLJSw__GeomW3P1__avx__256__single_8c}{nb\-\_\-kernel\-\_\-\-Elec\-Ew\-Sw\-\_\-\-Vdw\-L\-J\-Sw\-\_\-\-Geom\-W3\-P1\-\_\-avx\-\_\-256\-\_\-single.\-c} }{\pageref{nb__kernel__ElecEwSw__VdwLJSw__GeomW3P1__avx__256__single_8c}}{}
\item\contentsline{section}{src/gmxlib/nonbonded/nb\-\_\-kernel\-\_\-avx\-\_\-256\-\_\-single/\hyperlink{nb__kernel__ElecEwSw__VdwLJSw__GeomW3W3__avx__256__single_8c}{nb\-\_\-kernel\-\_\-\-Elec\-Ew\-Sw\-\_\-\-Vdw\-L\-J\-Sw\-\_\-\-Geom\-W3\-W3\-\_\-avx\-\_\-256\-\_\-single.\-c} }{\pageref{nb__kernel__ElecEwSw__VdwLJSw__GeomW3W3__avx__256__single_8c}}{}
\item\contentsline{section}{src/gmxlib/nonbonded/nb\-\_\-kernel\-\_\-avx\-\_\-256\-\_\-single/\hyperlink{nb__kernel__ElecEwSw__VdwLJSw__GeomW4P1__avx__256__single_8c}{nb\-\_\-kernel\-\_\-\-Elec\-Ew\-Sw\-\_\-\-Vdw\-L\-J\-Sw\-\_\-\-Geom\-W4\-P1\-\_\-avx\-\_\-256\-\_\-single.\-c} }{\pageref{nb__kernel__ElecEwSw__VdwLJSw__GeomW4P1__avx__256__single_8c}}{}
\item\contentsline{section}{src/gmxlib/nonbonded/nb\-\_\-kernel\-\_\-avx\-\_\-256\-\_\-single/\hyperlink{nb__kernel__ElecEwSw__VdwLJSw__GeomW4W4__avx__256__single_8c}{nb\-\_\-kernel\-\_\-\-Elec\-Ew\-Sw\-\_\-\-Vdw\-L\-J\-Sw\-\_\-\-Geom\-W4\-W4\-\_\-avx\-\_\-256\-\_\-single.\-c} }{\pageref{nb__kernel__ElecEwSw__VdwLJSw__GeomW4W4__avx__256__single_8c}}{}
\item\contentsline{section}{src/gmxlib/nonbonded/nb\-\_\-kernel\-\_\-avx\-\_\-256\-\_\-single/\hyperlink{nb__kernel__ElecEwSw__VdwNone__GeomP1P1__avx__256__single_8c}{nb\-\_\-kernel\-\_\-\-Elec\-Ew\-Sw\-\_\-\-Vdw\-None\-\_\-\-Geom\-P1\-P1\-\_\-avx\-\_\-256\-\_\-single.\-c} }{\pageref{nb__kernel__ElecEwSw__VdwNone__GeomP1P1__avx__256__single_8c}}{}
\item\contentsline{section}{src/gmxlib/nonbonded/nb\-\_\-kernel\-\_\-avx\-\_\-256\-\_\-single/\hyperlink{nb__kernel__ElecEwSw__VdwNone__GeomW3P1__avx__256__single_8c}{nb\-\_\-kernel\-\_\-\-Elec\-Ew\-Sw\-\_\-\-Vdw\-None\-\_\-\-Geom\-W3\-P1\-\_\-avx\-\_\-256\-\_\-single.\-c} }{\pageref{nb__kernel__ElecEwSw__VdwNone__GeomW3P1__avx__256__single_8c}}{}
\item\contentsline{section}{src/gmxlib/nonbonded/nb\-\_\-kernel\-\_\-avx\-\_\-256\-\_\-single/\hyperlink{nb__kernel__ElecEwSw__VdwNone__GeomW3W3__avx__256__single_8c}{nb\-\_\-kernel\-\_\-\-Elec\-Ew\-Sw\-\_\-\-Vdw\-None\-\_\-\-Geom\-W3\-W3\-\_\-avx\-\_\-256\-\_\-single.\-c} }{\pageref{nb__kernel__ElecEwSw__VdwNone__GeomW3W3__avx__256__single_8c}}{}
\item\contentsline{section}{src/gmxlib/nonbonded/nb\-\_\-kernel\-\_\-avx\-\_\-256\-\_\-single/\hyperlink{nb__kernel__ElecEwSw__VdwNone__GeomW4P1__avx__256__single_8c}{nb\-\_\-kernel\-\_\-\-Elec\-Ew\-Sw\-\_\-\-Vdw\-None\-\_\-\-Geom\-W4\-P1\-\_\-avx\-\_\-256\-\_\-single.\-c} }{\pageref{nb__kernel__ElecEwSw__VdwNone__GeomW4P1__avx__256__single_8c}}{}
\item\contentsline{section}{src/gmxlib/nonbonded/nb\-\_\-kernel\-\_\-avx\-\_\-256\-\_\-single/\hyperlink{nb__kernel__ElecEwSw__VdwNone__GeomW4W4__avx__256__single_8c}{nb\-\_\-kernel\-\_\-\-Elec\-Ew\-Sw\-\_\-\-Vdw\-None\-\_\-\-Geom\-W4\-W4\-\_\-avx\-\_\-256\-\_\-single.\-c} }{\pageref{nb__kernel__ElecEwSw__VdwNone__GeomW4W4__avx__256__single_8c}}{}
\item\contentsline{section}{src/gmxlib/nonbonded/nb\-\_\-kernel\-\_\-avx\-\_\-256\-\_\-single/\hyperlink{nb__kernel__ElecGB__VdwCSTab__GeomP1P1__avx__256__single_8c}{nb\-\_\-kernel\-\_\-\-Elec\-G\-B\-\_\-\-Vdw\-C\-S\-Tab\-\_\-\-Geom\-P1\-P1\-\_\-avx\-\_\-256\-\_\-single.\-c} }{\pageref{nb__kernel__ElecGB__VdwCSTab__GeomP1P1__avx__256__single_8c}}{}
\item\contentsline{section}{src/gmxlib/nonbonded/nb\-\_\-kernel\-\_\-avx\-\_\-256\-\_\-single/\hyperlink{nb__kernel__ElecGB__VdwLJ__GeomP1P1__avx__256__single_8c}{nb\-\_\-kernel\-\_\-\-Elec\-G\-B\-\_\-\-Vdw\-L\-J\-\_\-\-Geom\-P1\-P1\-\_\-avx\-\_\-256\-\_\-single.\-c} }{\pageref{nb__kernel__ElecGB__VdwLJ__GeomP1P1__avx__256__single_8c}}{}
\item\contentsline{section}{src/gmxlib/nonbonded/nb\-\_\-kernel\-\_\-avx\-\_\-256\-\_\-single/\hyperlink{nb__kernel__ElecGB__VdwNone__GeomP1P1__avx__256__single_8c}{nb\-\_\-kernel\-\_\-\-Elec\-G\-B\-\_\-\-Vdw\-None\-\_\-\-Geom\-P1\-P1\-\_\-avx\-\_\-256\-\_\-single.\-c} }{\pageref{nb__kernel__ElecGB__VdwNone__GeomP1P1__avx__256__single_8c}}{}
\item\contentsline{section}{src/gmxlib/nonbonded/nb\-\_\-kernel\-\_\-avx\-\_\-256\-\_\-single/\hyperlink{nb__kernel__ElecNone__VdwCSTab__GeomP1P1__avx__256__single_8c}{nb\-\_\-kernel\-\_\-\-Elec\-None\-\_\-\-Vdw\-C\-S\-Tab\-\_\-\-Geom\-P1\-P1\-\_\-avx\-\_\-256\-\_\-single.\-c} }{\pageref{nb__kernel__ElecNone__VdwCSTab__GeomP1P1__avx__256__single_8c}}{}
\item\contentsline{section}{src/gmxlib/nonbonded/nb\-\_\-kernel\-\_\-avx\-\_\-256\-\_\-single/\hyperlink{nb__kernel__ElecNone__VdwLJ__GeomP1P1__avx__256__single_8c}{nb\-\_\-kernel\-\_\-\-Elec\-None\-\_\-\-Vdw\-L\-J\-\_\-\-Geom\-P1\-P1\-\_\-avx\-\_\-256\-\_\-single.\-c} }{\pageref{nb__kernel__ElecNone__VdwLJ__GeomP1P1__avx__256__single_8c}}{}
\item\contentsline{section}{src/gmxlib/nonbonded/nb\-\_\-kernel\-\_\-avx\-\_\-256\-\_\-single/\hyperlink{nb__kernel__ElecNone__VdwLJSh__GeomP1P1__avx__256__single_8c}{nb\-\_\-kernel\-\_\-\-Elec\-None\-\_\-\-Vdw\-L\-J\-Sh\-\_\-\-Geom\-P1\-P1\-\_\-avx\-\_\-256\-\_\-single.\-c} }{\pageref{nb__kernel__ElecNone__VdwLJSh__GeomP1P1__avx__256__single_8c}}{}
\item\contentsline{section}{src/gmxlib/nonbonded/nb\-\_\-kernel\-\_\-avx\-\_\-256\-\_\-single/\hyperlink{nb__kernel__ElecNone__VdwLJSw__GeomP1P1__avx__256__single_8c}{nb\-\_\-kernel\-\_\-\-Elec\-None\-\_\-\-Vdw\-L\-J\-Sw\-\_\-\-Geom\-P1\-P1\-\_\-avx\-\_\-256\-\_\-single.\-c} }{\pageref{nb__kernel__ElecNone__VdwLJSw__GeomP1P1__avx__256__single_8c}}{}
\item\contentsline{section}{src/gmxlib/nonbonded/nb\-\_\-kernel\-\_\-avx\-\_\-256\-\_\-single/\hyperlink{nb__kernel__ElecRF__VdwCSTab__GeomP1P1__avx__256__single_8c}{nb\-\_\-kernel\-\_\-\-Elec\-R\-F\-\_\-\-Vdw\-C\-S\-Tab\-\_\-\-Geom\-P1\-P1\-\_\-avx\-\_\-256\-\_\-single.\-c} }{\pageref{nb__kernel__ElecRF__VdwCSTab__GeomP1P1__avx__256__single_8c}}{}
\item\contentsline{section}{src/gmxlib/nonbonded/nb\-\_\-kernel\-\_\-avx\-\_\-256\-\_\-single/\hyperlink{nb__kernel__ElecRF__VdwCSTab__GeomW3P1__avx__256__single_8c}{nb\-\_\-kernel\-\_\-\-Elec\-R\-F\-\_\-\-Vdw\-C\-S\-Tab\-\_\-\-Geom\-W3\-P1\-\_\-avx\-\_\-256\-\_\-single.\-c} }{\pageref{nb__kernel__ElecRF__VdwCSTab__GeomW3P1__avx__256__single_8c}}{}
\item\contentsline{section}{src/gmxlib/nonbonded/nb\-\_\-kernel\-\_\-avx\-\_\-256\-\_\-single/\hyperlink{nb__kernel__ElecRF__VdwCSTab__GeomW3W3__avx__256__single_8c}{nb\-\_\-kernel\-\_\-\-Elec\-R\-F\-\_\-\-Vdw\-C\-S\-Tab\-\_\-\-Geom\-W3\-W3\-\_\-avx\-\_\-256\-\_\-single.\-c} }{\pageref{nb__kernel__ElecRF__VdwCSTab__GeomW3W3__avx__256__single_8c}}{}
\item\contentsline{section}{src/gmxlib/nonbonded/nb\-\_\-kernel\-\_\-avx\-\_\-256\-\_\-single/\hyperlink{nb__kernel__ElecRF__VdwCSTab__GeomW4P1__avx__256__single_8c}{nb\-\_\-kernel\-\_\-\-Elec\-R\-F\-\_\-\-Vdw\-C\-S\-Tab\-\_\-\-Geom\-W4\-P1\-\_\-avx\-\_\-256\-\_\-single.\-c} }{\pageref{nb__kernel__ElecRF__VdwCSTab__GeomW4P1__avx__256__single_8c}}{}
\item\contentsline{section}{src/gmxlib/nonbonded/nb\-\_\-kernel\-\_\-avx\-\_\-256\-\_\-single/\hyperlink{nb__kernel__ElecRF__VdwCSTab__GeomW4W4__avx__256__single_8c}{nb\-\_\-kernel\-\_\-\-Elec\-R\-F\-\_\-\-Vdw\-C\-S\-Tab\-\_\-\-Geom\-W4\-W4\-\_\-avx\-\_\-256\-\_\-single.\-c} }{\pageref{nb__kernel__ElecRF__VdwCSTab__GeomW4W4__avx__256__single_8c}}{}
\item\contentsline{section}{src/gmxlib/nonbonded/nb\-\_\-kernel\-\_\-avx\-\_\-256\-\_\-single/\hyperlink{nb__kernel__ElecRF__VdwLJ__GeomP1P1__avx__256__single_8c}{nb\-\_\-kernel\-\_\-\-Elec\-R\-F\-\_\-\-Vdw\-L\-J\-\_\-\-Geom\-P1\-P1\-\_\-avx\-\_\-256\-\_\-single.\-c} }{\pageref{nb__kernel__ElecRF__VdwLJ__GeomP1P1__avx__256__single_8c}}{}
\item\contentsline{section}{src/gmxlib/nonbonded/nb\-\_\-kernel\-\_\-avx\-\_\-256\-\_\-single/\hyperlink{nb__kernel__ElecRF__VdwLJ__GeomW3P1__avx__256__single_8c}{nb\-\_\-kernel\-\_\-\-Elec\-R\-F\-\_\-\-Vdw\-L\-J\-\_\-\-Geom\-W3\-P1\-\_\-avx\-\_\-256\-\_\-single.\-c} }{\pageref{nb__kernel__ElecRF__VdwLJ__GeomW3P1__avx__256__single_8c}}{}
\item\contentsline{section}{src/gmxlib/nonbonded/nb\-\_\-kernel\-\_\-avx\-\_\-256\-\_\-single/\hyperlink{nb__kernel__ElecRF__VdwLJ__GeomW3W3__avx__256__single_8c}{nb\-\_\-kernel\-\_\-\-Elec\-R\-F\-\_\-\-Vdw\-L\-J\-\_\-\-Geom\-W3\-W3\-\_\-avx\-\_\-256\-\_\-single.\-c} }{\pageref{nb__kernel__ElecRF__VdwLJ__GeomW3W3__avx__256__single_8c}}{}
\item\contentsline{section}{src/gmxlib/nonbonded/nb\-\_\-kernel\-\_\-avx\-\_\-256\-\_\-single/\hyperlink{nb__kernel__ElecRF__VdwLJ__GeomW4P1__avx__256__single_8c}{nb\-\_\-kernel\-\_\-\-Elec\-R\-F\-\_\-\-Vdw\-L\-J\-\_\-\-Geom\-W4\-P1\-\_\-avx\-\_\-256\-\_\-single.\-c} }{\pageref{nb__kernel__ElecRF__VdwLJ__GeomW4P1__avx__256__single_8c}}{}
\item\contentsline{section}{src/gmxlib/nonbonded/nb\-\_\-kernel\-\_\-avx\-\_\-256\-\_\-single/\hyperlink{nb__kernel__ElecRF__VdwLJ__GeomW4W4__avx__256__single_8c}{nb\-\_\-kernel\-\_\-\-Elec\-R\-F\-\_\-\-Vdw\-L\-J\-\_\-\-Geom\-W4\-W4\-\_\-avx\-\_\-256\-\_\-single.\-c} }{\pageref{nb__kernel__ElecRF__VdwLJ__GeomW4W4__avx__256__single_8c}}{}
\item\contentsline{section}{src/gmxlib/nonbonded/nb\-\_\-kernel\-\_\-avx\-\_\-256\-\_\-single/\hyperlink{nb__kernel__ElecRF__VdwNone__GeomP1P1__avx__256__single_8c}{nb\-\_\-kernel\-\_\-\-Elec\-R\-F\-\_\-\-Vdw\-None\-\_\-\-Geom\-P1\-P1\-\_\-avx\-\_\-256\-\_\-single.\-c} }{\pageref{nb__kernel__ElecRF__VdwNone__GeomP1P1__avx__256__single_8c}}{}
\item\contentsline{section}{src/gmxlib/nonbonded/nb\-\_\-kernel\-\_\-avx\-\_\-256\-\_\-single/\hyperlink{nb__kernel__ElecRF__VdwNone__GeomW3P1__avx__256__single_8c}{nb\-\_\-kernel\-\_\-\-Elec\-R\-F\-\_\-\-Vdw\-None\-\_\-\-Geom\-W3\-P1\-\_\-avx\-\_\-256\-\_\-single.\-c} }{\pageref{nb__kernel__ElecRF__VdwNone__GeomW3P1__avx__256__single_8c}}{}
\item\contentsline{section}{src/gmxlib/nonbonded/nb\-\_\-kernel\-\_\-avx\-\_\-256\-\_\-single/\hyperlink{nb__kernel__ElecRF__VdwNone__GeomW3W3__avx__256__single_8c}{nb\-\_\-kernel\-\_\-\-Elec\-R\-F\-\_\-\-Vdw\-None\-\_\-\-Geom\-W3\-W3\-\_\-avx\-\_\-256\-\_\-single.\-c} }{\pageref{nb__kernel__ElecRF__VdwNone__GeomW3W3__avx__256__single_8c}}{}
\item\contentsline{section}{src/gmxlib/nonbonded/nb\-\_\-kernel\-\_\-avx\-\_\-256\-\_\-single/\hyperlink{nb__kernel__ElecRF__VdwNone__GeomW4P1__avx__256__single_8c}{nb\-\_\-kernel\-\_\-\-Elec\-R\-F\-\_\-\-Vdw\-None\-\_\-\-Geom\-W4\-P1\-\_\-avx\-\_\-256\-\_\-single.\-c} }{\pageref{nb__kernel__ElecRF__VdwNone__GeomW4P1__avx__256__single_8c}}{}
\item\contentsline{section}{src/gmxlib/nonbonded/nb\-\_\-kernel\-\_\-avx\-\_\-256\-\_\-single/\hyperlink{nb__kernel__ElecRF__VdwNone__GeomW4W4__avx__256__single_8c}{nb\-\_\-kernel\-\_\-\-Elec\-R\-F\-\_\-\-Vdw\-None\-\_\-\-Geom\-W4\-W4\-\_\-avx\-\_\-256\-\_\-single.\-c} }{\pageref{nb__kernel__ElecRF__VdwNone__GeomW4W4__avx__256__single_8c}}{}
\item\contentsline{section}{src/gmxlib/nonbonded/nb\-\_\-kernel\-\_\-avx\-\_\-256\-\_\-single/\hyperlink{nb__kernel__ElecRFCut__VdwCSTab__GeomP1P1__avx__256__single_8c}{nb\-\_\-kernel\-\_\-\-Elec\-R\-F\-Cut\-\_\-\-Vdw\-C\-S\-Tab\-\_\-\-Geom\-P1\-P1\-\_\-avx\-\_\-256\-\_\-single.\-c} }{\pageref{nb__kernel__ElecRFCut__VdwCSTab__GeomP1P1__avx__256__single_8c}}{}
\item\contentsline{section}{src/gmxlib/nonbonded/nb\-\_\-kernel\-\_\-avx\-\_\-256\-\_\-single/\hyperlink{nb__kernel__ElecRFCut__VdwCSTab__GeomW3P1__avx__256__single_8c}{nb\-\_\-kernel\-\_\-\-Elec\-R\-F\-Cut\-\_\-\-Vdw\-C\-S\-Tab\-\_\-\-Geom\-W3\-P1\-\_\-avx\-\_\-256\-\_\-single.\-c} }{\pageref{nb__kernel__ElecRFCut__VdwCSTab__GeomW3P1__avx__256__single_8c}}{}
\item\contentsline{section}{src/gmxlib/nonbonded/nb\-\_\-kernel\-\_\-avx\-\_\-256\-\_\-single/\hyperlink{nb__kernel__ElecRFCut__VdwCSTab__GeomW3W3__avx__256__single_8c}{nb\-\_\-kernel\-\_\-\-Elec\-R\-F\-Cut\-\_\-\-Vdw\-C\-S\-Tab\-\_\-\-Geom\-W3\-W3\-\_\-avx\-\_\-256\-\_\-single.\-c} }{\pageref{nb__kernel__ElecRFCut__VdwCSTab__GeomW3W3__avx__256__single_8c}}{}
\item\contentsline{section}{src/gmxlib/nonbonded/nb\-\_\-kernel\-\_\-avx\-\_\-256\-\_\-single/\hyperlink{nb__kernel__ElecRFCut__VdwCSTab__GeomW4P1__avx__256__single_8c}{nb\-\_\-kernel\-\_\-\-Elec\-R\-F\-Cut\-\_\-\-Vdw\-C\-S\-Tab\-\_\-\-Geom\-W4\-P1\-\_\-avx\-\_\-256\-\_\-single.\-c} }{\pageref{nb__kernel__ElecRFCut__VdwCSTab__GeomW4P1__avx__256__single_8c}}{}
\item\contentsline{section}{src/gmxlib/nonbonded/nb\-\_\-kernel\-\_\-avx\-\_\-256\-\_\-single/\hyperlink{nb__kernel__ElecRFCut__VdwCSTab__GeomW4W4__avx__256__single_8c}{nb\-\_\-kernel\-\_\-\-Elec\-R\-F\-Cut\-\_\-\-Vdw\-C\-S\-Tab\-\_\-\-Geom\-W4\-W4\-\_\-avx\-\_\-256\-\_\-single.\-c} }{\pageref{nb__kernel__ElecRFCut__VdwCSTab__GeomW4W4__avx__256__single_8c}}{}
\item\contentsline{section}{src/gmxlib/nonbonded/nb\-\_\-kernel\-\_\-avx\-\_\-256\-\_\-single/\hyperlink{nb__kernel__ElecRFCut__VdwLJSh__GeomP1P1__avx__256__single_8c}{nb\-\_\-kernel\-\_\-\-Elec\-R\-F\-Cut\-\_\-\-Vdw\-L\-J\-Sh\-\_\-\-Geom\-P1\-P1\-\_\-avx\-\_\-256\-\_\-single.\-c} }{\pageref{nb__kernel__ElecRFCut__VdwLJSh__GeomP1P1__avx__256__single_8c}}{}
\item\contentsline{section}{src/gmxlib/nonbonded/nb\-\_\-kernel\-\_\-avx\-\_\-256\-\_\-single/\hyperlink{nb__kernel__ElecRFCut__VdwLJSh__GeomW3P1__avx__256__single_8c}{nb\-\_\-kernel\-\_\-\-Elec\-R\-F\-Cut\-\_\-\-Vdw\-L\-J\-Sh\-\_\-\-Geom\-W3\-P1\-\_\-avx\-\_\-256\-\_\-single.\-c} }{\pageref{nb__kernel__ElecRFCut__VdwLJSh__GeomW3P1__avx__256__single_8c}}{}
\item\contentsline{section}{src/gmxlib/nonbonded/nb\-\_\-kernel\-\_\-avx\-\_\-256\-\_\-single/\hyperlink{nb__kernel__ElecRFCut__VdwLJSh__GeomW3W3__avx__256__single_8c}{nb\-\_\-kernel\-\_\-\-Elec\-R\-F\-Cut\-\_\-\-Vdw\-L\-J\-Sh\-\_\-\-Geom\-W3\-W3\-\_\-avx\-\_\-256\-\_\-single.\-c} }{\pageref{nb__kernel__ElecRFCut__VdwLJSh__GeomW3W3__avx__256__single_8c}}{}
\item\contentsline{section}{src/gmxlib/nonbonded/nb\-\_\-kernel\-\_\-avx\-\_\-256\-\_\-single/\hyperlink{nb__kernel__ElecRFCut__VdwLJSh__GeomW4P1__avx__256__single_8c}{nb\-\_\-kernel\-\_\-\-Elec\-R\-F\-Cut\-\_\-\-Vdw\-L\-J\-Sh\-\_\-\-Geom\-W4\-P1\-\_\-avx\-\_\-256\-\_\-single.\-c} }{\pageref{nb__kernel__ElecRFCut__VdwLJSh__GeomW4P1__avx__256__single_8c}}{}
\item\contentsline{section}{src/gmxlib/nonbonded/nb\-\_\-kernel\-\_\-avx\-\_\-256\-\_\-single/\hyperlink{nb__kernel__ElecRFCut__VdwLJSh__GeomW4W4__avx__256__single_8c}{nb\-\_\-kernel\-\_\-\-Elec\-R\-F\-Cut\-\_\-\-Vdw\-L\-J\-Sh\-\_\-\-Geom\-W4\-W4\-\_\-avx\-\_\-256\-\_\-single.\-c} }{\pageref{nb__kernel__ElecRFCut__VdwLJSh__GeomW4W4__avx__256__single_8c}}{}
\item\contentsline{section}{src/gmxlib/nonbonded/nb\-\_\-kernel\-\_\-avx\-\_\-256\-\_\-single/\hyperlink{nb__kernel__ElecRFCut__VdwLJSw__GeomP1P1__avx__256__single_8c}{nb\-\_\-kernel\-\_\-\-Elec\-R\-F\-Cut\-\_\-\-Vdw\-L\-J\-Sw\-\_\-\-Geom\-P1\-P1\-\_\-avx\-\_\-256\-\_\-single.\-c} }{\pageref{nb__kernel__ElecRFCut__VdwLJSw__GeomP1P1__avx__256__single_8c}}{}
\item\contentsline{section}{src/gmxlib/nonbonded/nb\-\_\-kernel\-\_\-avx\-\_\-256\-\_\-single/\hyperlink{nb__kernel__ElecRFCut__VdwLJSw__GeomW3P1__avx__256__single_8c}{nb\-\_\-kernel\-\_\-\-Elec\-R\-F\-Cut\-\_\-\-Vdw\-L\-J\-Sw\-\_\-\-Geom\-W3\-P1\-\_\-avx\-\_\-256\-\_\-single.\-c} }{\pageref{nb__kernel__ElecRFCut__VdwLJSw__GeomW3P1__avx__256__single_8c}}{}
\item\contentsline{section}{src/gmxlib/nonbonded/nb\-\_\-kernel\-\_\-avx\-\_\-256\-\_\-single/\hyperlink{nb__kernel__ElecRFCut__VdwLJSw__GeomW3W3__avx__256__single_8c}{nb\-\_\-kernel\-\_\-\-Elec\-R\-F\-Cut\-\_\-\-Vdw\-L\-J\-Sw\-\_\-\-Geom\-W3\-W3\-\_\-avx\-\_\-256\-\_\-single.\-c} }{\pageref{nb__kernel__ElecRFCut__VdwLJSw__GeomW3W3__avx__256__single_8c}}{}
\item\contentsline{section}{src/gmxlib/nonbonded/nb\-\_\-kernel\-\_\-avx\-\_\-256\-\_\-single/\hyperlink{nb__kernel__ElecRFCut__VdwLJSw__GeomW4P1__avx__256__single_8c}{nb\-\_\-kernel\-\_\-\-Elec\-R\-F\-Cut\-\_\-\-Vdw\-L\-J\-Sw\-\_\-\-Geom\-W4\-P1\-\_\-avx\-\_\-256\-\_\-single.\-c} }{\pageref{nb__kernel__ElecRFCut__VdwLJSw__GeomW4P1__avx__256__single_8c}}{}
\item\contentsline{section}{src/gmxlib/nonbonded/nb\-\_\-kernel\-\_\-avx\-\_\-256\-\_\-single/\hyperlink{nb__kernel__ElecRFCut__VdwLJSw__GeomW4W4__avx__256__single_8c}{nb\-\_\-kernel\-\_\-\-Elec\-R\-F\-Cut\-\_\-\-Vdw\-L\-J\-Sw\-\_\-\-Geom\-W4\-W4\-\_\-avx\-\_\-256\-\_\-single.\-c} }{\pageref{nb__kernel__ElecRFCut__VdwLJSw__GeomW4W4__avx__256__single_8c}}{}
\item\contentsline{section}{src/gmxlib/nonbonded/nb\-\_\-kernel\-\_\-avx\-\_\-256\-\_\-single/\hyperlink{nb__kernel__ElecRFCut__VdwNone__GeomP1P1__avx__256__single_8c}{nb\-\_\-kernel\-\_\-\-Elec\-R\-F\-Cut\-\_\-\-Vdw\-None\-\_\-\-Geom\-P1\-P1\-\_\-avx\-\_\-256\-\_\-single.\-c} }{\pageref{nb__kernel__ElecRFCut__VdwNone__GeomP1P1__avx__256__single_8c}}{}
\item\contentsline{section}{src/gmxlib/nonbonded/nb\-\_\-kernel\-\_\-avx\-\_\-256\-\_\-single/\hyperlink{nb__kernel__ElecRFCut__VdwNone__GeomW3P1__avx__256__single_8c}{nb\-\_\-kernel\-\_\-\-Elec\-R\-F\-Cut\-\_\-\-Vdw\-None\-\_\-\-Geom\-W3\-P1\-\_\-avx\-\_\-256\-\_\-single.\-c} }{\pageref{nb__kernel__ElecRFCut__VdwNone__GeomW3P1__avx__256__single_8c}}{}
\item\contentsline{section}{src/gmxlib/nonbonded/nb\-\_\-kernel\-\_\-avx\-\_\-256\-\_\-single/\hyperlink{nb__kernel__ElecRFCut__VdwNone__GeomW3W3__avx__256__single_8c}{nb\-\_\-kernel\-\_\-\-Elec\-R\-F\-Cut\-\_\-\-Vdw\-None\-\_\-\-Geom\-W3\-W3\-\_\-avx\-\_\-256\-\_\-single.\-c} }{\pageref{nb__kernel__ElecRFCut__VdwNone__GeomW3W3__avx__256__single_8c}}{}
\item\contentsline{section}{src/gmxlib/nonbonded/nb\-\_\-kernel\-\_\-avx\-\_\-256\-\_\-single/\hyperlink{nb__kernel__ElecRFCut__VdwNone__GeomW4P1__avx__256__single_8c}{nb\-\_\-kernel\-\_\-\-Elec\-R\-F\-Cut\-\_\-\-Vdw\-None\-\_\-\-Geom\-W4\-P1\-\_\-avx\-\_\-256\-\_\-single.\-c} }{\pageref{nb__kernel__ElecRFCut__VdwNone__GeomW4P1__avx__256__single_8c}}{}
\item\contentsline{section}{src/gmxlib/nonbonded/nb\-\_\-kernel\-\_\-avx\-\_\-256\-\_\-single/\hyperlink{nb__kernel__ElecRFCut__VdwNone__GeomW4W4__avx__256__single_8c}{nb\-\_\-kernel\-\_\-\-Elec\-R\-F\-Cut\-\_\-\-Vdw\-None\-\_\-\-Geom\-W4\-W4\-\_\-avx\-\_\-256\-\_\-single.\-c} }{\pageref{nb__kernel__ElecRFCut__VdwNone__GeomW4W4__avx__256__single_8c}}{}
\item\contentsline{section}{src/gmxlib/nonbonded/nb\-\_\-kernel\-\_\-c/\hyperlink{nb__kernel__allvsall_8c}{nb\-\_\-kernel\-\_\-allvsall.\-c} }{\pageref{nb__kernel__allvsall_8c}}{}
\item\contentsline{section}{src/gmxlib/nonbonded/nb\-\_\-kernel\-\_\-c/\hyperlink{nb__kernel__allvsall_8h}{nb\-\_\-kernel\-\_\-allvsall.\-h} }{\pageref{nb__kernel__allvsall_8h}}{}
\item\contentsline{section}{src/gmxlib/nonbonded/nb\-\_\-kernel\-\_\-c/\hyperlink{nb__kernel__allvsallgb_8c}{nb\-\_\-kernel\-\_\-allvsallgb.\-c} }{\pageref{nb__kernel__allvsallgb_8c}}{}
\item\contentsline{section}{src/gmxlib/nonbonded/nb\-\_\-kernel\-\_\-c/\hyperlink{nb__kernel__allvsallgb_8h}{nb\-\_\-kernel\-\_\-allvsallgb.\-h} }{\pageref{nb__kernel__allvsallgb_8h}}{}
\item\contentsline{section}{src/gmxlib/nonbonded/nb\-\_\-kernel\-\_\-c/\hyperlink{nb__kernel__c_8c}{nb\-\_\-kernel\-\_\-c.\-c} }{\pageref{nb__kernel__c_8c}}{}
\item\contentsline{section}{src/gmxlib/nonbonded/nb\-\_\-kernel\-\_\-c/\hyperlink{nb__kernel__c_8h}{nb\-\_\-kernel\-\_\-c.\-h} }{\pageref{nb__kernel__c_8h}}{}
\item\contentsline{section}{src/gmxlib/nonbonded/nb\-\_\-kernel\-\_\-c/\hyperlink{nb__kernel__ElecCoul__VdwBham__GeomP1P1__c_8c}{nb\-\_\-kernel\-\_\-\-Elec\-Coul\-\_\-\-Vdw\-Bham\-\_\-\-Geom\-P1\-P1\-\_\-c.\-c} }{\pageref{nb__kernel__ElecCoul__VdwBham__GeomP1P1__c_8c}}{}
\item\contentsline{section}{src/gmxlib/nonbonded/nb\-\_\-kernel\-\_\-c/\hyperlink{nb__kernel__ElecCoul__VdwBham__GeomW3P1__c_8c}{nb\-\_\-kernel\-\_\-\-Elec\-Coul\-\_\-\-Vdw\-Bham\-\_\-\-Geom\-W3\-P1\-\_\-c.\-c} }{\pageref{nb__kernel__ElecCoul__VdwBham__GeomW3P1__c_8c}}{}
\item\contentsline{section}{src/gmxlib/nonbonded/nb\-\_\-kernel\-\_\-c/\hyperlink{nb__kernel__ElecCoul__VdwBham__GeomW3W3__c_8c}{nb\-\_\-kernel\-\_\-\-Elec\-Coul\-\_\-\-Vdw\-Bham\-\_\-\-Geom\-W3\-W3\-\_\-c.\-c} }{\pageref{nb__kernel__ElecCoul__VdwBham__GeomW3W3__c_8c}}{}
\item\contentsline{section}{src/gmxlib/nonbonded/nb\-\_\-kernel\-\_\-c/\hyperlink{nb__kernel__ElecCoul__VdwBham__GeomW4P1__c_8c}{nb\-\_\-kernel\-\_\-\-Elec\-Coul\-\_\-\-Vdw\-Bham\-\_\-\-Geom\-W4\-P1\-\_\-c.\-c} }{\pageref{nb__kernel__ElecCoul__VdwBham__GeomW4P1__c_8c}}{}
\item\contentsline{section}{src/gmxlib/nonbonded/nb\-\_\-kernel\-\_\-c/\hyperlink{nb__kernel__ElecCoul__VdwBham__GeomW4W4__c_8c}{nb\-\_\-kernel\-\_\-\-Elec\-Coul\-\_\-\-Vdw\-Bham\-\_\-\-Geom\-W4\-W4\-\_\-c.\-c} }{\pageref{nb__kernel__ElecCoul__VdwBham__GeomW4W4__c_8c}}{}
\item\contentsline{section}{src/gmxlib/nonbonded/nb\-\_\-kernel\-\_\-c/\hyperlink{nb__kernel__ElecCoul__VdwCSTab__GeomP1P1__c_8c}{nb\-\_\-kernel\-\_\-\-Elec\-Coul\-\_\-\-Vdw\-C\-S\-Tab\-\_\-\-Geom\-P1\-P1\-\_\-c.\-c} }{\pageref{nb__kernel__ElecCoul__VdwCSTab__GeomP1P1__c_8c}}{}
\item\contentsline{section}{src/gmxlib/nonbonded/nb\-\_\-kernel\-\_\-c/\hyperlink{nb__kernel__ElecCoul__VdwCSTab__GeomW3P1__c_8c}{nb\-\_\-kernel\-\_\-\-Elec\-Coul\-\_\-\-Vdw\-C\-S\-Tab\-\_\-\-Geom\-W3\-P1\-\_\-c.\-c} }{\pageref{nb__kernel__ElecCoul__VdwCSTab__GeomW3P1__c_8c}}{}
\item\contentsline{section}{src/gmxlib/nonbonded/nb\-\_\-kernel\-\_\-c/\hyperlink{nb__kernel__ElecCoul__VdwCSTab__GeomW3W3__c_8c}{nb\-\_\-kernel\-\_\-\-Elec\-Coul\-\_\-\-Vdw\-C\-S\-Tab\-\_\-\-Geom\-W3\-W3\-\_\-c.\-c} }{\pageref{nb__kernel__ElecCoul__VdwCSTab__GeomW3W3__c_8c}}{}
\item\contentsline{section}{src/gmxlib/nonbonded/nb\-\_\-kernel\-\_\-c/\hyperlink{nb__kernel__ElecCoul__VdwCSTab__GeomW4P1__c_8c}{nb\-\_\-kernel\-\_\-\-Elec\-Coul\-\_\-\-Vdw\-C\-S\-Tab\-\_\-\-Geom\-W4\-P1\-\_\-c.\-c} }{\pageref{nb__kernel__ElecCoul__VdwCSTab__GeomW4P1__c_8c}}{}
\item\contentsline{section}{src/gmxlib/nonbonded/nb\-\_\-kernel\-\_\-c/\hyperlink{nb__kernel__ElecCoul__VdwCSTab__GeomW4W4__c_8c}{nb\-\_\-kernel\-\_\-\-Elec\-Coul\-\_\-\-Vdw\-C\-S\-Tab\-\_\-\-Geom\-W4\-W4\-\_\-c.\-c} }{\pageref{nb__kernel__ElecCoul__VdwCSTab__GeomW4W4__c_8c}}{}
\item\contentsline{section}{src/gmxlib/nonbonded/nb\-\_\-kernel\-\_\-c/\hyperlink{nb__kernel__ElecCoul__VdwLJ__GeomP1P1__c_8c}{nb\-\_\-kernel\-\_\-\-Elec\-Coul\-\_\-\-Vdw\-L\-J\-\_\-\-Geom\-P1\-P1\-\_\-c.\-c} }{\pageref{nb__kernel__ElecCoul__VdwLJ__GeomP1P1__c_8c}}{}
\item\contentsline{section}{src/gmxlib/nonbonded/nb\-\_\-kernel\-\_\-c/\hyperlink{nb__kernel__ElecCoul__VdwLJ__GeomW3P1__c_8c}{nb\-\_\-kernel\-\_\-\-Elec\-Coul\-\_\-\-Vdw\-L\-J\-\_\-\-Geom\-W3\-P1\-\_\-c.\-c} }{\pageref{nb__kernel__ElecCoul__VdwLJ__GeomW3P1__c_8c}}{}
\item\contentsline{section}{src/gmxlib/nonbonded/nb\-\_\-kernel\-\_\-c/\hyperlink{nb__kernel__ElecCoul__VdwLJ__GeomW3W3__c_8c}{nb\-\_\-kernel\-\_\-\-Elec\-Coul\-\_\-\-Vdw\-L\-J\-\_\-\-Geom\-W3\-W3\-\_\-c.\-c} }{\pageref{nb__kernel__ElecCoul__VdwLJ__GeomW3W3__c_8c}}{}
\item\contentsline{section}{src/gmxlib/nonbonded/nb\-\_\-kernel\-\_\-c/\hyperlink{nb__kernel__ElecCoul__VdwLJ__GeomW4P1__c_8c}{nb\-\_\-kernel\-\_\-\-Elec\-Coul\-\_\-\-Vdw\-L\-J\-\_\-\-Geom\-W4\-P1\-\_\-c.\-c} }{\pageref{nb__kernel__ElecCoul__VdwLJ__GeomW4P1__c_8c}}{}
\item\contentsline{section}{src/gmxlib/nonbonded/nb\-\_\-kernel\-\_\-c/\hyperlink{nb__kernel__ElecCoul__VdwLJ__GeomW4W4__c_8c}{nb\-\_\-kernel\-\_\-\-Elec\-Coul\-\_\-\-Vdw\-L\-J\-\_\-\-Geom\-W4\-W4\-\_\-c.\-c} }{\pageref{nb__kernel__ElecCoul__VdwLJ__GeomW4W4__c_8c}}{}
\item\contentsline{section}{src/gmxlib/nonbonded/nb\-\_\-kernel\-\_\-c/\hyperlink{nb__kernel__ElecCoul__VdwNone__GeomP1P1__c_8c}{nb\-\_\-kernel\-\_\-\-Elec\-Coul\-\_\-\-Vdw\-None\-\_\-\-Geom\-P1\-P1\-\_\-c.\-c} }{\pageref{nb__kernel__ElecCoul__VdwNone__GeomP1P1__c_8c}}{}
\item\contentsline{section}{src/gmxlib/nonbonded/nb\-\_\-kernel\-\_\-c/\hyperlink{nb__kernel__ElecCoul__VdwNone__GeomW3P1__c_8c}{nb\-\_\-kernel\-\_\-\-Elec\-Coul\-\_\-\-Vdw\-None\-\_\-\-Geom\-W3\-P1\-\_\-c.\-c} }{\pageref{nb__kernel__ElecCoul__VdwNone__GeomW3P1__c_8c}}{}
\item\contentsline{section}{src/gmxlib/nonbonded/nb\-\_\-kernel\-\_\-c/\hyperlink{nb__kernel__ElecCoul__VdwNone__GeomW3W3__c_8c}{nb\-\_\-kernel\-\_\-\-Elec\-Coul\-\_\-\-Vdw\-None\-\_\-\-Geom\-W3\-W3\-\_\-c.\-c} }{\pageref{nb__kernel__ElecCoul__VdwNone__GeomW3W3__c_8c}}{}
\item\contentsline{section}{src/gmxlib/nonbonded/nb\-\_\-kernel\-\_\-c/\hyperlink{nb__kernel__ElecCoul__VdwNone__GeomW4P1__c_8c}{nb\-\_\-kernel\-\_\-\-Elec\-Coul\-\_\-\-Vdw\-None\-\_\-\-Geom\-W4\-P1\-\_\-c.\-c} }{\pageref{nb__kernel__ElecCoul__VdwNone__GeomW4P1__c_8c}}{}
\item\contentsline{section}{src/gmxlib/nonbonded/nb\-\_\-kernel\-\_\-c/\hyperlink{nb__kernel__ElecCoul__VdwNone__GeomW4W4__c_8c}{nb\-\_\-kernel\-\_\-\-Elec\-Coul\-\_\-\-Vdw\-None\-\_\-\-Geom\-W4\-W4\-\_\-c.\-c} }{\pageref{nb__kernel__ElecCoul__VdwNone__GeomW4W4__c_8c}}{}
\item\contentsline{section}{src/gmxlib/nonbonded/nb\-\_\-kernel\-\_\-c/\hyperlink{nb__kernel__ElecCSTab__VdwBham__GeomP1P1__c_8c}{nb\-\_\-kernel\-\_\-\-Elec\-C\-S\-Tab\-\_\-\-Vdw\-Bham\-\_\-\-Geom\-P1\-P1\-\_\-c.\-c} }{\pageref{nb__kernel__ElecCSTab__VdwBham__GeomP1P1__c_8c}}{}
\item\contentsline{section}{src/gmxlib/nonbonded/nb\-\_\-kernel\-\_\-c/\hyperlink{nb__kernel__ElecCSTab__VdwBham__GeomW3P1__c_8c}{nb\-\_\-kernel\-\_\-\-Elec\-C\-S\-Tab\-\_\-\-Vdw\-Bham\-\_\-\-Geom\-W3\-P1\-\_\-c.\-c} }{\pageref{nb__kernel__ElecCSTab__VdwBham__GeomW3P1__c_8c}}{}
\item\contentsline{section}{src/gmxlib/nonbonded/nb\-\_\-kernel\-\_\-c/\hyperlink{nb__kernel__ElecCSTab__VdwBham__GeomW3W3__c_8c}{nb\-\_\-kernel\-\_\-\-Elec\-C\-S\-Tab\-\_\-\-Vdw\-Bham\-\_\-\-Geom\-W3\-W3\-\_\-c.\-c} }{\pageref{nb__kernel__ElecCSTab__VdwBham__GeomW3W3__c_8c}}{}
\item\contentsline{section}{src/gmxlib/nonbonded/nb\-\_\-kernel\-\_\-c/\hyperlink{nb__kernel__ElecCSTab__VdwBham__GeomW4P1__c_8c}{nb\-\_\-kernel\-\_\-\-Elec\-C\-S\-Tab\-\_\-\-Vdw\-Bham\-\_\-\-Geom\-W4\-P1\-\_\-c.\-c} }{\pageref{nb__kernel__ElecCSTab__VdwBham__GeomW4P1__c_8c}}{}
\item\contentsline{section}{src/gmxlib/nonbonded/nb\-\_\-kernel\-\_\-c/\hyperlink{nb__kernel__ElecCSTab__VdwBham__GeomW4W4__c_8c}{nb\-\_\-kernel\-\_\-\-Elec\-C\-S\-Tab\-\_\-\-Vdw\-Bham\-\_\-\-Geom\-W4\-W4\-\_\-c.\-c} }{\pageref{nb__kernel__ElecCSTab__VdwBham__GeomW4W4__c_8c}}{}
\item\contentsline{section}{src/gmxlib/nonbonded/nb\-\_\-kernel\-\_\-c/\hyperlink{nb__kernel__ElecCSTab__VdwCSTab__GeomP1P1__c_8c}{nb\-\_\-kernel\-\_\-\-Elec\-C\-S\-Tab\-\_\-\-Vdw\-C\-S\-Tab\-\_\-\-Geom\-P1\-P1\-\_\-c.\-c} }{\pageref{nb__kernel__ElecCSTab__VdwCSTab__GeomP1P1__c_8c}}{}
\item\contentsline{section}{src/gmxlib/nonbonded/nb\-\_\-kernel\-\_\-c/\hyperlink{nb__kernel__ElecCSTab__VdwCSTab__GeomW3P1__c_8c}{nb\-\_\-kernel\-\_\-\-Elec\-C\-S\-Tab\-\_\-\-Vdw\-C\-S\-Tab\-\_\-\-Geom\-W3\-P1\-\_\-c.\-c} }{\pageref{nb__kernel__ElecCSTab__VdwCSTab__GeomW3P1__c_8c}}{}
\item\contentsline{section}{src/gmxlib/nonbonded/nb\-\_\-kernel\-\_\-c/\hyperlink{nb__kernel__ElecCSTab__VdwCSTab__GeomW3W3__c_8c}{nb\-\_\-kernel\-\_\-\-Elec\-C\-S\-Tab\-\_\-\-Vdw\-C\-S\-Tab\-\_\-\-Geom\-W3\-W3\-\_\-c.\-c} }{\pageref{nb__kernel__ElecCSTab__VdwCSTab__GeomW3W3__c_8c}}{}
\item\contentsline{section}{src/gmxlib/nonbonded/nb\-\_\-kernel\-\_\-c/\hyperlink{nb__kernel__ElecCSTab__VdwCSTab__GeomW4P1__c_8c}{nb\-\_\-kernel\-\_\-\-Elec\-C\-S\-Tab\-\_\-\-Vdw\-C\-S\-Tab\-\_\-\-Geom\-W4\-P1\-\_\-c.\-c} }{\pageref{nb__kernel__ElecCSTab__VdwCSTab__GeomW4P1__c_8c}}{}
\item\contentsline{section}{src/gmxlib/nonbonded/nb\-\_\-kernel\-\_\-c/\hyperlink{nb__kernel__ElecCSTab__VdwCSTab__GeomW4W4__c_8c}{nb\-\_\-kernel\-\_\-\-Elec\-C\-S\-Tab\-\_\-\-Vdw\-C\-S\-Tab\-\_\-\-Geom\-W4\-W4\-\_\-c.\-c} }{\pageref{nb__kernel__ElecCSTab__VdwCSTab__GeomW4W4__c_8c}}{}
\item\contentsline{section}{src/gmxlib/nonbonded/nb\-\_\-kernel\-\_\-c/\hyperlink{nb__kernel__ElecCSTab__VdwLJ__GeomP1P1__c_8c}{nb\-\_\-kernel\-\_\-\-Elec\-C\-S\-Tab\-\_\-\-Vdw\-L\-J\-\_\-\-Geom\-P1\-P1\-\_\-c.\-c} }{\pageref{nb__kernel__ElecCSTab__VdwLJ__GeomP1P1__c_8c}}{}
\item\contentsline{section}{src/gmxlib/nonbonded/nb\-\_\-kernel\-\_\-c/\hyperlink{nb__kernel__ElecCSTab__VdwLJ__GeomW3P1__c_8c}{nb\-\_\-kernel\-\_\-\-Elec\-C\-S\-Tab\-\_\-\-Vdw\-L\-J\-\_\-\-Geom\-W3\-P1\-\_\-c.\-c} }{\pageref{nb__kernel__ElecCSTab__VdwLJ__GeomW3P1__c_8c}}{}
\item\contentsline{section}{src/gmxlib/nonbonded/nb\-\_\-kernel\-\_\-c/\hyperlink{nb__kernel__ElecCSTab__VdwLJ__GeomW3W3__c_8c}{nb\-\_\-kernel\-\_\-\-Elec\-C\-S\-Tab\-\_\-\-Vdw\-L\-J\-\_\-\-Geom\-W3\-W3\-\_\-c.\-c} }{\pageref{nb__kernel__ElecCSTab__VdwLJ__GeomW3W3__c_8c}}{}
\item\contentsline{section}{src/gmxlib/nonbonded/nb\-\_\-kernel\-\_\-c/\hyperlink{nb__kernel__ElecCSTab__VdwLJ__GeomW4P1__c_8c}{nb\-\_\-kernel\-\_\-\-Elec\-C\-S\-Tab\-\_\-\-Vdw\-L\-J\-\_\-\-Geom\-W4\-P1\-\_\-c.\-c} }{\pageref{nb__kernel__ElecCSTab__VdwLJ__GeomW4P1__c_8c}}{}
\item\contentsline{section}{src/gmxlib/nonbonded/nb\-\_\-kernel\-\_\-c/\hyperlink{nb__kernel__ElecCSTab__VdwLJ__GeomW4W4__c_8c}{nb\-\_\-kernel\-\_\-\-Elec\-C\-S\-Tab\-\_\-\-Vdw\-L\-J\-\_\-\-Geom\-W4\-W4\-\_\-c.\-c} }{\pageref{nb__kernel__ElecCSTab__VdwLJ__GeomW4W4__c_8c}}{}
\item\contentsline{section}{src/gmxlib/nonbonded/nb\-\_\-kernel\-\_\-c/\hyperlink{nb__kernel__ElecCSTab__VdwNone__GeomP1P1__c_8c}{nb\-\_\-kernel\-\_\-\-Elec\-C\-S\-Tab\-\_\-\-Vdw\-None\-\_\-\-Geom\-P1\-P1\-\_\-c.\-c} }{\pageref{nb__kernel__ElecCSTab__VdwNone__GeomP1P1__c_8c}}{}
\item\contentsline{section}{src/gmxlib/nonbonded/nb\-\_\-kernel\-\_\-c/\hyperlink{nb__kernel__ElecCSTab__VdwNone__GeomW3P1__c_8c}{nb\-\_\-kernel\-\_\-\-Elec\-C\-S\-Tab\-\_\-\-Vdw\-None\-\_\-\-Geom\-W3\-P1\-\_\-c.\-c} }{\pageref{nb__kernel__ElecCSTab__VdwNone__GeomW3P1__c_8c}}{}
\item\contentsline{section}{src/gmxlib/nonbonded/nb\-\_\-kernel\-\_\-c/\hyperlink{nb__kernel__ElecCSTab__VdwNone__GeomW3W3__c_8c}{nb\-\_\-kernel\-\_\-\-Elec\-C\-S\-Tab\-\_\-\-Vdw\-None\-\_\-\-Geom\-W3\-W3\-\_\-c.\-c} }{\pageref{nb__kernel__ElecCSTab__VdwNone__GeomW3W3__c_8c}}{}
\item\contentsline{section}{src/gmxlib/nonbonded/nb\-\_\-kernel\-\_\-c/\hyperlink{nb__kernel__ElecCSTab__VdwNone__GeomW4P1__c_8c}{nb\-\_\-kernel\-\_\-\-Elec\-C\-S\-Tab\-\_\-\-Vdw\-None\-\_\-\-Geom\-W4\-P1\-\_\-c.\-c} }{\pageref{nb__kernel__ElecCSTab__VdwNone__GeomW4P1__c_8c}}{}
\item\contentsline{section}{src/gmxlib/nonbonded/nb\-\_\-kernel\-\_\-c/\hyperlink{nb__kernel__ElecCSTab__VdwNone__GeomW4W4__c_8c}{nb\-\_\-kernel\-\_\-\-Elec\-C\-S\-Tab\-\_\-\-Vdw\-None\-\_\-\-Geom\-W4\-W4\-\_\-c.\-c} }{\pageref{nb__kernel__ElecCSTab__VdwNone__GeomW4W4__c_8c}}{}
\item\contentsline{section}{src/gmxlib/nonbonded/nb\-\_\-kernel\-\_\-c/\hyperlink{nb__kernel__ElecEw__VdwBham__GeomP1P1__c_8c}{nb\-\_\-kernel\-\_\-\-Elec\-Ew\-\_\-\-Vdw\-Bham\-\_\-\-Geom\-P1\-P1\-\_\-c.\-c} }{\pageref{nb__kernel__ElecEw__VdwBham__GeomP1P1__c_8c}}{}
\item\contentsline{section}{src/gmxlib/nonbonded/nb\-\_\-kernel\-\_\-c/\hyperlink{nb__kernel__ElecEw__VdwBham__GeomW3P1__c_8c}{nb\-\_\-kernel\-\_\-\-Elec\-Ew\-\_\-\-Vdw\-Bham\-\_\-\-Geom\-W3\-P1\-\_\-c.\-c} }{\pageref{nb__kernel__ElecEw__VdwBham__GeomW3P1__c_8c}}{}
\item\contentsline{section}{src/gmxlib/nonbonded/nb\-\_\-kernel\-\_\-c/\hyperlink{nb__kernel__ElecEw__VdwBham__GeomW3W3__c_8c}{nb\-\_\-kernel\-\_\-\-Elec\-Ew\-\_\-\-Vdw\-Bham\-\_\-\-Geom\-W3\-W3\-\_\-c.\-c} }{\pageref{nb__kernel__ElecEw__VdwBham__GeomW3W3__c_8c}}{}
\item\contentsline{section}{src/gmxlib/nonbonded/nb\-\_\-kernel\-\_\-c/\hyperlink{nb__kernel__ElecEw__VdwBham__GeomW4P1__c_8c}{nb\-\_\-kernel\-\_\-\-Elec\-Ew\-\_\-\-Vdw\-Bham\-\_\-\-Geom\-W4\-P1\-\_\-c.\-c} }{\pageref{nb__kernel__ElecEw__VdwBham__GeomW4P1__c_8c}}{}
\item\contentsline{section}{src/gmxlib/nonbonded/nb\-\_\-kernel\-\_\-c/\hyperlink{nb__kernel__ElecEw__VdwBham__GeomW4W4__c_8c}{nb\-\_\-kernel\-\_\-\-Elec\-Ew\-\_\-\-Vdw\-Bham\-\_\-\-Geom\-W4\-W4\-\_\-c.\-c} }{\pageref{nb__kernel__ElecEw__VdwBham__GeomW4W4__c_8c}}{}
\item\contentsline{section}{src/gmxlib/nonbonded/nb\-\_\-kernel\-\_\-c/\hyperlink{nb__kernel__ElecEw__VdwCSTab__GeomP1P1__c_8c}{nb\-\_\-kernel\-\_\-\-Elec\-Ew\-\_\-\-Vdw\-C\-S\-Tab\-\_\-\-Geom\-P1\-P1\-\_\-c.\-c} }{\pageref{nb__kernel__ElecEw__VdwCSTab__GeomP1P1__c_8c}}{}
\item\contentsline{section}{src/gmxlib/nonbonded/nb\-\_\-kernel\-\_\-c/\hyperlink{nb__kernel__ElecEw__VdwCSTab__GeomW3P1__c_8c}{nb\-\_\-kernel\-\_\-\-Elec\-Ew\-\_\-\-Vdw\-C\-S\-Tab\-\_\-\-Geom\-W3\-P1\-\_\-c.\-c} }{\pageref{nb__kernel__ElecEw__VdwCSTab__GeomW3P1__c_8c}}{}
\item\contentsline{section}{src/gmxlib/nonbonded/nb\-\_\-kernel\-\_\-c/\hyperlink{nb__kernel__ElecEw__VdwCSTab__GeomW3W3__c_8c}{nb\-\_\-kernel\-\_\-\-Elec\-Ew\-\_\-\-Vdw\-C\-S\-Tab\-\_\-\-Geom\-W3\-W3\-\_\-c.\-c} }{\pageref{nb__kernel__ElecEw__VdwCSTab__GeomW3W3__c_8c}}{}
\item\contentsline{section}{src/gmxlib/nonbonded/nb\-\_\-kernel\-\_\-c/\hyperlink{nb__kernel__ElecEw__VdwCSTab__GeomW4P1__c_8c}{nb\-\_\-kernel\-\_\-\-Elec\-Ew\-\_\-\-Vdw\-C\-S\-Tab\-\_\-\-Geom\-W4\-P1\-\_\-c.\-c} }{\pageref{nb__kernel__ElecEw__VdwCSTab__GeomW4P1__c_8c}}{}
\item\contentsline{section}{src/gmxlib/nonbonded/nb\-\_\-kernel\-\_\-c/\hyperlink{nb__kernel__ElecEw__VdwCSTab__GeomW4W4__c_8c}{nb\-\_\-kernel\-\_\-\-Elec\-Ew\-\_\-\-Vdw\-C\-S\-Tab\-\_\-\-Geom\-W4\-W4\-\_\-c.\-c} }{\pageref{nb__kernel__ElecEw__VdwCSTab__GeomW4W4__c_8c}}{}
\item\contentsline{section}{src/gmxlib/nonbonded/nb\-\_\-kernel\-\_\-c/\hyperlink{nb__kernel__ElecEw__VdwLJ__GeomP1P1__c_8c}{nb\-\_\-kernel\-\_\-\-Elec\-Ew\-\_\-\-Vdw\-L\-J\-\_\-\-Geom\-P1\-P1\-\_\-c.\-c} }{\pageref{nb__kernel__ElecEw__VdwLJ__GeomP1P1__c_8c}}{}
\item\contentsline{section}{src/gmxlib/nonbonded/nb\-\_\-kernel\-\_\-c/\hyperlink{nb__kernel__ElecEw__VdwLJ__GeomW3P1__c_8c}{nb\-\_\-kernel\-\_\-\-Elec\-Ew\-\_\-\-Vdw\-L\-J\-\_\-\-Geom\-W3\-P1\-\_\-c.\-c} }{\pageref{nb__kernel__ElecEw__VdwLJ__GeomW3P1__c_8c}}{}
\item\contentsline{section}{src/gmxlib/nonbonded/nb\-\_\-kernel\-\_\-c/\hyperlink{nb__kernel__ElecEw__VdwLJ__GeomW3W3__c_8c}{nb\-\_\-kernel\-\_\-\-Elec\-Ew\-\_\-\-Vdw\-L\-J\-\_\-\-Geom\-W3\-W3\-\_\-c.\-c} }{\pageref{nb__kernel__ElecEw__VdwLJ__GeomW3W3__c_8c}}{}
\item\contentsline{section}{src/gmxlib/nonbonded/nb\-\_\-kernel\-\_\-c/\hyperlink{nb__kernel__ElecEw__VdwLJ__GeomW4P1__c_8c}{nb\-\_\-kernel\-\_\-\-Elec\-Ew\-\_\-\-Vdw\-L\-J\-\_\-\-Geom\-W4\-P1\-\_\-c.\-c} }{\pageref{nb__kernel__ElecEw__VdwLJ__GeomW4P1__c_8c}}{}
\item\contentsline{section}{src/gmxlib/nonbonded/nb\-\_\-kernel\-\_\-c/\hyperlink{nb__kernel__ElecEw__VdwLJ__GeomW4W4__c_8c}{nb\-\_\-kernel\-\_\-\-Elec\-Ew\-\_\-\-Vdw\-L\-J\-\_\-\-Geom\-W4\-W4\-\_\-c.\-c} }{\pageref{nb__kernel__ElecEw__VdwLJ__GeomW4W4__c_8c}}{}
\item\contentsline{section}{src/gmxlib/nonbonded/nb\-\_\-kernel\-\_\-c/\hyperlink{nb__kernel__ElecEw__VdwNone__GeomP1P1__c_8c}{nb\-\_\-kernel\-\_\-\-Elec\-Ew\-\_\-\-Vdw\-None\-\_\-\-Geom\-P1\-P1\-\_\-c.\-c} }{\pageref{nb__kernel__ElecEw__VdwNone__GeomP1P1__c_8c}}{}
\item\contentsline{section}{src/gmxlib/nonbonded/nb\-\_\-kernel\-\_\-c/\hyperlink{nb__kernel__ElecEw__VdwNone__GeomW3P1__c_8c}{nb\-\_\-kernel\-\_\-\-Elec\-Ew\-\_\-\-Vdw\-None\-\_\-\-Geom\-W3\-P1\-\_\-c.\-c} }{\pageref{nb__kernel__ElecEw__VdwNone__GeomW3P1__c_8c}}{}
\item\contentsline{section}{src/gmxlib/nonbonded/nb\-\_\-kernel\-\_\-c/\hyperlink{nb__kernel__ElecEw__VdwNone__GeomW3W3__c_8c}{nb\-\_\-kernel\-\_\-\-Elec\-Ew\-\_\-\-Vdw\-None\-\_\-\-Geom\-W3\-W3\-\_\-c.\-c} }{\pageref{nb__kernel__ElecEw__VdwNone__GeomW3W3__c_8c}}{}
\item\contentsline{section}{src/gmxlib/nonbonded/nb\-\_\-kernel\-\_\-c/\hyperlink{nb__kernel__ElecEw__VdwNone__GeomW4P1__c_8c}{nb\-\_\-kernel\-\_\-\-Elec\-Ew\-\_\-\-Vdw\-None\-\_\-\-Geom\-W4\-P1\-\_\-c.\-c} }{\pageref{nb__kernel__ElecEw__VdwNone__GeomW4P1__c_8c}}{}
\item\contentsline{section}{src/gmxlib/nonbonded/nb\-\_\-kernel\-\_\-c/\hyperlink{nb__kernel__ElecEw__VdwNone__GeomW4W4__c_8c}{nb\-\_\-kernel\-\_\-\-Elec\-Ew\-\_\-\-Vdw\-None\-\_\-\-Geom\-W4\-W4\-\_\-c.\-c} }{\pageref{nb__kernel__ElecEw__VdwNone__GeomW4W4__c_8c}}{}
\item\contentsline{section}{src/gmxlib/nonbonded/nb\-\_\-kernel\-\_\-c/\hyperlink{nb__kernel__ElecEwSh__VdwBhamSh__GeomP1P1__c_8c}{nb\-\_\-kernel\-\_\-\-Elec\-Ew\-Sh\-\_\-\-Vdw\-Bham\-Sh\-\_\-\-Geom\-P1\-P1\-\_\-c.\-c} }{\pageref{nb__kernel__ElecEwSh__VdwBhamSh__GeomP1P1__c_8c}}{}
\item\contentsline{section}{src/gmxlib/nonbonded/nb\-\_\-kernel\-\_\-c/\hyperlink{nb__kernel__ElecEwSh__VdwBhamSh__GeomW3P1__c_8c}{nb\-\_\-kernel\-\_\-\-Elec\-Ew\-Sh\-\_\-\-Vdw\-Bham\-Sh\-\_\-\-Geom\-W3\-P1\-\_\-c.\-c} }{\pageref{nb__kernel__ElecEwSh__VdwBhamSh__GeomW3P1__c_8c}}{}
\item\contentsline{section}{src/gmxlib/nonbonded/nb\-\_\-kernel\-\_\-c/\hyperlink{nb__kernel__ElecEwSh__VdwBhamSh__GeomW3W3__c_8c}{nb\-\_\-kernel\-\_\-\-Elec\-Ew\-Sh\-\_\-\-Vdw\-Bham\-Sh\-\_\-\-Geom\-W3\-W3\-\_\-c.\-c} }{\pageref{nb__kernel__ElecEwSh__VdwBhamSh__GeomW3W3__c_8c}}{}
\item\contentsline{section}{src/gmxlib/nonbonded/nb\-\_\-kernel\-\_\-c/\hyperlink{nb__kernel__ElecEwSh__VdwBhamSh__GeomW4P1__c_8c}{nb\-\_\-kernel\-\_\-\-Elec\-Ew\-Sh\-\_\-\-Vdw\-Bham\-Sh\-\_\-\-Geom\-W4\-P1\-\_\-c.\-c} }{\pageref{nb__kernel__ElecEwSh__VdwBhamSh__GeomW4P1__c_8c}}{}
\item\contentsline{section}{src/gmxlib/nonbonded/nb\-\_\-kernel\-\_\-c/\hyperlink{nb__kernel__ElecEwSh__VdwBhamSh__GeomW4W4__c_8c}{nb\-\_\-kernel\-\_\-\-Elec\-Ew\-Sh\-\_\-\-Vdw\-Bham\-Sh\-\_\-\-Geom\-W4\-W4\-\_\-c.\-c} }{\pageref{nb__kernel__ElecEwSh__VdwBhamSh__GeomW4W4__c_8c}}{}
\item\contentsline{section}{src/gmxlib/nonbonded/nb\-\_\-kernel\-\_\-c/\hyperlink{nb__kernel__ElecEwSh__VdwLJSh__GeomP1P1__c_8c}{nb\-\_\-kernel\-\_\-\-Elec\-Ew\-Sh\-\_\-\-Vdw\-L\-J\-Sh\-\_\-\-Geom\-P1\-P1\-\_\-c.\-c} }{\pageref{nb__kernel__ElecEwSh__VdwLJSh__GeomP1P1__c_8c}}{}
\item\contentsline{section}{src/gmxlib/nonbonded/nb\-\_\-kernel\-\_\-c/\hyperlink{nb__kernel__ElecEwSh__VdwLJSh__GeomW3P1__c_8c}{nb\-\_\-kernel\-\_\-\-Elec\-Ew\-Sh\-\_\-\-Vdw\-L\-J\-Sh\-\_\-\-Geom\-W3\-P1\-\_\-c.\-c} }{\pageref{nb__kernel__ElecEwSh__VdwLJSh__GeomW3P1__c_8c}}{}
\item\contentsline{section}{src/gmxlib/nonbonded/nb\-\_\-kernel\-\_\-c/\hyperlink{nb__kernel__ElecEwSh__VdwLJSh__GeomW3W3__c_8c}{nb\-\_\-kernel\-\_\-\-Elec\-Ew\-Sh\-\_\-\-Vdw\-L\-J\-Sh\-\_\-\-Geom\-W3\-W3\-\_\-c.\-c} }{\pageref{nb__kernel__ElecEwSh__VdwLJSh__GeomW3W3__c_8c}}{}
\item\contentsline{section}{src/gmxlib/nonbonded/nb\-\_\-kernel\-\_\-c/\hyperlink{nb__kernel__ElecEwSh__VdwLJSh__GeomW4P1__c_8c}{nb\-\_\-kernel\-\_\-\-Elec\-Ew\-Sh\-\_\-\-Vdw\-L\-J\-Sh\-\_\-\-Geom\-W4\-P1\-\_\-c.\-c} }{\pageref{nb__kernel__ElecEwSh__VdwLJSh__GeomW4P1__c_8c}}{}
\item\contentsline{section}{src/gmxlib/nonbonded/nb\-\_\-kernel\-\_\-c/\hyperlink{nb__kernel__ElecEwSh__VdwLJSh__GeomW4W4__c_8c}{nb\-\_\-kernel\-\_\-\-Elec\-Ew\-Sh\-\_\-\-Vdw\-L\-J\-Sh\-\_\-\-Geom\-W4\-W4\-\_\-c.\-c} }{\pageref{nb__kernel__ElecEwSh__VdwLJSh__GeomW4W4__c_8c}}{}
\item\contentsline{section}{src/gmxlib/nonbonded/nb\-\_\-kernel\-\_\-c/\hyperlink{nb__kernel__ElecEwSh__VdwNone__GeomP1P1__c_8c}{nb\-\_\-kernel\-\_\-\-Elec\-Ew\-Sh\-\_\-\-Vdw\-None\-\_\-\-Geom\-P1\-P1\-\_\-c.\-c} }{\pageref{nb__kernel__ElecEwSh__VdwNone__GeomP1P1__c_8c}}{}
\item\contentsline{section}{src/gmxlib/nonbonded/nb\-\_\-kernel\-\_\-c/\hyperlink{nb__kernel__ElecEwSh__VdwNone__GeomW3P1__c_8c}{nb\-\_\-kernel\-\_\-\-Elec\-Ew\-Sh\-\_\-\-Vdw\-None\-\_\-\-Geom\-W3\-P1\-\_\-c.\-c} }{\pageref{nb__kernel__ElecEwSh__VdwNone__GeomW3P1__c_8c}}{}
\item\contentsline{section}{src/gmxlib/nonbonded/nb\-\_\-kernel\-\_\-c/\hyperlink{nb__kernel__ElecEwSh__VdwNone__GeomW3W3__c_8c}{nb\-\_\-kernel\-\_\-\-Elec\-Ew\-Sh\-\_\-\-Vdw\-None\-\_\-\-Geom\-W3\-W3\-\_\-c.\-c} }{\pageref{nb__kernel__ElecEwSh__VdwNone__GeomW3W3__c_8c}}{}
\item\contentsline{section}{src/gmxlib/nonbonded/nb\-\_\-kernel\-\_\-c/\hyperlink{nb__kernel__ElecEwSh__VdwNone__GeomW4P1__c_8c}{nb\-\_\-kernel\-\_\-\-Elec\-Ew\-Sh\-\_\-\-Vdw\-None\-\_\-\-Geom\-W4\-P1\-\_\-c.\-c} }{\pageref{nb__kernel__ElecEwSh__VdwNone__GeomW4P1__c_8c}}{}
\item\contentsline{section}{src/gmxlib/nonbonded/nb\-\_\-kernel\-\_\-c/\hyperlink{nb__kernel__ElecEwSh__VdwNone__GeomW4W4__c_8c}{nb\-\_\-kernel\-\_\-\-Elec\-Ew\-Sh\-\_\-\-Vdw\-None\-\_\-\-Geom\-W4\-W4\-\_\-c.\-c} }{\pageref{nb__kernel__ElecEwSh__VdwNone__GeomW4W4__c_8c}}{}
\item\contentsline{section}{src/gmxlib/nonbonded/nb\-\_\-kernel\-\_\-c/\hyperlink{nb__kernel__ElecEwSw__VdwBhamSw__GeomP1P1__c_8c}{nb\-\_\-kernel\-\_\-\-Elec\-Ew\-Sw\-\_\-\-Vdw\-Bham\-Sw\-\_\-\-Geom\-P1\-P1\-\_\-c.\-c} }{\pageref{nb__kernel__ElecEwSw__VdwBhamSw__GeomP1P1__c_8c}}{}
\item\contentsline{section}{src/gmxlib/nonbonded/nb\-\_\-kernel\-\_\-c/\hyperlink{nb__kernel__ElecEwSw__VdwBhamSw__GeomW3P1__c_8c}{nb\-\_\-kernel\-\_\-\-Elec\-Ew\-Sw\-\_\-\-Vdw\-Bham\-Sw\-\_\-\-Geom\-W3\-P1\-\_\-c.\-c} }{\pageref{nb__kernel__ElecEwSw__VdwBhamSw__GeomW3P1__c_8c}}{}
\item\contentsline{section}{src/gmxlib/nonbonded/nb\-\_\-kernel\-\_\-c/\hyperlink{nb__kernel__ElecEwSw__VdwBhamSw__GeomW3W3__c_8c}{nb\-\_\-kernel\-\_\-\-Elec\-Ew\-Sw\-\_\-\-Vdw\-Bham\-Sw\-\_\-\-Geom\-W3\-W3\-\_\-c.\-c} }{\pageref{nb__kernel__ElecEwSw__VdwBhamSw__GeomW3W3__c_8c}}{}
\item\contentsline{section}{src/gmxlib/nonbonded/nb\-\_\-kernel\-\_\-c/\hyperlink{nb__kernel__ElecEwSw__VdwBhamSw__GeomW4P1__c_8c}{nb\-\_\-kernel\-\_\-\-Elec\-Ew\-Sw\-\_\-\-Vdw\-Bham\-Sw\-\_\-\-Geom\-W4\-P1\-\_\-c.\-c} }{\pageref{nb__kernel__ElecEwSw__VdwBhamSw__GeomW4P1__c_8c}}{}
\item\contentsline{section}{src/gmxlib/nonbonded/nb\-\_\-kernel\-\_\-c/\hyperlink{nb__kernel__ElecEwSw__VdwBhamSw__GeomW4W4__c_8c}{nb\-\_\-kernel\-\_\-\-Elec\-Ew\-Sw\-\_\-\-Vdw\-Bham\-Sw\-\_\-\-Geom\-W4\-W4\-\_\-c.\-c} }{\pageref{nb__kernel__ElecEwSw__VdwBhamSw__GeomW4W4__c_8c}}{}
\item\contentsline{section}{src/gmxlib/nonbonded/nb\-\_\-kernel\-\_\-c/\hyperlink{nb__kernel__ElecEwSw__VdwLJSw__GeomP1P1__c_8c}{nb\-\_\-kernel\-\_\-\-Elec\-Ew\-Sw\-\_\-\-Vdw\-L\-J\-Sw\-\_\-\-Geom\-P1\-P1\-\_\-c.\-c} }{\pageref{nb__kernel__ElecEwSw__VdwLJSw__GeomP1P1__c_8c}}{}
\item\contentsline{section}{src/gmxlib/nonbonded/nb\-\_\-kernel\-\_\-c/\hyperlink{nb__kernel__ElecEwSw__VdwLJSw__GeomW3P1__c_8c}{nb\-\_\-kernel\-\_\-\-Elec\-Ew\-Sw\-\_\-\-Vdw\-L\-J\-Sw\-\_\-\-Geom\-W3\-P1\-\_\-c.\-c} }{\pageref{nb__kernel__ElecEwSw__VdwLJSw__GeomW3P1__c_8c}}{}
\item\contentsline{section}{src/gmxlib/nonbonded/nb\-\_\-kernel\-\_\-c/\hyperlink{nb__kernel__ElecEwSw__VdwLJSw__GeomW3W3__c_8c}{nb\-\_\-kernel\-\_\-\-Elec\-Ew\-Sw\-\_\-\-Vdw\-L\-J\-Sw\-\_\-\-Geom\-W3\-W3\-\_\-c.\-c} }{\pageref{nb__kernel__ElecEwSw__VdwLJSw__GeomW3W3__c_8c}}{}
\item\contentsline{section}{src/gmxlib/nonbonded/nb\-\_\-kernel\-\_\-c/\hyperlink{nb__kernel__ElecEwSw__VdwLJSw__GeomW4P1__c_8c}{nb\-\_\-kernel\-\_\-\-Elec\-Ew\-Sw\-\_\-\-Vdw\-L\-J\-Sw\-\_\-\-Geom\-W4\-P1\-\_\-c.\-c} }{\pageref{nb__kernel__ElecEwSw__VdwLJSw__GeomW4P1__c_8c}}{}
\item\contentsline{section}{src/gmxlib/nonbonded/nb\-\_\-kernel\-\_\-c/\hyperlink{nb__kernel__ElecEwSw__VdwLJSw__GeomW4W4__c_8c}{nb\-\_\-kernel\-\_\-\-Elec\-Ew\-Sw\-\_\-\-Vdw\-L\-J\-Sw\-\_\-\-Geom\-W4\-W4\-\_\-c.\-c} }{\pageref{nb__kernel__ElecEwSw__VdwLJSw__GeomW4W4__c_8c}}{}
\item\contentsline{section}{src/gmxlib/nonbonded/nb\-\_\-kernel\-\_\-c/\hyperlink{nb__kernel__ElecEwSw__VdwNone__GeomP1P1__c_8c}{nb\-\_\-kernel\-\_\-\-Elec\-Ew\-Sw\-\_\-\-Vdw\-None\-\_\-\-Geom\-P1\-P1\-\_\-c.\-c} }{\pageref{nb__kernel__ElecEwSw__VdwNone__GeomP1P1__c_8c}}{}
\item\contentsline{section}{src/gmxlib/nonbonded/nb\-\_\-kernel\-\_\-c/\hyperlink{nb__kernel__ElecEwSw__VdwNone__GeomW3P1__c_8c}{nb\-\_\-kernel\-\_\-\-Elec\-Ew\-Sw\-\_\-\-Vdw\-None\-\_\-\-Geom\-W3\-P1\-\_\-c.\-c} }{\pageref{nb__kernel__ElecEwSw__VdwNone__GeomW3P1__c_8c}}{}
\item\contentsline{section}{src/gmxlib/nonbonded/nb\-\_\-kernel\-\_\-c/\hyperlink{nb__kernel__ElecEwSw__VdwNone__GeomW3W3__c_8c}{nb\-\_\-kernel\-\_\-\-Elec\-Ew\-Sw\-\_\-\-Vdw\-None\-\_\-\-Geom\-W3\-W3\-\_\-c.\-c} }{\pageref{nb__kernel__ElecEwSw__VdwNone__GeomW3W3__c_8c}}{}
\item\contentsline{section}{src/gmxlib/nonbonded/nb\-\_\-kernel\-\_\-c/\hyperlink{nb__kernel__ElecEwSw__VdwNone__GeomW4P1__c_8c}{nb\-\_\-kernel\-\_\-\-Elec\-Ew\-Sw\-\_\-\-Vdw\-None\-\_\-\-Geom\-W4\-P1\-\_\-c.\-c} }{\pageref{nb__kernel__ElecEwSw__VdwNone__GeomW4P1__c_8c}}{}
\item\contentsline{section}{src/gmxlib/nonbonded/nb\-\_\-kernel\-\_\-c/\hyperlink{nb__kernel__ElecEwSw__VdwNone__GeomW4W4__c_8c}{nb\-\_\-kernel\-\_\-\-Elec\-Ew\-Sw\-\_\-\-Vdw\-None\-\_\-\-Geom\-W4\-W4\-\_\-c.\-c} }{\pageref{nb__kernel__ElecEwSw__VdwNone__GeomW4W4__c_8c}}{}
\item\contentsline{section}{src/gmxlib/nonbonded/nb\-\_\-kernel\-\_\-c/\hyperlink{nb__kernel__ElecGB__VdwBham__GeomP1P1__c_8c}{nb\-\_\-kernel\-\_\-\-Elec\-G\-B\-\_\-\-Vdw\-Bham\-\_\-\-Geom\-P1\-P1\-\_\-c.\-c} }{\pageref{nb__kernel__ElecGB__VdwBham__GeomP1P1__c_8c}}{}
\item\contentsline{section}{src/gmxlib/nonbonded/nb\-\_\-kernel\-\_\-c/\hyperlink{nb__kernel__ElecGB__VdwCSTab__GeomP1P1__c_8c}{nb\-\_\-kernel\-\_\-\-Elec\-G\-B\-\_\-\-Vdw\-C\-S\-Tab\-\_\-\-Geom\-P1\-P1\-\_\-c.\-c} }{\pageref{nb__kernel__ElecGB__VdwCSTab__GeomP1P1__c_8c}}{}
\item\contentsline{section}{src/gmxlib/nonbonded/nb\-\_\-kernel\-\_\-c/\hyperlink{nb__kernel__ElecGB__VdwLJ__GeomP1P1__c_8c}{nb\-\_\-kernel\-\_\-\-Elec\-G\-B\-\_\-\-Vdw\-L\-J\-\_\-\-Geom\-P1\-P1\-\_\-c.\-c} }{\pageref{nb__kernel__ElecGB__VdwLJ__GeomP1P1__c_8c}}{}
\item\contentsline{section}{src/gmxlib/nonbonded/nb\-\_\-kernel\-\_\-c/\hyperlink{nb__kernel__ElecGB__VdwNone__GeomP1P1__c_8c}{nb\-\_\-kernel\-\_\-\-Elec\-G\-B\-\_\-\-Vdw\-None\-\_\-\-Geom\-P1\-P1\-\_\-c.\-c} }{\pageref{nb__kernel__ElecGB__VdwNone__GeomP1P1__c_8c}}{}
\item\contentsline{section}{src/gmxlib/nonbonded/nb\-\_\-kernel\-\_\-c/\hyperlink{nb__kernel__ElecNone__VdwBham__GeomP1P1__c_8c}{nb\-\_\-kernel\-\_\-\-Elec\-None\-\_\-\-Vdw\-Bham\-\_\-\-Geom\-P1\-P1\-\_\-c.\-c} }{\pageref{nb__kernel__ElecNone__VdwBham__GeomP1P1__c_8c}}{}
\item\contentsline{section}{src/gmxlib/nonbonded/nb\-\_\-kernel\-\_\-c/\hyperlink{nb__kernel__ElecNone__VdwBhamSh__GeomP1P1__c_8c}{nb\-\_\-kernel\-\_\-\-Elec\-None\-\_\-\-Vdw\-Bham\-Sh\-\_\-\-Geom\-P1\-P1\-\_\-c.\-c} }{\pageref{nb__kernel__ElecNone__VdwBhamSh__GeomP1P1__c_8c}}{}
\item\contentsline{section}{src/gmxlib/nonbonded/nb\-\_\-kernel\-\_\-c/\hyperlink{nb__kernel__ElecNone__VdwBhamSw__GeomP1P1__c_8c}{nb\-\_\-kernel\-\_\-\-Elec\-None\-\_\-\-Vdw\-Bham\-Sw\-\_\-\-Geom\-P1\-P1\-\_\-c.\-c} }{\pageref{nb__kernel__ElecNone__VdwBhamSw__GeomP1P1__c_8c}}{}
\item\contentsline{section}{src/gmxlib/nonbonded/nb\-\_\-kernel\-\_\-c/\hyperlink{nb__kernel__ElecNone__VdwCSTab__GeomP1P1__c_8c}{nb\-\_\-kernel\-\_\-\-Elec\-None\-\_\-\-Vdw\-C\-S\-Tab\-\_\-\-Geom\-P1\-P1\-\_\-c.\-c} }{\pageref{nb__kernel__ElecNone__VdwCSTab__GeomP1P1__c_8c}}{}
\item\contentsline{section}{src/gmxlib/nonbonded/nb\-\_\-kernel\-\_\-c/\hyperlink{nb__kernel__ElecNone__VdwLJ__GeomP1P1__c_8c}{nb\-\_\-kernel\-\_\-\-Elec\-None\-\_\-\-Vdw\-L\-J\-\_\-\-Geom\-P1\-P1\-\_\-c.\-c} }{\pageref{nb__kernel__ElecNone__VdwLJ__GeomP1P1__c_8c}}{}
\item\contentsline{section}{src/gmxlib/nonbonded/nb\-\_\-kernel\-\_\-c/\hyperlink{nb__kernel__ElecNone__VdwLJSh__GeomP1P1__c_8c}{nb\-\_\-kernel\-\_\-\-Elec\-None\-\_\-\-Vdw\-L\-J\-Sh\-\_\-\-Geom\-P1\-P1\-\_\-c.\-c} }{\pageref{nb__kernel__ElecNone__VdwLJSh__GeomP1P1__c_8c}}{}
\item\contentsline{section}{src/gmxlib/nonbonded/nb\-\_\-kernel\-\_\-c/\hyperlink{nb__kernel__ElecNone__VdwLJSw__GeomP1P1__c_8c}{nb\-\_\-kernel\-\_\-\-Elec\-None\-\_\-\-Vdw\-L\-J\-Sw\-\_\-\-Geom\-P1\-P1\-\_\-c.\-c} }{\pageref{nb__kernel__ElecNone__VdwLJSw__GeomP1P1__c_8c}}{}
\item\contentsline{section}{src/gmxlib/nonbonded/nb\-\_\-kernel\-\_\-c/\hyperlink{nb__kernel__ElecRF__VdwBham__GeomP1P1__c_8c}{nb\-\_\-kernel\-\_\-\-Elec\-R\-F\-\_\-\-Vdw\-Bham\-\_\-\-Geom\-P1\-P1\-\_\-c.\-c} }{\pageref{nb__kernel__ElecRF__VdwBham__GeomP1P1__c_8c}}{}
\item\contentsline{section}{src/gmxlib/nonbonded/nb\-\_\-kernel\-\_\-c/\hyperlink{nb__kernel__ElecRF__VdwBham__GeomW3P1__c_8c}{nb\-\_\-kernel\-\_\-\-Elec\-R\-F\-\_\-\-Vdw\-Bham\-\_\-\-Geom\-W3\-P1\-\_\-c.\-c} }{\pageref{nb__kernel__ElecRF__VdwBham__GeomW3P1__c_8c}}{}
\item\contentsline{section}{src/gmxlib/nonbonded/nb\-\_\-kernel\-\_\-c/\hyperlink{nb__kernel__ElecRF__VdwBham__GeomW3W3__c_8c}{nb\-\_\-kernel\-\_\-\-Elec\-R\-F\-\_\-\-Vdw\-Bham\-\_\-\-Geom\-W3\-W3\-\_\-c.\-c} }{\pageref{nb__kernel__ElecRF__VdwBham__GeomW3W3__c_8c}}{}
\item\contentsline{section}{src/gmxlib/nonbonded/nb\-\_\-kernel\-\_\-c/\hyperlink{nb__kernel__ElecRF__VdwBham__GeomW4P1__c_8c}{nb\-\_\-kernel\-\_\-\-Elec\-R\-F\-\_\-\-Vdw\-Bham\-\_\-\-Geom\-W4\-P1\-\_\-c.\-c} }{\pageref{nb__kernel__ElecRF__VdwBham__GeomW4P1__c_8c}}{}
\item\contentsline{section}{src/gmxlib/nonbonded/nb\-\_\-kernel\-\_\-c/\hyperlink{nb__kernel__ElecRF__VdwBham__GeomW4W4__c_8c}{nb\-\_\-kernel\-\_\-\-Elec\-R\-F\-\_\-\-Vdw\-Bham\-\_\-\-Geom\-W4\-W4\-\_\-c.\-c} }{\pageref{nb__kernel__ElecRF__VdwBham__GeomW4W4__c_8c}}{}
\item\contentsline{section}{src/gmxlib/nonbonded/nb\-\_\-kernel\-\_\-c/\hyperlink{nb__kernel__ElecRF__VdwCSTab__GeomP1P1__c_8c}{nb\-\_\-kernel\-\_\-\-Elec\-R\-F\-\_\-\-Vdw\-C\-S\-Tab\-\_\-\-Geom\-P1\-P1\-\_\-c.\-c} }{\pageref{nb__kernel__ElecRF__VdwCSTab__GeomP1P1__c_8c}}{}
\item\contentsline{section}{src/gmxlib/nonbonded/nb\-\_\-kernel\-\_\-c/\hyperlink{nb__kernel__ElecRF__VdwCSTab__GeomW3P1__c_8c}{nb\-\_\-kernel\-\_\-\-Elec\-R\-F\-\_\-\-Vdw\-C\-S\-Tab\-\_\-\-Geom\-W3\-P1\-\_\-c.\-c} }{\pageref{nb__kernel__ElecRF__VdwCSTab__GeomW3P1__c_8c}}{}
\item\contentsline{section}{src/gmxlib/nonbonded/nb\-\_\-kernel\-\_\-c/\hyperlink{nb__kernel__ElecRF__VdwCSTab__GeomW3W3__c_8c}{nb\-\_\-kernel\-\_\-\-Elec\-R\-F\-\_\-\-Vdw\-C\-S\-Tab\-\_\-\-Geom\-W3\-W3\-\_\-c.\-c} }{\pageref{nb__kernel__ElecRF__VdwCSTab__GeomW3W3__c_8c}}{}
\item\contentsline{section}{src/gmxlib/nonbonded/nb\-\_\-kernel\-\_\-c/\hyperlink{nb__kernel__ElecRF__VdwCSTab__GeomW4P1__c_8c}{nb\-\_\-kernel\-\_\-\-Elec\-R\-F\-\_\-\-Vdw\-C\-S\-Tab\-\_\-\-Geom\-W4\-P1\-\_\-c.\-c} }{\pageref{nb__kernel__ElecRF__VdwCSTab__GeomW4P1__c_8c}}{}
\item\contentsline{section}{src/gmxlib/nonbonded/nb\-\_\-kernel\-\_\-c/\hyperlink{nb__kernel__ElecRF__VdwCSTab__GeomW4W4__c_8c}{nb\-\_\-kernel\-\_\-\-Elec\-R\-F\-\_\-\-Vdw\-C\-S\-Tab\-\_\-\-Geom\-W4\-W4\-\_\-c.\-c} }{\pageref{nb__kernel__ElecRF__VdwCSTab__GeomW4W4__c_8c}}{}
\item\contentsline{section}{src/gmxlib/nonbonded/nb\-\_\-kernel\-\_\-c/\hyperlink{nb__kernel__ElecRF__VdwLJ__GeomP1P1__c_8c}{nb\-\_\-kernel\-\_\-\-Elec\-R\-F\-\_\-\-Vdw\-L\-J\-\_\-\-Geom\-P1\-P1\-\_\-c.\-c} }{\pageref{nb__kernel__ElecRF__VdwLJ__GeomP1P1__c_8c}}{}
\item\contentsline{section}{src/gmxlib/nonbonded/nb\-\_\-kernel\-\_\-c/\hyperlink{nb__kernel__ElecRF__VdwLJ__GeomW3P1__c_8c}{nb\-\_\-kernel\-\_\-\-Elec\-R\-F\-\_\-\-Vdw\-L\-J\-\_\-\-Geom\-W3\-P1\-\_\-c.\-c} }{\pageref{nb__kernel__ElecRF__VdwLJ__GeomW3P1__c_8c}}{}
\item\contentsline{section}{src/gmxlib/nonbonded/nb\-\_\-kernel\-\_\-c/\hyperlink{nb__kernel__ElecRF__VdwLJ__GeomW3W3__c_8c}{nb\-\_\-kernel\-\_\-\-Elec\-R\-F\-\_\-\-Vdw\-L\-J\-\_\-\-Geom\-W3\-W3\-\_\-c.\-c} }{\pageref{nb__kernel__ElecRF__VdwLJ__GeomW3W3__c_8c}}{}
\item\contentsline{section}{src/gmxlib/nonbonded/nb\-\_\-kernel\-\_\-c/\hyperlink{nb__kernel__ElecRF__VdwLJ__GeomW4P1__c_8c}{nb\-\_\-kernel\-\_\-\-Elec\-R\-F\-\_\-\-Vdw\-L\-J\-\_\-\-Geom\-W4\-P1\-\_\-c.\-c} }{\pageref{nb__kernel__ElecRF__VdwLJ__GeomW4P1__c_8c}}{}
\item\contentsline{section}{src/gmxlib/nonbonded/nb\-\_\-kernel\-\_\-c/\hyperlink{nb__kernel__ElecRF__VdwLJ__GeomW4W4__c_8c}{nb\-\_\-kernel\-\_\-\-Elec\-R\-F\-\_\-\-Vdw\-L\-J\-\_\-\-Geom\-W4\-W4\-\_\-c.\-c} }{\pageref{nb__kernel__ElecRF__VdwLJ__GeomW4W4__c_8c}}{}
\item\contentsline{section}{src/gmxlib/nonbonded/nb\-\_\-kernel\-\_\-c/\hyperlink{nb__kernel__ElecRF__VdwNone__GeomP1P1__c_8c}{nb\-\_\-kernel\-\_\-\-Elec\-R\-F\-\_\-\-Vdw\-None\-\_\-\-Geom\-P1\-P1\-\_\-c.\-c} }{\pageref{nb__kernel__ElecRF__VdwNone__GeomP1P1__c_8c}}{}
\item\contentsline{section}{src/gmxlib/nonbonded/nb\-\_\-kernel\-\_\-c/\hyperlink{nb__kernel__ElecRF__VdwNone__GeomW3P1__c_8c}{nb\-\_\-kernel\-\_\-\-Elec\-R\-F\-\_\-\-Vdw\-None\-\_\-\-Geom\-W3\-P1\-\_\-c.\-c} }{\pageref{nb__kernel__ElecRF__VdwNone__GeomW3P1__c_8c}}{}
\item\contentsline{section}{src/gmxlib/nonbonded/nb\-\_\-kernel\-\_\-c/\hyperlink{nb__kernel__ElecRF__VdwNone__GeomW3W3__c_8c}{nb\-\_\-kernel\-\_\-\-Elec\-R\-F\-\_\-\-Vdw\-None\-\_\-\-Geom\-W3\-W3\-\_\-c.\-c} }{\pageref{nb__kernel__ElecRF__VdwNone__GeomW3W3__c_8c}}{}
\item\contentsline{section}{src/gmxlib/nonbonded/nb\-\_\-kernel\-\_\-c/\hyperlink{nb__kernel__ElecRF__VdwNone__GeomW4P1__c_8c}{nb\-\_\-kernel\-\_\-\-Elec\-R\-F\-\_\-\-Vdw\-None\-\_\-\-Geom\-W4\-P1\-\_\-c.\-c} }{\pageref{nb__kernel__ElecRF__VdwNone__GeomW4P1__c_8c}}{}
\item\contentsline{section}{src/gmxlib/nonbonded/nb\-\_\-kernel\-\_\-c/\hyperlink{nb__kernel__ElecRF__VdwNone__GeomW4W4__c_8c}{nb\-\_\-kernel\-\_\-\-Elec\-R\-F\-\_\-\-Vdw\-None\-\_\-\-Geom\-W4\-W4\-\_\-c.\-c} }{\pageref{nb__kernel__ElecRF__VdwNone__GeomW4W4__c_8c}}{}
\item\contentsline{section}{src/gmxlib/nonbonded/nb\-\_\-kernel\-\_\-c/\hyperlink{nb__kernel__ElecRFCut__VdwBhamSh__GeomP1P1__c_8c}{nb\-\_\-kernel\-\_\-\-Elec\-R\-F\-Cut\-\_\-\-Vdw\-Bham\-Sh\-\_\-\-Geom\-P1\-P1\-\_\-c.\-c} }{\pageref{nb__kernel__ElecRFCut__VdwBhamSh__GeomP1P1__c_8c}}{}
\item\contentsline{section}{src/gmxlib/nonbonded/nb\-\_\-kernel\-\_\-c/\hyperlink{nb__kernel__ElecRFCut__VdwBhamSh__GeomW3P1__c_8c}{nb\-\_\-kernel\-\_\-\-Elec\-R\-F\-Cut\-\_\-\-Vdw\-Bham\-Sh\-\_\-\-Geom\-W3\-P1\-\_\-c.\-c} }{\pageref{nb__kernel__ElecRFCut__VdwBhamSh__GeomW3P1__c_8c}}{}
\item\contentsline{section}{src/gmxlib/nonbonded/nb\-\_\-kernel\-\_\-c/\hyperlink{nb__kernel__ElecRFCut__VdwBhamSh__GeomW3W3__c_8c}{nb\-\_\-kernel\-\_\-\-Elec\-R\-F\-Cut\-\_\-\-Vdw\-Bham\-Sh\-\_\-\-Geom\-W3\-W3\-\_\-c.\-c} }{\pageref{nb__kernel__ElecRFCut__VdwBhamSh__GeomW3W3__c_8c}}{}
\item\contentsline{section}{src/gmxlib/nonbonded/nb\-\_\-kernel\-\_\-c/\hyperlink{nb__kernel__ElecRFCut__VdwBhamSh__GeomW4P1__c_8c}{nb\-\_\-kernel\-\_\-\-Elec\-R\-F\-Cut\-\_\-\-Vdw\-Bham\-Sh\-\_\-\-Geom\-W4\-P1\-\_\-c.\-c} }{\pageref{nb__kernel__ElecRFCut__VdwBhamSh__GeomW4P1__c_8c}}{}
\item\contentsline{section}{src/gmxlib/nonbonded/nb\-\_\-kernel\-\_\-c/\hyperlink{nb__kernel__ElecRFCut__VdwBhamSh__GeomW4W4__c_8c}{nb\-\_\-kernel\-\_\-\-Elec\-R\-F\-Cut\-\_\-\-Vdw\-Bham\-Sh\-\_\-\-Geom\-W4\-W4\-\_\-c.\-c} }{\pageref{nb__kernel__ElecRFCut__VdwBhamSh__GeomW4W4__c_8c}}{}
\item\contentsline{section}{src/gmxlib/nonbonded/nb\-\_\-kernel\-\_\-c/\hyperlink{nb__kernel__ElecRFCut__VdwBhamSw__GeomP1P1__c_8c}{nb\-\_\-kernel\-\_\-\-Elec\-R\-F\-Cut\-\_\-\-Vdw\-Bham\-Sw\-\_\-\-Geom\-P1\-P1\-\_\-c.\-c} }{\pageref{nb__kernel__ElecRFCut__VdwBhamSw__GeomP1P1__c_8c}}{}
\item\contentsline{section}{src/gmxlib/nonbonded/nb\-\_\-kernel\-\_\-c/\hyperlink{nb__kernel__ElecRFCut__VdwBhamSw__GeomW3P1__c_8c}{nb\-\_\-kernel\-\_\-\-Elec\-R\-F\-Cut\-\_\-\-Vdw\-Bham\-Sw\-\_\-\-Geom\-W3\-P1\-\_\-c.\-c} }{\pageref{nb__kernel__ElecRFCut__VdwBhamSw__GeomW3P1__c_8c}}{}
\item\contentsline{section}{src/gmxlib/nonbonded/nb\-\_\-kernel\-\_\-c/\hyperlink{nb__kernel__ElecRFCut__VdwBhamSw__GeomW3W3__c_8c}{nb\-\_\-kernel\-\_\-\-Elec\-R\-F\-Cut\-\_\-\-Vdw\-Bham\-Sw\-\_\-\-Geom\-W3\-W3\-\_\-c.\-c} }{\pageref{nb__kernel__ElecRFCut__VdwBhamSw__GeomW3W3__c_8c}}{}
\item\contentsline{section}{src/gmxlib/nonbonded/nb\-\_\-kernel\-\_\-c/\hyperlink{nb__kernel__ElecRFCut__VdwBhamSw__GeomW4P1__c_8c}{nb\-\_\-kernel\-\_\-\-Elec\-R\-F\-Cut\-\_\-\-Vdw\-Bham\-Sw\-\_\-\-Geom\-W4\-P1\-\_\-c.\-c} }{\pageref{nb__kernel__ElecRFCut__VdwBhamSw__GeomW4P1__c_8c}}{}
\item\contentsline{section}{src/gmxlib/nonbonded/nb\-\_\-kernel\-\_\-c/\hyperlink{nb__kernel__ElecRFCut__VdwBhamSw__GeomW4W4__c_8c}{nb\-\_\-kernel\-\_\-\-Elec\-R\-F\-Cut\-\_\-\-Vdw\-Bham\-Sw\-\_\-\-Geom\-W4\-W4\-\_\-c.\-c} }{\pageref{nb__kernel__ElecRFCut__VdwBhamSw__GeomW4W4__c_8c}}{}
\item\contentsline{section}{src/gmxlib/nonbonded/nb\-\_\-kernel\-\_\-c/\hyperlink{nb__kernel__ElecRFCut__VdwCSTab__GeomP1P1__c_8c}{nb\-\_\-kernel\-\_\-\-Elec\-R\-F\-Cut\-\_\-\-Vdw\-C\-S\-Tab\-\_\-\-Geom\-P1\-P1\-\_\-c.\-c} }{\pageref{nb__kernel__ElecRFCut__VdwCSTab__GeomP1P1__c_8c}}{}
\item\contentsline{section}{src/gmxlib/nonbonded/nb\-\_\-kernel\-\_\-c/\hyperlink{nb__kernel__ElecRFCut__VdwCSTab__GeomW3P1__c_8c}{nb\-\_\-kernel\-\_\-\-Elec\-R\-F\-Cut\-\_\-\-Vdw\-C\-S\-Tab\-\_\-\-Geom\-W3\-P1\-\_\-c.\-c} }{\pageref{nb__kernel__ElecRFCut__VdwCSTab__GeomW3P1__c_8c}}{}
\item\contentsline{section}{src/gmxlib/nonbonded/nb\-\_\-kernel\-\_\-c/\hyperlink{nb__kernel__ElecRFCut__VdwCSTab__GeomW3W3__c_8c}{nb\-\_\-kernel\-\_\-\-Elec\-R\-F\-Cut\-\_\-\-Vdw\-C\-S\-Tab\-\_\-\-Geom\-W3\-W3\-\_\-c.\-c} }{\pageref{nb__kernel__ElecRFCut__VdwCSTab__GeomW3W3__c_8c}}{}
\item\contentsline{section}{src/gmxlib/nonbonded/nb\-\_\-kernel\-\_\-c/\hyperlink{nb__kernel__ElecRFCut__VdwCSTab__GeomW4P1__c_8c}{nb\-\_\-kernel\-\_\-\-Elec\-R\-F\-Cut\-\_\-\-Vdw\-C\-S\-Tab\-\_\-\-Geom\-W4\-P1\-\_\-c.\-c} }{\pageref{nb__kernel__ElecRFCut__VdwCSTab__GeomW4P1__c_8c}}{}
\item\contentsline{section}{src/gmxlib/nonbonded/nb\-\_\-kernel\-\_\-c/\hyperlink{nb__kernel__ElecRFCut__VdwCSTab__GeomW4W4__c_8c}{nb\-\_\-kernel\-\_\-\-Elec\-R\-F\-Cut\-\_\-\-Vdw\-C\-S\-Tab\-\_\-\-Geom\-W4\-W4\-\_\-c.\-c} }{\pageref{nb__kernel__ElecRFCut__VdwCSTab__GeomW4W4__c_8c}}{}
\item\contentsline{section}{src/gmxlib/nonbonded/nb\-\_\-kernel\-\_\-c/\hyperlink{nb__kernel__ElecRFCut__VdwLJSh__GeomP1P1__c_8c}{nb\-\_\-kernel\-\_\-\-Elec\-R\-F\-Cut\-\_\-\-Vdw\-L\-J\-Sh\-\_\-\-Geom\-P1\-P1\-\_\-c.\-c} }{\pageref{nb__kernel__ElecRFCut__VdwLJSh__GeomP1P1__c_8c}}{}
\item\contentsline{section}{src/gmxlib/nonbonded/nb\-\_\-kernel\-\_\-c/\hyperlink{nb__kernel__ElecRFCut__VdwLJSh__GeomW3P1__c_8c}{nb\-\_\-kernel\-\_\-\-Elec\-R\-F\-Cut\-\_\-\-Vdw\-L\-J\-Sh\-\_\-\-Geom\-W3\-P1\-\_\-c.\-c} }{\pageref{nb__kernel__ElecRFCut__VdwLJSh__GeomW3P1__c_8c}}{}
\item\contentsline{section}{src/gmxlib/nonbonded/nb\-\_\-kernel\-\_\-c/\hyperlink{nb__kernel__ElecRFCut__VdwLJSh__GeomW3W3__c_8c}{nb\-\_\-kernel\-\_\-\-Elec\-R\-F\-Cut\-\_\-\-Vdw\-L\-J\-Sh\-\_\-\-Geom\-W3\-W3\-\_\-c.\-c} }{\pageref{nb__kernel__ElecRFCut__VdwLJSh__GeomW3W3__c_8c}}{}
\item\contentsline{section}{src/gmxlib/nonbonded/nb\-\_\-kernel\-\_\-c/\hyperlink{nb__kernel__ElecRFCut__VdwLJSh__GeomW4P1__c_8c}{nb\-\_\-kernel\-\_\-\-Elec\-R\-F\-Cut\-\_\-\-Vdw\-L\-J\-Sh\-\_\-\-Geom\-W4\-P1\-\_\-c.\-c} }{\pageref{nb__kernel__ElecRFCut__VdwLJSh__GeomW4P1__c_8c}}{}
\item\contentsline{section}{src/gmxlib/nonbonded/nb\-\_\-kernel\-\_\-c/\hyperlink{nb__kernel__ElecRFCut__VdwLJSh__GeomW4W4__c_8c}{nb\-\_\-kernel\-\_\-\-Elec\-R\-F\-Cut\-\_\-\-Vdw\-L\-J\-Sh\-\_\-\-Geom\-W4\-W4\-\_\-c.\-c} }{\pageref{nb__kernel__ElecRFCut__VdwLJSh__GeomW4W4__c_8c}}{}
\item\contentsline{section}{src/gmxlib/nonbonded/nb\-\_\-kernel\-\_\-c/\hyperlink{nb__kernel__ElecRFCut__VdwLJSw__GeomP1P1__c_8c}{nb\-\_\-kernel\-\_\-\-Elec\-R\-F\-Cut\-\_\-\-Vdw\-L\-J\-Sw\-\_\-\-Geom\-P1\-P1\-\_\-c.\-c} }{\pageref{nb__kernel__ElecRFCut__VdwLJSw__GeomP1P1__c_8c}}{}
\item\contentsline{section}{src/gmxlib/nonbonded/nb\-\_\-kernel\-\_\-c/\hyperlink{nb__kernel__ElecRFCut__VdwLJSw__GeomW3P1__c_8c}{nb\-\_\-kernel\-\_\-\-Elec\-R\-F\-Cut\-\_\-\-Vdw\-L\-J\-Sw\-\_\-\-Geom\-W3\-P1\-\_\-c.\-c} }{\pageref{nb__kernel__ElecRFCut__VdwLJSw__GeomW3P1__c_8c}}{}
\item\contentsline{section}{src/gmxlib/nonbonded/nb\-\_\-kernel\-\_\-c/\hyperlink{nb__kernel__ElecRFCut__VdwLJSw__GeomW3W3__c_8c}{nb\-\_\-kernel\-\_\-\-Elec\-R\-F\-Cut\-\_\-\-Vdw\-L\-J\-Sw\-\_\-\-Geom\-W3\-W3\-\_\-c.\-c} }{\pageref{nb__kernel__ElecRFCut__VdwLJSw__GeomW3W3__c_8c}}{}
\item\contentsline{section}{src/gmxlib/nonbonded/nb\-\_\-kernel\-\_\-c/\hyperlink{nb__kernel__ElecRFCut__VdwLJSw__GeomW4P1__c_8c}{nb\-\_\-kernel\-\_\-\-Elec\-R\-F\-Cut\-\_\-\-Vdw\-L\-J\-Sw\-\_\-\-Geom\-W4\-P1\-\_\-c.\-c} }{\pageref{nb__kernel__ElecRFCut__VdwLJSw__GeomW4P1__c_8c}}{}
\item\contentsline{section}{src/gmxlib/nonbonded/nb\-\_\-kernel\-\_\-c/\hyperlink{nb__kernel__ElecRFCut__VdwLJSw__GeomW4W4__c_8c}{nb\-\_\-kernel\-\_\-\-Elec\-R\-F\-Cut\-\_\-\-Vdw\-L\-J\-Sw\-\_\-\-Geom\-W4\-W4\-\_\-c.\-c} }{\pageref{nb__kernel__ElecRFCut__VdwLJSw__GeomW4W4__c_8c}}{}
\item\contentsline{section}{src/gmxlib/nonbonded/nb\-\_\-kernel\-\_\-c/\hyperlink{nb__kernel__ElecRFCut__VdwNone__GeomP1P1__c_8c}{nb\-\_\-kernel\-\_\-\-Elec\-R\-F\-Cut\-\_\-\-Vdw\-None\-\_\-\-Geom\-P1\-P1\-\_\-c.\-c} }{\pageref{nb__kernel__ElecRFCut__VdwNone__GeomP1P1__c_8c}}{}
\item\contentsline{section}{src/gmxlib/nonbonded/nb\-\_\-kernel\-\_\-c/\hyperlink{nb__kernel__ElecRFCut__VdwNone__GeomW3P1__c_8c}{nb\-\_\-kernel\-\_\-\-Elec\-R\-F\-Cut\-\_\-\-Vdw\-None\-\_\-\-Geom\-W3\-P1\-\_\-c.\-c} }{\pageref{nb__kernel__ElecRFCut__VdwNone__GeomW3P1__c_8c}}{}
\item\contentsline{section}{src/gmxlib/nonbonded/nb\-\_\-kernel\-\_\-c/\hyperlink{nb__kernel__ElecRFCut__VdwNone__GeomW3W3__c_8c}{nb\-\_\-kernel\-\_\-\-Elec\-R\-F\-Cut\-\_\-\-Vdw\-None\-\_\-\-Geom\-W3\-W3\-\_\-c.\-c} }{\pageref{nb__kernel__ElecRFCut__VdwNone__GeomW3W3__c_8c}}{}
\item\contentsline{section}{src/gmxlib/nonbonded/nb\-\_\-kernel\-\_\-c/\hyperlink{nb__kernel__ElecRFCut__VdwNone__GeomW4P1__c_8c}{nb\-\_\-kernel\-\_\-\-Elec\-R\-F\-Cut\-\_\-\-Vdw\-None\-\_\-\-Geom\-W4\-P1\-\_\-c.\-c} }{\pageref{nb__kernel__ElecRFCut__VdwNone__GeomW4P1__c_8c}}{}
\item\contentsline{section}{src/gmxlib/nonbonded/nb\-\_\-kernel\-\_\-c/\hyperlink{nb__kernel__ElecRFCut__VdwNone__GeomW4W4__c_8c}{nb\-\_\-kernel\-\_\-\-Elec\-R\-F\-Cut\-\_\-\-Vdw\-None\-\_\-\-Geom\-W4\-W4\-\_\-c.\-c} }{\pageref{nb__kernel__ElecRFCut__VdwNone__GeomW4W4__c_8c}}{}
\item\contentsline{section}{src/gmxlib/nonbonded/nb\-\_\-kernel\-\_\-sse2\-\_\-double/\hyperlink{kernelutil__x86__sse2__double_8h}{kernelutil\-\_\-x86\-\_\-sse2\-\_\-double.\-h} }{\pageref{kernelutil__x86__sse2__double_8h}}{}
\item\contentsline{section}{src/gmxlib/nonbonded/nb\-\_\-kernel\-\_\-sse2\-\_\-double/\hyperlink{nb__kernel__ElecCoul__VdwCSTab__GeomP1P1__sse2__double_8c}{nb\-\_\-kernel\-\_\-\-Elec\-Coul\-\_\-\-Vdw\-C\-S\-Tab\-\_\-\-Geom\-P1\-P1\-\_\-sse2\-\_\-double.\-c} }{\pageref{nb__kernel__ElecCoul__VdwCSTab__GeomP1P1__sse2__double_8c}}{}
\item\contentsline{section}{src/gmxlib/nonbonded/nb\-\_\-kernel\-\_\-sse2\-\_\-double/\hyperlink{nb__kernel__ElecCoul__VdwCSTab__GeomW3P1__sse2__double_8c}{nb\-\_\-kernel\-\_\-\-Elec\-Coul\-\_\-\-Vdw\-C\-S\-Tab\-\_\-\-Geom\-W3\-P1\-\_\-sse2\-\_\-double.\-c} }{\pageref{nb__kernel__ElecCoul__VdwCSTab__GeomW3P1__sse2__double_8c}}{}
\item\contentsline{section}{src/gmxlib/nonbonded/nb\-\_\-kernel\-\_\-sse2\-\_\-double/\hyperlink{nb__kernel__ElecCoul__VdwCSTab__GeomW3W3__sse2__double_8c}{nb\-\_\-kernel\-\_\-\-Elec\-Coul\-\_\-\-Vdw\-C\-S\-Tab\-\_\-\-Geom\-W3\-W3\-\_\-sse2\-\_\-double.\-c} }{\pageref{nb__kernel__ElecCoul__VdwCSTab__GeomW3W3__sse2__double_8c}}{}
\item\contentsline{section}{src/gmxlib/nonbonded/nb\-\_\-kernel\-\_\-sse2\-\_\-double/\hyperlink{nb__kernel__ElecCoul__VdwCSTab__GeomW4P1__sse2__double_8c}{nb\-\_\-kernel\-\_\-\-Elec\-Coul\-\_\-\-Vdw\-C\-S\-Tab\-\_\-\-Geom\-W4\-P1\-\_\-sse2\-\_\-double.\-c} }{\pageref{nb__kernel__ElecCoul__VdwCSTab__GeomW4P1__sse2__double_8c}}{}
\item\contentsline{section}{src/gmxlib/nonbonded/nb\-\_\-kernel\-\_\-sse2\-\_\-double/\hyperlink{nb__kernel__ElecCoul__VdwCSTab__GeomW4W4__sse2__double_8c}{nb\-\_\-kernel\-\_\-\-Elec\-Coul\-\_\-\-Vdw\-C\-S\-Tab\-\_\-\-Geom\-W4\-W4\-\_\-sse2\-\_\-double.\-c} }{\pageref{nb__kernel__ElecCoul__VdwCSTab__GeomW4W4__sse2__double_8c}}{}
\item\contentsline{section}{src/gmxlib/nonbonded/nb\-\_\-kernel\-\_\-sse2\-\_\-double/\hyperlink{nb__kernel__ElecCoul__VdwLJ__GeomP1P1__sse2__double_8c}{nb\-\_\-kernel\-\_\-\-Elec\-Coul\-\_\-\-Vdw\-L\-J\-\_\-\-Geom\-P1\-P1\-\_\-sse2\-\_\-double.\-c} }{\pageref{nb__kernel__ElecCoul__VdwLJ__GeomP1P1__sse2__double_8c}}{}
\item\contentsline{section}{src/gmxlib/nonbonded/nb\-\_\-kernel\-\_\-sse2\-\_\-double/\hyperlink{nb__kernel__ElecCoul__VdwLJ__GeomW3P1__sse2__double_8c}{nb\-\_\-kernel\-\_\-\-Elec\-Coul\-\_\-\-Vdw\-L\-J\-\_\-\-Geom\-W3\-P1\-\_\-sse2\-\_\-double.\-c} }{\pageref{nb__kernel__ElecCoul__VdwLJ__GeomW3P1__sse2__double_8c}}{}
\item\contentsline{section}{src/gmxlib/nonbonded/nb\-\_\-kernel\-\_\-sse2\-\_\-double/\hyperlink{nb__kernel__ElecCoul__VdwLJ__GeomW3W3__sse2__double_8c}{nb\-\_\-kernel\-\_\-\-Elec\-Coul\-\_\-\-Vdw\-L\-J\-\_\-\-Geom\-W3\-W3\-\_\-sse2\-\_\-double.\-c} }{\pageref{nb__kernel__ElecCoul__VdwLJ__GeomW3W3__sse2__double_8c}}{}
\item\contentsline{section}{src/gmxlib/nonbonded/nb\-\_\-kernel\-\_\-sse2\-\_\-double/\hyperlink{nb__kernel__ElecCoul__VdwLJ__GeomW4P1__sse2__double_8c}{nb\-\_\-kernel\-\_\-\-Elec\-Coul\-\_\-\-Vdw\-L\-J\-\_\-\-Geom\-W4\-P1\-\_\-sse2\-\_\-double.\-c} }{\pageref{nb__kernel__ElecCoul__VdwLJ__GeomW4P1__sse2__double_8c}}{}
\item\contentsline{section}{src/gmxlib/nonbonded/nb\-\_\-kernel\-\_\-sse2\-\_\-double/\hyperlink{nb__kernel__ElecCoul__VdwLJ__GeomW4W4__sse2__double_8c}{nb\-\_\-kernel\-\_\-\-Elec\-Coul\-\_\-\-Vdw\-L\-J\-\_\-\-Geom\-W4\-W4\-\_\-sse2\-\_\-double.\-c} }{\pageref{nb__kernel__ElecCoul__VdwLJ__GeomW4W4__sse2__double_8c}}{}
\item\contentsline{section}{src/gmxlib/nonbonded/nb\-\_\-kernel\-\_\-sse2\-\_\-double/\hyperlink{nb__kernel__ElecCoul__VdwNone__GeomP1P1__sse2__double_8c}{nb\-\_\-kernel\-\_\-\-Elec\-Coul\-\_\-\-Vdw\-None\-\_\-\-Geom\-P1\-P1\-\_\-sse2\-\_\-double.\-c} }{\pageref{nb__kernel__ElecCoul__VdwNone__GeomP1P1__sse2__double_8c}}{}
\item\contentsline{section}{src/gmxlib/nonbonded/nb\-\_\-kernel\-\_\-sse2\-\_\-double/\hyperlink{nb__kernel__ElecCoul__VdwNone__GeomW3P1__sse2__double_8c}{nb\-\_\-kernel\-\_\-\-Elec\-Coul\-\_\-\-Vdw\-None\-\_\-\-Geom\-W3\-P1\-\_\-sse2\-\_\-double.\-c} }{\pageref{nb__kernel__ElecCoul__VdwNone__GeomW3P1__sse2__double_8c}}{}
\item\contentsline{section}{src/gmxlib/nonbonded/nb\-\_\-kernel\-\_\-sse2\-\_\-double/\hyperlink{nb__kernel__ElecCoul__VdwNone__GeomW3W3__sse2__double_8c}{nb\-\_\-kernel\-\_\-\-Elec\-Coul\-\_\-\-Vdw\-None\-\_\-\-Geom\-W3\-W3\-\_\-sse2\-\_\-double.\-c} }{\pageref{nb__kernel__ElecCoul__VdwNone__GeomW3W3__sse2__double_8c}}{}
\item\contentsline{section}{src/gmxlib/nonbonded/nb\-\_\-kernel\-\_\-sse2\-\_\-double/\hyperlink{nb__kernel__ElecCoul__VdwNone__GeomW4P1__sse2__double_8c}{nb\-\_\-kernel\-\_\-\-Elec\-Coul\-\_\-\-Vdw\-None\-\_\-\-Geom\-W4\-P1\-\_\-sse2\-\_\-double.\-c} }{\pageref{nb__kernel__ElecCoul__VdwNone__GeomW4P1__sse2__double_8c}}{}
\item\contentsline{section}{src/gmxlib/nonbonded/nb\-\_\-kernel\-\_\-sse2\-\_\-double/\hyperlink{nb__kernel__ElecCoul__VdwNone__GeomW4W4__sse2__double_8c}{nb\-\_\-kernel\-\_\-\-Elec\-Coul\-\_\-\-Vdw\-None\-\_\-\-Geom\-W4\-W4\-\_\-sse2\-\_\-double.\-c} }{\pageref{nb__kernel__ElecCoul__VdwNone__GeomW4W4__sse2__double_8c}}{}
\item\contentsline{section}{src/gmxlib/nonbonded/nb\-\_\-kernel\-\_\-sse2\-\_\-double/\hyperlink{nb__kernel__ElecCSTab__VdwCSTab__GeomP1P1__sse2__double_8c}{nb\-\_\-kernel\-\_\-\-Elec\-C\-S\-Tab\-\_\-\-Vdw\-C\-S\-Tab\-\_\-\-Geom\-P1\-P1\-\_\-sse2\-\_\-double.\-c} }{\pageref{nb__kernel__ElecCSTab__VdwCSTab__GeomP1P1__sse2__double_8c}}{}
\item\contentsline{section}{src/gmxlib/nonbonded/nb\-\_\-kernel\-\_\-sse2\-\_\-double/\hyperlink{nb__kernel__ElecCSTab__VdwCSTab__GeomW3P1__sse2__double_8c}{nb\-\_\-kernel\-\_\-\-Elec\-C\-S\-Tab\-\_\-\-Vdw\-C\-S\-Tab\-\_\-\-Geom\-W3\-P1\-\_\-sse2\-\_\-double.\-c} }{\pageref{nb__kernel__ElecCSTab__VdwCSTab__GeomW3P1__sse2__double_8c}}{}
\item\contentsline{section}{src/gmxlib/nonbonded/nb\-\_\-kernel\-\_\-sse2\-\_\-double/\hyperlink{nb__kernel__ElecCSTab__VdwCSTab__GeomW3W3__sse2__double_8c}{nb\-\_\-kernel\-\_\-\-Elec\-C\-S\-Tab\-\_\-\-Vdw\-C\-S\-Tab\-\_\-\-Geom\-W3\-W3\-\_\-sse2\-\_\-double.\-c} }{\pageref{nb__kernel__ElecCSTab__VdwCSTab__GeomW3W3__sse2__double_8c}}{}
\item\contentsline{section}{src/gmxlib/nonbonded/nb\-\_\-kernel\-\_\-sse2\-\_\-double/\hyperlink{nb__kernel__ElecCSTab__VdwCSTab__GeomW4P1__sse2__double_8c}{nb\-\_\-kernel\-\_\-\-Elec\-C\-S\-Tab\-\_\-\-Vdw\-C\-S\-Tab\-\_\-\-Geom\-W4\-P1\-\_\-sse2\-\_\-double.\-c} }{\pageref{nb__kernel__ElecCSTab__VdwCSTab__GeomW4P1__sse2__double_8c}}{}
\item\contentsline{section}{src/gmxlib/nonbonded/nb\-\_\-kernel\-\_\-sse2\-\_\-double/\hyperlink{nb__kernel__ElecCSTab__VdwCSTab__GeomW4W4__sse2__double_8c}{nb\-\_\-kernel\-\_\-\-Elec\-C\-S\-Tab\-\_\-\-Vdw\-C\-S\-Tab\-\_\-\-Geom\-W4\-W4\-\_\-sse2\-\_\-double.\-c} }{\pageref{nb__kernel__ElecCSTab__VdwCSTab__GeomW4W4__sse2__double_8c}}{}
\item\contentsline{section}{src/gmxlib/nonbonded/nb\-\_\-kernel\-\_\-sse2\-\_\-double/\hyperlink{nb__kernel__ElecCSTab__VdwLJ__GeomP1P1__sse2__double_8c}{nb\-\_\-kernel\-\_\-\-Elec\-C\-S\-Tab\-\_\-\-Vdw\-L\-J\-\_\-\-Geom\-P1\-P1\-\_\-sse2\-\_\-double.\-c} }{\pageref{nb__kernel__ElecCSTab__VdwLJ__GeomP1P1__sse2__double_8c}}{}
\item\contentsline{section}{src/gmxlib/nonbonded/nb\-\_\-kernel\-\_\-sse2\-\_\-double/\hyperlink{nb__kernel__ElecCSTab__VdwLJ__GeomW3P1__sse2__double_8c}{nb\-\_\-kernel\-\_\-\-Elec\-C\-S\-Tab\-\_\-\-Vdw\-L\-J\-\_\-\-Geom\-W3\-P1\-\_\-sse2\-\_\-double.\-c} }{\pageref{nb__kernel__ElecCSTab__VdwLJ__GeomW3P1__sse2__double_8c}}{}
\item\contentsline{section}{src/gmxlib/nonbonded/nb\-\_\-kernel\-\_\-sse2\-\_\-double/\hyperlink{nb__kernel__ElecCSTab__VdwLJ__GeomW3W3__sse2__double_8c}{nb\-\_\-kernel\-\_\-\-Elec\-C\-S\-Tab\-\_\-\-Vdw\-L\-J\-\_\-\-Geom\-W3\-W3\-\_\-sse2\-\_\-double.\-c} }{\pageref{nb__kernel__ElecCSTab__VdwLJ__GeomW3W3__sse2__double_8c}}{}
\item\contentsline{section}{src/gmxlib/nonbonded/nb\-\_\-kernel\-\_\-sse2\-\_\-double/\hyperlink{nb__kernel__ElecCSTab__VdwLJ__GeomW4P1__sse2__double_8c}{nb\-\_\-kernel\-\_\-\-Elec\-C\-S\-Tab\-\_\-\-Vdw\-L\-J\-\_\-\-Geom\-W4\-P1\-\_\-sse2\-\_\-double.\-c} }{\pageref{nb__kernel__ElecCSTab__VdwLJ__GeomW4P1__sse2__double_8c}}{}
\item\contentsline{section}{src/gmxlib/nonbonded/nb\-\_\-kernel\-\_\-sse2\-\_\-double/\hyperlink{nb__kernel__ElecCSTab__VdwLJ__GeomW4W4__sse2__double_8c}{nb\-\_\-kernel\-\_\-\-Elec\-C\-S\-Tab\-\_\-\-Vdw\-L\-J\-\_\-\-Geom\-W4\-W4\-\_\-sse2\-\_\-double.\-c} }{\pageref{nb__kernel__ElecCSTab__VdwLJ__GeomW4W4__sse2__double_8c}}{}
\item\contentsline{section}{src/gmxlib/nonbonded/nb\-\_\-kernel\-\_\-sse2\-\_\-double/\hyperlink{nb__kernel__ElecCSTab__VdwNone__GeomP1P1__sse2__double_8c}{nb\-\_\-kernel\-\_\-\-Elec\-C\-S\-Tab\-\_\-\-Vdw\-None\-\_\-\-Geom\-P1\-P1\-\_\-sse2\-\_\-double.\-c} }{\pageref{nb__kernel__ElecCSTab__VdwNone__GeomP1P1__sse2__double_8c}}{}
\item\contentsline{section}{src/gmxlib/nonbonded/nb\-\_\-kernel\-\_\-sse2\-\_\-double/\hyperlink{nb__kernel__ElecCSTab__VdwNone__GeomW3P1__sse2__double_8c}{nb\-\_\-kernel\-\_\-\-Elec\-C\-S\-Tab\-\_\-\-Vdw\-None\-\_\-\-Geom\-W3\-P1\-\_\-sse2\-\_\-double.\-c} }{\pageref{nb__kernel__ElecCSTab__VdwNone__GeomW3P1__sse2__double_8c}}{}
\item\contentsline{section}{src/gmxlib/nonbonded/nb\-\_\-kernel\-\_\-sse2\-\_\-double/\hyperlink{nb__kernel__ElecCSTab__VdwNone__GeomW3W3__sse2__double_8c}{nb\-\_\-kernel\-\_\-\-Elec\-C\-S\-Tab\-\_\-\-Vdw\-None\-\_\-\-Geom\-W3\-W3\-\_\-sse2\-\_\-double.\-c} }{\pageref{nb__kernel__ElecCSTab__VdwNone__GeomW3W3__sse2__double_8c}}{}
\item\contentsline{section}{src/gmxlib/nonbonded/nb\-\_\-kernel\-\_\-sse2\-\_\-double/\hyperlink{nb__kernel__ElecCSTab__VdwNone__GeomW4P1__sse2__double_8c}{nb\-\_\-kernel\-\_\-\-Elec\-C\-S\-Tab\-\_\-\-Vdw\-None\-\_\-\-Geom\-W4\-P1\-\_\-sse2\-\_\-double.\-c} }{\pageref{nb__kernel__ElecCSTab__VdwNone__GeomW4P1__sse2__double_8c}}{}
\item\contentsline{section}{src/gmxlib/nonbonded/nb\-\_\-kernel\-\_\-sse2\-\_\-double/\hyperlink{nb__kernel__ElecCSTab__VdwNone__GeomW4W4__sse2__double_8c}{nb\-\_\-kernel\-\_\-\-Elec\-C\-S\-Tab\-\_\-\-Vdw\-None\-\_\-\-Geom\-W4\-W4\-\_\-sse2\-\_\-double.\-c} }{\pageref{nb__kernel__ElecCSTab__VdwNone__GeomW4W4__sse2__double_8c}}{}
\item\contentsline{section}{src/gmxlib/nonbonded/nb\-\_\-kernel\-\_\-sse2\-\_\-double/\hyperlink{nb__kernel__ElecEw__VdwCSTab__GeomP1P1__sse2__double_8c}{nb\-\_\-kernel\-\_\-\-Elec\-Ew\-\_\-\-Vdw\-C\-S\-Tab\-\_\-\-Geom\-P1\-P1\-\_\-sse2\-\_\-double.\-c} }{\pageref{nb__kernel__ElecEw__VdwCSTab__GeomP1P1__sse2__double_8c}}{}
\item\contentsline{section}{src/gmxlib/nonbonded/nb\-\_\-kernel\-\_\-sse2\-\_\-double/\hyperlink{nb__kernel__ElecEw__VdwCSTab__GeomW3P1__sse2__double_8c}{nb\-\_\-kernel\-\_\-\-Elec\-Ew\-\_\-\-Vdw\-C\-S\-Tab\-\_\-\-Geom\-W3\-P1\-\_\-sse2\-\_\-double.\-c} }{\pageref{nb__kernel__ElecEw__VdwCSTab__GeomW3P1__sse2__double_8c}}{}
\item\contentsline{section}{src/gmxlib/nonbonded/nb\-\_\-kernel\-\_\-sse2\-\_\-double/\hyperlink{nb__kernel__ElecEw__VdwCSTab__GeomW3W3__sse2__double_8c}{nb\-\_\-kernel\-\_\-\-Elec\-Ew\-\_\-\-Vdw\-C\-S\-Tab\-\_\-\-Geom\-W3\-W3\-\_\-sse2\-\_\-double.\-c} }{\pageref{nb__kernel__ElecEw__VdwCSTab__GeomW3W3__sse2__double_8c}}{}
\item\contentsline{section}{src/gmxlib/nonbonded/nb\-\_\-kernel\-\_\-sse2\-\_\-double/\hyperlink{nb__kernel__ElecEw__VdwCSTab__GeomW4P1__sse2__double_8c}{nb\-\_\-kernel\-\_\-\-Elec\-Ew\-\_\-\-Vdw\-C\-S\-Tab\-\_\-\-Geom\-W4\-P1\-\_\-sse2\-\_\-double.\-c} }{\pageref{nb__kernel__ElecEw__VdwCSTab__GeomW4P1__sse2__double_8c}}{}
\item\contentsline{section}{src/gmxlib/nonbonded/nb\-\_\-kernel\-\_\-sse2\-\_\-double/\hyperlink{nb__kernel__ElecEw__VdwCSTab__GeomW4W4__sse2__double_8c}{nb\-\_\-kernel\-\_\-\-Elec\-Ew\-\_\-\-Vdw\-C\-S\-Tab\-\_\-\-Geom\-W4\-W4\-\_\-sse2\-\_\-double.\-c} }{\pageref{nb__kernel__ElecEw__VdwCSTab__GeomW4W4__sse2__double_8c}}{}
\item\contentsline{section}{src/gmxlib/nonbonded/nb\-\_\-kernel\-\_\-sse2\-\_\-double/\hyperlink{nb__kernel__ElecEw__VdwLJ__GeomP1P1__sse2__double_8c}{nb\-\_\-kernel\-\_\-\-Elec\-Ew\-\_\-\-Vdw\-L\-J\-\_\-\-Geom\-P1\-P1\-\_\-sse2\-\_\-double.\-c} }{\pageref{nb__kernel__ElecEw__VdwLJ__GeomP1P1__sse2__double_8c}}{}
\item\contentsline{section}{src/gmxlib/nonbonded/nb\-\_\-kernel\-\_\-sse2\-\_\-double/\hyperlink{nb__kernel__ElecEw__VdwLJ__GeomW3P1__sse2__double_8c}{nb\-\_\-kernel\-\_\-\-Elec\-Ew\-\_\-\-Vdw\-L\-J\-\_\-\-Geom\-W3\-P1\-\_\-sse2\-\_\-double.\-c} }{\pageref{nb__kernel__ElecEw__VdwLJ__GeomW3P1__sse2__double_8c}}{}
\item\contentsline{section}{src/gmxlib/nonbonded/nb\-\_\-kernel\-\_\-sse2\-\_\-double/\hyperlink{nb__kernel__ElecEw__VdwLJ__GeomW3W3__sse2__double_8c}{nb\-\_\-kernel\-\_\-\-Elec\-Ew\-\_\-\-Vdw\-L\-J\-\_\-\-Geom\-W3\-W3\-\_\-sse2\-\_\-double.\-c} }{\pageref{nb__kernel__ElecEw__VdwLJ__GeomW3W3__sse2__double_8c}}{}
\item\contentsline{section}{src/gmxlib/nonbonded/nb\-\_\-kernel\-\_\-sse2\-\_\-double/\hyperlink{nb__kernel__ElecEw__VdwLJ__GeomW4P1__sse2__double_8c}{nb\-\_\-kernel\-\_\-\-Elec\-Ew\-\_\-\-Vdw\-L\-J\-\_\-\-Geom\-W4\-P1\-\_\-sse2\-\_\-double.\-c} }{\pageref{nb__kernel__ElecEw__VdwLJ__GeomW4P1__sse2__double_8c}}{}
\item\contentsline{section}{src/gmxlib/nonbonded/nb\-\_\-kernel\-\_\-sse2\-\_\-double/\hyperlink{nb__kernel__ElecEw__VdwLJ__GeomW4W4__sse2__double_8c}{nb\-\_\-kernel\-\_\-\-Elec\-Ew\-\_\-\-Vdw\-L\-J\-\_\-\-Geom\-W4\-W4\-\_\-sse2\-\_\-double.\-c} }{\pageref{nb__kernel__ElecEw__VdwLJ__GeomW4W4__sse2__double_8c}}{}
\item\contentsline{section}{src/gmxlib/nonbonded/nb\-\_\-kernel\-\_\-sse2\-\_\-double/\hyperlink{nb__kernel__ElecEw__VdwNone__GeomP1P1__sse2__double_8c}{nb\-\_\-kernel\-\_\-\-Elec\-Ew\-\_\-\-Vdw\-None\-\_\-\-Geom\-P1\-P1\-\_\-sse2\-\_\-double.\-c} }{\pageref{nb__kernel__ElecEw__VdwNone__GeomP1P1__sse2__double_8c}}{}
\item\contentsline{section}{src/gmxlib/nonbonded/nb\-\_\-kernel\-\_\-sse2\-\_\-double/\hyperlink{nb__kernel__ElecEw__VdwNone__GeomW3P1__sse2__double_8c}{nb\-\_\-kernel\-\_\-\-Elec\-Ew\-\_\-\-Vdw\-None\-\_\-\-Geom\-W3\-P1\-\_\-sse2\-\_\-double.\-c} }{\pageref{nb__kernel__ElecEw__VdwNone__GeomW3P1__sse2__double_8c}}{}
\item\contentsline{section}{src/gmxlib/nonbonded/nb\-\_\-kernel\-\_\-sse2\-\_\-double/\hyperlink{nb__kernel__ElecEw__VdwNone__GeomW3W3__sse2__double_8c}{nb\-\_\-kernel\-\_\-\-Elec\-Ew\-\_\-\-Vdw\-None\-\_\-\-Geom\-W3\-W3\-\_\-sse2\-\_\-double.\-c} }{\pageref{nb__kernel__ElecEw__VdwNone__GeomW3W3__sse2__double_8c}}{}
\item\contentsline{section}{src/gmxlib/nonbonded/nb\-\_\-kernel\-\_\-sse2\-\_\-double/\hyperlink{nb__kernel__ElecEw__VdwNone__GeomW4P1__sse2__double_8c}{nb\-\_\-kernel\-\_\-\-Elec\-Ew\-\_\-\-Vdw\-None\-\_\-\-Geom\-W4\-P1\-\_\-sse2\-\_\-double.\-c} }{\pageref{nb__kernel__ElecEw__VdwNone__GeomW4P1__sse2__double_8c}}{}
\item\contentsline{section}{src/gmxlib/nonbonded/nb\-\_\-kernel\-\_\-sse2\-\_\-double/\hyperlink{nb__kernel__ElecEw__VdwNone__GeomW4W4__sse2__double_8c}{nb\-\_\-kernel\-\_\-\-Elec\-Ew\-\_\-\-Vdw\-None\-\_\-\-Geom\-W4\-W4\-\_\-sse2\-\_\-double.\-c} }{\pageref{nb__kernel__ElecEw__VdwNone__GeomW4W4__sse2__double_8c}}{}
\item\contentsline{section}{src/gmxlib/nonbonded/nb\-\_\-kernel\-\_\-sse2\-\_\-double/\hyperlink{nb__kernel__ElecEwSh__VdwLJSh__GeomP1P1__sse2__double_8c}{nb\-\_\-kernel\-\_\-\-Elec\-Ew\-Sh\-\_\-\-Vdw\-L\-J\-Sh\-\_\-\-Geom\-P1\-P1\-\_\-sse2\-\_\-double.\-c} }{\pageref{nb__kernel__ElecEwSh__VdwLJSh__GeomP1P1__sse2__double_8c}}{}
\item\contentsline{section}{src/gmxlib/nonbonded/nb\-\_\-kernel\-\_\-sse2\-\_\-double/\hyperlink{nb__kernel__ElecEwSh__VdwLJSh__GeomW3P1__sse2__double_8c}{nb\-\_\-kernel\-\_\-\-Elec\-Ew\-Sh\-\_\-\-Vdw\-L\-J\-Sh\-\_\-\-Geom\-W3\-P1\-\_\-sse2\-\_\-double.\-c} }{\pageref{nb__kernel__ElecEwSh__VdwLJSh__GeomW3P1__sse2__double_8c}}{}
\item\contentsline{section}{src/gmxlib/nonbonded/nb\-\_\-kernel\-\_\-sse2\-\_\-double/\hyperlink{nb__kernel__ElecEwSh__VdwLJSh__GeomW3W3__sse2__double_8c}{nb\-\_\-kernel\-\_\-\-Elec\-Ew\-Sh\-\_\-\-Vdw\-L\-J\-Sh\-\_\-\-Geom\-W3\-W3\-\_\-sse2\-\_\-double.\-c} }{\pageref{nb__kernel__ElecEwSh__VdwLJSh__GeomW3W3__sse2__double_8c}}{}
\item\contentsline{section}{src/gmxlib/nonbonded/nb\-\_\-kernel\-\_\-sse2\-\_\-double/\hyperlink{nb__kernel__ElecEwSh__VdwLJSh__GeomW4P1__sse2__double_8c}{nb\-\_\-kernel\-\_\-\-Elec\-Ew\-Sh\-\_\-\-Vdw\-L\-J\-Sh\-\_\-\-Geom\-W4\-P1\-\_\-sse2\-\_\-double.\-c} }{\pageref{nb__kernel__ElecEwSh__VdwLJSh__GeomW4P1__sse2__double_8c}}{}
\item\contentsline{section}{src/gmxlib/nonbonded/nb\-\_\-kernel\-\_\-sse2\-\_\-double/\hyperlink{nb__kernel__ElecEwSh__VdwLJSh__GeomW4W4__sse2__double_8c}{nb\-\_\-kernel\-\_\-\-Elec\-Ew\-Sh\-\_\-\-Vdw\-L\-J\-Sh\-\_\-\-Geom\-W4\-W4\-\_\-sse2\-\_\-double.\-c} }{\pageref{nb__kernel__ElecEwSh__VdwLJSh__GeomW4W4__sse2__double_8c}}{}
\item\contentsline{section}{src/gmxlib/nonbonded/nb\-\_\-kernel\-\_\-sse2\-\_\-double/\hyperlink{nb__kernel__ElecEwSh__VdwNone__GeomP1P1__sse2__double_8c}{nb\-\_\-kernel\-\_\-\-Elec\-Ew\-Sh\-\_\-\-Vdw\-None\-\_\-\-Geom\-P1\-P1\-\_\-sse2\-\_\-double.\-c} }{\pageref{nb__kernel__ElecEwSh__VdwNone__GeomP1P1__sse2__double_8c}}{}
\item\contentsline{section}{src/gmxlib/nonbonded/nb\-\_\-kernel\-\_\-sse2\-\_\-double/\hyperlink{nb__kernel__ElecEwSh__VdwNone__GeomW3P1__sse2__double_8c}{nb\-\_\-kernel\-\_\-\-Elec\-Ew\-Sh\-\_\-\-Vdw\-None\-\_\-\-Geom\-W3\-P1\-\_\-sse2\-\_\-double.\-c} }{\pageref{nb__kernel__ElecEwSh__VdwNone__GeomW3P1__sse2__double_8c}}{}
\item\contentsline{section}{src/gmxlib/nonbonded/nb\-\_\-kernel\-\_\-sse2\-\_\-double/\hyperlink{nb__kernel__ElecEwSh__VdwNone__GeomW3W3__sse2__double_8c}{nb\-\_\-kernel\-\_\-\-Elec\-Ew\-Sh\-\_\-\-Vdw\-None\-\_\-\-Geom\-W3\-W3\-\_\-sse2\-\_\-double.\-c} }{\pageref{nb__kernel__ElecEwSh__VdwNone__GeomW3W3__sse2__double_8c}}{}
\item\contentsline{section}{src/gmxlib/nonbonded/nb\-\_\-kernel\-\_\-sse2\-\_\-double/\hyperlink{nb__kernel__ElecEwSh__VdwNone__GeomW4P1__sse2__double_8c}{nb\-\_\-kernel\-\_\-\-Elec\-Ew\-Sh\-\_\-\-Vdw\-None\-\_\-\-Geom\-W4\-P1\-\_\-sse2\-\_\-double.\-c} }{\pageref{nb__kernel__ElecEwSh__VdwNone__GeomW4P1__sse2__double_8c}}{}
\item\contentsline{section}{src/gmxlib/nonbonded/nb\-\_\-kernel\-\_\-sse2\-\_\-double/\hyperlink{nb__kernel__ElecEwSh__VdwNone__GeomW4W4__sse2__double_8c}{nb\-\_\-kernel\-\_\-\-Elec\-Ew\-Sh\-\_\-\-Vdw\-None\-\_\-\-Geom\-W4\-W4\-\_\-sse2\-\_\-double.\-c} }{\pageref{nb__kernel__ElecEwSh__VdwNone__GeomW4W4__sse2__double_8c}}{}
\item\contentsline{section}{src/gmxlib/nonbonded/nb\-\_\-kernel\-\_\-sse2\-\_\-double/\hyperlink{nb__kernel__ElecEwSw__VdwLJSw__GeomP1P1__sse2__double_8c}{nb\-\_\-kernel\-\_\-\-Elec\-Ew\-Sw\-\_\-\-Vdw\-L\-J\-Sw\-\_\-\-Geom\-P1\-P1\-\_\-sse2\-\_\-double.\-c} }{\pageref{nb__kernel__ElecEwSw__VdwLJSw__GeomP1P1__sse2__double_8c}}{}
\item\contentsline{section}{src/gmxlib/nonbonded/nb\-\_\-kernel\-\_\-sse2\-\_\-double/\hyperlink{nb__kernel__ElecEwSw__VdwLJSw__GeomW3P1__sse2__double_8c}{nb\-\_\-kernel\-\_\-\-Elec\-Ew\-Sw\-\_\-\-Vdw\-L\-J\-Sw\-\_\-\-Geom\-W3\-P1\-\_\-sse2\-\_\-double.\-c} }{\pageref{nb__kernel__ElecEwSw__VdwLJSw__GeomW3P1__sse2__double_8c}}{}
\item\contentsline{section}{src/gmxlib/nonbonded/nb\-\_\-kernel\-\_\-sse2\-\_\-double/\hyperlink{nb__kernel__ElecEwSw__VdwLJSw__GeomW3W3__sse2__double_8c}{nb\-\_\-kernel\-\_\-\-Elec\-Ew\-Sw\-\_\-\-Vdw\-L\-J\-Sw\-\_\-\-Geom\-W3\-W3\-\_\-sse2\-\_\-double.\-c} }{\pageref{nb__kernel__ElecEwSw__VdwLJSw__GeomW3W3__sse2__double_8c}}{}
\item\contentsline{section}{src/gmxlib/nonbonded/nb\-\_\-kernel\-\_\-sse2\-\_\-double/\hyperlink{nb__kernel__ElecEwSw__VdwLJSw__GeomW4P1__sse2__double_8c}{nb\-\_\-kernel\-\_\-\-Elec\-Ew\-Sw\-\_\-\-Vdw\-L\-J\-Sw\-\_\-\-Geom\-W4\-P1\-\_\-sse2\-\_\-double.\-c} }{\pageref{nb__kernel__ElecEwSw__VdwLJSw__GeomW4P1__sse2__double_8c}}{}
\item\contentsline{section}{src/gmxlib/nonbonded/nb\-\_\-kernel\-\_\-sse2\-\_\-double/\hyperlink{nb__kernel__ElecEwSw__VdwLJSw__GeomW4W4__sse2__double_8c}{nb\-\_\-kernel\-\_\-\-Elec\-Ew\-Sw\-\_\-\-Vdw\-L\-J\-Sw\-\_\-\-Geom\-W4\-W4\-\_\-sse2\-\_\-double.\-c} }{\pageref{nb__kernel__ElecEwSw__VdwLJSw__GeomW4W4__sse2__double_8c}}{}
\item\contentsline{section}{src/gmxlib/nonbonded/nb\-\_\-kernel\-\_\-sse2\-\_\-double/\hyperlink{nb__kernel__ElecEwSw__VdwNone__GeomP1P1__sse2__double_8c}{nb\-\_\-kernel\-\_\-\-Elec\-Ew\-Sw\-\_\-\-Vdw\-None\-\_\-\-Geom\-P1\-P1\-\_\-sse2\-\_\-double.\-c} }{\pageref{nb__kernel__ElecEwSw__VdwNone__GeomP1P1__sse2__double_8c}}{}
\item\contentsline{section}{src/gmxlib/nonbonded/nb\-\_\-kernel\-\_\-sse2\-\_\-double/\hyperlink{nb__kernel__ElecEwSw__VdwNone__GeomW3P1__sse2__double_8c}{nb\-\_\-kernel\-\_\-\-Elec\-Ew\-Sw\-\_\-\-Vdw\-None\-\_\-\-Geom\-W3\-P1\-\_\-sse2\-\_\-double.\-c} }{\pageref{nb__kernel__ElecEwSw__VdwNone__GeomW3P1__sse2__double_8c}}{}
\item\contentsline{section}{src/gmxlib/nonbonded/nb\-\_\-kernel\-\_\-sse2\-\_\-double/\hyperlink{nb__kernel__ElecEwSw__VdwNone__GeomW3W3__sse2__double_8c}{nb\-\_\-kernel\-\_\-\-Elec\-Ew\-Sw\-\_\-\-Vdw\-None\-\_\-\-Geom\-W3\-W3\-\_\-sse2\-\_\-double.\-c} }{\pageref{nb__kernel__ElecEwSw__VdwNone__GeomW3W3__sse2__double_8c}}{}
\item\contentsline{section}{src/gmxlib/nonbonded/nb\-\_\-kernel\-\_\-sse2\-\_\-double/\hyperlink{nb__kernel__ElecEwSw__VdwNone__GeomW4P1__sse2__double_8c}{nb\-\_\-kernel\-\_\-\-Elec\-Ew\-Sw\-\_\-\-Vdw\-None\-\_\-\-Geom\-W4\-P1\-\_\-sse2\-\_\-double.\-c} }{\pageref{nb__kernel__ElecEwSw__VdwNone__GeomW4P1__sse2__double_8c}}{}
\item\contentsline{section}{src/gmxlib/nonbonded/nb\-\_\-kernel\-\_\-sse2\-\_\-double/\hyperlink{nb__kernel__ElecEwSw__VdwNone__GeomW4W4__sse2__double_8c}{nb\-\_\-kernel\-\_\-\-Elec\-Ew\-Sw\-\_\-\-Vdw\-None\-\_\-\-Geom\-W4\-W4\-\_\-sse2\-\_\-double.\-c} }{\pageref{nb__kernel__ElecEwSw__VdwNone__GeomW4W4__sse2__double_8c}}{}
\item\contentsline{section}{src/gmxlib/nonbonded/nb\-\_\-kernel\-\_\-sse2\-\_\-double/\hyperlink{nb__kernel__ElecGB__VdwCSTab__GeomP1P1__sse2__double_8c}{nb\-\_\-kernel\-\_\-\-Elec\-G\-B\-\_\-\-Vdw\-C\-S\-Tab\-\_\-\-Geom\-P1\-P1\-\_\-sse2\-\_\-double.\-c} }{\pageref{nb__kernel__ElecGB__VdwCSTab__GeomP1P1__sse2__double_8c}}{}
\item\contentsline{section}{src/gmxlib/nonbonded/nb\-\_\-kernel\-\_\-sse2\-\_\-double/\hyperlink{nb__kernel__ElecGB__VdwLJ__GeomP1P1__sse2__double_8c}{nb\-\_\-kernel\-\_\-\-Elec\-G\-B\-\_\-\-Vdw\-L\-J\-\_\-\-Geom\-P1\-P1\-\_\-sse2\-\_\-double.\-c} }{\pageref{nb__kernel__ElecGB__VdwLJ__GeomP1P1__sse2__double_8c}}{}
\item\contentsline{section}{src/gmxlib/nonbonded/nb\-\_\-kernel\-\_\-sse2\-\_\-double/\hyperlink{nb__kernel__ElecGB__VdwNone__GeomP1P1__sse2__double_8c}{nb\-\_\-kernel\-\_\-\-Elec\-G\-B\-\_\-\-Vdw\-None\-\_\-\-Geom\-P1\-P1\-\_\-sse2\-\_\-double.\-c} }{\pageref{nb__kernel__ElecGB__VdwNone__GeomP1P1__sse2__double_8c}}{}
\item\contentsline{section}{src/gmxlib/nonbonded/nb\-\_\-kernel\-\_\-sse2\-\_\-double/\hyperlink{nb__kernel__ElecNone__VdwCSTab__GeomP1P1__sse2__double_8c}{nb\-\_\-kernel\-\_\-\-Elec\-None\-\_\-\-Vdw\-C\-S\-Tab\-\_\-\-Geom\-P1\-P1\-\_\-sse2\-\_\-double.\-c} }{\pageref{nb__kernel__ElecNone__VdwCSTab__GeomP1P1__sse2__double_8c}}{}
\item\contentsline{section}{src/gmxlib/nonbonded/nb\-\_\-kernel\-\_\-sse2\-\_\-double/\hyperlink{nb__kernel__ElecNone__VdwLJ__GeomP1P1__sse2__double_8c}{nb\-\_\-kernel\-\_\-\-Elec\-None\-\_\-\-Vdw\-L\-J\-\_\-\-Geom\-P1\-P1\-\_\-sse2\-\_\-double.\-c} }{\pageref{nb__kernel__ElecNone__VdwLJ__GeomP1P1__sse2__double_8c}}{}
\item\contentsline{section}{src/gmxlib/nonbonded/nb\-\_\-kernel\-\_\-sse2\-\_\-double/\hyperlink{nb__kernel__ElecNone__VdwLJSh__GeomP1P1__sse2__double_8c}{nb\-\_\-kernel\-\_\-\-Elec\-None\-\_\-\-Vdw\-L\-J\-Sh\-\_\-\-Geom\-P1\-P1\-\_\-sse2\-\_\-double.\-c} }{\pageref{nb__kernel__ElecNone__VdwLJSh__GeomP1P1__sse2__double_8c}}{}
\item\contentsline{section}{src/gmxlib/nonbonded/nb\-\_\-kernel\-\_\-sse2\-\_\-double/\hyperlink{nb__kernel__ElecNone__VdwLJSw__GeomP1P1__sse2__double_8c}{nb\-\_\-kernel\-\_\-\-Elec\-None\-\_\-\-Vdw\-L\-J\-Sw\-\_\-\-Geom\-P1\-P1\-\_\-sse2\-\_\-double.\-c} }{\pageref{nb__kernel__ElecNone__VdwLJSw__GeomP1P1__sse2__double_8c}}{}
\item\contentsline{section}{src/gmxlib/nonbonded/nb\-\_\-kernel\-\_\-sse2\-\_\-double/\hyperlink{nb__kernel__ElecRF__VdwCSTab__GeomP1P1__sse2__double_8c}{nb\-\_\-kernel\-\_\-\-Elec\-R\-F\-\_\-\-Vdw\-C\-S\-Tab\-\_\-\-Geom\-P1\-P1\-\_\-sse2\-\_\-double.\-c} }{\pageref{nb__kernel__ElecRF__VdwCSTab__GeomP1P1__sse2__double_8c}}{}
\item\contentsline{section}{src/gmxlib/nonbonded/nb\-\_\-kernel\-\_\-sse2\-\_\-double/\hyperlink{nb__kernel__ElecRF__VdwCSTab__GeomW3P1__sse2__double_8c}{nb\-\_\-kernel\-\_\-\-Elec\-R\-F\-\_\-\-Vdw\-C\-S\-Tab\-\_\-\-Geom\-W3\-P1\-\_\-sse2\-\_\-double.\-c} }{\pageref{nb__kernel__ElecRF__VdwCSTab__GeomW3P1__sse2__double_8c}}{}
\item\contentsline{section}{src/gmxlib/nonbonded/nb\-\_\-kernel\-\_\-sse2\-\_\-double/\hyperlink{nb__kernel__ElecRF__VdwCSTab__GeomW3W3__sse2__double_8c}{nb\-\_\-kernel\-\_\-\-Elec\-R\-F\-\_\-\-Vdw\-C\-S\-Tab\-\_\-\-Geom\-W3\-W3\-\_\-sse2\-\_\-double.\-c} }{\pageref{nb__kernel__ElecRF__VdwCSTab__GeomW3W3__sse2__double_8c}}{}
\item\contentsline{section}{src/gmxlib/nonbonded/nb\-\_\-kernel\-\_\-sse2\-\_\-double/\hyperlink{nb__kernel__ElecRF__VdwCSTab__GeomW4P1__sse2__double_8c}{nb\-\_\-kernel\-\_\-\-Elec\-R\-F\-\_\-\-Vdw\-C\-S\-Tab\-\_\-\-Geom\-W4\-P1\-\_\-sse2\-\_\-double.\-c} }{\pageref{nb__kernel__ElecRF__VdwCSTab__GeomW4P1__sse2__double_8c}}{}
\item\contentsline{section}{src/gmxlib/nonbonded/nb\-\_\-kernel\-\_\-sse2\-\_\-double/\hyperlink{nb__kernel__ElecRF__VdwCSTab__GeomW4W4__sse2__double_8c}{nb\-\_\-kernel\-\_\-\-Elec\-R\-F\-\_\-\-Vdw\-C\-S\-Tab\-\_\-\-Geom\-W4\-W4\-\_\-sse2\-\_\-double.\-c} }{\pageref{nb__kernel__ElecRF__VdwCSTab__GeomW4W4__sse2__double_8c}}{}
\item\contentsline{section}{src/gmxlib/nonbonded/nb\-\_\-kernel\-\_\-sse2\-\_\-double/\hyperlink{nb__kernel__ElecRF__VdwLJ__GeomP1P1__sse2__double_8c}{nb\-\_\-kernel\-\_\-\-Elec\-R\-F\-\_\-\-Vdw\-L\-J\-\_\-\-Geom\-P1\-P1\-\_\-sse2\-\_\-double.\-c} }{\pageref{nb__kernel__ElecRF__VdwLJ__GeomP1P1__sse2__double_8c}}{}
\item\contentsline{section}{src/gmxlib/nonbonded/nb\-\_\-kernel\-\_\-sse2\-\_\-double/\hyperlink{nb__kernel__ElecRF__VdwLJ__GeomW3P1__sse2__double_8c}{nb\-\_\-kernel\-\_\-\-Elec\-R\-F\-\_\-\-Vdw\-L\-J\-\_\-\-Geom\-W3\-P1\-\_\-sse2\-\_\-double.\-c} }{\pageref{nb__kernel__ElecRF__VdwLJ__GeomW3P1__sse2__double_8c}}{}
\item\contentsline{section}{src/gmxlib/nonbonded/nb\-\_\-kernel\-\_\-sse2\-\_\-double/\hyperlink{nb__kernel__ElecRF__VdwLJ__GeomW3W3__sse2__double_8c}{nb\-\_\-kernel\-\_\-\-Elec\-R\-F\-\_\-\-Vdw\-L\-J\-\_\-\-Geom\-W3\-W3\-\_\-sse2\-\_\-double.\-c} }{\pageref{nb__kernel__ElecRF__VdwLJ__GeomW3W3__sse2__double_8c}}{}
\item\contentsline{section}{src/gmxlib/nonbonded/nb\-\_\-kernel\-\_\-sse2\-\_\-double/\hyperlink{nb__kernel__ElecRF__VdwLJ__GeomW4P1__sse2__double_8c}{nb\-\_\-kernel\-\_\-\-Elec\-R\-F\-\_\-\-Vdw\-L\-J\-\_\-\-Geom\-W4\-P1\-\_\-sse2\-\_\-double.\-c} }{\pageref{nb__kernel__ElecRF__VdwLJ__GeomW4P1__sse2__double_8c}}{}
\item\contentsline{section}{src/gmxlib/nonbonded/nb\-\_\-kernel\-\_\-sse2\-\_\-double/\hyperlink{nb__kernel__ElecRF__VdwLJ__GeomW4W4__sse2__double_8c}{nb\-\_\-kernel\-\_\-\-Elec\-R\-F\-\_\-\-Vdw\-L\-J\-\_\-\-Geom\-W4\-W4\-\_\-sse2\-\_\-double.\-c} }{\pageref{nb__kernel__ElecRF__VdwLJ__GeomW4W4__sse2__double_8c}}{}
\item\contentsline{section}{src/gmxlib/nonbonded/nb\-\_\-kernel\-\_\-sse2\-\_\-double/\hyperlink{nb__kernel__ElecRF__VdwNone__GeomP1P1__sse2__double_8c}{nb\-\_\-kernel\-\_\-\-Elec\-R\-F\-\_\-\-Vdw\-None\-\_\-\-Geom\-P1\-P1\-\_\-sse2\-\_\-double.\-c} }{\pageref{nb__kernel__ElecRF__VdwNone__GeomP1P1__sse2__double_8c}}{}
\item\contentsline{section}{src/gmxlib/nonbonded/nb\-\_\-kernel\-\_\-sse2\-\_\-double/\hyperlink{nb__kernel__ElecRF__VdwNone__GeomW3P1__sse2__double_8c}{nb\-\_\-kernel\-\_\-\-Elec\-R\-F\-\_\-\-Vdw\-None\-\_\-\-Geom\-W3\-P1\-\_\-sse2\-\_\-double.\-c} }{\pageref{nb__kernel__ElecRF__VdwNone__GeomW3P1__sse2__double_8c}}{}
\item\contentsline{section}{src/gmxlib/nonbonded/nb\-\_\-kernel\-\_\-sse2\-\_\-double/\hyperlink{nb__kernel__ElecRF__VdwNone__GeomW3W3__sse2__double_8c}{nb\-\_\-kernel\-\_\-\-Elec\-R\-F\-\_\-\-Vdw\-None\-\_\-\-Geom\-W3\-W3\-\_\-sse2\-\_\-double.\-c} }{\pageref{nb__kernel__ElecRF__VdwNone__GeomW3W3__sse2__double_8c}}{}
\item\contentsline{section}{src/gmxlib/nonbonded/nb\-\_\-kernel\-\_\-sse2\-\_\-double/\hyperlink{nb__kernel__ElecRF__VdwNone__GeomW4P1__sse2__double_8c}{nb\-\_\-kernel\-\_\-\-Elec\-R\-F\-\_\-\-Vdw\-None\-\_\-\-Geom\-W4\-P1\-\_\-sse2\-\_\-double.\-c} }{\pageref{nb__kernel__ElecRF__VdwNone__GeomW4P1__sse2__double_8c}}{}
\item\contentsline{section}{src/gmxlib/nonbonded/nb\-\_\-kernel\-\_\-sse2\-\_\-double/\hyperlink{nb__kernel__ElecRF__VdwNone__GeomW4W4__sse2__double_8c}{nb\-\_\-kernel\-\_\-\-Elec\-R\-F\-\_\-\-Vdw\-None\-\_\-\-Geom\-W4\-W4\-\_\-sse2\-\_\-double.\-c} }{\pageref{nb__kernel__ElecRF__VdwNone__GeomW4W4__sse2__double_8c}}{}
\item\contentsline{section}{src/gmxlib/nonbonded/nb\-\_\-kernel\-\_\-sse2\-\_\-double/\hyperlink{nb__kernel__ElecRFCut__VdwCSTab__GeomP1P1__sse2__double_8c}{nb\-\_\-kernel\-\_\-\-Elec\-R\-F\-Cut\-\_\-\-Vdw\-C\-S\-Tab\-\_\-\-Geom\-P1\-P1\-\_\-sse2\-\_\-double.\-c} }{\pageref{nb__kernel__ElecRFCut__VdwCSTab__GeomP1P1__sse2__double_8c}}{}
\item\contentsline{section}{src/gmxlib/nonbonded/nb\-\_\-kernel\-\_\-sse2\-\_\-double/\hyperlink{nb__kernel__ElecRFCut__VdwCSTab__GeomW3P1__sse2__double_8c}{nb\-\_\-kernel\-\_\-\-Elec\-R\-F\-Cut\-\_\-\-Vdw\-C\-S\-Tab\-\_\-\-Geom\-W3\-P1\-\_\-sse2\-\_\-double.\-c} }{\pageref{nb__kernel__ElecRFCut__VdwCSTab__GeomW3P1__sse2__double_8c}}{}
\item\contentsline{section}{src/gmxlib/nonbonded/nb\-\_\-kernel\-\_\-sse2\-\_\-double/\hyperlink{nb__kernel__ElecRFCut__VdwCSTab__GeomW3W3__sse2__double_8c}{nb\-\_\-kernel\-\_\-\-Elec\-R\-F\-Cut\-\_\-\-Vdw\-C\-S\-Tab\-\_\-\-Geom\-W3\-W3\-\_\-sse2\-\_\-double.\-c} }{\pageref{nb__kernel__ElecRFCut__VdwCSTab__GeomW3W3__sse2__double_8c}}{}
\item\contentsline{section}{src/gmxlib/nonbonded/nb\-\_\-kernel\-\_\-sse2\-\_\-double/\hyperlink{nb__kernel__ElecRFCut__VdwCSTab__GeomW4P1__sse2__double_8c}{nb\-\_\-kernel\-\_\-\-Elec\-R\-F\-Cut\-\_\-\-Vdw\-C\-S\-Tab\-\_\-\-Geom\-W4\-P1\-\_\-sse2\-\_\-double.\-c} }{\pageref{nb__kernel__ElecRFCut__VdwCSTab__GeomW4P1__sse2__double_8c}}{}
\item\contentsline{section}{src/gmxlib/nonbonded/nb\-\_\-kernel\-\_\-sse2\-\_\-double/\hyperlink{nb__kernel__ElecRFCut__VdwCSTab__GeomW4W4__sse2__double_8c}{nb\-\_\-kernel\-\_\-\-Elec\-R\-F\-Cut\-\_\-\-Vdw\-C\-S\-Tab\-\_\-\-Geom\-W4\-W4\-\_\-sse2\-\_\-double.\-c} }{\pageref{nb__kernel__ElecRFCut__VdwCSTab__GeomW4W4__sse2__double_8c}}{}
\item\contentsline{section}{src/gmxlib/nonbonded/nb\-\_\-kernel\-\_\-sse2\-\_\-double/\hyperlink{nb__kernel__ElecRFCut__VdwLJSh__GeomP1P1__sse2__double_8c}{nb\-\_\-kernel\-\_\-\-Elec\-R\-F\-Cut\-\_\-\-Vdw\-L\-J\-Sh\-\_\-\-Geom\-P1\-P1\-\_\-sse2\-\_\-double.\-c} }{\pageref{nb__kernel__ElecRFCut__VdwLJSh__GeomP1P1__sse2__double_8c}}{}
\item\contentsline{section}{src/gmxlib/nonbonded/nb\-\_\-kernel\-\_\-sse2\-\_\-double/\hyperlink{nb__kernel__ElecRFCut__VdwLJSh__GeomW3P1__sse2__double_8c}{nb\-\_\-kernel\-\_\-\-Elec\-R\-F\-Cut\-\_\-\-Vdw\-L\-J\-Sh\-\_\-\-Geom\-W3\-P1\-\_\-sse2\-\_\-double.\-c} }{\pageref{nb__kernel__ElecRFCut__VdwLJSh__GeomW3P1__sse2__double_8c}}{}
\item\contentsline{section}{src/gmxlib/nonbonded/nb\-\_\-kernel\-\_\-sse2\-\_\-double/\hyperlink{nb__kernel__ElecRFCut__VdwLJSh__GeomW3W3__sse2__double_8c}{nb\-\_\-kernel\-\_\-\-Elec\-R\-F\-Cut\-\_\-\-Vdw\-L\-J\-Sh\-\_\-\-Geom\-W3\-W3\-\_\-sse2\-\_\-double.\-c} }{\pageref{nb__kernel__ElecRFCut__VdwLJSh__GeomW3W3__sse2__double_8c}}{}
\item\contentsline{section}{src/gmxlib/nonbonded/nb\-\_\-kernel\-\_\-sse2\-\_\-double/\hyperlink{nb__kernel__ElecRFCut__VdwLJSh__GeomW4P1__sse2__double_8c}{nb\-\_\-kernel\-\_\-\-Elec\-R\-F\-Cut\-\_\-\-Vdw\-L\-J\-Sh\-\_\-\-Geom\-W4\-P1\-\_\-sse2\-\_\-double.\-c} }{\pageref{nb__kernel__ElecRFCut__VdwLJSh__GeomW4P1__sse2__double_8c}}{}
\item\contentsline{section}{src/gmxlib/nonbonded/nb\-\_\-kernel\-\_\-sse2\-\_\-double/\hyperlink{nb__kernel__ElecRFCut__VdwLJSh__GeomW4W4__sse2__double_8c}{nb\-\_\-kernel\-\_\-\-Elec\-R\-F\-Cut\-\_\-\-Vdw\-L\-J\-Sh\-\_\-\-Geom\-W4\-W4\-\_\-sse2\-\_\-double.\-c} }{\pageref{nb__kernel__ElecRFCut__VdwLJSh__GeomW4W4__sse2__double_8c}}{}
\item\contentsline{section}{src/gmxlib/nonbonded/nb\-\_\-kernel\-\_\-sse2\-\_\-double/\hyperlink{nb__kernel__ElecRFCut__VdwLJSw__GeomP1P1__sse2__double_8c}{nb\-\_\-kernel\-\_\-\-Elec\-R\-F\-Cut\-\_\-\-Vdw\-L\-J\-Sw\-\_\-\-Geom\-P1\-P1\-\_\-sse2\-\_\-double.\-c} }{\pageref{nb__kernel__ElecRFCut__VdwLJSw__GeomP1P1__sse2__double_8c}}{}
\item\contentsline{section}{src/gmxlib/nonbonded/nb\-\_\-kernel\-\_\-sse2\-\_\-double/\hyperlink{nb__kernel__ElecRFCut__VdwLJSw__GeomW3P1__sse2__double_8c}{nb\-\_\-kernel\-\_\-\-Elec\-R\-F\-Cut\-\_\-\-Vdw\-L\-J\-Sw\-\_\-\-Geom\-W3\-P1\-\_\-sse2\-\_\-double.\-c} }{\pageref{nb__kernel__ElecRFCut__VdwLJSw__GeomW3P1__sse2__double_8c}}{}
\item\contentsline{section}{src/gmxlib/nonbonded/nb\-\_\-kernel\-\_\-sse2\-\_\-double/\hyperlink{nb__kernel__ElecRFCut__VdwLJSw__GeomW3W3__sse2__double_8c}{nb\-\_\-kernel\-\_\-\-Elec\-R\-F\-Cut\-\_\-\-Vdw\-L\-J\-Sw\-\_\-\-Geom\-W3\-W3\-\_\-sse2\-\_\-double.\-c} }{\pageref{nb__kernel__ElecRFCut__VdwLJSw__GeomW3W3__sse2__double_8c}}{}
\item\contentsline{section}{src/gmxlib/nonbonded/nb\-\_\-kernel\-\_\-sse2\-\_\-double/\hyperlink{nb__kernel__ElecRFCut__VdwLJSw__GeomW4P1__sse2__double_8c}{nb\-\_\-kernel\-\_\-\-Elec\-R\-F\-Cut\-\_\-\-Vdw\-L\-J\-Sw\-\_\-\-Geom\-W4\-P1\-\_\-sse2\-\_\-double.\-c} }{\pageref{nb__kernel__ElecRFCut__VdwLJSw__GeomW4P1__sse2__double_8c}}{}
\item\contentsline{section}{src/gmxlib/nonbonded/nb\-\_\-kernel\-\_\-sse2\-\_\-double/\hyperlink{nb__kernel__ElecRFCut__VdwLJSw__GeomW4W4__sse2__double_8c}{nb\-\_\-kernel\-\_\-\-Elec\-R\-F\-Cut\-\_\-\-Vdw\-L\-J\-Sw\-\_\-\-Geom\-W4\-W4\-\_\-sse2\-\_\-double.\-c} }{\pageref{nb__kernel__ElecRFCut__VdwLJSw__GeomW4W4__sse2__double_8c}}{}
\item\contentsline{section}{src/gmxlib/nonbonded/nb\-\_\-kernel\-\_\-sse2\-\_\-double/\hyperlink{nb__kernel__ElecRFCut__VdwNone__GeomP1P1__sse2__double_8c}{nb\-\_\-kernel\-\_\-\-Elec\-R\-F\-Cut\-\_\-\-Vdw\-None\-\_\-\-Geom\-P1\-P1\-\_\-sse2\-\_\-double.\-c} }{\pageref{nb__kernel__ElecRFCut__VdwNone__GeomP1P1__sse2__double_8c}}{}
\item\contentsline{section}{src/gmxlib/nonbonded/nb\-\_\-kernel\-\_\-sse2\-\_\-double/\hyperlink{nb__kernel__ElecRFCut__VdwNone__GeomW3P1__sse2__double_8c}{nb\-\_\-kernel\-\_\-\-Elec\-R\-F\-Cut\-\_\-\-Vdw\-None\-\_\-\-Geom\-W3\-P1\-\_\-sse2\-\_\-double.\-c} }{\pageref{nb__kernel__ElecRFCut__VdwNone__GeomW3P1__sse2__double_8c}}{}
\item\contentsline{section}{src/gmxlib/nonbonded/nb\-\_\-kernel\-\_\-sse2\-\_\-double/\hyperlink{nb__kernel__ElecRFCut__VdwNone__GeomW3W3__sse2__double_8c}{nb\-\_\-kernel\-\_\-\-Elec\-R\-F\-Cut\-\_\-\-Vdw\-None\-\_\-\-Geom\-W3\-W3\-\_\-sse2\-\_\-double.\-c} }{\pageref{nb__kernel__ElecRFCut__VdwNone__GeomW3W3__sse2__double_8c}}{}
\item\contentsline{section}{src/gmxlib/nonbonded/nb\-\_\-kernel\-\_\-sse2\-\_\-double/\hyperlink{nb__kernel__ElecRFCut__VdwNone__GeomW4P1__sse2__double_8c}{nb\-\_\-kernel\-\_\-\-Elec\-R\-F\-Cut\-\_\-\-Vdw\-None\-\_\-\-Geom\-W4\-P1\-\_\-sse2\-\_\-double.\-c} }{\pageref{nb__kernel__ElecRFCut__VdwNone__GeomW4P1__sse2__double_8c}}{}
\item\contentsline{section}{src/gmxlib/nonbonded/nb\-\_\-kernel\-\_\-sse2\-\_\-double/\hyperlink{nb__kernel__ElecRFCut__VdwNone__GeomW4W4__sse2__double_8c}{nb\-\_\-kernel\-\_\-\-Elec\-R\-F\-Cut\-\_\-\-Vdw\-None\-\_\-\-Geom\-W4\-W4\-\_\-sse2\-\_\-double.\-c} }{\pageref{nb__kernel__ElecRFCut__VdwNone__GeomW4W4__sse2__double_8c}}{}
\item\contentsline{section}{src/gmxlib/nonbonded/nb\-\_\-kernel\-\_\-sse2\-\_\-double/\hyperlink{nb__kernel__sse2__double_8c}{nb\-\_\-kernel\-\_\-sse2\-\_\-double.\-c} }{\pageref{nb__kernel__sse2__double_8c}}{}
\item\contentsline{section}{src/gmxlib/nonbonded/nb\-\_\-kernel\-\_\-sse2\-\_\-double/\hyperlink{nb__kernel__sse2__double_8h}{nb\-\_\-kernel\-\_\-sse2\-\_\-double.\-h} }{\pageref{nb__kernel__sse2__double_8h}}{}
\item\contentsline{section}{src/gmxlib/nonbonded/nb\-\_\-kernel\-\_\-sse2\-\_\-single/\hyperlink{kernelutil__x86__sse2__single_8h}{kernelutil\-\_\-x86\-\_\-sse2\-\_\-single.\-h} }{\pageref{kernelutil__x86__sse2__single_8h}}{}
\item\contentsline{section}{src/gmxlib/nonbonded/nb\-\_\-kernel\-\_\-sse2\-\_\-single/\hyperlink{nb__kernel__ElecCoul__VdwCSTab__GeomP1P1__sse2__single_8c}{nb\-\_\-kernel\-\_\-\-Elec\-Coul\-\_\-\-Vdw\-C\-S\-Tab\-\_\-\-Geom\-P1\-P1\-\_\-sse2\-\_\-single.\-c} }{\pageref{nb__kernel__ElecCoul__VdwCSTab__GeomP1P1__sse2__single_8c}}{}
\item\contentsline{section}{src/gmxlib/nonbonded/nb\-\_\-kernel\-\_\-sse2\-\_\-single/\hyperlink{nb__kernel__ElecCoul__VdwCSTab__GeomW3P1__sse2__single_8c}{nb\-\_\-kernel\-\_\-\-Elec\-Coul\-\_\-\-Vdw\-C\-S\-Tab\-\_\-\-Geom\-W3\-P1\-\_\-sse2\-\_\-single.\-c} }{\pageref{nb__kernel__ElecCoul__VdwCSTab__GeomW3P1__sse2__single_8c}}{}
\item\contentsline{section}{src/gmxlib/nonbonded/nb\-\_\-kernel\-\_\-sse2\-\_\-single/\hyperlink{nb__kernel__ElecCoul__VdwCSTab__GeomW3W3__sse2__single_8c}{nb\-\_\-kernel\-\_\-\-Elec\-Coul\-\_\-\-Vdw\-C\-S\-Tab\-\_\-\-Geom\-W3\-W3\-\_\-sse2\-\_\-single.\-c} }{\pageref{nb__kernel__ElecCoul__VdwCSTab__GeomW3W3__sse2__single_8c}}{}
\item\contentsline{section}{src/gmxlib/nonbonded/nb\-\_\-kernel\-\_\-sse2\-\_\-single/\hyperlink{nb__kernel__ElecCoul__VdwCSTab__GeomW4P1__sse2__single_8c}{nb\-\_\-kernel\-\_\-\-Elec\-Coul\-\_\-\-Vdw\-C\-S\-Tab\-\_\-\-Geom\-W4\-P1\-\_\-sse2\-\_\-single.\-c} }{\pageref{nb__kernel__ElecCoul__VdwCSTab__GeomW4P1__sse2__single_8c}}{}
\item\contentsline{section}{src/gmxlib/nonbonded/nb\-\_\-kernel\-\_\-sse2\-\_\-single/\hyperlink{nb__kernel__ElecCoul__VdwCSTab__GeomW4W4__sse2__single_8c}{nb\-\_\-kernel\-\_\-\-Elec\-Coul\-\_\-\-Vdw\-C\-S\-Tab\-\_\-\-Geom\-W4\-W4\-\_\-sse2\-\_\-single.\-c} }{\pageref{nb__kernel__ElecCoul__VdwCSTab__GeomW4W4__sse2__single_8c}}{}
\item\contentsline{section}{src/gmxlib/nonbonded/nb\-\_\-kernel\-\_\-sse2\-\_\-single/\hyperlink{nb__kernel__ElecCoul__VdwLJ__GeomP1P1__sse2__single_8c}{nb\-\_\-kernel\-\_\-\-Elec\-Coul\-\_\-\-Vdw\-L\-J\-\_\-\-Geom\-P1\-P1\-\_\-sse2\-\_\-single.\-c} }{\pageref{nb__kernel__ElecCoul__VdwLJ__GeomP1P1__sse2__single_8c}}{}
\item\contentsline{section}{src/gmxlib/nonbonded/nb\-\_\-kernel\-\_\-sse2\-\_\-single/\hyperlink{nb__kernel__ElecCoul__VdwLJ__GeomW3P1__sse2__single_8c}{nb\-\_\-kernel\-\_\-\-Elec\-Coul\-\_\-\-Vdw\-L\-J\-\_\-\-Geom\-W3\-P1\-\_\-sse2\-\_\-single.\-c} }{\pageref{nb__kernel__ElecCoul__VdwLJ__GeomW3P1__sse2__single_8c}}{}
\item\contentsline{section}{src/gmxlib/nonbonded/nb\-\_\-kernel\-\_\-sse2\-\_\-single/\hyperlink{nb__kernel__ElecCoul__VdwLJ__GeomW3W3__sse2__single_8c}{nb\-\_\-kernel\-\_\-\-Elec\-Coul\-\_\-\-Vdw\-L\-J\-\_\-\-Geom\-W3\-W3\-\_\-sse2\-\_\-single.\-c} }{\pageref{nb__kernel__ElecCoul__VdwLJ__GeomW3W3__sse2__single_8c}}{}
\item\contentsline{section}{src/gmxlib/nonbonded/nb\-\_\-kernel\-\_\-sse2\-\_\-single/\hyperlink{nb__kernel__ElecCoul__VdwLJ__GeomW4P1__sse2__single_8c}{nb\-\_\-kernel\-\_\-\-Elec\-Coul\-\_\-\-Vdw\-L\-J\-\_\-\-Geom\-W4\-P1\-\_\-sse2\-\_\-single.\-c} }{\pageref{nb__kernel__ElecCoul__VdwLJ__GeomW4P1__sse2__single_8c}}{}
\item\contentsline{section}{src/gmxlib/nonbonded/nb\-\_\-kernel\-\_\-sse2\-\_\-single/\hyperlink{nb__kernel__ElecCoul__VdwLJ__GeomW4W4__sse2__single_8c}{nb\-\_\-kernel\-\_\-\-Elec\-Coul\-\_\-\-Vdw\-L\-J\-\_\-\-Geom\-W4\-W4\-\_\-sse2\-\_\-single.\-c} }{\pageref{nb__kernel__ElecCoul__VdwLJ__GeomW4W4__sse2__single_8c}}{}
\item\contentsline{section}{src/gmxlib/nonbonded/nb\-\_\-kernel\-\_\-sse2\-\_\-single/\hyperlink{nb__kernel__ElecCoul__VdwNone__GeomP1P1__sse2__single_8c}{nb\-\_\-kernel\-\_\-\-Elec\-Coul\-\_\-\-Vdw\-None\-\_\-\-Geom\-P1\-P1\-\_\-sse2\-\_\-single.\-c} }{\pageref{nb__kernel__ElecCoul__VdwNone__GeomP1P1__sse2__single_8c}}{}
\item\contentsline{section}{src/gmxlib/nonbonded/nb\-\_\-kernel\-\_\-sse2\-\_\-single/\hyperlink{nb__kernel__ElecCoul__VdwNone__GeomW3P1__sse2__single_8c}{nb\-\_\-kernel\-\_\-\-Elec\-Coul\-\_\-\-Vdw\-None\-\_\-\-Geom\-W3\-P1\-\_\-sse2\-\_\-single.\-c} }{\pageref{nb__kernel__ElecCoul__VdwNone__GeomW3P1__sse2__single_8c}}{}
\item\contentsline{section}{src/gmxlib/nonbonded/nb\-\_\-kernel\-\_\-sse2\-\_\-single/\hyperlink{nb__kernel__ElecCoul__VdwNone__GeomW3W3__sse2__single_8c}{nb\-\_\-kernel\-\_\-\-Elec\-Coul\-\_\-\-Vdw\-None\-\_\-\-Geom\-W3\-W3\-\_\-sse2\-\_\-single.\-c} }{\pageref{nb__kernel__ElecCoul__VdwNone__GeomW3W3__sse2__single_8c}}{}
\item\contentsline{section}{src/gmxlib/nonbonded/nb\-\_\-kernel\-\_\-sse2\-\_\-single/\hyperlink{nb__kernel__ElecCoul__VdwNone__GeomW4P1__sse2__single_8c}{nb\-\_\-kernel\-\_\-\-Elec\-Coul\-\_\-\-Vdw\-None\-\_\-\-Geom\-W4\-P1\-\_\-sse2\-\_\-single.\-c} }{\pageref{nb__kernel__ElecCoul__VdwNone__GeomW4P1__sse2__single_8c}}{}
\item\contentsline{section}{src/gmxlib/nonbonded/nb\-\_\-kernel\-\_\-sse2\-\_\-single/\hyperlink{nb__kernel__ElecCoul__VdwNone__GeomW4W4__sse2__single_8c}{nb\-\_\-kernel\-\_\-\-Elec\-Coul\-\_\-\-Vdw\-None\-\_\-\-Geom\-W4\-W4\-\_\-sse2\-\_\-single.\-c} }{\pageref{nb__kernel__ElecCoul__VdwNone__GeomW4W4__sse2__single_8c}}{}
\item\contentsline{section}{src/gmxlib/nonbonded/nb\-\_\-kernel\-\_\-sse2\-\_\-single/\hyperlink{nb__kernel__ElecCSTab__VdwCSTab__GeomP1P1__sse2__single_8c}{nb\-\_\-kernel\-\_\-\-Elec\-C\-S\-Tab\-\_\-\-Vdw\-C\-S\-Tab\-\_\-\-Geom\-P1\-P1\-\_\-sse2\-\_\-single.\-c} }{\pageref{nb__kernel__ElecCSTab__VdwCSTab__GeomP1P1__sse2__single_8c}}{}
\item\contentsline{section}{src/gmxlib/nonbonded/nb\-\_\-kernel\-\_\-sse2\-\_\-single/\hyperlink{nb__kernel__ElecCSTab__VdwCSTab__GeomW3P1__sse2__single_8c}{nb\-\_\-kernel\-\_\-\-Elec\-C\-S\-Tab\-\_\-\-Vdw\-C\-S\-Tab\-\_\-\-Geom\-W3\-P1\-\_\-sse2\-\_\-single.\-c} }{\pageref{nb__kernel__ElecCSTab__VdwCSTab__GeomW3P1__sse2__single_8c}}{}
\item\contentsline{section}{src/gmxlib/nonbonded/nb\-\_\-kernel\-\_\-sse2\-\_\-single/\hyperlink{nb__kernel__ElecCSTab__VdwCSTab__GeomW3W3__sse2__single_8c}{nb\-\_\-kernel\-\_\-\-Elec\-C\-S\-Tab\-\_\-\-Vdw\-C\-S\-Tab\-\_\-\-Geom\-W3\-W3\-\_\-sse2\-\_\-single.\-c} }{\pageref{nb__kernel__ElecCSTab__VdwCSTab__GeomW3W3__sse2__single_8c}}{}
\item\contentsline{section}{src/gmxlib/nonbonded/nb\-\_\-kernel\-\_\-sse2\-\_\-single/\hyperlink{nb__kernel__ElecCSTab__VdwCSTab__GeomW4P1__sse2__single_8c}{nb\-\_\-kernel\-\_\-\-Elec\-C\-S\-Tab\-\_\-\-Vdw\-C\-S\-Tab\-\_\-\-Geom\-W4\-P1\-\_\-sse2\-\_\-single.\-c} }{\pageref{nb__kernel__ElecCSTab__VdwCSTab__GeomW4P1__sse2__single_8c}}{}
\item\contentsline{section}{src/gmxlib/nonbonded/nb\-\_\-kernel\-\_\-sse2\-\_\-single/\hyperlink{nb__kernel__ElecCSTab__VdwCSTab__GeomW4W4__sse2__single_8c}{nb\-\_\-kernel\-\_\-\-Elec\-C\-S\-Tab\-\_\-\-Vdw\-C\-S\-Tab\-\_\-\-Geom\-W4\-W4\-\_\-sse2\-\_\-single.\-c} }{\pageref{nb__kernel__ElecCSTab__VdwCSTab__GeomW4W4__sse2__single_8c}}{}
\item\contentsline{section}{src/gmxlib/nonbonded/nb\-\_\-kernel\-\_\-sse2\-\_\-single/\hyperlink{nb__kernel__ElecCSTab__VdwLJ__GeomP1P1__sse2__single_8c}{nb\-\_\-kernel\-\_\-\-Elec\-C\-S\-Tab\-\_\-\-Vdw\-L\-J\-\_\-\-Geom\-P1\-P1\-\_\-sse2\-\_\-single.\-c} }{\pageref{nb__kernel__ElecCSTab__VdwLJ__GeomP1P1__sse2__single_8c}}{}
\item\contentsline{section}{src/gmxlib/nonbonded/nb\-\_\-kernel\-\_\-sse2\-\_\-single/\hyperlink{nb__kernel__ElecCSTab__VdwLJ__GeomW3P1__sse2__single_8c}{nb\-\_\-kernel\-\_\-\-Elec\-C\-S\-Tab\-\_\-\-Vdw\-L\-J\-\_\-\-Geom\-W3\-P1\-\_\-sse2\-\_\-single.\-c} }{\pageref{nb__kernel__ElecCSTab__VdwLJ__GeomW3P1__sse2__single_8c}}{}
\item\contentsline{section}{src/gmxlib/nonbonded/nb\-\_\-kernel\-\_\-sse2\-\_\-single/\hyperlink{nb__kernel__ElecCSTab__VdwLJ__GeomW3W3__sse2__single_8c}{nb\-\_\-kernel\-\_\-\-Elec\-C\-S\-Tab\-\_\-\-Vdw\-L\-J\-\_\-\-Geom\-W3\-W3\-\_\-sse2\-\_\-single.\-c} }{\pageref{nb__kernel__ElecCSTab__VdwLJ__GeomW3W3__sse2__single_8c}}{}
\item\contentsline{section}{src/gmxlib/nonbonded/nb\-\_\-kernel\-\_\-sse2\-\_\-single/\hyperlink{nb__kernel__ElecCSTab__VdwLJ__GeomW4P1__sse2__single_8c}{nb\-\_\-kernel\-\_\-\-Elec\-C\-S\-Tab\-\_\-\-Vdw\-L\-J\-\_\-\-Geom\-W4\-P1\-\_\-sse2\-\_\-single.\-c} }{\pageref{nb__kernel__ElecCSTab__VdwLJ__GeomW4P1__sse2__single_8c}}{}
\item\contentsline{section}{src/gmxlib/nonbonded/nb\-\_\-kernel\-\_\-sse2\-\_\-single/\hyperlink{nb__kernel__ElecCSTab__VdwLJ__GeomW4W4__sse2__single_8c}{nb\-\_\-kernel\-\_\-\-Elec\-C\-S\-Tab\-\_\-\-Vdw\-L\-J\-\_\-\-Geom\-W4\-W4\-\_\-sse2\-\_\-single.\-c} }{\pageref{nb__kernel__ElecCSTab__VdwLJ__GeomW4W4__sse2__single_8c}}{}
\item\contentsline{section}{src/gmxlib/nonbonded/nb\-\_\-kernel\-\_\-sse2\-\_\-single/\hyperlink{nb__kernel__ElecCSTab__VdwNone__GeomP1P1__sse2__single_8c}{nb\-\_\-kernel\-\_\-\-Elec\-C\-S\-Tab\-\_\-\-Vdw\-None\-\_\-\-Geom\-P1\-P1\-\_\-sse2\-\_\-single.\-c} }{\pageref{nb__kernel__ElecCSTab__VdwNone__GeomP1P1__sse2__single_8c}}{}
\item\contentsline{section}{src/gmxlib/nonbonded/nb\-\_\-kernel\-\_\-sse2\-\_\-single/\hyperlink{nb__kernel__ElecCSTab__VdwNone__GeomW3P1__sse2__single_8c}{nb\-\_\-kernel\-\_\-\-Elec\-C\-S\-Tab\-\_\-\-Vdw\-None\-\_\-\-Geom\-W3\-P1\-\_\-sse2\-\_\-single.\-c} }{\pageref{nb__kernel__ElecCSTab__VdwNone__GeomW3P1__sse2__single_8c}}{}
\item\contentsline{section}{src/gmxlib/nonbonded/nb\-\_\-kernel\-\_\-sse2\-\_\-single/\hyperlink{nb__kernel__ElecCSTab__VdwNone__GeomW3W3__sse2__single_8c}{nb\-\_\-kernel\-\_\-\-Elec\-C\-S\-Tab\-\_\-\-Vdw\-None\-\_\-\-Geom\-W3\-W3\-\_\-sse2\-\_\-single.\-c} }{\pageref{nb__kernel__ElecCSTab__VdwNone__GeomW3W3__sse2__single_8c}}{}
\item\contentsline{section}{src/gmxlib/nonbonded/nb\-\_\-kernel\-\_\-sse2\-\_\-single/\hyperlink{nb__kernel__ElecCSTab__VdwNone__GeomW4P1__sse2__single_8c}{nb\-\_\-kernel\-\_\-\-Elec\-C\-S\-Tab\-\_\-\-Vdw\-None\-\_\-\-Geom\-W4\-P1\-\_\-sse2\-\_\-single.\-c} }{\pageref{nb__kernel__ElecCSTab__VdwNone__GeomW4P1__sse2__single_8c}}{}
\item\contentsline{section}{src/gmxlib/nonbonded/nb\-\_\-kernel\-\_\-sse2\-\_\-single/\hyperlink{nb__kernel__ElecCSTab__VdwNone__GeomW4W4__sse2__single_8c}{nb\-\_\-kernel\-\_\-\-Elec\-C\-S\-Tab\-\_\-\-Vdw\-None\-\_\-\-Geom\-W4\-W4\-\_\-sse2\-\_\-single.\-c} }{\pageref{nb__kernel__ElecCSTab__VdwNone__GeomW4W4__sse2__single_8c}}{}
\item\contentsline{section}{src/gmxlib/nonbonded/nb\-\_\-kernel\-\_\-sse2\-\_\-single/\hyperlink{nb__kernel__ElecEw__VdwCSTab__GeomP1P1__sse2__single_8c}{nb\-\_\-kernel\-\_\-\-Elec\-Ew\-\_\-\-Vdw\-C\-S\-Tab\-\_\-\-Geom\-P1\-P1\-\_\-sse2\-\_\-single.\-c} }{\pageref{nb__kernel__ElecEw__VdwCSTab__GeomP1P1__sse2__single_8c}}{}
\item\contentsline{section}{src/gmxlib/nonbonded/nb\-\_\-kernel\-\_\-sse2\-\_\-single/\hyperlink{nb__kernel__ElecEw__VdwCSTab__GeomW3P1__sse2__single_8c}{nb\-\_\-kernel\-\_\-\-Elec\-Ew\-\_\-\-Vdw\-C\-S\-Tab\-\_\-\-Geom\-W3\-P1\-\_\-sse2\-\_\-single.\-c} }{\pageref{nb__kernel__ElecEw__VdwCSTab__GeomW3P1__sse2__single_8c}}{}
\item\contentsline{section}{src/gmxlib/nonbonded/nb\-\_\-kernel\-\_\-sse2\-\_\-single/\hyperlink{nb__kernel__ElecEw__VdwCSTab__GeomW3W3__sse2__single_8c}{nb\-\_\-kernel\-\_\-\-Elec\-Ew\-\_\-\-Vdw\-C\-S\-Tab\-\_\-\-Geom\-W3\-W3\-\_\-sse2\-\_\-single.\-c} }{\pageref{nb__kernel__ElecEw__VdwCSTab__GeomW3W3__sse2__single_8c}}{}
\item\contentsline{section}{src/gmxlib/nonbonded/nb\-\_\-kernel\-\_\-sse2\-\_\-single/\hyperlink{nb__kernel__ElecEw__VdwCSTab__GeomW4P1__sse2__single_8c}{nb\-\_\-kernel\-\_\-\-Elec\-Ew\-\_\-\-Vdw\-C\-S\-Tab\-\_\-\-Geom\-W4\-P1\-\_\-sse2\-\_\-single.\-c} }{\pageref{nb__kernel__ElecEw__VdwCSTab__GeomW4P1__sse2__single_8c}}{}
\item\contentsline{section}{src/gmxlib/nonbonded/nb\-\_\-kernel\-\_\-sse2\-\_\-single/\hyperlink{nb__kernel__ElecEw__VdwCSTab__GeomW4W4__sse2__single_8c}{nb\-\_\-kernel\-\_\-\-Elec\-Ew\-\_\-\-Vdw\-C\-S\-Tab\-\_\-\-Geom\-W4\-W4\-\_\-sse2\-\_\-single.\-c} }{\pageref{nb__kernel__ElecEw__VdwCSTab__GeomW4W4__sse2__single_8c}}{}
\item\contentsline{section}{src/gmxlib/nonbonded/nb\-\_\-kernel\-\_\-sse2\-\_\-single/\hyperlink{nb__kernel__ElecEw__VdwLJ__GeomP1P1__sse2__single_8c}{nb\-\_\-kernel\-\_\-\-Elec\-Ew\-\_\-\-Vdw\-L\-J\-\_\-\-Geom\-P1\-P1\-\_\-sse2\-\_\-single.\-c} }{\pageref{nb__kernel__ElecEw__VdwLJ__GeomP1P1__sse2__single_8c}}{}
\item\contentsline{section}{src/gmxlib/nonbonded/nb\-\_\-kernel\-\_\-sse2\-\_\-single/\hyperlink{nb__kernel__ElecEw__VdwLJ__GeomW3P1__sse2__single_8c}{nb\-\_\-kernel\-\_\-\-Elec\-Ew\-\_\-\-Vdw\-L\-J\-\_\-\-Geom\-W3\-P1\-\_\-sse2\-\_\-single.\-c} }{\pageref{nb__kernel__ElecEw__VdwLJ__GeomW3P1__sse2__single_8c}}{}
\item\contentsline{section}{src/gmxlib/nonbonded/nb\-\_\-kernel\-\_\-sse2\-\_\-single/\hyperlink{nb__kernel__ElecEw__VdwLJ__GeomW3W3__sse2__single_8c}{nb\-\_\-kernel\-\_\-\-Elec\-Ew\-\_\-\-Vdw\-L\-J\-\_\-\-Geom\-W3\-W3\-\_\-sse2\-\_\-single.\-c} }{\pageref{nb__kernel__ElecEw__VdwLJ__GeomW3W3__sse2__single_8c}}{}
\item\contentsline{section}{src/gmxlib/nonbonded/nb\-\_\-kernel\-\_\-sse2\-\_\-single/\hyperlink{nb__kernel__ElecEw__VdwLJ__GeomW4P1__sse2__single_8c}{nb\-\_\-kernel\-\_\-\-Elec\-Ew\-\_\-\-Vdw\-L\-J\-\_\-\-Geom\-W4\-P1\-\_\-sse2\-\_\-single.\-c} }{\pageref{nb__kernel__ElecEw__VdwLJ__GeomW4P1__sse2__single_8c}}{}
\item\contentsline{section}{src/gmxlib/nonbonded/nb\-\_\-kernel\-\_\-sse2\-\_\-single/\hyperlink{nb__kernel__ElecEw__VdwLJ__GeomW4W4__sse2__single_8c}{nb\-\_\-kernel\-\_\-\-Elec\-Ew\-\_\-\-Vdw\-L\-J\-\_\-\-Geom\-W4\-W4\-\_\-sse2\-\_\-single.\-c} }{\pageref{nb__kernel__ElecEw__VdwLJ__GeomW4W4__sse2__single_8c}}{}
\item\contentsline{section}{src/gmxlib/nonbonded/nb\-\_\-kernel\-\_\-sse2\-\_\-single/\hyperlink{nb__kernel__ElecEw__VdwNone__GeomP1P1__sse2__single_8c}{nb\-\_\-kernel\-\_\-\-Elec\-Ew\-\_\-\-Vdw\-None\-\_\-\-Geom\-P1\-P1\-\_\-sse2\-\_\-single.\-c} }{\pageref{nb__kernel__ElecEw__VdwNone__GeomP1P1__sse2__single_8c}}{}
\item\contentsline{section}{src/gmxlib/nonbonded/nb\-\_\-kernel\-\_\-sse2\-\_\-single/\hyperlink{nb__kernel__ElecEw__VdwNone__GeomW3P1__sse2__single_8c}{nb\-\_\-kernel\-\_\-\-Elec\-Ew\-\_\-\-Vdw\-None\-\_\-\-Geom\-W3\-P1\-\_\-sse2\-\_\-single.\-c} }{\pageref{nb__kernel__ElecEw__VdwNone__GeomW3P1__sse2__single_8c}}{}
\item\contentsline{section}{src/gmxlib/nonbonded/nb\-\_\-kernel\-\_\-sse2\-\_\-single/\hyperlink{nb__kernel__ElecEw__VdwNone__GeomW3W3__sse2__single_8c}{nb\-\_\-kernel\-\_\-\-Elec\-Ew\-\_\-\-Vdw\-None\-\_\-\-Geom\-W3\-W3\-\_\-sse2\-\_\-single.\-c} }{\pageref{nb__kernel__ElecEw__VdwNone__GeomW3W3__sse2__single_8c}}{}
\item\contentsline{section}{src/gmxlib/nonbonded/nb\-\_\-kernel\-\_\-sse2\-\_\-single/\hyperlink{nb__kernel__ElecEw__VdwNone__GeomW4P1__sse2__single_8c}{nb\-\_\-kernel\-\_\-\-Elec\-Ew\-\_\-\-Vdw\-None\-\_\-\-Geom\-W4\-P1\-\_\-sse2\-\_\-single.\-c} }{\pageref{nb__kernel__ElecEw__VdwNone__GeomW4P1__sse2__single_8c}}{}
\item\contentsline{section}{src/gmxlib/nonbonded/nb\-\_\-kernel\-\_\-sse2\-\_\-single/\hyperlink{nb__kernel__ElecEw__VdwNone__GeomW4W4__sse2__single_8c}{nb\-\_\-kernel\-\_\-\-Elec\-Ew\-\_\-\-Vdw\-None\-\_\-\-Geom\-W4\-W4\-\_\-sse2\-\_\-single.\-c} }{\pageref{nb__kernel__ElecEw__VdwNone__GeomW4W4__sse2__single_8c}}{}
\item\contentsline{section}{src/gmxlib/nonbonded/nb\-\_\-kernel\-\_\-sse2\-\_\-single/\hyperlink{nb__kernel__ElecEwSh__VdwLJSh__GeomP1P1__sse2__single_8c}{nb\-\_\-kernel\-\_\-\-Elec\-Ew\-Sh\-\_\-\-Vdw\-L\-J\-Sh\-\_\-\-Geom\-P1\-P1\-\_\-sse2\-\_\-single.\-c} }{\pageref{nb__kernel__ElecEwSh__VdwLJSh__GeomP1P1__sse2__single_8c}}{}
\item\contentsline{section}{src/gmxlib/nonbonded/nb\-\_\-kernel\-\_\-sse2\-\_\-single/\hyperlink{nb__kernel__ElecEwSh__VdwLJSh__GeomW3P1__sse2__single_8c}{nb\-\_\-kernel\-\_\-\-Elec\-Ew\-Sh\-\_\-\-Vdw\-L\-J\-Sh\-\_\-\-Geom\-W3\-P1\-\_\-sse2\-\_\-single.\-c} }{\pageref{nb__kernel__ElecEwSh__VdwLJSh__GeomW3P1__sse2__single_8c}}{}
\item\contentsline{section}{src/gmxlib/nonbonded/nb\-\_\-kernel\-\_\-sse2\-\_\-single/\hyperlink{nb__kernel__ElecEwSh__VdwLJSh__GeomW3W3__sse2__single_8c}{nb\-\_\-kernel\-\_\-\-Elec\-Ew\-Sh\-\_\-\-Vdw\-L\-J\-Sh\-\_\-\-Geom\-W3\-W3\-\_\-sse2\-\_\-single.\-c} }{\pageref{nb__kernel__ElecEwSh__VdwLJSh__GeomW3W3__sse2__single_8c}}{}
\item\contentsline{section}{src/gmxlib/nonbonded/nb\-\_\-kernel\-\_\-sse2\-\_\-single/\hyperlink{nb__kernel__ElecEwSh__VdwLJSh__GeomW4P1__sse2__single_8c}{nb\-\_\-kernel\-\_\-\-Elec\-Ew\-Sh\-\_\-\-Vdw\-L\-J\-Sh\-\_\-\-Geom\-W4\-P1\-\_\-sse2\-\_\-single.\-c} }{\pageref{nb__kernel__ElecEwSh__VdwLJSh__GeomW4P1__sse2__single_8c}}{}
\item\contentsline{section}{src/gmxlib/nonbonded/nb\-\_\-kernel\-\_\-sse2\-\_\-single/\hyperlink{nb__kernel__ElecEwSh__VdwLJSh__GeomW4W4__sse2__single_8c}{nb\-\_\-kernel\-\_\-\-Elec\-Ew\-Sh\-\_\-\-Vdw\-L\-J\-Sh\-\_\-\-Geom\-W4\-W4\-\_\-sse2\-\_\-single.\-c} }{\pageref{nb__kernel__ElecEwSh__VdwLJSh__GeomW4W4__sse2__single_8c}}{}
\item\contentsline{section}{src/gmxlib/nonbonded/nb\-\_\-kernel\-\_\-sse2\-\_\-single/\hyperlink{nb__kernel__ElecEwSh__VdwNone__GeomP1P1__sse2__single_8c}{nb\-\_\-kernel\-\_\-\-Elec\-Ew\-Sh\-\_\-\-Vdw\-None\-\_\-\-Geom\-P1\-P1\-\_\-sse2\-\_\-single.\-c} }{\pageref{nb__kernel__ElecEwSh__VdwNone__GeomP1P1__sse2__single_8c}}{}
\item\contentsline{section}{src/gmxlib/nonbonded/nb\-\_\-kernel\-\_\-sse2\-\_\-single/\hyperlink{nb__kernel__ElecEwSh__VdwNone__GeomW3P1__sse2__single_8c}{nb\-\_\-kernel\-\_\-\-Elec\-Ew\-Sh\-\_\-\-Vdw\-None\-\_\-\-Geom\-W3\-P1\-\_\-sse2\-\_\-single.\-c} }{\pageref{nb__kernel__ElecEwSh__VdwNone__GeomW3P1__sse2__single_8c}}{}
\item\contentsline{section}{src/gmxlib/nonbonded/nb\-\_\-kernel\-\_\-sse2\-\_\-single/\hyperlink{nb__kernel__ElecEwSh__VdwNone__GeomW3W3__sse2__single_8c}{nb\-\_\-kernel\-\_\-\-Elec\-Ew\-Sh\-\_\-\-Vdw\-None\-\_\-\-Geom\-W3\-W3\-\_\-sse2\-\_\-single.\-c} }{\pageref{nb__kernel__ElecEwSh__VdwNone__GeomW3W3__sse2__single_8c}}{}
\item\contentsline{section}{src/gmxlib/nonbonded/nb\-\_\-kernel\-\_\-sse2\-\_\-single/\hyperlink{nb__kernel__ElecEwSh__VdwNone__GeomW4P1__sse2__single_8c}{nb\-\_\-kernel\-\_\-\-Elec\-Ew\-Sh\-\_\-\-Vdw\-None\-\_\-\-Geom\-W4\-P1\-\_\-sse2\-\_\-single.\-c} }{\pageref{nb__kernel__ElecEwSh__VdwNone__GeomW4P1__sse2__single_8c}}{}
\item\contentsline{section}{src/gmxlib/nonbonded/nb\-\_\-kernel\-\_\-sse2\-\_\-single/\hyperlink{nb__kernel__ElecEwSh__VdwNone__GeomW4W4__sse2__single_8c}{nb\-\_\-kernel\-\_\-\-Elec\-Ew\-Sh\-\_\-\-Vdw\-None\-\_\-\-Geom\-W4\-W4\-\_\-sse2\-\_\-single.\-c} }{\pageref{nb__kernel__ElecEwSh__VdwNone__GeomW4W4__sse2__single_8c}}{}
\item\contentsline{section}{src/gmxlib/nonbonded/nb\-\_\-kernel\-\_\-sse2\-\_\-single/\hyperlink{nb__kernel__ElecEwSw__VdwLJSw__GeomP1P1__sse2__single_8c}{nb\-\_\-kernel\-\_\-\-Elec\-Ew\-Sw\-\_\-\-Vdw\-L\-J\-Sw\-\_\-\-Geom\-P1\-P1\-\_\-sse2\-\_\-single.\-c} }{\pageref{nb__kernel__ElecEwSw__VdwLJSw__GeomP1P1__sse2__single_8c}}{}
\item\contentsline{section}{src/gmxlib/nonbonded/nb\-\_\-kernel\-\_\-sse2\-\_\-single/\hyperlink{nb__kernel__ElecEwSw__VdwLJSw__GeomW3P1__sse2__single_8c}{nb\-\_\-kernel\-\_\-\-Elec\-Ew\-Sw\-\_\-\-Vdw\-L\-J\-Sw\-\_\-\-Geom\-W3\-P1\-\_\-sse2\-\_\-single.\-c} }{\pageref{nb__kernel__ElecEwSw__VdwLJSw__GeomW3P1__sse2__single_8c}}{}
\item\contentsline{section}{src/gmxlib/nonbonded/nb\-\_\-kernel\-\_\-sse2\-\_\-single/\hyperlink{nb__kernel__ElecEwSw__VdwLJSw__GeomW3W3__sse2__single_8c}{nb\-\_\-kernel\-\_\-\-Elec\-Ew\-Sw\-\_\-\-Vdw\-L\-J\-Sw\-\_\-\-Geom\-W3\-W3\-\_\-sse2\-\_\-single.\-c} }{\pageref{nb__kernel__ElecEwSw__VdwLJSw__GeomW3W3__sse2__single_8c}}{}
\item\contentsline{section}{src/gmxlib/nonbonded/nb\-\_\-kernel\-\_\-sse2\-\_\-single/\hyperlink{nb__kernel__ElecEwSw__VdwLJSw__GeomW4P1__sse2__single_8c}{nb\-\_\-kernel\-\_\-\-Elec\-Ew\-Sw\-\_\-\-Vdw\-L\-J\-Sw\-\_\-\-Geom\-W4\-P1\-\_\-sse2\-\_\-single.\-c} }{\pageref{nb__kernel__ElecEwSw__VdwLJSw__GeomW4P1__sse2__single_8c}}{}
\item\contentsline{section}{src/gmxlib/nonbonded/nb\-\_\-kernel\-\_\-sse2\-\_\-single/\hyperlink{nb__kernel__ElecEwSw__VdwLJSw__GeomW4W4__sse2__single_8c}{nb\-\_\-kernel\-\_\-\-Elec\-Ew\-Sw\-\_\-\-Vdw\-L\-J\-Sw\-\_\-\-Geom\-W4\-W4\-\_\-sse2\-\_\-single.\-c} }{\pageref{nb__kernel__ElecEwSw__VdwLJSw__GeomW4W4__sse2__single_8c}}{}
\item\contentsline{section}{src/gmxlib/nonbonded/nb\-\_\-kernel\-\_\-sse2\-\_\-single/\hyperlink{nb__kernel__ElecEwSw__VdwNone__GeomP1P1__sse2__single_8c}{nb\-\_\-kernel\-\_\-\-Elec\-Ew\-Sw\-\_\-\-Vdw\-None\-\_\-\-Geom\-P1\-P1\-\_\-sse2\-\_\-single.\-c} }{\pageref{nb__kernel__ElecEwSw__VdwNone__GeomP1P1__sse2__single_8c}}{}
\item\contentsline{section}{src/gmxlib/nonbonded/nb\-\_\-kernel\-\_\-sse2\-\_\-single/\hyperlink{nb__kernel__ElecEwSw__VdwNone__GeomW3P1__sse2__single_8c}{nb\-\_\-kernel\-\_\-\-Elec\-Ew\-Sw\-\_\-\-Vdw\-None\-\_\-\-Geom\-W3\-P1\-\_\-sse2\-\_\-single.\-c} }{\pageref{nb__kernel__ElecEwSw__VdwNone__GeomW3P1__sse2__single_8c}}{}
\item\contentsline{section}{src/gmxlib/nonbonded/nb\-\_\-kernel\-\_\-sse2\-\_\-single/\hyperlink{nb__kernel__ElecEwSw__VdwNone__GeomW3W3__sse2__single_8c}{nb\-\_\-kernel\-\_\-\-Elec\-Ew\-Sw\-\_\-\-Vdw\-None\-\_\-\-Geom\-W3\-W3\-\_\-sse2\-\_\-single.\-c} }{\pageref{nb__kernel__ElecEwSw__VdwNone__GeomW3W3__sse2__single_8c}}{}
\item\contentsline{section}{src/gmxlib/nonbonded/nb\-\_\-kernel\-\_\-sse2\-\_\-single/\hyperlink{nb__kernel__ElecEwSw__VdwNone__GeomW4P1__sse2__single_8c}{nb\-\_\-kernel\-\_\-\-Elec\-Ew\-Sw\-\_\-\-Vdw\-None\-\_\-\-Geom\-W4\-P1\-\_\-sse2\-\_\-single.\-c} }{\pageref{nb__kernel__ElecEwSw__VdwNone__GeomW4P1__sse2__single_8c}}{}
\item\contentsline{section}{src/gmxlib/nonbonded/nb\-\_\-kernel\-\_\-sse2\-\_\-single/\hyperlink{nb__kernel__ElecEwSw__VdwNone__GeomW4W4__sse2__single_8c}{nb\-\_\-kernel\-\_\-\-Elec\-Ew\-Sw\-\_\-\-Vdw\-None\-\_\-\-Geom\-W4\-W4\-\_\-sse2\-\_\-single.\-c} }{\pageref{nb__kernel__ElecEwSw__VdwNone__GeomW4W4__sse2__single_8c}}{}
\item\contentsline{section}{src/gmxlib/nonbonded/nb\-\_\-kernel\-\_\-sse2\-\_\-single/\hyperlink{nb__kernel__ElecGB__VdwCSTab__GeomP1P1__sse2__single_8c}{nb\-\_\-kernel\-\_\-\-Elec\-G\-B\-\_\-\-Vdw\-C\-S\-Tab\-\_\-\-Geom\-P1\-P1\-\_\-sse2\-\_\-single.\-c} }{\pageref{nb__kernel__ElecGB__VdwCSTab__GeomP1P1__sse2__single_8c}}{}
\item\contentsline{section}{src/gmxlib/nonbonded/nb\-\_\-kernel\-\_\-sse2\-\_\-single/\hyperlink{nb__kernel__ElecGB__VdwLJ__GeomP1P1__sse2__single_8c}{nb\-\_\-kernel\-\_\-\-Elec\-G\-B\-\_\-\-Vdw\-L\-J\-\_\-\-Geom\-P1\-P1\-\_\-sse2\-\_\-single.\-c} }{\pageref{nb__kernel__ElecGB__VdwLJ__GeomP1P1__sse2__single_8c}}{}
\item\contentsline{section}{src/gmxlib/nonbonded/nb\-\_\-kernel\-\_\-sse2\-\_\-single/\hyperlink{nb__kernel__ElecGB__VdwNone__GeomP1P1__sse2__single_8c}{nb\-\_\-kernel\-\_\-\-Elec\-G\-B\-\_\-\-Vdw\-None\-\_\-\-Geom\-P1\-P1\-\_\-sse2\-\_\-single.\-c} }{\pageref{nb__kernel__ElecGB__VdwNone__GeomP1P1__sse2__single_8c}}{}
\item\contentsline{section}{src/gmxlib/nonbonded/nb\-\_\-kernel\-\_\-sse2\-\_\-single/\hyperlink{nb__kernel__ElecNone__VdwCSTab__GeomP1P1__sse2__single_8c}{nb\-\_\-kernel\-\_\-\-Elec\-None\-\_\-\-Vdw\-C\-S\-Tab\-\_\-\-Geom\-P1\-P1\-\_\-sse2\-\_\-single.\-c} }{\pageref{nb__kernel__ElecNone__VdwCSTab__GeomP1P1__sse2__single_8c}}{}
\item\contentsline{section}{src/gmxlib/nonbonded/nb\-\_\-kernel\-\_\-sse2\-\_\-single/\hyperlink{nb__kernel__ElecNone__VdwLJ__GeomP1P1__sse2__single_8c}{nb\-\_\-kernel\-\_\-\-Elec\-None\-\_\-\-Vdw\-L\-J\-\_\-\-Geom\-P1\-P1\-\_\-sse2\-\_\-single.\-c} }{\pageref{nb__kernel__ElecNone__VdwLJ__GeomP1P1__sse2__single_8c}}{}
\item\contentsline{section}{src/gmxlib/nonbonded/nb\-\_\-kernel\-\_\-sse2\-\_\-single/\hyperlink{nb__kernel__ElecNone__VdwLJSh__GeomP1P1__sse2__single_8c}{nb\-\_\-kernel\-\_\-\-Elec\-None\-\_\-\-Vdw\-L\-J\-Sh\-\_\-\-Geom\-P1\-P1\-\_\-sse2\-\_\-single.\-c} }{\pageref{nb__kernel__ElecNone__VdwLJSh__GeomP1P1__sse2__single_8c}}{}
\item\contentsline{section}{src/gmxlib/nonbonded/nb\-\_\-kernel\-\_\-sse2\-\_\-single/\hyperlink{nb__kernel__ElecNone__VdwLJSw__GeomP1P1__sse2__single_8c}{nb\-\_\-kernel\-\_\-\-Elec\-None\-\_\-\-Vdw\-L\-J\-Sw\-\_\-\-Geom\-P1\-P1\-\_\-sse2\-\_\-single.\-c} }{\pageref{nb__kernel__ElecNone__VdwLJSw__GeomP1P1__sse2__single_8c}}{}
\item\contentsline{section}{src/gmxlib/nonbonded/nb\-\_\-kernel\-\_\-sse2\-\_\-single/\hyperlink{nb__kernel__ElecRF__VdwCSTab__GeomP1P1__sse2__single_8c}{nb\-\_\-kernel\-\_\-\-Elec\-R\-F\-\_\-\-Vdw\-C\-S\-Tab\-\_\-\-Geom\-P1\-P1\-\_\-sse2\-\_\-single.\-c} }{\pageref{nb__kernel__ElecRF__VdwCSTab__GeomP1P1__sse2__single_8c}}{}
\item\contentsline{section}{src/gmxlib/nonbonded/nb\-\_\-kernel\-\_\-sse2\-\_\-single/\hyperlink{nb__kernel__ElecRF__VdwCSTab__GeomW3P1__sse2__single_8c}{nb\-\_\-kernel\-\_\-\-Elec\-R\-F\-\_\-\-Vdw\-C\-S\-Tab\-\_\-\-Geom\-W3\-P1\-\_\-sse2\-\_\-single.\-c} }{\pageref{nb__kernel__ElecRF__VdwCSTab__GeomW3P1__sse2__single_8c}}{}
\item\contentsline{section}{src/gmxlib/nonbonded/nb\-\_\-kernel\-\_\-sse2\-\_\-single/\hyperlink{nb__kernel__ElecRF__VdwCSTab__GeomW3W3__sse2__single_8c}{nb\-\_\-kernel\-\_\-\-Elec\-R\-F\-\_\-\-Vdw\-C\-S\-Tab\-\_\-\-Geom\-W3\-W3\-\_\-sse2\-\_\-single.\-c} }{\pageref{nb__kernel__ElecRF__VdwCSTab__GeomW3W3__sse2__single_8c}}{}
\item\contentsline{section}{src/gmxlib/nonbonded/nb\-\_\-kernel\-\_\-sse2\-\_\-single/\hyperlink{nb__kernel__ElecRF__VdwCSTab__GeomW4P1__sse2__single_8c}{nb\-\_\-kernel\-\_\-\-Elec\-R\-F\-\_\-\-Vdw\-C\-S\-Tab\-\_\-\-Geom\-W4\-P1\-\_\-sse2\-\_\-single.\-c} }{\pageref{nb__kernel__ElecRF__VdwCSTab__GeomW4P1__sse2__single_8c}}{}
\item\contentsline{section}{src/gmxlib/nonbonded/nb\-\_\-kernel\-\_\-sse2\-\_\-single/\hyperlink{nb__kernel__ElecRF__VdwCSTab__GeomW4W4__sse2__single_8c}{nb\-\_\-kernel\-\_\-\-Elec\-R\-F\-\_\-\-Vdw\-C\-S\-Tab\-\_\-\-Geom\-W4\-W4\-\_\-sse2\-\_\-single.\-c} }{\pageref{nb__kernel__ElecRF__VdwCSTab__GeomW4W4__sse2__single_8c}}{}
\item\contentsline{section}{src/gmxlib/nonbonded/nb\-\_\-kernel\-\_\-sse2\-\_\-single/\hyperlink{nb__kernel__ElecRF__VdwLJ__GeomP1P1__sse2__single_8c}{nb\-\_\-kernel\-\_\-\-Elec\-R\-F\-\_\-\-Vdw\-L\-J\-\_\-\-Geom\-P1\-P1\-\_\-sse2\-\_\-single.\-c} }{\pageref{nb__kernel__ElecRF__VdwLJ__GeomP1P1__sse2__single_8c}}{}
\item\contentsline{section}{src/gmxlib/nonbonded/nb\-\_\-kernel\-\_\-sse2\-\_\-single/\hyperlink{nb__kernel__ElecRF__VdwLJ__GeomW3P1__sse2__single_8c}{nb\-\_\-kernel\-\_\-\-Elec\-R\-F\-\_\-\-Vdw\-L\-J\-\_\-\-Geom\-W3\-P1\-\_\-sse2\-\_\-single.\-c} }{\pageref{nb__kernel__ElecRF__VdwLJ__GeomW3P1__sse2__single_8c}}{}
\item\contentsline{section}{src/gmxlib/nonbonded/nb\-\_\-kernel\-\_\-sse2\-\_\-single/\hyperlink{nb__kernel__ElecRF__VdwLJ__GeomW3W3__sse2__single_8c}{nb\-\_\-kernel\-\_\-\-Elec\-R\-F\-\_\-\-Vdw\-L\-J\-\_\-\-Geom\-W3\-W3\-\_\-sse2\-\_\-single.\-c} }{\pageref{nb__kernel__ElecRF__VdwLJ__GeomW3W3__sse2__single_8c}}{}
\item\contentsline{section}{src/gmxlib/nonbonded/nb\-\_\-kernel\-\_\-sse2\-\_\-single/\hyperlink{nb__kernel__ElecRF__VdwLJ__GeomW4P1__sse2__single_8c}{nb\-\_\-kernel\-\_\-\-Elec\-R\-F\-\_\-\-Vdw\-L\-J\-\_\-\-Geom\-W4\-P1\-\_\-sse2\-\_\-single.\-c} }{\pageref{nb__kernel__ElecRF__VdwLJ__GeomW4P1__sse2__single_8c}}{}
\item\contentsline{section}{src/gmxlib/nonbonded/nb\-\_\-kernel\-\_\-sse2\-\_\-single/\hyperlink{nb__kernel__ElecRF__VdwLJ__GeomW4W4__sse2__single_8c}{nb\-\_\-kernel\-\_\-\-Elec\-R\-F\-\_\-\-Vdw\-L\-J\-\_\-\-Geom\-W4\-W4\-\_\-sse2\-\_\-single.\-c} }{\pageref{nb__kernel__ElecRF__VdwLJ__GeomW4W4__sse2__single_8c}}{}
\item\contentsline{section}{src/gmxlib/nonbonded/nb\-\_\-kernel\-\_\-sse2\-\_\-single/\hyperlink{nb__kernel__ElecRF__VdwNone__GeomP1P1__sse2__single_8c}{nb\-\_\-kernel\-\_\-\-Elec\-R\-F\-\_\-\-Vdw\-None\-\_\-\-Geom\-P1\-P1\-\_\-sse2\-\_\-single.\-c} }{\pageref{nb__kernel__ElecRF__VdwNone__GeomP1P1__sse2__single_8c}}{}
\item\contentsline{section}{src/gmxlib/nonbonded/nb\-\_\-kernel\-\_\-sse2\-\_\-single/\hyperlink{nb__kernel__ElecRF__VdwNone__GeomW3P1__sse2__single_8c}{nb\-\_\-kernel\-\_\-\-Elec\-R\-F\-\_\-\-Vdw\-None\-\_\-\-Geom\-W3\-P1\-\_\-sse2\-\_\-single.\-c} }{\pageref{nb__kernel__ElecRF__VdwNone__GeomW3P1__sse2__single_8c}}{}
\item\contentsline{section}{src/gmxlib/nonbonded/nb\-\_\-kernel\-\_\-sse2\-\_\-single/\hyperlink{nb__kernel__ElecRF__VdwNone__GeomW3W3__sse2__single_8c}{nb\-\_\-kernel\-\_\-\-Elec\-R\-F\-\_\-\-Vdw\-None\-\_\-\-Geom\-W3\-W3\-\_\-sse2\-\_\-single.\-c} }{\pageref{nb__kernel__ElecRF__VdwNone__GeomW3W3__sse2__single_8c}}{}
\item\contentsline{section}{src/gmxlib/nonbonded/nb\-\_\-kernel\-\_\-sse2\-\_\-single/\hyperlink{nb__kernel__ElecRF__VdwNone__GeomW4P1__sse2__single_8c}{nb\-\_\-kernel\-\_\-\-Elec\-R\-F\-\_\-\-Vdw\-None\-\_\-\-Geom\-W4\-P1\-\_\-sse2\-\_\-single.\-c} }{\pageref{nb__kernel__ElecRF__VdwNone__GeomW4P1__sse2__single_8c}}{}
\item\contentsline{section}{src/gmxlib/nonbonded/nb\-\_\-kernel\-\_\-sse2\-\_\-single/\hyperlink{nb__kernel__ElecRF__VdwNone__GeomW4W4__sse2__single_8c}{nb\-\_\-kernel\-\_\-\-Elec\-R\-F\-\_\-\-Vdw\-None\-\_\-\-Geom\-W4\-W4\-\_\-sse2\-\_\-single.\-c} }{\pageref{nb__kernel__ElecRF__VdwNone__GeomW4W4__sse2__single_8c}}{}
\item\contentsline{section}{src/gmxlib/nonbonded/nb\-\_\-kernel\-\_\-sse2\-\_\-single/\hyperlink{nb__kernel__ElecRFCut__VdwCSTab__GeomP1P1__sse2__single_8c}{nb\-\_\-kernel\-\_\-\-Elec\-R\-F\-Cut\-\_\-\-Vdw\-C\-S\-Tab\-\_\-\-Geom\-P1\-P1\-\_\-sse2\-\_\-single.\-c} }{\pageref{nb__kernel__ElecRFCut__VdwCSTab__GeomP1P1__sse2__single_8c}}{}
\item\contentsline{section}{src/gmxlib/nonbonded/nb\-\_\-kernel\-\_\-sse2\-\_\-single/\hyperlink{nb__kernel__ElecRFCut__VdwCSTab__GeomW3P1__sse2__single_8c}{nb\-\_\-kernel\-\_\-\-Elec\-R\-F\-Cut\-\_\-\-Vdw\-C\-S\-Tab\-\_\-\-Geom\-W3\-P1\-\_\-sse2\-\_\-single.\-c} }{\pageref{nb__kernel__ElecRFCut__VdwCSTab__GeomW3P1__sse2__single_8c}}{}
\item\contentsline{section}{src/gmxlib/nonbonded/nb\-\_\-kernel\-\_\-sse2\-\_\-single/\hyperlink{nb__kernel__ElecRFCut__VdwCSTab__GeomW3W3__sse2__single_8c}{nb\-\_\-kernel\-\_\-\-Elec\-R\-F\-Cut\-\_\-\-Vdw\-C\-S\-Tab\-\_\-\-Geom\-W3\-W3\-\_\-sse2\-\_\-single.\-c} }{\pageref{nb__kernel__ElecRFCut__VdwCSTab__GeomW3W3__sse2__single_8c}}{}
\item\contentsline{section}{src/gmxlib/nonbonded/nb\-\_\-kernel\-\_\-sse2\-\_\-single/\hyperlink{nb__kernel__ElecRFCut__VdwCSTab__GeomW4P1__sse2__single_8c}{nb\-\_\-kernel\-\_\-\-Elec\-R\-F\-Cut\-\_\-\-Vdw\-C\-S\-Tab\-\_\-\-Geom\-W4\-P1\-\_\-sse2\-\_\-single.\-c} }{\pageref{nb__kernel__ElecRFCut__VdwCSTab__GeomW4P1__sse2__single_8c}}{}
\item\contentsline{section}{src/gmxlib/nonbonded/nb\-\_\-kernel\-\_\-sse2\-\_\-single/\hyperlink{nb__kernel__ElecRFCut__VdwCSTab__GeomW4W4__sse2__single_8c}{nb\-\_\-kernel\-\_\-\-Elec\-R\-F\-Cut\-\_\-\-Vdw\-C\-S\-Tab\-\_\-\-Geom\-W4\-W4\-\_\-sse2\-\_\-single.\-c} }{\pageref{nb__kernel__ElecRFCut__VdwCSTab__GeomW4W4__sse2__single_8c}}{}
\item\contentsline{section}{src/gmxlib/nonbonded/nb\-\_\-kernel\-\_\-sse2\-\_\-single/\hyperlink{nb__kernel__ElecRFCut__VdwLJSh__GeomP1P1__sse2__single_8c}{nb\-\_\-kernel\-\_\-\-Elec\-R\-F\-Cut\-\_\-\-Vdw\-L\-J\-Sh\-\_\-\-Geom\-P1\-P1\-\_\-sse2\-\_\-single.\-c} }{\pageref{nb__kernel__ElecRFCut__VdwLJSh__GeomP1P1__sse2__single_8c}}{}
\item\contentsline{section}{src/gmxlib/nonbonded/nb\-\_\-kernel\-\_\-sse2\-\_\-single/\hyperlink{nb__kernel__ElecRFCut__VdwLJSh__GeomW3P1__sse2__single_8c}{nb\-\_\-kernel\-\_\-\-Elec\-R\-F\-Cut\-\_\-\-Vdw\-L\-J\-Sh\-\_\-\-Geom\-W3\-P1\-\_\-sse2\-\_\-single.\-c} }{\pageref{nb__kernel__ElecRFCut__VdwLJSh__GeomW3P1__sse2__single_8c}}{}
\item\contentsline{section}{src/gmxlib/nonbonded/nb\-\_\-kernel\-\_\-sse2\-\_\-single/\hyperlink{nb__kernel__ElecRFCut__VdwLJSh__GeomW3W3__sse2__single_8c}{nb\-\_\-kernel\-\_\-\-Elec\-R\-F\-Cut\-\_\-\-Vdw\-L\-J\-Sh\-\_\-\-Geom\-W3\-W3\-\_\-sse2\-\_\-single.\-c} }{\pageref{nb__kernel__ElecRFCut__VdwLJSh__GeomW3W3__sse2__single_8c}}{}
\item\contentsline{section}{src/gmxlib/nonbonded/nb\-\_\-kernel\-\_\-sse2\-\_\-single/\hyperlink{nb__kernel__ElecRFCut__VdwLJSh__GeomW4P1__sse2__single_8c}{nb\-\_\-kernel\-\_\-\-Elec\-R\-F\-Cut\-\_\-\-Vdw\-L\-J\-Sh\-\_\-\-Geom\-W4\-P1\-\_\-sse2\-\_\-single.\-c} }{\pageref{nb__kernel__ElecRFCut__VdwLJSh__GeomW4P1__sse2__single_8c}}{}
\item\contentsline{section}{src/gmxlib/nonbonded/nb\-\_\-kernel\-\_\-sse2\-\_\-single/\hyperlink{nb__kernel__ElecRFCut__VdwLJSh__GeomW4W4__sse2__single_8c}{nb\-\_\-kernel\-\_\-\-Elec\-R\-F\-Cut\-\_\-\-Vdw\-L\-J\-Sh\-\_\-\-Geom\-W4\-W4\-\_\-sse2\-\_\-single.\-c} }{\pageref{nb__kernel__ElecRFCut__VdwLJSh__GeomW4W4__sse2__single_8c}}{}
\item\contentsline{section}{src/gmxlib/nonbonded/nb\-\_\-kernel\-\_\-sse2\-\_\-single/\hyperlink{nb__kernel__ElecRFCut__VdwLJSw__GeomP1P1__sse2__single_8c}{nb\-\_\-kernel\-\_\-\-Elec\-R\-F\-Cut\-\_\-\-Vdw\-L\-J\-Sw\-\_\-\-Geom\-P1\-P1\-\_\-sse2\-\_\-single.\-c} }{\pageref{nb__kernel__ElecRFCut__VdwLJSw__GeomP1P1__sse2__single_8c}}{}
\item\contentsline{section}{src/gmxlib/nonbonded/nb\-\_\-kernel\-\_\-sse2\-\_\-single/\hyperlink{nb__kernel__ElecRFCut__VdwLJSw__GeomW3P1__sse2__single_8c}{nb\-\_\-kernel\-\_\-\-Elec\-R\-F\-Cut\-\_\-\-Vdw\-L\-J\-Sw\-\_\-\-Geom\-W3\-P1\-\_\-sse2\-\_\-single.\-c} }{\pageref{nb__kernel__ElecRFCut__VdwLJSw__GeomW3P1__sse2__single_8c}}{}
\item\contentsline{section}{src/gmxlib/nonbonded/nb\-\_\-kernel\-\_\-sse2\-\_\-single/\hyperlink{nb__kernel__ElecRFCut__VdwLJSw__GeomW3W3__sse2__single_8c}{nb\-\_\-kernel\-\_\-\-Elec\-R\-F\-Cut\-\_\-\-Vdw\-L\-J\-Sw\-\_\-\-Geom\-W3\-W3\-\_\-sse2\-\_\-single.\-c} }{\pageref{nb__kernel__ElecRFCut__VdwLJSw__GeomW3W3__sse2__single_8c}}{}
\item\contentsline{section}{src/gmxlib/nonbonded/nb\-\_\-kernel\-\_\-sse2\-\_\-single/\hyperlink{nb__kernel__ElecRFCut__VdwLJSw__GeomW4P1__sse2__single_8c}{nb\-\_\-kernel\-\_\-\-Elec\-R\-F\-Cut\-\_\-\-Vdw\-L\-J\-Sw\-\_\-\-Geom\-W4\-P1\-\_\-sse2\-\_\-single.\-c} }{\pageref{nb__kernel__ElecRFCut__VdwLJSw__GeomW4P1__sse2__single_8c}}{}
\item\contentsline{section}{src/gmxlib/nonbonded/nb\-\_\-kernel\-\_\-sse2\-\_\-single/\hyperlink{nb__kernel__ElecRFCut__VdwLJSw__GeomW4W4__sse2__single_8c}{nb\-\_\-kernel\-\_\-\-Elec\-R\-F\-Cut\-\_\-\-Vdw\-L\-J\-Sw\-\_\-\-Geom\-W4\-W4\-\_\-sse2\-\_\-single.\-c} }{\pageref{nb__kernel__ElecRFCut__VdwLJSw__GeomW4W4__sse2__single_8c}}{}
\item\contentsline{section}{src/gmxlib/nonbonded/nb\-\_\-kernel\-\_\-sse2\-\_\-single/\hyperlink{nb__kernel__ElecRFCut__VdwNone__GeomP1P1__sse2__single_8c}{nb\-\_\-kernel\-\_\-\-Elec\-R\-F\-Cut\-\_\-\-Vdw\-None\-\_\-\-Geom\-P1\-P1\-\_\-sse2\-\_\-single.\-c} }{\pageref{nb__kernel__ElecRFCut__VdwNone__GeomP1P1__sse2__single_8c}}{}
\item\contentsline{section}{src/gmxlib/nonbonded/nb\-\_\-kernel\-\_\-sse2\-\_\-single/\hyperlink{nb__kernel__ElecRFCut__VdwNone__GeomW3P1__sse2__single_8c}{nb\-\_\-kernel\-\_\-\-Elec\-R\-F\-Cut\-\_\-\-Vdw\-None\-\_\-\-Geom\-W3\-P1\-\_\-sse2\-\_\-single.\-c} }{\pageref{nb__kernel__ElecRFCut__VdwNone__GeomW3P1__sse2__single_8c}}{}
\item\contentsline{section}{src/gmxlib/nonbonded/nb\-\_\-kernel\-\_\-sse2\-\_\-single/\hyperlink{nb__kernel__ElecRFCut__VdwNone__GeomW3W3__sse2__single_8c}{nb\-\_\-kernel\-\_\-\-Elec\-R\-F\-Cut\-\_\-\-Vdw\-None\-\_\-\-Geom\-W3\-W3\-\_\-sse2\-\_\-single.\-c} }{\pageref{nb__kernel__ElecRFCut__VdwNone__GeomW3W3__sse2__single_8c}}{}
\item\contentsline{section}{src/gmxlib/nonbonded/nb\-\_\-kernel\-\_\-sse2\-\_\-single/\hyperlink{nb__kernel__ElecRFCut__VdwNone__GeomW4P1__sse2__single_8c}{nb\-\_\-kernel\-\_\-\-Elec\-R\-F\-Cut\-\_\-\-Vdw\-None\-\_\-\-Geom\-W4\-P1\-\_\-sse2\-\_\-single.\-c} }{\pageref{nb__kernel__ElecRFCut__VdwNone__GeomW4P1__sse2__single_8c}}{}
\item\contentsline{section}{src/gmxlib/nonbonded/nb\-\_\-kernel\-\_\-sse2\-\_\-single/\hyperlink{nb__kernel__ElecRFCut__VdwNone__GeomW4W4__sse2__single_8c}{nb\-\_\-kernel\-\_\-\-Elec\-R\-F\-Cut\-\_\-\-Vdw\-None\-\_\-\-Geom\-W4\-W4\-\_\-sse2\-\_\-single.\-c} }{\pageref{nb__kernel__ElecRFCut__VdwNone__GeomW4W4__sse2__single_8c}}{}
\item\contentsline{section}{src/gmxlib/nonbonded/nb\-\_\-kernel\-\_\-sse2\-\_\-single/\hyperlink{nb__kernel__sse2__single_8c}{nb\-\_\-kernel\-\_\-sse2\-\_\-single.\-c} }{\pageref{nb__kernel__sse2__single_8c}}{}
\item\contentsline{section}{src/gmxlib/nonbonded/nb\-\_\-kernel\-\_\-sse2\-\_\-single/\hyperlink{nb__kernel__sse2__single_8h}{nb\-\_\-kernel\-\_\-sse2\-\_\-single.\-h} }{\pageref{nb__kernel__sse2__single_8h}}{}
\item\contentsline{section}{src/gmxlib/nonbonded/nb\-\_\-kernel\-\_\-sse4\-\_\-1\-\_\-double/\hyperlink{kernelutil__x86__sse4__1__double_8h}{kernelutil\-\_\-x86\-\_\-sse4\-\_\-1\-\_\-double.\-h} }{\pageref{kernelutil__x86__sse4__1__double_8h}}{}
\item\contentsline{section}{src/gmxlib/nonbonded/nb\-\_\-kernel\-\_\-sse4\-\_\-1\-\_\-double/\hyperlink{nb__kernel__ElecCoul__VdwCSTab__GeomP1P1__sse4__1__double_8c}{nb\-\_\-kernel\-\_\-\-Elec\-Coul\-\_\-\-Vdw\-C\-S\-Tab\-\_\-\-Geom\-P1\-P1\-\_\-sse4\-\_\-1\-\_\-double.\-c} }{\pageref{nb__kernel__ElecCoul__VdwCSTab__GeomP1P1__sse4__1__double_8c}}{}
\item\contentsline{section}{src/gmxlib/nonbonded/nb\-\_\-kernel\-\_\-sse4\-\_\-1\-\_\-double/\hyperlink{nb__kernel__ElecCoul__VdwCSTab__GeomW3P1__sse4__1__double_8c}{nb\-\_\-kernel\-\_\-\-Elec\-Coul\-\_\-\-Vdw\-C\-S\-Tab\-\_\-\-Geom\-W3\-P1\-\_\-sse4\-\_\-1\-\_\-double.\-c} }{\pageref{nb__kernel__ElecCoul__VdwCSTab__GeomW3P1__sse4__1__double_8c}}{}
\item\contentsline{section}{src/gmxlib/nonbonded/nb\-\_\-kernel\-\_\-sse4\-\_\-1\-\_\-double/\hyperlink{nb__kernel__ElecCoul__VdwCSTab__GeomW3W3__sse4__1__double_8c}{nb\-\_\-kernel\-\_\-\-Elec\-Coul\-\_\-\-Vdw\-C\-S\-Tab\-\_\-\-Geom\-W3\-W3\-\_\-sse4\-\_\-1\-\_\-double.\-c} }{\pageref{nb__kernel__ElecCoul__VdwCSTab__GeomW3W3__sse4__1__double_8c}}{}
\item\contentsline{section}{src/gmxlib/nonbonded/nb\-\_\-kernel\-\_\-sse4\-\_\-1\-\_\-double/\hyperlink{nb__kernel__ElecCoul__VdwCSTab__GeomW4P1__sse4__1__double_8c}{nb\-\_\-kernel\-\_\-\-Elec\-Coul\-\_\-\-Vdw\-C\-S\-Tab\-\_\-\-Geom\-W4\-P1\-\_\-sse4\-\_\-1\-\_\-double.\-c} }{\pageref{nb__kernel__ElecCoul__VdwCSTab__GeomW4P1__sse4__1__double_8c}}{}
\item\contentsline{section}{src/gmxlib/nonbonded/nb\-\_\-kernel\-\_\-sse4\-\_\-1\-\_\-double/\hyperlink{nb__kernel__ElecCoul__VdwCSTab__GeomW4W4__sse4__1__double_8c}{nb\-\_\-kernel\-\_\-\-Elec\-Coul\-\_\-\-Vdw\-C\-S\-Tab\-\_\-\-Geom\-W4\-W4\-\_\-sse4\-\_\-1\-\_\-double.\-c} }{\pageref{nb__kernel__ElecCoul__VdwCSTab__GeomW4W4__sse4__1__double_8c}}{}
\item\contentsline{section}{src/gmxlib/nonbonded/nb\-\_\-kernel\-\_\-sse4\-\_\-1\-\_\-double/\hyperlink{nb__kernel__ElecCoul__VdwLJ__GeomP1P1__sse4__1__double_8c}{nb\-\_\-kernel\-\_\-\-Elec\-Coul\-\_\-\-Vdw\-L\-J\-\_\-\-Geom\-P1\-P1\-\_\-sse4\-\_\-1\-\_\-double.\-c} }{\pageref{nb__kernel__ElecCoul__VdwLJ__GeomP1P1__sse4__1__double_8c}}{}
\item\contentsline{section}{src/gmxlib/nonbonded/nb\-\_\-kernel\-\_\-sse4\-\_\-1\-\_\-double/\hyperlink{nb__kernel__ElecCoul__VdwLJ__GeomW3P1__sse4__1__double_8c}{nb\-\_\-kernel\-\_\-\-Elec\-Coul\-\_\-\-Vdw\-L\-J\-\_\-\-Geom\-W3\-P1\-\_\-sse4\-\_\-1\-\_\-double.\-c} }{\pageref{nb__kernel__ElecCoul__VdwLJ__GeomW3P1__sse4__1__double_8c}}{}
\item\contentsline{section}{src/gmxlib/nonbonded/nb\-\_\-kernel\-\_\-sse4\-\_\-1\-\_\-double/\hyperlink{nb__kernel__ElecCoul__VdwLJ__GeomW3W3__sse4__1__double_8c}{nb\-\_\-kernel\-\_\-\-Elec\-Coul\-\_\-\-Vdw\-L\-J\-\_\-\-Geom\-W3\-W3\-\_\-sse4\-\_\-1\-\_\-double.\-c} }{\pageref{nb__kernel__ElecCoul__VdwLJ__GeomW3W3__sse4__1__double_8c}}{}
\item\contentsline{section}{src/gmxlib/nonbonded/nb\-\_\-kernel\-\_\-sse4\-\_\-1\-\_\-double/\hyperlink{nb__kernel__ElecCoul__VdwLJ__GeomW4P1__sse4__1__double_8c}{nb\-\_\-kernel\-\_\-\-Elec\-Coul\-\_\-\-Vdw\-L\-J\-\_\-\-Geom\-W4\-P1\-\_\-sse4\-\_\-1\-\_\-double.\-c} }{\pageref{nb__kernel__ElecCoul__VdwLJ__GeomW4P1__sse4__1__double_8c}}{}
\item\contentsline{section}{src/gmxlib/nonbonded/nb\-\_\-kernel\-\_\-sse4\-\_\-1\-\_\-double/\hyperlink{nb__kernel__ElecCoul__VdwLJ__GeomW4W4__sse4__1__double_8c}{nb\-\_\-kernel\-\_\-\-Elec\-Coul\-\_\-\-Vdw\-L\-J\-\_\-\-Geom\-W4\-W4\-\_\-sse4\-\_\-1\-\_\-double.\-c} }{\pageref{nb__kernel__ElecCoul__VdwLJ__GeomW4W4__sse4__1__double_8c}}{}
\item\contentsline{section}{src/gmxlib/nonbonded/nb\-\_\-kernel\-\_\-sse4\-\_\-1\-\_\-double/\hyperlink{nb__kernel__ElecCoul__VdwNone__GeomP1P1__sse4__1__double_8c}{nb\-\_\-kernel\-\_\-\-Elec\-Coul\-\_\-\-Vdw\-None\-\_\-\-Geom\-P1\-P1\-\_\-sse4\-\_\-1\-\_\-double.\-c} }{\pageref{nb__kernel__ElecCoul__VdwNone__GeomP1P1__sse4__1__double_8c}}{}
\item\contentsline{section}{src/gmxlib/nonbonded/nb\-\_\-kernel\-\_\-sse4\-\_\-1\-\_\-double/\hyperlink{nb__kernel__ElecCoul__VdwNone__GeomW3P1__sse4__1__double_8c}{nb\-\_\-kernel\-\_\-\-Elec\-Coul\-\_\-\-Vdw\-None\-\_\-\-Geom\-W3\-P1\-\_\-sse4\-\_\-1\-\_\-double.\-c} }{\pageref{nb__kernel__ElecCoul__VdwNone__GeomW3P1__sse4__1__double_8c}}{}
\item\contentsline{section}{src/gmxlib/nonbonded/nb\-\_\-kernel\-\_\-sse4\-\_\-1\-\_\-double/\hyperlink{nb__kernel__ElecCoul__VdwNone__GeomW3W3__sse4__1__double_8c}{nb\-\_\-kernel\-\_\-\-Elec\-Coul\-\_\-\-Vdw\-None\-\_\-\-Geom\-W3\-W3\-\_\-sse4\-\_\-1\-\_\-double.\-c} }{\pageref{nb__kernel__ElecCoul__VdwNone__GeomW3W3__sse4__1__double_8c}}{}
\item\contentsline{section}{src/gmxlib/nonbonded/nb\-\_\-kernel\-\_\-sse4\-\_\-1\-\_\-double/\hyperlink{nb__kernel__ElecCoul__VdwNone__GeomW4P1__sse4__1__double_8c}{nb\-\_\-kernel\-\_\-\-Elec\-Coul\-\_\-\-Vdw\-None\-\_\-\-Geom\-W4\-P1\-\_\-sse4\-\_\-1\-\_\-double.\-c} }{\pageref{nb__kernel__ElecCoul__VdwNone__GeomW4P1__sse4__1__double_8c}}{}
\item\contentsline{section}{src/gmxlib/nonbonded/nb\-\_\-kernel\-\_\-sse4\-\_\-1\-\_\-double/\hyperlink{nb__kernel__ElecCoul__VdwNone__GeomW4W4__sse4__1__double_8c}{nb\-\_\-kernel\-\_\-\-Elec\-Coul\-\_\-\-Vdw\-None\-\_\-\-Geom\-W4\-W4\-\_\-sse4\-\_\-1\-\_\-double.\-c} }{\pageref{nb__kernel__ElecCoul__VdwNone__GeomW4W4__sse4__1__double_8c}}{}
\item\contentsline{section}{src/gmxlib/nonbonded/nb\-\_\-kernel\-\_\-sse4\-\_\-1\-\_\-double/\hyperlink{nb__kernel__ElecCSTab__VdwCSTab__GeomP1P1__sse4__1__double_8c}{nb\-\_\-kernel\-\_\-\-Elec\-C\-S\-Tab\-\_\-\-Vdw\-C\-S\-Tab\-\_\-\-Geom\-P1\-P1\-\_\-sse4\-\_\-1\-\_\-double.\-c} }{\pageref{nb__kernel__ElecCSTab__VdwCSTab__GeomP1P1__sse4__1__double_8c}}{}
\item\contentsline{section}{src/gmxlib/nonbonded/nb\-\_\-kernel\-\_\-sse4\-\_\-1\-\_\-double/\hyperlink{nb__kernel__ElecCSTab__VdwCSTab__GeomW3P1__sse4__1__double_8c}{nb\-\_\-kernel\-\_\-\-Elec\-C\-S\-Tab\-\_\-\-Vdw\-C\-S\-Tab\-\_\-\-Geom\-W3\-P1\-\_\-sse4\-\_\-1\-\_\-double.\-c} }{\pageref{nb__kernel__ElecCSTab__VdwCSTab__GeomW3P1__sse4__1__double_8c}}{}
\item\contentsline{section}{src/gmxlib/nonbonded/nb\-\_\-kernel\-\_\-sse4\-\_\-1\-\_\-double/\hyperlink{nb__kernel__ElecCSTab__VdwCSTab__GeomW3W3__sse4__1__double_8c}{nb\-\_\-kernel\-\_\-\-Elec\-C\-S\-Tab\-\_\-\-Vdw\-C\-S\-Tab\-\_\-\-Geom\-W3\-W3\-\_\-sse4\-\_\-1\-\_\-double.\-c} }{\pageref{nb__kernel__ElecCSTab__VdwCSTab__GeomW3W3__sse4__1__double_8c}}{}
\item\contentsline{section}{src/gmxlib/nonbonded/nb\-\_\-kernel\-\_\-sse4\-\_\-1\-\_\-double/\hyperlink{nb__kernel__ElecCSTab__VdwCSTab__GeomW4P1__sse4__1__double_8c}{nb\-\_\-kernel\-\_\-\-Elec\-C\-S\-Tab\-\_\-\-Vdw\-C\-S\-Tab\-\_\-\-Geom\-W4\-P1\-\_\-sse4\-\_\-1\-\_\-double.\-c} }{\pageref{nb__kernel__ElecCSTab__VdwCSTab__GeomW4P1__sse4__1__double_8c}}{}
\item\contentsline{section}{src/gmxlib/nonbonded/nb\-\_\-kernel\-\_\-sse4\-\_\-1\-\_\-double/\hyperlink{nb__kernel__ElecCSTab__VdwCSTab__GeomW4W4__sse4__1__double_8c}{nb\-\_\-kernel\-\_\-\-Elec\-C\-S\-Tab\-\_\-\-Vdw\-C\-S\-Tab\-\_\-\-Geom\-W4\-W4\-\_\-sse4\-\_\-1\-\_\-double.\-c} }{\pageref{nb__kernel__ElecCSTab__VdwCSTab__GeomW4W4__sse4__1__double_8c}}{}
\item\contentsline{section}{src/gmxlib/nonbonded/nb\-\_\-kernel\-\_\-sse4\-\_\-1\-\_\-double/\hyperlink{nb__kernel__ElecCSTab__VdwLJ__GeomP1P1__sse4__1__double_8c}{nb\-\_\-kernel\-\_\-\-Elec\-C\-S\-Tab\-\_\-\-Vdw\-L\-J\-\_\-\-Geom\-P1\-P1\-\_\-sse4\-\_\-1\-\_\-double.\-c} }{\pageref{nb__kernel__ElecCSTab__VdwLJ__GeomP1P1__sse4__1__double_8c}}{}
\item\contentsline{section}{src/gmxlib/nonbonded/nb\-\_\-kernel\-\_\-sse4\-\_\-1\-\_\-double/\hyperlink{nb__kernel__ElecCSTab__VdwLJ__GeomW3P1__sse4__1__double_8c}{nb\-\_\-kernel\-\_\-\-Elec\-C\-S\-Tab\-\_\-\-Vdw\-L\-J\-\_\-\-Geom\-W3\-P1\-\_\-sse4\-\_\-1\-\_\-double.\-c} }{\pageref{nb__kernel__ElecCSTab__VdwLJ__GeomW3P1__sse4__1__double_8c}}{}
\item\contentsline{section}{src/gmxlib/nonbonded/nb\-\_\-kernel\-\_\-sse4\-\_\-1\-\_\-double/\hyperlink{nb__kernel__ElecCSTab__VdwLJ__GeomW3W3__sse4__1__double_8c}{nb\-\_\-kernel\-\_\-\-Elec\-C\-S\-Tab\-\_\-\-Vdw\-L\-J\-\_\-\-Geom\-W3\-W3\-\_\-sse4\-\_\-1\-\_\-double.\-c} }{\pageref{nb__kernel__ElecCSTab__VdwLJ__GeomW3W3__sse4__1__double_8c}}{}
\item\contentsline{section}{src/gmxlib/nonbonded/nb\-\_\-kernel\-\_\-sse4\-\_\-1\-\_\-double/\hyperlink{nb__kernel__ElecCSTab__VdwLJ__GeomW4P1__sse4__1__double_8c}{nb\-\_\-kernel\-\_\-\-Elec\-C\-S\-Tab\-\_\-\-Vdw\-L\-J\-\_\-\-Geom\-W4\-P1\-\_\-sse4\-\_\-1\-\_\-double.\-c} }{\pageref{nb__kernel__ElecCSTab__VdwLJ__GeomW4P1__sse4__1__double_8c}}{}
\item\contentsline{section}{src/gmxlib/nonbonded/nb\-\_\-kernel\-\_\-sse4\-\_\-1\-\_\-double/\hyperlink{nb__kernel__ElecCSTab__VdwLJ__GeomW4W4__sse4__1__double_8c}{nb\-\_\-kernel\-\_\-\-Elec\-C\-S\-Tab\-\_\-\-Vdw\-L\-J\-\_\-\-Geom\-W4\-W4\-\_\-sse4\-\_\-1\-\_\-double.\-c} }{\pageref{nb__kernel__ElecCSTab__VdwLJ__GeomW4W4__sse4__1__double_8c}}{}
\item\contentsline{section}{src/gmxlib/nonbonded/nb\-\_\-kernel\-\_\-sse4\-\_\-1\-\_\-double/\hyperlink{nb__kernel__ElecCSTab__VdwNone__GeomP1P1__sse4__1__double_8c}{nb\-\_\-kernel\-\_\-\-Elec\-C\-S\-Tab\-\_\-\-Vdw\-None\-\_\-\-Geom\-P1\-P1\-\_\-sse4\-\_\-1\-\_\-double.\-c} }{\pageref{nb__kernel__ElecCSTab__VdwNone__GeomP1P1__sse4__1__double_8c}}{}
\item\contentsline{section}{src/gmxlib/nonbonded/nb\-\_\-kernel\-\_\-sse4\-\_\-1\-\_\-double/\hyperlink{nb__kernel__ElecCSTab__VdwNone__GeomW3P1__sse4__1__double_8c}{nb\-\_\-kernel\-\_\-\-Elec\-C\-S\-Tab\-\_\-\-Vdw\-None\-\_\-\-Geom\-W3\-P1\-\_\-sse4\-\_\-1\-\_\-double.\-c} }{\pageref{nb__kernel__ElecCSTab__VdwNone__GeomW3P1__sse4__1__double_8c}}{}
\item\contentsline{section}{src/gmxlib/nonbonded/nb\-\_\-kernel\-\_\-sse4\-\_\-1\-\_\-double/\hyperlink{nb__kernel__ElecCSTab__VdwNone__GeomW3W3__sse4__1__double_8c}{nb\-\_\-kernel\-\_\-\-Elec\-C\-S\-Tab\-\_\-\-Vdw\-None\-\_\-\-Geom\-W3\-W3\-\_\-sse4\-\_\-1\-\_\-double.\-c} }{\pageref{nb__kernel__ElecCSTab__VdwNone__GeomW3W3__sse4__1__double_8c}}{}
\item\contentsline{section}{src/gmxlib/nonbonded/nb\-\_\-kernel\-\_\-sse4\-\_\-1\-\_\-double/\hyperlink{nb__kernel__ElecCSTab__VdwNone__GeomW4P1__sse4__1__double_8c}{nb\-\_\-kernel\-\_\-\-Elec\-C\-S\-Tab\-\_\-\-Vdw\-None\-\_\-\-Geom\-W4\-P1\-\_\-sse4\-\_\-1\-\_\-double.\-c} }{\pageref{nb__kernel__ElecCSTab__VdwNone__GeomW4P1__sse4__1__double_8c}}{}
\item\contentsline{section}{src/gmxlib/nonbonded/nb\-\_\-kernel\-\_\-sse4\-\_\-1\-\_\-double/\hyperlink{nb__kernel__ElecCSTab__VdwNone__GeomW4W4__sse4__1__double_8c}{nb\-\_\-kernel\-\_\-\-Elec\-C\-S\-Tab\-\_\-\-Vdw\-None\-\_\-\-Geom\-W4\-W4\-\_\-sse4\-\_\-1\-\_\-double.\-c} }{\pageref{nb__kernel__ElecCSTab__VdwNone__GeomW4W4__sse4__1__double_8c}}{}
\item\contentsline{section}{src/gmxlib/nonbonded/nb\-\_\-kernel\-\_\-sse4\-\_\-1\-\_\-double/\hyperlink{nb__kernel__ElecEw__VdwCSTab__GeomP1P1__sse4__1__double_8c}{nb\-\_\-kernel\-\_\-\-Elec\-Ew\-\_\-\-Vdw\-C\-S\-Tab\-\_\-\-Geom\-P1\-P1\-\_\-sse4\-\_\-1\-\_\-double.\-c} }{\pageref{nb__kernel__ElecEw__VdwCSTab__GeomP1P1__sse4__1__double_8c}}{}
\item\contentsline{section}{src/gmxlib/nonbonded/nb\-\_\-kernel\-\_\-sse4\-\_\-1\-\_\-double/\hyperlink{nb__kernel__ElecEw__VdwCSTab__GeomW3P1__sse4__1__double_8c}{nb\-\_\-kernel\-\_\-\-Elec\-Ew\-\_\-\-Vdw\-C\-S\-Tab\-\_\-\-Geom\-W3\-P1\-\_\-sse4\-\_\-1\-\_\-double.\-c} }{\pageref{nb__kernel__ElecEw__VdwCSTab__GeomW3P1__sse4__1__double_8c}}{}
\item\contentsline{section}{src/gmxlib/nonbonded/nb\-\_\-kernel\-\_\-sse4\-\_\-1\-\_\-double/\hyperlink{nb__kernel__ElecEw__VdwCSTab__GeomW3W3__sse4__1__double_8c}{nb\-\_\-kernel\-\_\-\-Elec\-Ew\-\_\-\-Vdw\-C\-S\-Tab\-\_\-\-Geom\-W3\-W3\-\_\-sse4\-\_\-1\-\_\-double.\-c} }{\pageref{nb__kernel__ElecEw__VdwCSTab__GeomW3W3__sse4__1__double_8c}}{}
\item\contentsline{section}{src/gmxlib/nonbonded/nb\-\_\-kernel\-\_\-sse4\-\_\-1\-\_\-double/\hyperlink{nb__kernel__ElecEw__VdwCSTab__GeomW4P1__sse4__1__double_8c}{nb\-\_\-kernel\-\_\-\-Elec\-Ew\-\_\-\-Vdw\-C\-S\-Tab\-\_\-\-Geom\-W4\-P1\-\_\-sse4\-\_\-1\-\_\-double.\-c} }{\pageref{nb__kernel__ElecEw__VdwCSTab__GeomW4P1__sse4__1__double_8c}}{}
\item\contentsline{section}{src/gmxlib/nonbonded/nb\-\_\-kernel\-\_\-sse4\-\_\-1\-\_\-double/\hyperlink{nb__kernel__ElecEw__VdwCSTab__GeomW4W4__sse4__1__double_8c}{nb\-\_\-kernel\-\_\-\-Elec\-Ew\-\_\-\-Vdw\-C\-S\-Tab\-\_\-\-Geom\-W4\-W4\-\_\-sse4\-\_\-1\-\_\-double.\-c} }{\pageref{nb__kernel__ElecEw__VdwCSTab__GeomW4W4__sse4__1__double_8c}}{}
\item\contentsline{section}{src/gmxlib/nonbonded/nb\-\_\-kernel\-\_\-sse4\-\_\-1\-\_\-double/\hyperlink{nb__kernel__ElecEw__VdwLJ__GeomP1P1__sse4__1__double_8c}{nb\-\_\-kernel\-\_\-\-Elec\-Ew\-\_\-\-Vdw\-L\-J\-\_\-\-Geom\-P1\-P1\-\_\-sse4\-\_\-1\-\_\-double.\-c} }{\pageref{nb__kernel__ElecEw__VdwLJ__GeomP1P1__sse4__1__double_8c}}{}
\item\contentsline{section}{src/gmxlib/nonbonded/nb\-\_\-kernel\-\_\-sse4\-\_\-1\-\_\-double/\hyperlink{nb__kernel__ElecEw__VdwLJ__GeomW3P1__sse4__1__double_8c}{nb\-\_\-kernel\-\_\-\-Elec\-Ew\-\_\-\-Vdw\-L\-J\-\_\-\-Geom\-W3\-P1\-\_\-sse4\-\_\-1\-\_\-double.\-c} }{\pageref{nb__kernel__ElecEw__VdwLJ__GeomW3P1__sse4__1__double_8c}}{}
\item\contentsline{section}{src/gmxlib/nonbonded/nb\-\_\-kernel\-\_\-sse4\-\_\-1\-\_\-double/\hyperlink{nb__kernel__ElecEw__VdwLJ__GeomW3W3__sse4__1__double_8c}{nb\-\_\-kernel\-\_\-\-Elec\-Ew\-\_\-\-Vdw\-L\-J\-\_\-\-Geom\-W3\-W3\-\_\-sse4\-\_\-1\-\_\-double.\-c} }{\pageref{nb__kernel__ElecEw__VdwLJ__GeomW3W3__sse4__1__double_8c}}{}
\item\contentsline{section}{src/gmxlib/nonbonded/nb\-\_\-kernel\-\_\-sse4\-\_\-1\-\_\-double/\hyperlink{nb__kernel__ElecEw__VdwLJ__GeomW4P1__sse4__1__double_8c}{nb\-\_\-kernel\-\_\-\-Elec\-Ew\-\_\-\-Vdw\-L\-J\-\_\-\-Geom\-W4\-P1\-\_\-sse4\-\_\-1\-\_\-double.\-c} }{\pageref{nb__kernel__ElecEw__VdwLJ__GeomW4P1__sse4__1__double_8c}}{}
\item\contentsline{section}{src/gmxlib/nonbonded/nb\-\_\-kernel\-\_\-sse4\-\_\-1\-\_\-double/\hyperlink{nb__kernel__ElecEw__VdwLJ__GeomW4W4__sse4__1__double_8c}{nb\-\_\-kernel\-\_\-\-Elec\-Ew\-\_\-\-Vdw\-L\-J\-\_\-\-Geom\-W4\-W4\-\_\-sse4\-\_\-1\-\_\-double.\-c} }{\pageref{nb__kernel__ElecEw__VdwLJ__GeomW4W4__sse4__1__double_8c}}{}
\item\contentsline{section}{src/gmxlib/nonbonded/nb\-\_\-kernel\-\_\-sse4\-\_\-1\-\_\-double/\hyperlink{nb__kernel__ElecEw__VdwNone__GeomP1P1__sse4__1__double_8c}{nb\-\_\-kernel\-\_\-\-Elec\-Ew\-\_\-\-Vdw\-None\-\_\-\-Geom\-P1\-P1\-\_\-sse4\-\_\-1\-\_\-double.\-c} }{\pageref{nb__kernel__ElecEw__VdwNone__GeomP1P1__sse4__1__double_8c}}{}
\item\contentsline{section}{src/gmxlib/nonbonded/nb\-\_\-kernel\-\_\-sse4\-\_\-1\-\_\-double/\hyperlink{nb__kernel__ElecEw__VdwNone__GeomW3P1__sse4__1__double_8c}{nb\-\_\-kernel\-\_\-\-Elec\-Ew\-\_\-\-Vdw\-None\-\_\-\-Geom\-W3\-P1\-\_\-sse4\-\_\-1\-\_\-double.\-c} }{\pageref{nb__kernel__ElecEw__VdwNone__GeomW3P1__sse4__1__double_8c}}{}
\item\contentsline{section}{src/gmxlib/nonbonded/nb\-\_\-kernel\-\_\-sse4\-\_\-1\-\_\-double/\hyperlink{nb__kernel__ElecEw__VdwNone__GeomW3W3__sse4__1__double_8c}{nb\-\_\-kernel\-\_\-\-Elec\-Ew\-\_\-\-Vdw\-None\-\_\-\-Geom\-W3\-W3\-\_\-sse4\-\_\-1\-\_\-double.\-c} }{\pageref{nb__kernel__ElecEw__VdwNone__GeomW3W3__sse4__1__double_8c}}{}
\item\contentsline{section}{src/gmxlib/nonbonded/nb\-\_\-kernel\-\_\-sse4\-\_\-1\-\_\-double/\hyperlink{nb__kernel__ElecEw__VdwNone__GeomW4P1__sse4__1__double_8c}{nb\-\_\-kernel\-\_\-\-Elec\-Ew\-\_\-\-Vdw\-None\-\_\-\-Geom\-W4\-P1\-\_\-sse4\-\_\-1\-\_\-double.\-c} }{\pageref{nb__kernel__ElecEw__VdwNone__GeomW4P1__sse4__1__double_8c}}{}
\item\contentsline{section}{src/gmxlib/nonbonded/nb\-\_\-kernel\-\_\-sse4\-\_\-1\-\_\-double/\hyperlink{nb__kernel__ElecEw__VdwNone__GeomW4W4__sse4__1__double_8c}{nb\-\_\-kernel\-\_\-\-Elec\-Ew\-\_\-\-Vdw\-None\-\_\-\-Geom\-W4\-W4\-\_\-sse4\-\_\-1\-\_\-double.\-c} }{\pageref{nb__kernel__ElecEw__VdwNone__GeomW4W4__sse4__1__double_8c}}{}
\item\contentsline{section}{src/gmxlib/nonbonded/nb\-\_\-kernel\-\_\-sse4\-\_\-1\-\_\-double/\hyperlink{nb__kernel__ElecEwSh__VdwLJSh__GeomP1P1__sse4__1__double_8c}{nb\-\_\-kernel\-\_\-\-Elec\-Ew\-Sh\-\_\-\-Vdw\-L\-J\-Sh\-\_\-\-Geom\-P1\-P1\-\_\-sse4\-\_\-1\-\_\-double.\-c} }{\pageref{nb__kernel__ElecEwSh__VdwLJSh__GeomP1P1__sse4__1__double_8c}}{}
\item\contentsline{section}{src/gmxlib/nonbonded/nb\-\_\-kernel\-\_\-sse4\-\_\-1\-\_\-double/\hyperlink{nb__kernel__ElecEwSh__VdwLJSh__GeomW3P1__sse4__1__double_8c}{nb\-\_\-kernel\-\_\-\-Elec\-Ew\-Sh\-\_\-\-Vdw\-L\-J\-Sh\-\_\-\-Geom\-W3\-P1\-\_\-sse4\-\_\-1\-\_\-double.\-c} }{\pageref{nb__kernel__ElecEwSh__VdwLJSh__GeomW3P1__sse4__1__double_8c}}{}
\item\contentsline{section}{src/gmxlib/nonbonded/nb\-\_\-kernel\-\_\-sse4\-\_\-1\-\_\-double/\hyperlink{nb__kernel__ElecEwSh__VdwLJSh__GeomW3W3__sse4__1__double_8c}{nb\-\_\-kernel\-\_\-\-Elec\-Ew\-Sh\-\_\-\-Vdw\-L\-J\-Sh\-\_\-\-Geom\-W3\-W3\-\_\-sse4\-\_\-1\-\_\-double.\-c} }{\pageref{nb__kernel__ElecEwSh__VdwLJSh__GeomW3W3__sse4__1__double_8c}}{}
\item\contentsline{section}{src/gmxlib/nonbonded/nb\-\_\-kernel\-\_\-sse4\-\_\-1\-\_\-double/\hyperlink{nb__kernel__ElecEwSh__VdwLJSh__GeomW4P1__sse4__1__double_8c}{nb\-\_\-kernel\-\_\-\-Elec\-Ew\-Sh\-\_\-\-Vdw\-L\-J\-Sh\-\_\-\-Geom\-W4\-P1\-\_\-sse4\-\_\-1\-\_\-double.\-c} }{\pageref{nb__kernel__ElecEwSh__VdwLJSh__GeomW4P1__sse4__1__double_8c}}{}
\item\contentsline{section}{src/gmxlib/nonbonded/nb\-\_\-kernel\-\_\-sse4\-\_\-1\-\_\-double/\hyperlink{nb__kernel__ElecEwSh__VdwLJSh__GeomW4W4__sse4__1__double_8c}{nb\-\_\-kernel\-\_\-\-Elec\-Ew\-Sh\-\_\-\-Vdw\-L\-J\-Sh\-\_\-\-Geom\-W4\-W4\-\_\-sse4\-\_\-1\-\_\-double.\-c} }{\pageref{nb__kernel__ElecEwSh__VdwLJSh__GeomW4W4__sse4__1__double_8c}}{}
\item\contentsline{section}{src/gmxlib/nonbonded/nb\-\_\-kernel\-\_\-sse4\-\_\-1\-\_\-double/\hyperlink{nb__kernel__ElecEwSh__VdwNone__GeomP1P1__sse4__1__double_8c}{nb\-\_\-kernel\-\_\-\-Elec\-Ew\-Sh\-\_\-\-Vdw\-None\-\_\-\-Geom\-P1\-P1\-\_\-sse4\-\_\-1\-\_\-double.\-c} }{\pageref{nb__kernel__ElecEwSh__VdwNone__GeomP1P1__sse4__1__double_8c}}{}
\item\contentsline{section}{src/gmxlib/nonbonded/nb\-\_\-kernel\-\_\-sse4\-\_\-1\-\_\-double/\hyperlink{nb__kernel__ElecEwSh__VdwNone__GeomW3P1__sse4__1__double_8c}{nb\-\_\-kernel\-\_\-\-Elec\-Ew\-Sh\-\_\-\-Vdw\-None\-\_\-\-Geom\-W3\-P1\-\_\-sse4\-\_\-1\-\_\-double.\-c} }{\pageref{nb__kernel__ElecEwSh__VdwNone__GeomW3P1__sse4__1__double_8c}}{}
\item\contentsline{section}{src/gmxlib/nonbonded/nb\-\_\-kernel\-\_\-sse4\-\_\-1\-\_\-double/\hyperlink{nb__kernel__ElecEwSh__VdwNone__GeomW3W3__sse4__1__double_8c}{nb\-\_\-kernel\-\_\-\-Elec\-Ew\-Sh\-\_\-\-Vdw\-None\-\_\-\-Geom\-W3\-W3\-\_\-sse4\-\_\-1\-\_\-double.\-c} }{\pageref{nb__kernel__ElecEwSh__VdwNone__GeomW3W3__sse4__1__double_8c}}{}
\item\contentsline{section}{src/gmxlib/nonbonded/nb\-\_\-kernel\-\_\-sse4\-\_\-1\-\_\-double/\hyperlink{nb__kernel__ElecEwSh__VdwNone__GeomW4P1__sse4__1__double_8c}{nb\-\_\-kernel\-\_\-\-Elec\-Ew\-Sh\-\_\-\-Vdw\-None\-\_\-\-Geom\-W4\-P1\-\_\-sse4\-\_\-1\-\_\-double.\-c} }{\pageref{nb__kernel__ElecEwSh__VdwNone__GeomW4P1__sse4__1__double_8c}}{}
\item\contentsline{section}{src/gmxlib/nonbonded/nb\-\_\-kernel\-\_\-sse4\-\_\-1\-\_\-double/\hyperlink{nb__kernel__ElecEwSh__VdwNone__GeomW4W4__sse4__1__double_8c}{nb\-\_\-kernel\-\_\-\-Elec\-Ew\-Sh\-\_\-\-Vdw\-None\-\_\-\-Geom\-W4\-W4\-\_\-sse4\-\_\-1\-\_\-double.\-c} }{\pageref{nb__kernel__ElecEwSh__VdwNone__GeomW4W4__sse4__1__double_8c}}{}
\item\contentsline{section}{src/gmxlib/nonbonded/nb\-\_\-kernel\-\_\-sse4\-\_\-1\-\_\-double/\hyperlink{nb__kernel__ElecEwSw__VdwLJSw__GeomP1P1__sse4__1__double_8c}{nb\-\_\-kernel\-\_\-\-Elec\-Ew\-Sw\-\_\-\-Vdw\-L\-J\-Sw\-\_\-\-Geom\-P1\-P1\-\_\-sse4\-\_\-1\-\_\-double.\-c} }{\pageref{nb__kernel__ElecEwSw__VdwLJSw__GeomP1P1__sse4__1__double_8c}}{}
\item\contentsline{section}{src/gmxlib/nonbonded/nb\-\_\-kernel\-\_\-sse4\-\_\-1\-\_\-double/\hyperlink{nb__kernel__ElecEwSw__VdwLJSw__GeomW3P1__sse4__1__double_8c}{nb\-\_\-kernel\-\_\-\-Elec\-Ew\-Sw\-\_\-\-Vdw\-L\-J\-Sw\-\_\-\-Geom\-W3\-P1\-\_\-sse4\-\_\-1\-\_\-double.\-c} }{\pageref{nb__kernel__ElecEwSw__VdwLJSw__GeomW3P1__sse4__1__double_8c}}{}
\item\contentsline{section}{src/gmxlib/nonbonded/nb\-\_\-kernel\-\_\-sse4\-\_\-1\-\_\-double/\hyperlink{nb__kernel__ElecEwSw__VdwLJSw__GeomW3W3__sse4__1__double_8c}{nb\-\_\-kernel\-\_\-\-Elec\-Ew\-Sw\-\_\-\-Vdw\-L\-J\-Sw\-\_\-\-Geom\-W3\-W3\-\_\-sse4\-\_\-1\-\_\-double.\-c} }{\pageref{nb__kernel__ElecEwSw__VdwLJSw__GeomW3W3__sse4__1__double_8c}}{}
\item\contentsline{section}{src/gmxlib/nonbonded/nb\-\_\-kernel\-\_\-sse4\-\_\-1\-\_\-double/\hyperlink{nb__kernel__ElecEwSw__VdwLJSw__GeomW4P1__sse4__1__double_8c}{nb\-\_\-kernel\-\_\-\-Elec\-Ew\-Sw\-\_\-\-Vdw\-L\-J\-Sw\-\_\-\-Geom\-W4\-P1\-\_\-sse4\-\_\-1\-\_\-double.\-c} }{\pageref{nb__kernel__ElecEwSw__VdwLJSw__GeomW4P1__sse4__1__double_8c}}{}
\item\contentsline{section}{src/gmxlib/nonbonded/nb\-\_\-kernel\-\_\-sse4\-\_\-1\-\_\-double/\hyperlink{nb__kernel__ElecEwSw__VdwLJSw__GeomW4W4__sse4__1__double_8c}{nb\-\_\-kernel\-\_\-\-Elec\-Ew\-Sw\-\_\-\-Vdw\-L\-J\-Sw\-\_\-\-Geom\-W4\-W4\-\_\-sse4\-\_\-1\-\_\-double.\-c} }{\pageref{nb__kernel__ElecEwSw__VdwLJSw__GeomW4W4__sse4__1__double_8c}}{}
\item\contentsline{section}{src/gmxlib/nonbonded/nb\-\_\-kernel\-\_\-sse4\-\_\-1\-\_\-double/\hyperlink{nb__kernel__ElecEwSw__VdwNone__GeomP1P1__sse4__1__double_8c}{nb\-\_\-kernel\-\_\-\-Elec\-Ew\-Sw\-\_\-\-Vdw\-None\-\_\-\-Geom\-P1\-P1\-\_\-sse4\-\_\-1\-\_\-double.\-c} }{\pageref{nb__kernel__ElecEwSw__VdwNone__GeomP1P1__sse4__1__double_8c}}{}
\item\contentsline{section}{src/gmxlib/nonbonded/nb\-\_\-kernel\-\_\-sse4\-\_\-1\-\_\-double/\hyperlink{nb__kernel__ElecEwSw__VdwNone__GeomW3P1__sse4__1__double_8c}{nb\-\_\-kernel\-\_\-\-Elec\-Ew\-Sw\-\_\-\-Vdw\-None\-\_\-\-Geom\-W3\-P1\-\_\-sse4\-\_\-1\-\_\-double.\-c} }{\pageref{nb__kernel__ElecEwSw__VdwNone__GeomW3P1__sse4__1__double_8c}}{}
\item\contentsline{section}{src/gmxlib/nonbonded/nb\-\_\-kernel\-\_\-sse4\-\_\-1\-\_\-double/\hyperlink{nb__kernel__ElecEwSw__VdwNone__GeomW3W3__sse4__1__double_8c}{nb\-\_\-kernel\-\_\-\-Elec\-Ew\-Sw\-\_\-\-Vdw\-None\-\_\-\-Geom\-W3\-W3\-\_\-sse4\-\_\-1\-\_\-double.\-c} }{\pageref{nb__kernel__ElecEwSw__VdwNone__GeomW3W3__sse4__1__double_8c}}{}
\item\contentsline{section}{src/gmxlib/nonbonded/nb\-\_\-kernel\-\_\-sse4\-\_\-1\-\_\-double/\hyperlink{nb__kernel__ElecEwSw__VdwNone__GeomW4P1__sse4__1__double_8c}{nb\-\_\-kernel\-\_\-\-Elec\-Ew\-Sw\-\_\-\-Vdw\-None\-\_\-\-Geom\-W4\-P1\-\_\-sse4\-\_\-1\-\_\-double.\-c} }{\pageref{nb__kernel__ElecEwSw__VdwNone__GeomW4P1__sse4__1__double_8c}}{}
\item\contentsline{section}{src/gmxlib/nonbonded/nb\-\_\-kernel\-\_\-sse4\-\_\-1\-\_\-double/\hyperlink{nb__kernel__ElecEwSw__VdwNone__GeomW4W4__sse4__1__double_8c}{nb\-\_\-kernel\-\_\-\-Elec\-Ew\-Sw\-\_\-\-Vdw\-None\-\_\-\-Geom\-W4\-W4\-\_\-sse4\-\_\-1\-\_\-double.\-c} }{\pageref{nb__kernel__ElecEwSw__VdwNone__GeomW4W4__sse4__1__double_8c}}{}
\item\contentsline{section}{src/gmxlib/nonbonded/nb\-\_\-kernel\-\_\-sse4\-\_\-1\-\_\-double/\hyperlink{nb__kernel__ElecGB__VdwCSTab__GeomP1P1__sse4__1__double_8c}{nb\-\_\-kernel\-\_\-\-Elec\-G\-B\-\_\-\-Vdw\-C\-S\-Tab\-\_\-\-Geom\-P1\-P1\-\_\-sse4\-\_\-1\-\_\-double.\-c} }{\pageref{nb__kernel__ElecGB__VdwCSTab__GeomP1P1__sse4__1__double_8c}}{}
\item\contentsline{section}{src/gmxlib/nonbonded/nb\-\_\-kernel\-\_\-sse4\-\_\-1\-\_\-double/\hyperlink{nb__kernel__ElecGB__VdwLJ__GeomP1P1__sse4__1__double_8c}{nb\-\_\-kernel\-\_\-\-Elec\-G\-B\-\_\-\-Vdw\-L\-J\-\_\-\-Geom\-P1\-P1\-\_\-sse4\-\_\-1\-\_\-double.\-c} }{\pageref{nb__kernel__ElecGB__VdwLJ__GeomP1P1__sse4__1__double_8c}}{}
\item\contentsline{section}{src/gmxlib/nonbonded/nb\-\_\-kernel\-\_\-sse4\-\_\-1\-\_\-double/\hyperlink{nb__kernel__ElecGB__VdwNone__GeomP1P1__sse4__1__double_8c}{nb\-\_\-kernel\-\_\-\-Elec\-G\-B\-\_\-\-Vdw\-None\-\_\-\-Geom\-P1\-P1\-\_\-sse4\-\_\-1\-\_\-double.\-c} }{\pageref{nb__kernel__ElecGB__VdwNone__GeomP1P1__sse4__1__double_8c}}{}
\item\contentsline{section}{src/gmxlib/nonbonded/nb\-\_\-kernel\-\_\-sse4\-\_\-1\-\_\-double/\hyperlink{nb__kernel__ElecNone__VdwCSTab__GeomP1P1__sse4__1__double_8c}{nb\-\_\-kernel\-\_\-\-Elec\-None\-\_\-\-Vdw\-C\-S\-Tab\-\_\-\-Geom\-P1\-P1\-\_\-sse4\-\_\-1\-\_\-double.\-c} }{\pageref{nb__kernel__ElecNone__VdwCSTab__GeomP1P1__sse4__1__double_8c}}{}
\item\contentsline{section}{src/gmxlib/nonbonded/nb\-\_\-kernel\-\_\-sse4\-\_\-1\-\_\-double/\hyperlink{nb__kernel__ElecNone__VdwLJ__GeomP1P1__sse4__1__double_8c}{nb\-\_\-kernel\-\_\-\-Elec\-None\-\_\-\-Vdw\-L\-J\-\_\-\-Geom\-P1\-P1\-\_\-sse4\-\_\-1\-\_\-double.\-c} }{\pageref{nb__kernel__ElecNone__VdwLJ__GeomP1P1__sse4__1__double_8c}}{}
\item\contentsline{section}{src/gmxlib/nonbonded/nb\-\_\-kernel\-\_\-sse4\-\_\-1\-\_\-double/\hyperlink{nb__kernel__ElecNone__VdwLJSh__GeomP1P1__sse4__1__double_8c}{nb\-\_\-kernel\-\_\-\-Elec\-None\-\_\-\-Vdw\-L\-J\-Sh\-\_\-\-Geom\-P1\-P1\-\_\-sse4\-\_\-1\-\_\-double.\-c} }{\pageref{nb__kernel__ElecNone__VdwLJSh__GeomP1P1__sse4__1__double_8c}}{}
\item\contentsline{section}{src/gmxlib/nonbonded/nb\-\_\-kernel\-\_\-sse4\-\_\-1\-\_\-double/\hyperlink{nb__kernel__ElecNone__VdwLJSw__GeomP1P1__sse4__1__double_8c}{nb\-\_\-kernel\-\_\-\-Elec\-None\-\_\-\-Vdw\-L\-J\-Sw\-\_\-\-Geom\-P1\-P1\-\_\-sse4\-\_\-1\-\_\-double.\-c} }{\pageref{nb__kernel__ElecNone__VdwLJSw__GeomP1P1__sse4__1__double_8c}}{}
\item\contentsline{section}{src/gmxlib/nonbonded/nb\-\_\-kernel\-\_\-sse4\-\_\-1\-\_\-double/\hyperlink{nb__kernel__ElecRF__VdwCSTab__GeomP1P1__sse4__1__double_8c}{nb\-\_\-kernel\-\_\-\-Elec\-R\-F\-\_\-\-Vdw\-C\-S\-Tab\-\_\-\-Geom\-P1\-P1\-\_\-sse4\-\_\-1\-\_\-double.\-c} }{\pageref{nb__kernel__ElecRF__VdwCSTab__GeomP1P1__sse4__1__double_8c}}{}
\item\contentsline{section}{src/gmxlib/nonbonded/nb\-\_\-kernel\-\_\-sse4\-\_\-1\-\_\-double/\hyperlink{nb__kernel__ElecRF__VdwCSTab__GeomW3P1__sse4__1__double_8c}{nb\-\_\-kernel\-\_\-\-Elec\-R\-F\-\_\-\-Vdw\-C\-S\-Tab\-\_\-\-Geom\-W3\-P1\-\_\-sse4\-\_\-1\-\_\-double.\-c} }{\pageref{nb__kernel__ElecRF__VdwCSTab__GeomW3P1__sse4__1__double_8c}}{}
\item\contentsline{section}{src/gmxlib/nonbonded/nb\-\_\-kernel\-\_\-sse4\-\_\-1\-\_\-double/\hyperlink{nb__kernel__ElecRF__VdwCSTab__GeomW3W3__sse4__1__double_8c}{nb\-\_\-kernel\-\_\-\-Elec\-R\-F\-\_\-\-Vdw\-C\-S\-Tab\-\_\-\-Geom\-W3\-W3\-\_\-sse4\-\_\-1\-\_\-double.\-c} }{\pageref{nb__kernel__ElecRF__VdwCSTab__GeomW3W3__sse4__1__double_8c}}{}
\item\contentsline{section}{src/gmxlib/nonbonded/nb\-\_\-kernel\-\_\-sse4\-\_\-1\-\_\-double/\hyperlink{nb__kernel__ElecRF__VdwCSTab__GeomW4P1__sse4__1__double_8c}{nb\-\_\-kernel\-\_\-\-Elec\-R\-F\-\_\-\-Vdw\-C\-S\-Tab\-\_\-\-Geom\-W4\-P1\-\_\-sse4\-\_\-1\-\_\-double.\-c} }{\pageref{nb__kernel__ElecRF__VdwCSTab__GeomW4P1__sse4__1__double_8c}}{}
\item\contentsline{section}{src/gmxlib/nonbonded/nb\-\_\-kernel\-\_\-sse4\-\_\-1\-\_\-double/\hyperlink{nb__kernel__ElecRF__VdwCSTab__GeomW4W4__sse4__1__double_8c}{nb\-\_\-kernel\-\_\-\-Elec\-R\-F\-\_\-\-Vdw\-C\-S\-Tab\-\_\-\-Geom\-W4\-W4\-\_\-sse4\-\_\-1\-\_\-double.\-c} }{\pageref{nb__kernel__ElecRF__VdwCSTab__GeomW4W4__sse4__1__double_8c}}{}
\item\contentsline{section}{src/gmxlib/nonbonded/nb\-\_\-kernel\-\_\-sse4\-\_\-1\-\_\-double/\hyperlink{nb__kernel__ElecRF__VdwLJ__GeomP1P1__sse4__1__double_8c}{nb\-\_\-kernel\-\_\-\-Elec\-R\-F\-\_\-\-Vdw\-L\-J\-\_\-\-Geom\-P1\-P1\-\_\-sse4\-\_\-1\-\_\-double.\-c} }{\pageref{nb__kernel__ElecRF__VdwLJ__GeomP1P1__sse4__1__double_8c}}{}
\item\contentsline{section}{src/gmxlib/nonbonded/nb\-\_\-kernel\-\_\-sse4\-\_\-1\-\_\-double/\hyperlink{nb__kernel__ElecRF__VdwLJ__GeomW3P1__sse4__1__double_8c}{nb\-\_\-kernel\-\_\-\-Elec\-R\-F\-\_\-\-Vdw\-L\-J\-\_\-\-Geom\-W3\-P1\-\_\-sse4\-\_\-1\-\_\-double.\-c} }{\pageref{nb__kernel__ElecRF__VdwLJ__GeomW3P1__sse4__1__double_8c}}{}
\item\contentsline{section}{src/gmxlib/nonbonded/nb\-\_\-kernel\-\_\-sse4\-\_\-1\-\_\-double/\hyperlink{nb__kernel__ElecRF__VdwLJ__GeomW3W3__sse4__1__double_8c}{nb\-\_\-kernel\-\_\-\-Elec\-R\-F\-\_\-\-Vdw\-L\-J\-\_\-\-Geom\-W3\-W3\-\_\-sse4\-\_\-1\-\_\-double.\-c} }{\pageref{nb__kernel__ElecRF__VdwLJ__GeomW3W3__sse4__1__double_8c}}{}
\item\contentsline{section}{src/gmxlib/nonbonded/nb\-\_\-kernel\-\_\-sse4\-\_\-1\-\_\-double/\hyperlink{nb__kernel__ElecRF__VdwLJ__GeomW4P1__sse4__1__double_8c}{nb\-\_\-kernel\-\_\-\-Elec\-R\-F\-\_\-\-Vdw\-L\-J\-\_\-\-Geom\-W4\-P1\-\_\-sse4\-\_\-1\-\_\-double.\-c} }{\pageref{nb__kernel__ElecRF__VdwLJ__GeomW4P1__sse4__1__double_8c}}{}
\item\contentsline{section}{src/gmxlib/nonbonded/nb\-\_\-kernel\-\_\-sse4\-\_\-1\-\_\-double/\hyperlink{nb__kernel__ElecRF__VdwLJ__GeomW4W4__sse4__1__double_8c}{nb\-\_\-kernel\-\_\-\-Elec\-R\-F\-\_\-\-Vdw\-L\-J\-\_\-\-Geom\-W4\-W4\-\_\-sse4\-\_\-1\-\_\-double.\-c} }{\pageref{nb__kernel__ElecRF__VdwLJ__GeomW4W4__sse4__1__double_8c}}{}
\item\contentsline{section}{src/gmxlib/nonbonded/nb\-\_\-kernel\-\_\-sse4\-\_\-1\-\_\-double/\hyperlink{nb__kernel__ElecRF__VdwNone__GeomP1P1__sse4__1__double_8c}{nb\-\_\-kernel\-\_\-\-Elec\-R\-F\-\_\-\-Vdw\-None\-\_\-\-Geom\-P1\-P1\-\_\-sse4\-\_\-1\-\_\-double.\-c} }{\pageref{nb__kernel__ElecRF__VdwNone__GeomP1P1__sse4__1__double_8c}}{}
\item\contentsline{section}{src/gmxlib/nonbonded/nb\-\_\-kernel\-\_\-sse4\-\_\-1\-\_\-double/\hyperlink{nb__kernel__ElecRF__VdwNone__GeomW3P1__sse4__1__double_8c}{nb\-\_\-kernel\-\_\-\-Elec\-R\-F\-\_\-\-Vdw\-None\-\_\-\-Geom\-W3\-P1\-\_\-sse4\-\_\-1\-\_\-double.\-c} }{\pageref{nb__kernel__ElecRF__VdwNone__GeomW3P1__sse4__1__double_8c}}{}
\item\contentsline{section}{src/gmxlib/nonbonded/nb\-\_\-kernel\-\_\-sse4\-\_\-1\-\_\-double/\hyperlink{nb__kernel__ElecRF__VdwNone__GeomW3W3__sse4__1__double_8c}{nb\-\_\-kernel\-\_\-\-Elec\-R\-F\-\_\-\-Vdw\-None\-\_\-\-Geom\-W3\-W3\-\_\-sse4\-\_\-1\-\_\-double.\-c} }{\pageref{nb__kernel__ElecRF__VdwNone__GeomW3W3__sse4__1__double_8c}}{}
\item\contentsline{section}{src/gmxlib/nonbonded/nb\-\_\-kernel\-\_\-sse4\-\_\-1\-\_\-double/\hyperlink{nb__kernel__ElecRF__VdwNone__GeomW4P1__sse4__1__double_8c}{nb\-\_\-kernel\-\_\-\-Elec\-R\-F\-\_\-\-Vdw\-None\-\_\-\-Geom\-W4\-P1\-\_\-sse4\-\_\-1\-\_\-double.\-c} }{\pageref{nb__kernel__ElecRF__VdwNone__GeomW4P1__sse4__1__double_8c}}{}
\item\contentsline{section}{src/gmxlib/nonbonded/nb\-\_\-kernel\-\_\-sse4\-\_\-1\-\_\-double/\hyperlink{nb__kernel__ElecRF__VdwNone__GeomW4W4__sse4__1__double_8c}{nb\-\_\-kernel\-\_\-\-Elec\-R\-F\-\_\-\-Vdw\-None\-\_\-\-Geom\-W4\-W4\-\_\-sse4\-\_\-1\-\_\-double.\-c} }{\pageref{nb__kernel__ElecRF__VdwNone__GeomW4W4__sse4__1__double_8c}}{}
\item\contentsline{section}{src/gmxlib/nonbonded/nb\-\_\-kernel\-\_\-sse4\-\_\-1\-\_\-double/\hyperlink{nb__kernel__ElecRFCut__VdwCSTab__GeomP1P1__sse4__1__double_8c}{nb\-\_\-kernel\-\_\-\-Elec\-R\-F\-Cut\-\_\-\-Vdw\-C\-S\-Tab\-\_\-\-Geom\-P1\-P1\-\_\-sse4\-\_\-1\-\_\-double.\-c} }{\pageref{nb__kernel__ElecRFCut__VdwCSTab__GeomP1P1__sse4__1__double_8c}}{}
\item\contentsline{section}{src/gmxlib/nonbonded/nb\-\_\-kernel\-\_\-sse4\-\_\-1\-\_\-double/\hyperlink{nb__kernel__ElecRFCut__VdwCSTab__GeomW3P1__sse4__1__double_8c}{nb\-\_\-kernel\-\_\-\-Elec\-R\-F\-Cut\-\_\-\-Vdw\-C\-S\-Tab\-\_\-\-Geom\-W3\-P1\-\_\-sse4\-\_\-1\-\_\-double.\-c} }{\pageref{nb__kernel__ElecRFCut__VdwCSTab__GeomW3P1__sse4__1__double_8c}}{}
\item\contentsline{section}{src/gmxlib/nonbonded/nb\-\_\-kernel\-\_\-sse4\-\_\-1\-\_\-double/\hyperlink{nb__kernel__ElecRFCut__VdwCSTab__GeomW3W3__sse4__1__double_8c}{nb\-\_\-kernel\-\_\-\-Elec\-R\-F\-Cut\-\_\-\-Vdw\-C\-S\-Tab\-\_\-\-Geom\-W3\-W3\-\_\-sse4\-\_\-1\-\_\-double.\-c} }{\pageref{nb__kernel__ElecRFCut__VdwCSTab__GeomW3W3__sse4__1__double_8c}}{}
\item\contentsline{section}{src/gmxlib/nonbonded/nb\-\_\-kernel\-\_\-sse4\-\_\-1\-\_\-double/\hyperlink{nb__kernel__ElecRFCut__VdwCSTab__GeomW4P1__sse4__1__double_8c}{nb\-\_\-kernel\-\_\-\-Elec\-R\-F\-Cut\-\_\-\-Vdw\-C\-S\-Tab\-\_\-\-Geom\-W4\-P1\-\_\-sse4\-\_\-1\-\_\-double.\-c} }{\pageref{nb__kernel__ElecRFCut__VdwCSTab__GeomW4P1__sse4__1__double_8c}}{}
\item\contentsline{section}{src/gmxlib/nonbonded/nb\-\_\-kernel\-\_\-sse4\-\_\-1\-\_\-double/\hyperlink{nb__kernel__ElecRFCut__VdwCSTab__GeomW4W4__sse4__1__double_8c}{nb\-\_\-kernel\-\_\-\-Elec\-R\-F\-Cut\-\_\-\-Vdw\-C\-S\-Tab\-\_\-\-Geom\-W4\-W4\-\_\-sse4\-\_\-1\-\_\-double.\-c} }{\pageref{nb__kernel__ElecRFCut__VdwCSTab__GeomW4W4__sse4__1__double_8c}}{}
\item\contentsline{section}{src/gmxlib/nonbonded/nb\-\_\-kernel\-\_\-sse4\-\_\-1\-\_\-double/\hyperlink{nb__kernel__ElecRFCut__VdwLJSh__GeomP1P1__sse4__1__double_8c}{nb\-\_\-kernel\-\_\-\-Elec\-R\-F\-Cut\-\_\-\-Vdw\-L\-J\-Sh\-\_\-\-Geom\-P1\-P1\-\_\-sse4\-\_\-1\-\_\-double.\-c} }{\pageref{nb__kernel__ElecRFCut__VdwLJSh__GeomP1P1__sse4__1__double_8c}}{}
\item\contentsline{section}{src/gmxlib/nonbonded/nb\-\_\-kernel\-\_\-sse4\-\_\-1\-\_\-double/\hyperlink{nb__kernel__ElecRFCut__VdwLJSh__GeomW3P1__sse4__1__double_8c}{nb\-\_\-kernel\-\_\-\-Elec\-R\-F\-Cut\-\_\-\-Vdw\-L\-J\-Sh\-\_\-\-Geom\-W3\-P1\-\_\-sse4\-\_\-1\-\_\-double.\-c} }{\pageref{nb__kernel__ElecRFCut__VdwLJSh__GeomW3P1__sse4__1__double_8c}}{}
\item\contentsline{section}{src/gmxlib/nonbonded/nb\-\_\-kernel\-\_\-sse4\-\_\-1\-\_\-double/\hyperlink{nb__kernel__ElecRFCut__VdwLJSh__GeomW3W3__sse4__1__double_8c}{nb\-\_\-kernel\-\_\-\-Elec\-R\-F\-Cut\-\_\-\-Vdw\-L\-J\-Sh\-\_\-\-Geom\-W3\-W3\-\_\-sse4\-\_\-1\-\_\-double.\-c} }{\pageref{nb__kernel__ElecRFCut__VdwLJSh__GeomW3W3__sse4__1__double_8c}}{}
\item\contentsline{section}{src/gmxlib/nonbonded/nb\-\_\-kernel\-\_\-sse4\-\_\-1\-\_\-double/\hyperlink{nb__kernel__ElecRFCut__VdwLJSh__GeomW4P1__sse4__1__double_8c}{nb\-\_\-kernel\-\_\-\-Elec\-R\-F\-Cut\-\_\-\-Vdw\-L\-J\-Sh\-\_\-\-Geom\-W4\-P1\-\_\-sse4\-\_\-1\-\_\-double.\-c} }{\pageref{nb__kernel__ElecRFCut__VdwLJSh__GeomW4P1__sse4__1__double_8c}}{}
\item\contentsline{section}{src/gmxlib/nonbonded/nb\-\_\-kernel\-\_\-sse4\-\_\-1\-\_\-double/\hyperlink{nb__kernel__ElecRFCut__VdwLJSh__GeomW4W4__sse4__1__double_8c}{nb\-\_\-kernel\-\_\-\-Elec\-R\-F\-Cut\-\_\-\-Vdw\-L\-J\-Sh\-\_\-\-Geom\-W4\-W4\-\_\-sse4\-\_\-1\-\_\-double.\-c} }{\pageref{nb__kernel__ElecRFCut__VdwLJSh__GeomW4W4__sse4__1__double_8c}}{}
\item\contentsline{section}{src/gmxlib/nonbonded/nb\-\_\-kernel\-\_\-sse4\-\_\-1\-\_\-double/\hyperlink{nb__kernel__ElecRFCut__VdwLJSw__GeomP1P1__sse4__1__double_8c}{nb\-\_\-kernel\-\_\-\-Elec\-R\-F\-Cut\-\_\-\-Vdw\-L\-J\-Sw\-\_\-\-Geom\-P1\-P1\-\_\-sse4\-\_\-1\-\_\-double.\-c} }{\pageref{nb__kernel__ElecRFCut__VdwLJSw__GeomP1P1__sse4__1__double_8c}}{}
\item\contentsline{section}{src/gmxlib/nonbonded/nb\-\_\-kernel\-\_\-sse4\-\_\-1\-\_\-double/\hyperlink{nb__kernel__ElecRFCut__VdwLJSw__GeomW3P1__sse4__1__double_8c}{nb\-\_\-kernel\-\_\-\-Elec\-R\-F\-Cut\-\_\-\-Vdw\-L\-J\-Sw\-\_\-\-Geom\-W3\-P1\-\_\-sse4\-\_\-1\-\_\-double.\-c} }{\pageref{nb__kernel__ElecRFCut__VdwLJSw__GeomW3P1__sse4__1__double_8c}}{}
\item\contentsline{section}{src/gmxlib/nonbonded/nb\-\_\-kernel\-\_\-sse4\-\_\-1\-\_\-double/\hyperlink{nb__kernel__ElecRFCut__VdwLJSw__GeomW3W3__sse4__1__double_8c}{nb\-\_\-kernel\-\_\-\-Elec\-R\-F\-Cut\-\_\-\-Vdw\-L\-J\-Sw\-\_\-\-Geom\-W3\-W3\-\_\-sse4\-\_\-1\-\_\-double.\-c} }{\pageref{nb__kernel__ElecRFCut__VdwLJSw__GeomW3W3__sse4__1__double_8c}}{}
\item\contentsline{section}{src/gmxlib/nonbonded/nb\-\_\-kernel\-\_\-sse4\-\_\-1\-\_\-double/\hyperlink{nb__kernel__ElecRFCut__VdwLJSw__GeomW4P1__sse4__1__double_8c}{nb\-\_\-kernel\-\_\-\-Elec\-R\-F\-Cut\-\_\-\-Vdw\-L\-J\-Sw\-\_\-\-Geom\-W4\-P1\-\_\-sse4\-\_\-1\-\_\-double.\-c} }{\pageref{nb__kernel__ElecRFCut__VdwLJSw__GeomW4P1__sse4__1__double_8c}}{}
\item\contentsline{section}{src/gmxlib/nonbonded/nb\-\_\-kernel\-\_\-sse4\-\_\-1\-\_\-double/\hyperlink{nb__kernel__ElecRFCut__VdwLJSw__GeomW4W4__sse4__1__double_8c}{nb\-\_\-kernel\-\_\-\-Elec\-R\-F\-Cut\-\_\-\-Vdw\-L\-J\-Sw\-\_\-\-Geom\-W4\-W4\-\_\-sse4\-\_\-1\-\_\-double.\-c} }{\pageref{nb__kernel__ElecRFCut__VdwLJSw__GeomW4W4__sse4__1__double_8c}}{}
\item\contentsline{section}{src/gmxlib/nonbonded/nb\-\_\-kernel\-\_\-sse4\-\_\-1\-\_\-double/\hyperlink{nb__kernel__ElecRFCut__VdwNone__GeomP1P1__sse4__1__double_8c}{nb\-\_\-kernel\-\_\-\-Elec\-R\-F\-Cut\-\_\-\-Vdw\-None\-\_\-\-Geom\-P1\-P1\-\_\-sse4\-\_\-1\-\_\-double.\-c} }{\pageref{nb__kernel__ElecRFCut__VdwNone__GeomP1P1__sse4__1__double_8c}}{}
\item\contentsline{section}{src/gmxlib/nonbonded/nb\-\_\-kernel\-\_\-sse4\-\_\-1\-\_\-double/\hyperlink{nb__kernel__ElecRFCut__VdwNone__GeomW3P1__sse4__1__double_8c}{nb\-\_\-kernel\-\_\-\-Elec\-R\-F\-Cut\-\_\-\-Vdw\-None\-\_\-\-Geom\-W3\-P1\-\_\-sse4\-\_\-1\-\_\-double.\-c} }{\pageref{nb__kernel__ElecRFCut__VdwNone__GeomW3P1__sse4__1__double_8c}}{}
\item\contentsline{section}{src/gmxlib/nonbonded/nb\-\_\-kernel\-\_\-sse4\-\_\-1\-\_\-double/\hyperlink{nb__kernel__ElecRFCut__VdwNone__GeomW3W3__sse4__1__double_8c}{nb\-\_\-kernel\-\_\-\-Elec\-R\-F\-Cut\-\_\-\-Vdw\-None\-\_\-\-Geom\-W3\-W3\-\_\-sse4\-\_\-1\-\_\-double.\-c} }{\pageref{nb__kernel__ElecRFCut__VdwNone__GeomW3W3__sse4__1__double_8c}}{}
\item\contentsline{section}{src/gmxlib/nonbonded/nb\-\_\-kernel\-\_\-sse4\-\_\-1\-\_\-double/\hyperlink{nb__kernel__ElecRFCut__VdwNone__GeomW4P1__sse4__1__double_8c}{nb\-\_\-kernel\-\_\-\-Elec\-R\-F\-Cut\-\_\-\-Vdw\-None\-\_\-\-Geom\-W4\-P1\-\_\-sse4\-\_\-1\-\_\-double.\-c} }{\pageref{nb__kernel__ElecRFCut__VdwNone__GeomW4P1__sse4__1__double_8c}}{}
\item\contentsline{section}{src/gmxlib/nonbonded/nb\-\_\-kernel\-\_\-sse4\-\_\-1\-\_\-double/\hyperlink{nb__kernel__ElecRFCut__VdwNone__GeomW4W4__sse4__1__double_8c}{nb\-\_\-kernel\-\_\-\-Elec\-R\-F\-Cut\-\_\-\-Vdw\-None\-\_\-\-Geom\-W4\-W4\-\_\-sse4\-\_\-1\-\_\-double.\-c} }{\pageref{nb__kernel__ElecRFCut__VdwNone__GeomW4W4__sse4__1__double_8c}}{}
\item\contentsline{section}{src/gmxlib/nonbonded/nb\-\_\-kernel\-\_\-sse4\-\_\-1\-\_\-double/\hyperlink{nb__kernel__sse4__1__double_8c}{nb\-\_\-kernel\-\_\-sse4\-\_\-1\-\_\-double.\-c} }{\pageref{nb__kernel__sse4__1__double_8c}}{}
\item\contentsline{section}{src/gmxlib/nonbonded/nb\-\_\-kernel\-\_\-sse4\-\_\-1\-\_\-double/\hyperlink{nb__kernel__sse4__1__double_8h}{nb\-\_\-kernel\-\_\-sse4\-\_\-1\-\_\-double.\-h} }{\pageref{nb__kernel__sse4__1__double_8h}}{}
\item\contentsline{section}{src/gmxlib/nonbonded/nb\-\_\-kernel\-\_\-sse4\-\_\-1\-\_\-single/\hyperlink{kernelutil__x86__sse4__1__single_8h}{kernelutil\-\_\-x86\-\_\-sse4\-\_\-1\-\_\-single.\-h} }{\pageref{kernelutil__x86__sse4__1__single_8h}}{}
\item\contentsline{section}{src/gmxlib/nonbonded/nb\-\_\-kernel\-\_\-sse4\-\_\-1\-\_\-single/\hyperlink{nb__kernel__ElecCoul__VdwCSTab__GeomP1P1__sse4__1__single_8c}{nb\-\_\-kernel\-\_\-\-Elec\-Coul\-\_\-\-Vdw\-C\-S\-Tab\-\_\-\-Geom\-P1\-P1\-\_\-sse4\-\_\-1\-\_\-single.\-c} }{\pageref{nb__kernel__ElecCoul__VdwCSTab__GeomP1P1__sse4__1__single_8c}}{}
\item\contentsline{section}{src/gmxlib/nonbonded/nb\-\_\-kernel\-\_\-sse4\-\_\-1\-\_\-single/\hyperlink{nb__kernel__ElecCoul__VdwCSTab__GeomW3P1__sse4__1__single_8c}{nb\-\_\-kernel\-\_\-\-Elec\-Coul\-\_\-\-Vdw\-C\-S\-Tab\-\_\-\-Geom\-W3\-P1\-\_\-sse4\-\_\-1\-\_\-single.\-c} }{\pageref{nb__kernel__ElecCoul__VdwCSTab__GeomW3P1__sse4__1__single_8c}}{}
\item\contentsline{section}{src/gmxlib/nonbonded/nb\-\_\-kernel\-\_\-sse4\-\_\-1\-\_\-single/\hyperlink{nb__kernel__ElecCoul__VdwCSTab__GeomW3W3__sse4__1__single_8c}{nb\-\_\-kernel\-\_\-\-Elec\-Coul\-\_\-\-Vdw\-C\-S\-Tab\-\_\-\-Geom\-W3\-W3\-\_\-sse4\-\_\-1\-\_\-single.\-c} }{\pageref{nb__kernel__ElecCoul__VdwCSTab__GeomW3W3__sse4__1__single_8c}}{}
\item\contentsline{section}{src/gmxlib/nonbonded/nb\-\_\-kernel\-\_\-sse4\-\_\-1\-\_\-single/\hyperlink{nb__kernel__ElecCoul__VdwCSTab__GeomW4P1__sse4__1__single_8c}{nb\-\_\-kernel\-\_\-\-Elec\-Coul\-\_\-\-Vdw\-C\-S\-Tab\-\_\-\-Geom\-W4\-P1\-\_\-sse4\-\_\-1\-\_\-single.\-c} }{\pageref{nb__kernel__ElecCoul__VdwCSTab__GeomW4P1__sse4__1__single_8c}}{}
\item\contentsline{section}{src/gmxlib/nonbonded/nb\-\_\-kernel\-\_\-sse4\-\_\-1\-\_\-single/\hyperlink{nb__kernel__ElecCoul__VdwCSTab__GeomW4W4__sse4__1__single_8c}{nb\-\_\-kernel\-\_\-\-Elec\-Coul\-\_\-\-Vdw\-C\-S\-Tab\-\_\-\-Geom\-W4\-W4\-\_\-sse4\-\_\-1\-\_\-single.\-c} }{\pageref{nb__kernel__ElecCoul__VdwCSTab__GeomW4W4__sse4__1__single_8c}}{}
\item\contentsline{section}{src/gmxlib/nonbonded/nb\-\_\-kernel\-\_\-sse4\-\_\-1\-\_\-single/\hyperlink{nb__kernel__ElecCoul__VdwLJ__GeomP1P1__sse4__1__single_8c}{nb\-\_\-kernel\-\_\-\-Elec\-Coul\-\_\-\-Vdw\-L\-J\-\_\-\-Geom\-P1\-P1\-\_\-sse4\-\_\-1\-\_\-single.\-c} }{\pageref{nb__kernel__ElecCoul__VdwLJ__GeomP1P1__sse4__1__single_8c}}{}
\item\contentsline{section}{src/gmxlib/nonbonded/nb\-\_\-kernel\-\_\-sse4\-\_\-1\-\_\-single/\hyperlink{nb__kernel__ElecCoul__VdwLJ__GeomW3P1__sse4__1__single_8c}{nb\-\_\-kernel\-\_\-\-Elec\-Coul\-\_\-\-Vdw\-L\-J\-\_\-\-Geom\-W3\-P1\-\_\-sse4\-\_\-1\-\_\-single.\-c} }{\pageref{nb__kernel__ElecCoul__VdwLJ__GeomW3P1__sse4__1__single_8c}}{}
\item\contentsline{section}{src/gmxlib/nonbonded/nb\-\_\-kernel\-\_\-sse4\-\_\-1\-\_\-single/\hyperlink{nb__kernel__ElecCoul__VdwLJ__GeomW3W3__sse4__1__single_8c}{nb\-\_\-kernel\-\_\-\-Elec\-Coul\-\_\-\-Vdw\-L\-J\-\_\-\-Geom\-W3\-W3\-\_\-sse4\-\_\-1\-\_\-single.\-c} }{\pageref{nb__kernel__ElecCoul__VdwLJ__GeomW3W3__sse4__1__single_8c}}{}
\item\contentsline{section}{src/gmxlib/nonbonded/nb\-\_\-kernel\-\_\-sse4\-\_\-1\-\_\-single/\hyperlink{nb__kernel__ElecCoul__VdwLJ__GeomW4P1__sse4__1__single_8c}{nb\-\_\-kernel\-\_\-\-Elec\-Coul\-\_\-\-Vdw\-L\-J\-\_\-\-Geom\-W4\-P1\-\_\-sse4\-\_\-1\-\_\-single.\-c} }{\pageref{nb__kernel__ElecCoul__VdwLJ__GeomW4P1__sse4__1__single_8c}}{}
\item\contentsline{section}{src/gmxlib/nonbonded/nb\-\_\-kernel\-\_\-sse4\-\_\-1\-\_\-single/\hyperlink{nb__kernel__ElecCoul__VdwLJ__GeomW4W4__sse4__1__single_8c}{nb\-\_\-kernel\-\_\-\-Elec\-Coul\-\_\-\-Vdw\-L\-J\-\_\-\-Geom\-W4\-W4\-\_\-sse4\-\_\-1\-\_\-single.\-c} }{\pageref{nb__kernel__ElecCoul__VdwLJ__GeomW4W4__sse4__1__single_8c}}{}
\item\contentsline{section}{src/gmxlib/nonbonded/nb\-\_\-kernel\-\_\-sse4\-\_\-1\-\_\-single/\hyperlink{nb__kernel__ElecCoul__VdwNone__GeomP1P1__sse4__1__single_8c}{nb\-\_\-kernel\-\_\-\-Elec\-Coul\-\_\-\-Vdw\-None\-\_\-\-Geom\-P1\-P1\-\_\-sse4\-\_\-1\-\_\-single.\-c} }{\pageref{nb__kernel__ElecCoul__VdwNone__GeomP1P1__sse4__1__single_8c}}{}
\item\contentsline{section}{src/gmxlib/nonbonded/nb\-\_\-kernel\-\_\-sse4\-\_\-1\-\_\-single/\hyperlink{nb__kernel__ElecCoul__VdwNone__GeomW3P1__sse4__1__single_8c}{nb\-\_\-kernel\-\_\-\-Elec\-Coul\-\_\-\-Vdw\-None\-\_\-\-Geom\-W3\-P1\-\_\-sse4\-\_\-1\-\_\-single.\-c} }{\pageref{nb__kernel__ElecCoul__VdwNone__GeomW3P1__sse4__1__single_8c}}{}
\item\contentsline{section}{src/gmxlib/nonbonded/nb\-\_\-kernel\-\_\-sse4\-\_\-1\-\_\-single/\hyperlink{nb__kernel__ElecCoul__VdwNone__GeomW3W3__sse4__1__single_8c}{nb\-\_\-kernel\-\_\-\-Elec\-Coul\-\_\-\-Vdw\-None\-\_\-\-Geom\-W3\-W3\-\_\-sse4\-\_\-1\-\_\-single.\-c} }{\pageref{nb__kernel__ElecCoul__VdwNone__GeomW3W3__sse4__1__single_8c}}{}
\item\contentsline{section}{src/gmxlib/nonbonded/nb\-\_\-kernel\-\_\-sse4\-\_\-1\-\_\-single/\hyperlink{nb__kernel__ElecCoul__VdwNone__GeomW4P1__sse4__1__single_8c}{nb\-\_\-kernel\-\_\-\-Elec\-Coul\-\_\-\-Vdw\-None\-\_\-\-Geom\-W4\-P1\-\_\-sse4\-\_\-1\-\_\-single.\-c} }{\pageref{nb__kernel__ElecCoul__VdwNone__GeomW4P1__sse4__1__single_8c}}{}
\item\contentsline{section}{src/gmxlib/nonbonded/nb\-\_\-kernel\-\_\-sse4\-\_\-1\-\_\-single/\hyperlink{nb__kernel__ElecCoul__VdwNone__GeomW4W4__sse4__1__single_8c}{nb\-\_\-kernel\-\_\-\-Elec\-Coul\-\_\-\-Vdw\-None\-\_\-\-Geom\-W4\-W4\-\_\-sse4\-\_\-1\-\_\-single.\-c} }{\pageref{nb__kernel__ElecCoul__VdwNone__GeomW4W4__sse4__1__single_8c}}{}
\item\contentsline{section}{src/gmxlib/nonbonded/nb\-\_\-kernel\-\_\-sse4\-\_\-1\-\_\-single/\hyperlink{nb__kernel__ElecCSTab__VdwCSTab__GeomP1P1__sse4__1__single_8c}{nb\-\_\-kernel\-\_\-\-Elec\-C\-S\-Tab\-\_\-\-Vdw\-C\-S\-Tab\-\_\-\-Geom\-P1\-P1\-\_\-sse4\-\_\-1\-\_\-single.\-c} }{\pageref{nb__kernel__ElecCSTab__VdwCSTab__GeomP1P1__sse4__1__single_8c}}{}
\item\contentsline{section}{src/gmxlib/nonbonded/nb\-\_\-kernel\-\_\-sse4\-\_\-1\-\_\-single/\hyperlink{nb__kernel__ElecCSTab__VdwCSTab__GeomW3P1__sse4__1__single_8c}{nb\-\_\-kernel\-\_\-\-Elec\-C\-S\-Tab\-\_\-\-Vdw\-C\-S\-Tab\-\_\-\-Geom\-W3\-P1\-\_\-sse4\-\_\-1\-\_\-single.\-c} }{\pageref{nb__kernel__ElecCSTab__VdwCSTab__GeomW3P1__sse4__1__single_8c}}{}
\item\contentsline{section}{src/gmxlib/nonbonded/nb\-\_\-kernel\-\_\-sse4\-\_\-1\-\_\-single/\hyperlink{nb__kernel__ElecCSTab__VdwCSTab__GeomW3W3__sse4__1__single_8c}{nb\-\_\-kernel\-\_\-\-Elec\-C\-S\-Tab\-\_\-\-Vdw\-C\-S\-Tab\-\_\-\-Geom\-W3\-W3\-\_\-sse4\-\_\-1\-\_\-single.\-c} }{\pageref{nb__kernel__ElecCSTab__VdwCSTab__GeomW3W3__sse4__1__single_8c}}{}
\item\contentsline{section}{src/gmxlib/nonbonded/nb\-\_\-kernel\-\_\-sse4\-\_\-1\-\_\-single/\hyperlink{nb__kernel__ElecCSTab__VdwCSTab__GeomW4P1__sse4__1__single_8c}{nb\-\_\-kernel\-\_\-\-Elec\-C\-S\-Tab\-\_\-\-Vdw\-C\-S\-Tab\-\_\-\-Geom\-W4\-P1\-\_\-sse4\-\_\-1\-\_\-single.\-c} }{\pageref{nb__kernel__ElecCSTab__VdwCSTab__GeomW4P1__sse4__1__single_8c}}{}
\item\contentsline{section}{src/gmxlib/nonbonded/nb\-\_\-kernel\-\_\-sse4\-\_\-1\-\_\-single/\hyperlink{nb__kernel__ElecCSTab__VdwCSTab__GeomW4W4__sse4__1__single_8c}{nb\-\_\-kernel\-\_\-\-Elec\-C\-S\-Tab\-\_\-\-Vdw\-C\-S\-Tab\-\_\-\-Geom\-W4\-W4\-\_\-sse4\-\_\-1\-\_\-single.\-c} }{\pageref{nb__kernel__ElecCSTab__VdwCSTab__GeomW4W4__sse4__1__single_8c}}{}
\item\contentsline{section}{src/gmxlib/nonbonded/nb\-\_\-kernel\-\_\-sse4\-\_\-1\-\_\-single/\hyperlink{nb__kernel__ElecCSTab__VdwLJ__GeomP1P1__sse4__1__single_8c}{nb\-\_\-kernel\-\_\-\-Elec\-C\-S\-Tab\-\_\-\-Vdw\-L\-J\-\_\-\-Geom\-P1\-P1\-\_\-sse4\-\_\-1\-\_\-single.\-c} }{\pageref{nb__kernel__ElecCSTab__VdwLJ__GeomP1P1__sse4__1__single_8c}}{}
\item\contentsline{section}{src/gmxlib/nonbonded/nb\-\_\-kernel\-\_\-sse4\-\_\-1\-\_\-single/\hyperlink{nb__kernel__ElecCSTab__VdwLJ__GeomW3P1__sse4__1__single_8c}{nb\-\_\-kernel\-\_\-\-Elec\-C\-S\-Tab\-\_\-\-Vdw\-L\-J\-\_\-\-Geom\-W3\-P1\-\_\-sse4\-\_\-1\-\_\-single.\-c} }{\pageref{nb__kernel__ElecCSTab__VdwLJ__GeomW3P1__sse4__1__single_8c}}{}
\item\contentsline{section}{src/gmxlib/nonbonded/nb\-\_\-kernel\-\_\-sse4\-\_\-1\-\_\-single/\hyperlink{nb__kernel__ElecCSTab__VdwLJ__GeomW3W3__sse4__1__single_8c}{nb\-\_\-kernel\-\_\-\-Elec\-C\-S\-Tab\-\_\-\-Vdw\-L\-J\-\_\-\-Geom\-W3\-W3\-\_\-sse4\-\_\-1\-\_\-single.\-c} }{\pageref{nb__kernel__ElecCSTab__VdwLJ__GeomW3W3__sse4__1__single_8c}}{}
\item\contentsline{section}{src/gmxlib/nonbonded/nb\-\_\-kernel\-\_\-sse4\-\_\-1\-\_\-single/\hyperlink{nb__kernel__ElecCSTab__VdwLJ__GeomW4P1__sse4__1__single_8c}{nb\-\_\-kernel\-\_\-\-Elec\-C\-S\-Tab\-\_\-\-Vdw\-L\-J\-\_\-\-Geom\-W4\-P1\-\_\-sse4\-\_\-1\-\_\-single.\-c} }{\pageref{nb__kernel__ElecCSTab__VdwLJ__GeomW4P1__sse4__1__single_8c}}{}
\item\contentsline{section}{src/gmxlib/nonbonded/nb\-\_\-kernel\-\_\-sse4\-\_\-1\-\_\-single/\hyperlink{nb__kernel__ElecCSTab__VdwLJ__GeomW4W4__sse4__1__single_8c}{nb\-\_\-kernel\-\_\-\-Elec\-C\-S\-Tab\-\_\-\-Vdw\-L\-J\-\_\-\-Geom\-W4\-W4\-\_\-sse4\-\_\-1\-\_\-single.\-c} }{\pageref{nb__kernel__ElecCSTab__VdwLJ__GeomW4W4__sse4__1__single_8c}}{}
\item\contentsline{section}{src/gmxlib/nonbonded/nb\-\_\-kernel\-\_\-sse4\-\_\-1\-\_\-single/\hyperlink{nb__kernel__ElecCSTab__VdwNone__GeomP1P1__sse4__1__single_8c}{nb\-\_\-kernel\-\_\-\-Elec\-C\-S\-Tab\-\_\-\-Vdw\-None\-\_\-\-Geom\-P1\-P1\-\_\-sse4\-\_\-1\-\_\-single.\-c} }{\pageref{nb__kernel__ElecCSTab__VdwNone__GeomP1P1__sse4__1__single_8c}}{}
\item\contentsline{section}{src/gmxlib/nonbonded/nb\-\_\-kernel\-\_\-sse4\-\_\-1\-\_\-single/\hyperlink{nb__kernel__ElecCSTab__VdwNone__GeomW3P1__sse4__1__single_8c}{nb\-\_\-kernel\-\_\-\-Elec\-C\-S\-Tab\-\_\-\-Vdw\-None\-\_\-\-Geom\-W3\-P1\-\_\-sse4\-\_\-1\-\_\-single.\-c} }{\pageref{nb__kernel__ElecCSTab__VdwNone__GeomW3P1__sse4__1__single_8c}}{}
\item\contentsline{section}{src/gmxlib/nonbonded/nb\-\_\-kernel\-\_\-sse4\-\_\-1\-\_\-single/\hyperlink{nb__kernel__ElecCSTab__VdwNone__GeomW3W3__sse4__1__single_8c}{nb\-\_\-kernel\-\_\-\-Elec\-C\-S\-Tab\-\_\-\-Vdw\-None\-\_\-\-Geom\-W3\-W3\-\_\-sse4\-\_\-1\-\_\-single.\-c} }{\pageref{nb__kernel__ElecCSTab__VdwNone__GeomW3W3__sse4__1__single_8c}}{}
\item\contentsline{section}{src/gmxlib/nonbonded/nb\-\_\-kernel\-\_\-sse4\-\_\-1\-\_\-single/\hyperlink{nb__kernel__ElecCSTab__VdwNone__GeomW4P1__sse4__1__single_8c}{nb\-\_\-kernel\-\_\-\-Elec\-C\-S\-Tab\-\_\-\-Vdw\-None\-\_\-\-Geom\-W4\-P1\-\_\-sse4\-\_\-1\-\_\-single.\-c} }{\pageref{nb__kernel__ElecCSTab__VdwNone__GeomW4P1__sse4__1__single_8c}}{}
\item\contentsline{section}{src/gmxlib/nonbonded/nb\-\_\-kernel\-\_\-sse4\-\_\-1\-\_\-single/\hyperlink{nb__kernel__ElecCSTab__VdwNone__GeomW4W4__sse4__1__single_8c}{nb\-\_\-kernel\-\_\-\-Elec\-C\-S\-Tab\-\_\-\-Vdw\-None\-\_\-\-Geom\-W4\-W4\-\_\-sse4\-\_\-1\-\_\-single.\-c} }{\pageref{nb__kernel__ElecCSTab__VdwNone__GeomW4W4__sse4__1__single_8c}}{}
\item\contentsline{section}{src/gmxlib/nonbonded/nb\-\_\-kernel\-\_\-sse4\-\_\-1\-\_\-single/\hyperlink{nb__kernel__ElecEw__VdwCSTab__GeomP1P1__sse4__1__single_8c}{nb\-\_\-kernel\-\_\-\-Elec\-Ew\-\_\-\-Vdw\-C\-S\-Tab\-\_\-\-Geom\-P1\-P1\-\_\-sse4\-\_\-1\-\_\-single.\-c} }{\pageref{nb__kernel__ElecEw__VdwCSTab__GeomP1P1__sse4__1__single_8c}}{}
\item\contentsline{section}{src/gmxlib/nonbonded/nb\-\_\-kernel\-\_\-sse4\-\_\-1\-\_\-single/\hyperlink{nb__kernel__ElecEw__VdwCSTab__GeomW3P1__sse4__1__single_8c}{nb\-\_\-kernel\-\_\-\-Elec\-Ew\-\_\-\-Vdw\-C\-S\-Tab\-\_\-\-Geom\-W3\-P1\-\_\-sse4\-\_\-1\-\_\-single.\-c} }{\pageref{nb__kernel__ElecEw__VdwCSTab__GeomW3P1__sse4__1__single_8c}}{}
\item\contentsline{section}{src/gmxlib/nonbonded/nb\-\_\-kernel\-\_\-sse4\-\_\-1\-\_\-single/\hyperlink{nb__kernel__ElecEw__VdwCSTab__GeomW3W3__sse4__1__single_8c}{nb\-\_\-kernel\-\_\-\-Elec\-Ew\-\_\-\-Vdw\-C\-S\-Tab\-\_\-\-Geom\-W3\-W3\-\_\-sse4\-\_\-1\-\_\-single.\-c} }{\pageref{nb__kernel__ElecEw__VdwCSTab__GeomW3W3__sse4__1__single_8c}}{}
\item\contentsline{section}{src/gmxlib/nonbonded/nb\-\_\-kernel\-\_\-sse4\-\_\-1\-\_\-single/\hyperlink{nb__kernel__ElecEw__VdwCSTab__GeomW4P1__sse4__1__single_8c}{nb\-\_\-kernel\-\_\-\-Elec\-Ew\-\_\-\-Vdw\-C\-S\-Tab\-\_\-\-Geom\-W4\-P1\-\_\-sse4\-\_\-1\-\_\-single.\-c} }{\pageref{nb__kernel__ElecEw__VdwCSTab__GeomW4P1__sse4__1__single_8c}}{}
\item\contentsline{section}{src/gmxlib/nonbonded/nb\-\_\-kernel\-\_\-sse4\-\_\-1\-\_\-single/\hyperlink{nb__kernel__ElecEw__VdwCSTab__GeomW4W4__sse4__1__single_8c}{nb\-\_\-kernel\-\_\-\-Elec\-Ew\-\_\-\-Vdw\-C\-S\-Tab\-\_\-\-Geom\-W4\-W4\-\_\-sse4\-\_\-1\-\_\-single.\-c} }{\pageref{nb__kernel__ElecEw__VdwCSTab__GeomW4W4__sse4__1__single_8c}}{}
\item\contentsline{section}{src/gmxlib/nonbonded/nb\-\_\-kernel\-\_\-sse4\-\_\-1\-\_\-single/\hyperlink{nb__kernel__ElecEw__VdwLJ__GeomP1P1__sse4__1__single_8c}{nb\-\_\-kernel\-\_\-\-Elec\-Ew\-\_\-\-Vdw\-L\-J\-\_\-\-Geom\-P1\-P1\-\_\-sse4\-\_\-1\-\_\-single.\-c} }{\pageref{nb__kernel__ElecEw__VdwLJ__GeomP1P1__sse4__1__single_8c}}{}
\item\contentsline{section}{src/gmxlib/nonbonded/nb\-\_\-kernel\-\_\-sse4\-\_\-1\-\_\-single/\hyperlink{nb__kernel__ElecEw__VdwLJ__GeomW3P1__sse4__1__single_8c}{nb\-\_\-kernel\-\_\-\-Elec\-Ew\-\_\-\-Vdw\-L\-J\-\_\-\-Geom\-W3\-P1\-\_\-sse4\-\_\-1\-\_\-single.\-c} }{\pageref{nb__kernel__ElecEw__VdwLJ__GeomW3P1__sse4__1__single_8c}}{}
\item\contentsline{section}{src/gmxlib/nonbonded/nb\-\_\-kernel\-\_\-sse4\-\_\-1\-\_\-single/\hyperlink{nb__kernel__ElecEw__VdwLJ__GeomW3W3__sse4__1__single_8c}{nb\-\_\-kernel\-\_\-\-Elec\-Ew\-\_\-\-Vdw\-L\-J\-\_\-\-Geom\-W3\-W3\-\_\-sse4\-\_\-1\-\_\-single.\-c} }{\pageref{nb__kernel__ElecEw__VdwLJ__GeomW3W3__sse4__1__single_8c}}{}
\item\contentsline{section}{src/gmxlib/nonbonded/nb\-\_\-kernel\-\_\-sse4\-\_\-1\-\_\-single/\hyperlink{nb__kernel__ElecEw__VdwLJ__GeomW4P1__sse4__1__single_8c}{nb\-\_\-kernel\-\_\-\-Elec\-Ew\-\_\-\-Vdw\-L\-J\-\_\-\-Geom\-W4\-P1\-\_\-sse4\-\_\-1\-\_\-single.\-c} }{\pageref{nb__kernel__ElecEw__VdwLJ__GeomW4P1__sse4__1__single_8c}}{}
\item\contentsline{section}{src/gmxlib/nonbonded/nb\-\_\-kernel\-\_\-sse4\-\_\-1\-\_\-single/\hyperlink{nb__kernel__ElecEw__VdwLJ__GeomW4W4__sse4__1__single_8c}{nb\-\_\-kernel\-\_\-\-Elec\-Ew\-\_\-\-Vdw\-L\-J\-\_\-\-Geom\-W4\-W4\-\_\-sse4\-\_\-1\-\_\-single.\-c} }{\pageref{nb__kernel__ElecEw__VdwLJ__GeomW4W4__sse4__1__single_8c}}{}
\item\contentsline{section}{src/gmxlib/nonbonded/nb\-\_\-kernel\-\_\-sse4\-\_\-1\-\_\-single/\hyperlink{nb__kernel__ElecEw__VdwNone__GeomP1P1__sse4__1__single_8c}{nb\-\_\-kernel\-\_\-\-Elec\-Ew\-\_\-\-Vdw\-None\-\_\-\-Geom\-P1\-P1\-\_\-sse4\-\_\-1\-\_\-single.\-c} }{\pageref{nb__kernel__ElecEw__VdwNone__GeomP1P1__sse4__1__single_8c}}{}
\item\contentsline{section}{src/gmxlib/nonbonded/nb\-\_\-kernel\-\_\-sse4\-\_\-1\-\_\-single/\hyperlink{nb__kernel__ElecEw__VdwNone__GeomW3P1__sse4__1__single_8c}{nb\-\_\-kernel\-\_\-\-Elec\-Ew\-\_\-\-Vdw\-None\-\_\-\-Geom\-W3\-P1\-\_\-sse4\-\_\-1\-\_\-single.\-c} }{\pageref{nb__kernel__ElecEw__VdwNone__GeomW3P1__sse4__1__single_8c}}{}
\item\contentsline{section}{src/gmxlib/nonbonded/nb\-\_\-kernel\-\_\-sse4\-\_\-1\-\_\-single/\hyperlink{nb__kernel__ElecEw__VdwNone__GeomW3W3__sse4__1__single_8c}{nb\-\_\-kernel\-\_\-\-Elec\-Ew\-\_\-\-Vdw\-None\-\_\-\-Geom\-W3\-W3\-\_\-sse4\-\_\-1\-\_\-single.\-c} }{\pageref{nb__kernel__ElecEw__VdwNone__GeomW3W3__sse4__1__single_8c}}{}
\item\contentsline{section}{src/gmxlib/nonbonded/nb\-\_\-kernel\-\_\-sse4\-\_\-1\-\_\-single/\hyperlink{nb__kernel__ElecEw__VdwNone__GeomW4P1__sse4__1__single_8c}{nb\-\_\-kernel\-\_\-\-Elec\-Ew\-\_\-\-Vdw\-None\-\_\-\-Geom\-W4\-P1\-\_\-sse4\-\_\-1\-\_\-single.\-c} }{\pageref{nb__kernel__ElecEw__VdwNone__GeomW4P1__sse4__1__single_8c}}{}
\item\contentsline{section}{src/gmxlib/nonbonded/nb\-\_\-kernel\-\_\-sse4\-\_\-1\-\_\-single/\hyperlink{nb__kernel__ElecEw__VdwNone__GeomW4W4__sse4__1__single_8c}{nb\-\_\-kernel\-\_\-\-Elec\-Ew\-\_\-\-Vdw\-None\-\_\-\-Geom\-W4\-W4\-\_\-sse4\-\_\-1\-\_\-single.\-c} }{\pageref{nb__kernel__ElecEw__VdwNone__GeomW4W4__sse4__1__single_8c}}{}
\item\contentsline{section}{src/gmxlib/nonbonded/nb\-\_\-kernel\-\_\-sse4\-\_\-1\-\_\-single/\hyperlink{nb__kernel__ElecEwSh__VdwLJSh__GeomP1P1__sse4__1__single_8c}{nb\-\_\-kernel\-\_\-\-Elec\-Ew\-Sh\-\_\-\-Vdw\-L\-J\-Sh\-\_\-\-Geom\-P1\-P1\-\_\-sse4\-\_\-1\-\_\-single.\-c} }{\pageref{nb__kernel__ElecEwSh__VdwLJSh__GeomP1P1__sse4__1__single_8c}}{}
\item\contentsline{section}{src/gmxlib/nonbonded/nb\-\_\-kernel\-\_\-sse4\-\_\-1\-\_\-single/\hyperlink{nb__kernel__ElecEwSh__VdwLJSh__GeomW3P1__sse4__1__single_8c}{nb\-\_\-kernel\-\_\-\-Elec\-Ew\-Sh\-\_\-\-Vdw\-L\-J\-Sh\-\_\-\-Geom\-W3\-P1\-\_\-sse4\-\_\-1\-\_\-single.\-c} }{\pageref{nb__kernel__ElecEwSh__VdwLJSh__GeomW3P1__sse4__1__single_8c}}{}
\item\contentsline{section}{src/gmxlib/nonbonded/nb\-\_\-kernel\-\_\-sse4\-\_\-1\-\_\-single/\hyperlink{nb__kernel__ElecEwSh__VdwLJSh__GeomW3W3__sse4__1__single_8c}{nb\-\_\-kernel\-\_\-\-Elec\-Ew\-Sh\-\_\-\-Vdw\-L\-J\-Sh\-\_\-\-Geom\-W3\-W3\-\_\-sse4\-\_\-1\-\_\-single.\-c} }{\pageref{nb__kernel__ElecEwSh__VdwLJSh__GeomW3W3__sse4__1__single_8c}}{}
\item\contentsline{section}{src/gmxlib/nonbonded/nb\-\_\-kernel\-\_\-sse4\-\_\-1\-\_\-single/\hyperlink{nb__kernel__ElecEwSh__VdwLJSh__GeomW4P1__sse4__1__single_8c}{nb\-\_\-kernel\-\_\-\-Elec\-Ew\-Sh\-\_\-\-Vdw\-L\-J\-Sh\-\_\-\-Geom\-W4\-P1\-\_\-sse4\-\_\-1\-\_\-single.\-c} }{\pageref{nb__kernel__ElecEwSh__VdwLJSh__GeomW4P1__sse4__1__single_8c}}{}
\item\contentsline{section}{src/gmxlib/nonbonded/nb\-\_\-kernel\-\_\-sse4\-\_\-1\-\_\-single/\hyperlink{nb__kernel__ElecEwSh__VdwLJSh__GeomW4W4__sse4__1__single_8c}{nb\-\_\-kernel\-\_\-\-Elec\-Ew\-Sh\-\_\-\-Vdw\-L\-J\-Sh\-\_\-\-Geom\-W4\-W4\-\_\-sse4\-\_\-1\-\_\-single.\-c} }{\pageref{nb__kernel__ElecEwSh__VdwLJSh__GeomW4W4__sse4__1__single_8c}}{}
\item\contentsline{section}{src/gmxlib/nonbonded/nb\-\_\-kernel\-\_\-sse4\-\_\-1\-\_\-single/\hyperlink{nb__kernel__ElecEwSh__VdwNone__GeomP1P1__sse4__1__single_8c}{nb\-\_\-kernel\-\_\-\-Elec\-Ew\-Sh\-\_\-\-Vdw\-None\-\_\-\-Geom\-P1\-P1\-\_\-sse4\-\_\-1\-\_\-single.\-c} }{\pageref{nb__kernel__ElecEwSh__VdwNone__GeomP1P1__sse4__1__single_8c}}{}
\item\contentsline{section}{src/gmxlib/nonbonded/nb\-\_\-kernel\-\_\-sse4\-\_\-1\-\_\-single/\hyperlink{nb__kernel__ElecEwSh__VdwNone__GeomW3P1__sse4__1__single_8c}{nb\-\_\-kernel\-\_\-\-Elec\-Ew\-Sh\-\_\-\-Vdw\-None\-\_\-\-Geom\-W3\-P1\-\_\-sse4\-\_\-1\-\_\-single.\-c} }{\pageref{nb__kernel__ElecEwSh__VdwNone__GeomW3P1__sse4__1__single_8c}}{}
\item\contentsline{section}{src/gmxlib/nonbonded/nb\-\_\-kernel\-\_\-sse4\-\_\-1\-\_\-single/\hyperlink{nb__kernel__ElecEwSh__VdwNone__GeomW3W3__sse4__1__single_8c}{nb\-\_\-kernel\-\_\-\-Elec\-Ew\-Sh\-\_\-\-Vdw\-None\-\_\-\-Geom\-W3\-W3\-\_\-sse4\-\_\-1\-\_\-single.\-c} }{\pageref{nb__kernel__ElecEwSh__VdwNone__GeomW3W3__sse4__1__single_8c}}{}
\item\contentsline{section}{src/gmxlib/nonbonded/nb\-\_\-kernel\-\_\-sse4\-\_\-1\-\_\-single/\hyperlink{nb__kernel__ElecEwSh__VdwNone__GeomW4P1__sse4__1__single_8c}{nb\-\_\-kernel\-\_\-\-Elec\-Ew\-Sh\-\_\-\-Vdw\-None\-\_\-\-Geom\-W4\-P1\-\_\-sse4\-\_\-1\-\_\-single.\-c} }{\pageref{nb__kernel__ElecEwSh__VdwNone__GeomW4P1__sse4__1__single_8c}}{}
\item\contentsline{section}{src/gmxlib/nonbonded/nb\-\_\-kernel\-\_\-sse4\-\_\-1\-\_\-single/\hyperlink{nb__kernel__ElecEwSh__VdwNone__GeomW4W4__sse4__1__single_8c}{nb\-\_\-kernel\-\_\-\-Elec\-Ew\-Sh\-\_\-\-Vdw\-None\-\_\-\-Geom\-W4\-W4\-\_\-sse4\-\_\-1\-\_\-single.\-c} }{\pageref{nb__kernel__ElecEwSh__VdwNone__GeomW4W4__sse4__1__single_8c}}{}
\item\contentsline{section}{src/gmxlib/nonbonded/nb\-\_\-kernel\-\_\-sse4\-\_\-1\-\_\-single/\hyperlink{nb__kernel__ElecEwSw__VdwLJSw__GeomP1P1__sse4__1__single_8c}{nb\-\_\-kernel\-\_\-\-Elec\-Ew\-Sw\-\_\-\-Vdw\-L\-J\-Sw\-\_\-\-Geom\-P1\-P1\-\_\-sse4\-\_\-1\-\_\-single.\-c} }{\pageref{nb__kernel__ElecEwSw__VdwLJSw__GeomP1P1__sse4__1__single_8c}}{}
\item\contentsline{section}{src/gmxlib/nonbonded/nb\-\_\-kernel\-\_\-sse4\-\_\-1\-\_\-single/\hyperlink{nb__kernel__ElecEwSw__VdwLJSw__GeomW3P1__sse4__1__single_8c}{nb\-\_\-kernel\-\_\-\-Elec\-Ew\-Sw\-\_\-\-Vdw\-L\-J\-Sw\-\_\-\-Geom\-W3\-P1\-\_\-sse4\-\_\-1\-\_\-single.\-c} }{\pageref{nb__kernel__ElecEwSw__VdwLJSw__GeomW3P1__sse4__1__single_8c}}{}
\item\contentsline{section}{src/gmxlib/nonbonded/nb\-\_\-kernel\-\_\-sse4\-\_\-1\-\_\-single/\hyperlink{nb__kernel__ElecEwSw__VdwLJSw__GeomW3W3__sse4__1__single_8c}{nb\-\_\-kernel\-\_\-\-Elec\-Ew\-Sw\-\_\-\-Vdw\-L\-J\-Sw\-\_\-\-Geom\-W3\-W3\-\_\-sse4\-\_\-1\-\_\-single.\-c} }{\pageref{nb__kernel__ElecEwSw__VdwLJSw__GeomW3W3__sse4__1__single_8c}}{}
\item\contentsline{section}{src/gmxlib/nonbonded/nb\-\_\-kernel\-\_\-sse4\-\_\-1\-\_\-single/\hyperlink{nb__kernel__ElecEwSw__VdwLJSw__GeomW4P1__sse4__1__single_8c}{nb\-\_\-kernel\-\_\-\-Elec\-Ew\-Sw\-\_\-\-Vdw\-L\-J\-Sw\-\_\-\-Geom\-W4\-P1\-\_\-sse4\-\_\-1\-\_\-single.\-c} }{\pageref{nb__kernel__ElecEwSw__VdwLJSw__GeomW4P1__sse4__1__single_8c}}{}
\item\contentsline{section}{src/gmxlib/nonbonded/nb\-\_\-kernel\-\_\-sse4\-\_\-1\-\_\-single/\hyperlink{nb__kernel__ElecEwSw__VdwLJSw__GeomW4W4__sse4__1__single_8c}{nb\-\_\-kernel\-\_\-\-Elec\-Ew\-Sw\-\_\-\-Vdw\-L\-J\-Sw\-\_\-\-Geom\-W4\-W4\-\_\-sse4\-\_\-1\-\_\-single.\-c} }{\pageref{nb__kernel__ElecEwSw__VdwLJSw__GeomW4W4__sse4__1__single_8c}}{}
\item\contentsline{section}{src/gmxlib/nonbonded/nb\-\_\-kernel\-\_\-sse4\-\_\-1\-\_\-single/\hyperlink{nb__kernel__ElecEwSw__VdwNone__GeomP1P1__sse4__1__single_8c}{nb\-\_\-kernel\-\_\-\-Elec\-Ew\-Sw\-\_\-\-Vdw\-None\-\_\-\-Geom\-P1\-P1\-\_\-sse4\-\_\-1\-\_\-single.\-c} }{\pageref{nb__kernel__ElecEwSw__VdwNone__GeomP1P1__sse4__1__single_8c}}{}
\item\contentsline{section}{src/gmxlib/nonbonded/nb\-\_\-kernel\-\_\-sse4\-\_\-1\-\_\-single/\hyperlink{nb__kernel__ElecEwSw__VdwNone__GeomW3P1__sse4__1__single_8c}{nb\-\_\-kernel\-\_\-\-Elec\-Ew\-Sw\-\_\-\-Vdw\-None\-\_\-\-Geom\-W3\-P1\-\_\-sse4\-\_\-1\-\_\-single.\-c} }{\pageref{nb__kernel__ElecEwSw__VdwNone__GeomW3P1__sse4__1__single_8c}}{}
\item\contentsline{section}{src/gmxlib/nonbonded/nb\-\_\-kernel\-\_\-sse4\-\_\-1\-\_\-single/\hyperlink{nb__kernel__ElecEwSw__VdwNone__GeomW3W3__sse4__1__single_8c}{nb\-\_\-kernel\-\_\-\-Elec\-Ew\-Sw\-\_\-\-Vdw\-None\-\_\-\-Geom\-W3\-W3\-\_\-sse4\-\_\-1\-\_\-single.\-c} }{\pageref{nb__kernel__ElecEwSw__VdwNone__GeomW3W3__sse4__1__single_8c}}{}
\item\contentsline{section}{src/gmxlib/nonbonded/nb\-\_\-kernel\-\_\-sse4\-\_\-1\-\_\-single/\hyperlink{nb__kernel__ElecEwSw__VdwNone__GeomW4P1__sse4__1__single_8c}{nb\-\_\-kernel\-\_\-\-Elec\-Ew\-Sw\-\_\-\-Vdw\-None\-\_\-\-Geom\-W4\-P1\-\_\-sse4\-\_\-1\-\_\-single.\-c} }{\pageref{nb__kernel__ElecEwSw__VdwNone__GeomW4P1__sse4__1__single_8c}}{}
\item\contentsline{section}{src/gmxlib/nonbonded/nb\-\_\-kernel\-\_\-sse4\-\_\-1\-\_\-single/\hyperlink{nb__kernel__ElecEwSw__VdwNone__GeomW4W4__sse4__1__single_8c}{nb\-\_\-kernel\-\_\-\-Elec\-Ew\-Sw\-\_\-\-Vdw\-None\-\_\-\-Geom\-W4\-W4\-\_\-sse4\-\_\-1\-\_\-single.\-c} }{\pageref{nb__kernel__ElecEwSw__VdwNone__GeomW4W4__sse4__1__single_8c}}{}
\item\contentsline{section}{src/gmxlib/nonbonded/nb\-\_\-kernel\-\_\-sse4\-\_\-1\-\_\-single/\hyperlink{nb__kernel__ElecGB__VdwCSTab__GeomP1P1__sse4__1__single_8c}{nb\-\_\-kernel\-\_\-\-Elec\-G\-B\-\_\-\-Vdw\-C\-S\-Tab\-\_\-\-Geom\-P1\-P1\-\_\-sse4\-\_\-1\-\_\-single.\-c} }{\pageref{nb__kernel__ElecGB__VdwCSTab__GeomP1P1__sse4__1__single_8c}}{}
\item\contentsline{section}{src/gmxlib/nonbonded/nb\-\_\-kernel\-\_\-sse4\-\_\-1\-\_\-single/\hyperlink{nb__kernel__ElecGB__VdwLJ__GeomP1P1__sse4__1__single_8c}{nb\-\_\-kernel\-\_\-\-Elec\-G\-B\-\_\-\-Vdw\-L\-J\-\_\-\-Geom\-P1\-P1\-\_\-sse4\-\_\-1\-\_\-single.\-c} }{\pageref{nb__kernel__ElecGB__VdwLJ__GeomP1P1__sse4__1__single_8c}}{}
\item\contentsline{section}{src/gmxlib/nonbonded/nb\-\_\-kernel\-\_\-sse4\-\_\-1\-\_\-single/\hyperlink{nb__kernel__ElecGB__VdwNone__GeomP1P1__sse4__1__single_8c}{nb\-\_\-kernel\-\_\-\-Elec\-G\-B\-\_\-\-Vdw\-None\-\_\-\-Geom\-P1\-P1\-\_\-sse4\-\_\-1\-\_\-single.\-c} }{\pageref{nb__kernel__ElecGB__VdwNone__GeomP1P1__sse4__1__single_8c}}{}
\item\contentsline{section}{src/gmxlib/nonbonded/nb\-\_\-kernel\-\_\-sse4\-\_\-1\-\_\-single/\hyperlink{nb__kernel__ElecNone__VdwCSTab__GeomP1P1__sse4__1__single_8c}{nb\-\_\-kernel\-\_\-\-Elec\-None\-\_\-\-Vdw\-C\-S\-Tab\-\_\-\-Geom\-P1\-P1\-\_\-sse4\-\_\-1\-\_\-single.\-c} }{\pageref{nb__kernel__ElecNone__VdwCSTab__GeomP1P1__sse4__1__single_8c}}{}
\item\contentsline{section}{src/gmxlib/nonbonded/nb\-\_\-kernel\-\_\-sse4\-\_\-1\-\_\-single/\hyperlink{nb__kernel__ElecNone__VdwLJ__GeomP1P1__sse4__1__single_8c}{nb\-\_\-kernel\-\_\-\-Elec\-None\-\_\-\-Vdw\-L\-J\-\_\-\-Geom\-P1\-P1\-\_\-sse4\-\_\-1\-\_\-single.\-c} }{\pageref{nb__kernel__ElecNone__VdwLJ__GeomP1P1__sse4__1__single_8c}}{}
\item\contentsline{section}{src/gmxlib/nonbonded/nb\-\_\-kernel\-\_\-sse4\-\_\-1\-\_\-single/\hyperlink{nb__kernel__ElecNone__VdwLJSh__GeomP1P1__sse4__1__single_8c}{nb\-\_\-kernel\-\_\-\-Elec\-None\-\_\-\-Vdw\-L\-J\-Sh\-\_\-\-Geom\-P1\-P1\-\_\-sse4\-\_\-1\-\_\-single.\-c} }{\pageref{nb__kernel__ElecNone__VdwLJSh__GeomP1P1__sse4__1__single_8c}}{}
\item\contentsline{section}{src/gmxlib/nonbonded/nb\-\_\-kernel\-\_\-sse4\-\_\-1\-\_\-single/\hyperlink{nb__kernel__ElecNone__VdwLJSw__GeomP1P1__sse4__1__single_8c}{nb\-\_\-kernel\-\_\-\-Elec\-None\-\_\-\-Vdw\-L\-J\-Sw\-\_\-\-Geom\-P1\-P1\-\_\-sse4\-\_\-1\-\_\-single.\-c} }{\pageref{nb__kernel__ElecNone__VdwLJSw__GeomP1P1__sse4__1__single_8c}}{}
\item\contentsline{section}{src/gmxlib/nonbonded/nb\-\_\-kernel\-\_\-sse4\-\_\-1\-\_\-single/\hyperlink{nb__kernel__ElecRF__VdwCSTab__GeomP1P1__sse4__1__single_8c}{nb\-\_\-kernel\-\_\-\-Elec\-R\-F\-\_\-\-Vdw\-C\-S\-Tab\-\_\-\-Geom\-P1\-P1\-\_\-sse4\-\_\-1\-\_\-single.\-c} }{\pageref{nb__kernel__ElecRF__VdwCSTab__GeomP1P1__sse4__1__single_8c}}{}
\item\contentsline{section}{src/gmxlib/nonbonded/nb\-\_\-kernel\-\_\-sse4\-\_\-1\-\_\-single/\hyperlink{nb__kernel__ElecRF__VdwCSTab__GeomW3P1__sse4__1__single_8c}{nb\-\_\-kernel\-\_\-\-Elec\-R\-F\-\_\-\-Vdw\-C\-S\-Tab\-\_\-\-Geom\-W3\-P1\-\_\-sse4\-\_\-1\-\_\-single.\-c} }{\pageref{nb__kernel__ElecRF__VdwCSTab__GeomW3P1__sse4__1__single_8c}}{}
\item\contentsline{section}{src/gmxlib/nonbonded/nb\-\_\-kernel\-\_\-sse4\-\_\-1\-\_\-single/\hyperlink{nb__kernel__ElecRF__VdwCSTab__GeomW3W3__sse4__1__single_8c}{nb\-\_\-kernel\-\_\-\-Elec\-R\-F\-\_\-\-Vdw\-C\-S\-Tab\-\_\-\-Geom\-W3\-W3\-\_\-sse4\-\_\-1\-\_\-single.\-c} }{\pageref{nb__kernel__ElecRF__VdwCSTab__GeomW3W3__sse4__1__single_8c}}{}
\item\contentsline{section}{src/gmxlib/nonbonded/nb\-\_\-kernel\-\_\-sse4\-\_\-1\-\_\-single/\hyperlink{nb__kernel__ElecRF__VdwCSTab__GeomW4P1__sse4__1__single_8c}{nb\-\_\-kernel\-\_\-\-Elec\-R\-F\-\_\-\-Vdw\-C\-S\-Tab\-\_\-\-Geom\-W4\-P1\-\_\-sse4\-\_\-1\-\_\-single.\-c} }{\pageref{nb__kernel__ElecRF__VdwCSTab__GeomW4P1__sse4__1__single_8c}}{}
\item\contentsline{section}{src/gmxlib/nonbonded/nb\-\_\-kernel\-\_\-sse4\-\_\-1\-\_\-single/\hyperlink{nb__kernel__ElecRF__VdwCSTab__GeomW4W4__sse4__1__single_8c}{nb\-\_\-kernel\-\_\-\-Elec\-R\-F\-\_\-\-Vdw\-C\-S\-Tab\-\_\-\-Geom\-W4\-W4\-\_\-sse4\-\_\-1\-\_\-single.\-c} }{\pageref{nb__kernel__ElecRF__VdwCSTab__GeomW4W4__sse4__1__single_8c}}{}
\item\contentsline{section}{src/gmxlib/nonbonded/nb\-\_\-kernel\-\_\-sse4\-\_\-1\-\_\-single/\hyperlink{nb__kernel__ElecRF__VdwLJ__GeomP1P1__sse4__1__single_8c}{nb\-\_\-kernel\-\_\-\-Elec\-R\-F\-\_\-\-Vdw\-L\-J\-\_\-\-Geom\-P1\-P1\-\_\-sse4\-\_\-1\-\_\-single.\-c} }{\pageref{nb__kernel__ElecRF__VdwLJ__GeomP1P1__sse4__1__single_8c}}{}
\item\contentsline{section}{src/gmxlib/nonbonded/nb\-\_\-kernel\-\_\-sse4\-\_\-1\-\_\-single/\hyperlink{nb__kernel__ElecRF__VdwLJ__GeomW3P1__sse4__1__single_8c}{nb\-\_\-kernel\-\_\-\-Elec\-R\-F\-\_\-\-Vdw\-L\-J\-\_\-\-Geom\-W3\-P1\-\_\-sse4\-\_\-1\-\_\-single.\-c} }{\pageref{nb__kernel__ElecRF__VdwLJ__GeomW3P1__sse4__1__single_8c}}{}
\item\contentsline{section}{src/gmxlib/nonbonded/nb\-\_\-kernel\-\_\-sse4\-\_\-1\-\_\-single/\hyperlink{nb__kernel__ElecRF__VdwLJ__GeomW3W3__sse4__1__single_8c}{nb\-\_\-kernel\-\_\-\-Elec\-R\-F\-\_\-\-Vdw\-L\-J\-\_\-\-Geom\-W3\-W3\-\_\-sse4\-\_\-1\-\_\-single.\-c} }{\pageref{nb__kernel__ElecRF__VdwLJ__GeomW3W3__sse4__1__single_8c}}{}
\item\contentsline{section}{src/gmxlib/nonbonded/nb\-\_\-kernel\-\_\-sse4\-\_\-1\-\_\-single/\hyperlink{nb__kernel__ElecRF__VdwLJ__GeomW4P1__sse4__1__single_8c}{nb\-\_\-kernel\-\_\-\-Elec\-R\-F\-\_\-\-Vdw\-L\-J\-\_\-\-Geom\-W4\-P1\-\_\-sse4\-\_\-1\-\_\-single.\-c} }{\pageref{nb__kernel__ElecRF__VdwLJ__GeomW4P1__sse4__1__single_8c}}{}
\item\contentsline{section}{src/gmxlib/nonbonded/nb\-\_\-kernel\-\_\-sse4\-\_\-1\-\_\-single/\hyperlink{nb__kernel__ElecRF__VdwLJ__GeomW4W4__sse4__1__single_8c}{nb\-\_\-kernel\-\_\-\-Elec\-R\-F\-\_\-\-Vdw\-L\-J\-\_\-\-Geom\-W4\-W4\-\_\-sse4\-\_\-1\-\_\-single.\-c} }{\pageref{nb__kernel__ElecRF__VdwLJ__GeomW4W4__sse4__1__single_8c}}{}
\item\contentsline{section}{src/gmxlib/nonbonded/nb\-\_\-kernel\-\_\-sse4\-\_\-1\-\_\-single/\hyperlink{nb__kernel__ElecRF__VdwNone__GeomP1P1__sse4__1__single_8c}{nb\-\_\-kernel\-\_\-\-Elec\-R\-F\-\_\-\-Vdw\-None\-\_\-\-Geom\-P1\-P1\-\_\-sse4\-\_\-1\-\_\-single.\-c} }{\pageref{nb__kernel__ElecRF__VdwNone__GeomP1P1__sse4__1__single_8c}}{}
\item\contentsline{section}{src/gmxlib/nonbonded/nb\-\_\-kernel\-\_\-sse4\-\_\-1\-\_\-single/\hyperlink{nb__kernel__ElecRF__VdwNone__GeomW3P1__sse4__1__single_8c}{nb\-\_\-kernel\-\_\-\-Elec\-R\-F\-\_\-\-Vdw\-None\-\_\-\-Geom\-W3\-P1\-\_\-sse4\-\_\-1\-\_\-single.\-c} }{\pageref{nb__kernel__ElecRF__VdwNone__GeomW3P1__sse4__1__single_8c}}{}
\item\contentsline{section}{src/gmxlib/nonbonded/nb\-\_\-kernel\-\_\-sse4\-\_\-1\-\_\-single/\hyperlink{nb__kernel__ElecRF__VdwNone__GeomW3W3__sse4__1__single_8c}{nb\-\_\-kernel\-\_\-\-Elec\-R\-F\-\_\-\-Vdw\-None\-\_\-\-Geom\-W3\-W3\-\_\-sse4\-\_\-1\-\_\-single.\-c} }{\pageref{nb__kernel__ElecRF__VdwNone__GeomW3W3__sse4__1__single_8c}}{}
\item\contentsline{section}{src/gmxlib/nonbonded/nb\-\_\-kernel\-\_\-sse4\-\_\-1\-\_\-single/\hyperlink{nb__kernel__ElecRF__VdwNone__GeomW4P1__sse4__1__single_8c}{nb\-\_\-kernel\-\_\-\-Elec\-R\-F\-\_\-\-Vdw\-None\-\_\-\-Geom\-W4\-P1\-\_\-sse4\-\_\-1\-\_\-single.\-c} }{\pageref{nb__kernel__ElecRF__VdwNone__GeomW4P1__sse4__1__single_8c}}{}
\item\contentsline{section}{src/gmxlib/nonbonded/nb\-\_\-kernel\-\_\-sse4\-\_\-1\-\_\-single/\hyperlink{nb__kernel__ElecRF__VdwNone__GeomW4W4__sse4__1__single_8c}{nb\-\_\-kernel\-\_\-\-Elec\-R\-F\-\_\-\-Vdw\-None\-\_\-\-Geom\-W4\-W4\-\_\-sse4\-\_\-1\-\_\-single.\-c} }{\pageref{nb__kernel__ElecRF__VdwNone__GeomW4W4__sse4__1__single_8c}}{}
\item\contentsline{section}{src/gmxlib/nonbonded/nb\-\_\-kernel\-\_\-sse4\-\_\-1\-\_\-single/\hyperlink{nb__kernel__ElecRFCut__VdwCSTab__GeomP1P1__sse4__1__single_8c}{nb\-\_\-kernel\-\_\-\-Elec\-R\-F\-Cut\-\_\-\-Vdw\-C\-S\-Tab\-\_\-\-Geom\-P1\-P1\-\_\-sse4\-\_\-1\-\_\-single.\-c} }{\pageref{nb__kernel__ElecRFCut__VdwCSTab__GeomP1P1__sse4__1__single_8c}}{}
\item\contentsline{section}{src/gmxlib/nonbonded/nb\-\_\-kernel\-\_\-sse4\-\_\-1\-\_\-single/\hyperlink{nb__kernel__ElecRFCut__VdwCSTab__GeomW3P1__sse4__1__single_8c}{nb\-\_\-kernel\-\_\-\-Elec\-R\-F\-Cut\-\_\-\-Vdw\-C\-S\-Tab\-\_\-\-Geom\-W3\-P1\-\_\-sse4\-\_\-1\-\_\-single.\-c} }{\pageref{nb__kernel__ElecRFCut__VdwCSTab__GeomW3P1__sse4__1__single_8c}}{}
\item\contentsline{section}{src/gmxlib/nonbonded/nb\-\_\-kernel\-\_\-sse4\-\_\-1\-\_\-single/\hyperlink{nb__kernel__ElecRFCut__VdwCSTab__GeomW3W3__sse4__1__single_8c}{nb\-\_\-kernel\-\_\-\-Elec\-R\-F\-Cut\-\_\-\-Vdw\-C\-S\-Tab\-\_\-\-Geom\-W3\-W3\-\_\-sse4\-\_\-1\-\_\-single.\-c} }{\pageref{nb__kernel__ElecRFCut__VdwCSTab__GeomW3W3__sse4__1__single_8c}}{}
\item\contentsline{section}{src/gmxlib/nonbonded/nb\-\_\-kernel\-\_\-sse4\-\_\-1\-\_\-single/\hyperlink{nb__kernel__ElecRFCut__VdwCSTab__GeomW4P1__sse4__1__single_8c}{nb\-\_\-kernel\-\_\-\-Elec\-R\-F\-Cut\-\_\-\-Vdw\-C\-S\-Tab\-\_\-\-Geom\-W4\-P1\-\_\-sse4\-\_\-1\-\_\-single.\-c} }{\pageref{nb__kernel__ElecRFCut__VdwCSTab__GeomW4P1__sse4__1__single_8c}}{}
\item\contentsline{section}{src/gmxlib/nonbonded/nb\-\_\-kernel\-\_\-sse4\-\_\-1\-\_\-single/\hyperlink{nb__kernel__ElecRFCut__VdwCSTab__GeomW4W4__sse4__1__single_8c}{nb\-\_\-kernel\-\_\-\-Elec\-R\-F\-Cut\-\_\-\-Vdw\-C\-S\-Tab\-\_\-\-Geom\-W4\-W4\-\_\-sse4\-\_\-1\-\_\-single.\-c} }{\pageref{nb__kernel__ElecRFCut__VdwCSTab__GeomW4W4__sse4__1__single_8c}}{}
\item\contentsline{section}{src/gmxlib/nonbonded/nb\-\_\-kernel\-\_\-sse4\-\_\-1\-\_\-single/\hyperlink{nb__kernel__ElecRFCut__VdwLJSh__GeomP1P1__sse4__1__single_8c}{nb\-\_\-kernel\-\_\-\-Elec\-R\-F\-Cut\-\_\-\-Vdw\-L\-J\-Sh\-\_\-\-Geom\-P1\-P1\-\_\-sse4\-\_\-1\-\_\-single.\-c} }{\pageref{nb__kernel__ElecRFCut__VdwLJSh__GeomP1P1__sse4__1__single_8c}}{}
\item\contentsline{section}{src/gmxlib/nonbonded/nb\-\_\-kernel\-\_\-sse4\-\_\-1\-\_\-single/\hyperlink{nb__kernel__ElecRFCut__VdwLJSh__GeomW3P1__sse4__1__single_8c}{nb\-\_\-kernel\-\_\-\-Elec\-R\-F\-Cut\-\_\-\-Vdw\-L\-J\-Sh\-\_\-\-Geom\-W3\-P1\-\_\-sse4\-\_\-1\-\_\-single.\-c} }{\pageref{nb__kernel__ElecRFCut__VdwLJSh__GeomW3P1__sse4__1__single_8c}}{}
\item\contentsline{section}{src/gmxlib/nonbonded/nb\-\_\-kernel\-\_\-sse4\-\_\-1\-\_\-single/\hyperlink{nb__kernel__ElecRFCut__VdwLJSh__GeomW3W3__sse4__1__single_8c}{nb\-\_\-kernel\-\_\-\-Elec\-R\-F\-Cut\-\_\-\-Vdw\-L\-J\-Sh\-\_\-\-Geom\-W3\-W3\-\_\-sse4\-\_\-1\-\_\-single.\-c} }{\pageref{nb__kernel__ElecRFCut__VdwLJSh__GeomW3W3__sse4__1__single_8c}}{}
\item\contentsline{section}{src/gmxlib/nonbonded/nb\-\_\-kernel\-\_\-sse4\-\_\-1\-\_\-single/\hyperlink{nb__kernel__ElecRFCut__VdwLJSh__GeomW4P1__sse4__1__single_8c}{nb\-\_\-kernel\-\_\-\-Elec\-R\-F\-Cut\-\_\-\-Vdw\-L\-J\-Sh\-\_\-\-Geom\-W4\-P1\-\_\-sse4\-\_\-1\-\_\-single.\-c} }{\pageref{nb__kernel__ElecRFCut__VdwLJSh__GeomW4P1__sse4__1__single_8c}}{}
\item\contentsline{section}{src/gmxlib/nonbonded/nb\-\_\-kernel\-\_\-sse4\-\_\-1\-\_\-single/\hyperlink{nb__kernel__ElecRFCut__VdwLJSh__GeomW4W4__sse4__1__single_8c}{nb\-\_\-kernel\-\_\-\-Elec\-R\-F\-Cut\-\_\-\-Vdw\-L\-J\-Sh\-\_\-\-Geom\-W4\-W4\-\_\-sse4\-\_\-1\-\_\-single.\-c} }{\pageref{nb__kernel__ElecRFCut__VdwLJSh__GeomW4W4__sse4__1__single_8c}}{}
\item\contentsline{section}{src/gmxlib/nonbonded/nb\-\_\-kernel\-\_\-sse4\-\_\-1\-\_\-single/\hyperlink{nb__kernel__ElecRFCut__VdwLJSw__GeomP1P1__sse4__1__single_8c}{nb\-\_\-kernel\-\_\-\-Elec\-R\-F\-Cut\-\_\-\-Vdw\-L\-J\-Sw\-\_\-\-Geom\-P1\-P1\-\_\-sse4\-\_\-1\-\_\-single.\-c} }{\pageref{nb__kernel__ElecRFCut__VdwLJSw__GeomP1P1__sse4__1__single_8c}}{}
\item\contentsline{section}{src/gmxlib/nonbonded/nb\-\_\-kernel\-\_\-sse4\-\_\-1\-\_\-single/\hyperlink{nb__kernel__ElecRFCut__VdwLJSw__GeomW3P1__sse4__1__single_8c}{nb\-\_\-kernel\-\_\-\-Elec\-R\-F\-Cut\-\_\-\-Vdw\-L\-J\-Sw\-\_\-\-Geom\-W3\-P1\-\_\-sse4\-\_\-1\-\_\-single.\-c} }{\pageref{nb__kernel__ElecRFCut__VdwLJSw__GeomW3P1__sse4__1__single_8c}}{}
\item\contentsline{section}{src/gmxlib/nonbonded/nb\-\_\-kernel\-\_\-sse4\-\_\-1\-\_\-single/\hyperlink{nb__kernel__ElecRFCut__VdwLJSw__GeomW3W3__sse4__1__single_8c}{nb\-\_\-kernel\-\_\-\-Elec\-R\-F\-Cut\-\_\-\-Vdw\-L\-J\-Sw\-\_\-\-Geom\-W3\-W3\-\_\-sse4\-\_\-1\-\_\-single.\-c} }{\pageref{nb__kernel__ElecRFCut__VdwLJSw__GeomW3W3__sse4__1__single_8c}}{}
\item\contentsline{section}{src/gmxlib/nonbonded/nb\-\_\-kernel\-\_\-sse4\-\_\-1\-\_\-single/\hyperlink{nb__kernel__ElecRFCut__VdwLJSw__GeomW4P1__sse4__1__single_8c}{nb\-\_\-kernel\-\_\-\-Elec\-R\-F\-Cut\-\_\-\-Vdw\-L\-J\-Sw\-\_\-\-Geom\-W4\-P1\-\_\-sse4\-\_\-1\-\_\-single.\-c} }{\pageref{nb__kernel__ElecRFCut__VdwLJSw__GeomW4P1__sse4__1__single_8c}}{}
\item\contentsline{section}{src/gmxlib/nonbonded/nb\-\_\-kernel\-\_\-sse4\-\_\-1\-\_\-single/\hyperlink{nb__kernel__ElecRFCut__VdwLJSw__GeomW4W4__sse4__1__single_8c}{nb\-\_\-kernel\-\_\-\-Elec\-R\-F\-Cut\-\_\-\-Vdw\-L\-J\-Sw\-\_\-\-Geom\-W4\-W4\-\_\-sse4\-\_\-1\-\_\-single.\-c} }{\pageref{nb__kernel__ElecRFCut__VdwLJSw__GeomW4W4__sse4__1__single_8c}}{}
\item\contentsline{section}{src/gmxlib/nonbonded/nb\-\_\-kernel\-\_\-sse4\-\_\-1\-\_\-single/\hyperlink{nb__kernel__ElecRFCut__VdwNone__GeomP1P1__sse4__1__single_8c}{nb\-\_\-kernel\-\_\-\-Elec\-R\-F\-Cut\-\_\-\-Vdw\-None\-\_\-\-Geom\-P1\-P1\-\_\-sse4\-\_\-1\-\_\-single.\-c} }{\pageref{nb__kernel__ElecRFCut__VdwNone__GeomP1P1__sse4__1__single_8c}}{}
\item\contentsline{section}{src/gmxlib/nonbonded/nb\-\_\-kernel\-\_\-sse4\-\_\-1\-\_\-single/\hyperlink{nb__kernel__ElecRFCut__VdwNone__GeomW3P1__sse4__1__single_8c}{nb\-\_\-kernel\-\_\-\-Elec\-R\-F\-Cut\-\_\-\-Vdw\-None\-\_\-\-Geom\-W3\-P1\-\_\-sse4\-\_\-1\-\_\-single.\-c} }{\pageref{nb__kernel__ElecRFCut__VdwNone__GeomW3P1__sse4__1__single_8c}}{}
\item\contentsline{section}{src/gmxlib/nonbonded/nb\-\_\-kernel\-\_\-sse4\-\_\-1\-\_\-single/\hyperlink{nb__kernel__ElecRFCut__VdwNone__GeomW3W3__sse4__1__single_8c}{nb\-\_\-kernel\-\_\-\-Elec\-R\-F\-Cut\-\_\-\-Vdw\-None\-\_\-\-Geom\-W3\-W3\-\_\-sse4\-\_\-1\-\_\-single.\-c} }{\pageref{nb__kernel__ElecRFCut__VdwNone__GeomW3W3__sse4__1__single_8c}}{}
\item\contentsline{section}{src/gmxlib/nonbonded/nb\-\_\-kernel\-\_\-sse4\-\_\-1\-\_\-single/\hyperlink{nb__kernel__ElecRFCut__VdwNone__GeomW4P1__sse4__1__single_8c}{nb\-\_\-kernel\-\_\-\-Elec\-R\-F\-Cut\-\_\-\-Vdw\-None\-\_\-\-Geom\-W4\-P1\-\_\-sse4\-\_\-1\-\_\-single.\-c} }{\pageref{nb__kernel__ElecRFCut__VdwNone__GeomW4P1__sse4__1__single_8c}}{}
\item\contentsline{section}{src/gmxlib/nonbonded/nb\-\_\-kernel\-\_\-sse4\-\_\-1\-\_\-single/\hyperlink{nb__kernel__ElecRFCut__VdwNone__GeomW4W4__sse4__1__single_8c}{nb\-\_\-kernel\-\_\-\-Elec\-R\-F\-Cut\-\_\-\-Vdw\-None\-\_\-\-Geom\-W4\-W4\-\_\-sse4\-\_\-1\-\_\-single.\-c} }{\pageref{nb__kernel__ElecRFCut__VdwNone__GeomW4W4__sse4__1__single_8c}}{}
\item\contentsline{section}{src/gmxlib/nonbonded/nb\-\_\-kernel\-\_\-sse4\-\_\-1\-\_\-single/\hyperlink{nb__kernel__sse4__1__single_8c}{nb\-\_\-kernel\-\_\-sse4\-\_\-1\-\_\-single.\-c} }{\pageref{nb__kernel__sse4__1__single_8c}}{}
\item\contentsline{section}{src/gmxlib/nonbonded/nb\-\_\-kernel\-\_\-sse4\-\_\-1\-\_\-single/\hyperlink{nb__kernel__sse4__1__single_8h}{nb\-\_\-kernel\-\_\-sse4\-\_\-1\-\_\-single.\-h} }{\pageref{nb__kernel__sse4__1__single_8h}}{}
\item\contentsline{section}{src/gmxlib/selection/\hyperlink{compiler_8c}{compiler.\-c} }{\pageref{compiler_8c}}{}
\item\contentsline{section}{src/gmxlib/selection/\hyperlink{evaluate_8c}{evaluate.\-c} }{\pageref{evaluate_8c}}{}
\item\contentsline{section}{src/gmxlib/selection/\hyperlink{evaluate_8h}{evaluate.\-h} }{\pageref{evaluate_8h}}{}
\item\contentsline{section}{src/gmxlib/selection/\hyperlink{keywords_8h}{keywords.\-h} }{\pageref{keywords_8h}}{}
\item\contentsline{section}{src/gmxlib/selection/\hyperlink{mempool_8c}{mempool.\-c} }{\pageref{mempool_8c}}{}
\item\contentsline{section}{src/gmxlib/selection/\hyperlink{mempool_8h}{mempool.\-h} }{\pageref{mempool_8h}}{}
\item\contentsline{section}{src/gmxlib/selection/\hyperlink{params_8c}{params.\-c} }{\pageref{params_8c}}{}
\item\contentsline{section}{src/gmxlib/selection/\hyperlink{parser_8c}{parser.\-c} }{\pageref{parser_8c}}{}
\item\contentsline{section}{src/gmxlib/selection/\hyperlink{parser_8h}{parser.\-h} }{\pageref{parser_8h}}{}
\item\contentsline{section}{src/gmxlib/selection/\hyperlink{parsetree_8c}{parsetree.\-c} }{\pageref{parsetree_8c}}{}
\item\contentsline{section}{src/gmxlib/selection/\hyperlink{parsetree_8h}{parsetree.\-h} }{\pageref{parsetree_8h}}{}
\item\contentsline{section}{src/gmxlib/selection/\hyperlink{scanner_8c}{scanner.\-c} }{\pageref{scanner_8c}}{}
\item\contentsline{section}{src/gmxlib/selection/\hyperlink{scanner_8h}{scanner.\-h} }{\pageref{scanner_8h}}{}
\item\contentsline{section}{src/gmxlib/selection/\hyperlink{scanner__flex_8h}{scanner\-\_\-flex.\-h} }{\pageref{scanner__flex_8h}}{}
\item\contentsline{section}{src/gmxlib/selection/\hyperlink{scanner__internal_8c}{scanner\-\_\-internal.\-c} }{\pageref{scanner__internal_8c}}{}
\item\contentsline{section}{src/gmxlib/selection/\hyperlink{scanner__internal_8h}{scanner\-\_\-internal.\-h} }{\pageref{scanner__internal_8h}}{}
\item\contentsline{section}{src/gmxlib/selection/\hyperlink{selcollection_8h}{selcollection.\-h} }{\pageref{selcollection_8h}}{}
\item\contentsline{section}{src/gmxlib/selection/\hyperlink{selection_8c}{selection.\-c} }{\pageref{selection_8c}}{}
\item\contentsline{section}{src/gmxlib/selection/\hyperlink{selelem_8c}{selelem.\-c} }{\pageref{selelem_8c}}{}
\item\contentsline{section}{src/gmxlib/selection/\hyperlink{selelem_8h}{selelem.\-h} }{\pageref{selelem_8h}}{}
\item\contentsline{section}{src/gmxlib/selection/\hyperlink{selhelp_8c}{selhelp.\-c} }{\pageref{selhelp_8c}}{}
\item\contentsline{section}{src/gmxlib/selection/\hyperlink{selhelp_8h}{selhelp.\-h} }{\pageref{selhelp_8h}}{}
\item\contentsline{section}{src/gmxlib/selection/\hyperlink{selmethod_8c}{selmethod.\-c} }{\pageref{selmethod_8c}}{}
\item\contentsline{section}{src/gmxlib/selection/\hyperlink{selvalue_8c}{selvalue.\-c} }{\pageref{selvalue_8c}}{}
\item\contentsline{section}{src/gmxlib/selection/\hyperlink{sm__compare_8c}{sm\-\_\-compare.\-c} }{\pageref{sm__compare_8c}}{}
\item\contentsline{section}{src/gmxlib/selection/\hyperlink{sm__distance_8c}{sm\-\_\-distance.\-c} }{\pageref{sm__distance_8c}}{}
\item\contentsline{section}{src/gmxlib/selection/\hyperlink{sm__insolidangle_8c}{sm\-\_\-insolidangle.\-c} }{\pageref{sm__insolidangle_8c}}{}
\item\contentsline{section}{src/gmxlib/selection/\hyperlink{sm__keywords_8c}{sm\-\_\-keywords.\-c} }{\pageref{sm__keywords_8c}}{}
\item\contentsline{section}{src/gmxlib/selection/\hyperlink{sm__merge_8c}{sm\-\_\-merge.\-c} }{\pageref{sm__merge_8c}}{}
\item\contentsline{section}{src/gmxlib/selection/\hyperlink{sm__permute_8c}{sm\-\_\-permute.\-c} }{\pageref{sm__permute_8c}}{}
\item\contentsline{section}{src/gmxlib/selection/\hyperlink{sm__position_8c}{sm\-\_\-position.\-c} }{\pageref{sm__position_8c}}{}
\item\contentsline{section}{src/gmxlib/selection/\hyperlink{sm__same_8c}{sm\-\_\-same.\-c} }{\pageref{sm__same_8c}}{}
\item\contentsline{section}{src/gmxlib/selection/\hyperlink{sm__simple_8c}{sm\-\_\-simple.\-c} }{\pageref{sm__simple_8c}}{}
\item\contentsline{section}{src/gmxlib/selection/\hyperlink{symrec_8c}{symrec.\-c} }{\pageref{symrec_8c}}{}
\item\contentsline{section}{src/gmxlib/selection/\hyperlink{symrec_8h}{symrec.\-h} }{\pageref{symrec_8h}}{}
\item\contentsline{section}{src/gmxlib/selection/\hyperlink{test__selection_8c}{test\-\_\-selection.\-c} }{\pageref{test__selection_8c}}{}
\item\contentsline{section}{src/gmxlib/statistics/\hyperlink{gmx__statistics_8c}{gmx\-\_\-statistics.\-c} }{\pageref{gmx__statistics_8c}}{}
\item\contentsline{section}{src/gmxlib/statistics/\hyperlink{gmx__statistics__test_8c}{gmx\-\_\-statistics\-\_\-test.\-c} }{\pageref{gmx__statistics__test_8c}}{}
\item\contentsline{section}{src/gmxlib/statistics/\hyperlink{histogram_8c}{histogram.\-c} }{\pageref{histogram_8c}}{}
\item\contentsline{section}{src/gmxlib/thread\-\_\-mpi/\hyperlink{alltoall_8c}{alltoall.\-c} }{\pageref{alltoall_8c}}{}
\item\contentsline{section}{src/gmxlib/thread\-\_\-mpi/\hyperlink{barrier_8c}{barrier.\-c} }{\pageref{barrier_8c}}{}
\item\contentsline{section}{src/gmxlib/thread\-\_\-mpi/\hyperlink{bcast_8c}{bcast.\-c} }{\pageref{bcast_8c}}{}
\item\contentsline{section}{src/gmxlib/thread\-\_\-mpi/\hyperlink{collective_8c}{collective.\-c} }{\pageref{collective_8c}}{}
\item\contentsline{section}{src/gmxlib/thread\-\_\-mpi/\hyperlink{src_2gmxlib_2thread__mpi_2collective_8h}{collective.\-h} }{\pageref{src_2gmxlib_2thread__mpi_2collective_8h}}{}
\item\contentsline{section}{src/gmxlib/thread\-\_\-mpi/\hyperlink{comm_8c}{comm.\-c} }{\pageref{comm_8c}}{}
\item\contentsline{section}{src/gmxlib/thread\-\_\-mpi/\hyperlink{errhandler_8c}{errhandler.\-c} }{\pageref{errhandler_8c}}{}
\item\contentsline{section}{src/gmxlib/thread\-\_\-mpi/\hyperlink{event_8c}{event.\-c} }{\pageref{event_8c}}{}
\item\contentsline{section}{src/gmxlib/thread\-\_\-mpi/\hyperlink{gather_8c}{gather.\-c} }{\pageref{gather_8c}}{}
\item\contentsline{section}{src/gmxlib/thread\-\_\-mpi/\hyperlink{group_8c}{group.\-c} }{\pageref{group_8c}}{}
\item\contentsline{section}{src/gmxlib/thread\-\_\-mpi/\hyperlink{impl_8h}{impl.\-h} }{\pageref{impl_8h}}{}
\item\contentsline{section}{src/gmxlib/thread\-\_\-mpi/\hyperlink{list_8c}{list.\-c} }{\pageref{list_8c}}{}
\item\contentsline{section}{src/gmxlib/thread\-\_\-mpi/\hyperlink{lock_8c}{lock.\-c} }{\pageref{lock_8c}}{}
\item\contentsline{section}{src/gmxlib/thread\-\_\-mpi/\hyperlink{numa__malloc_8c}{numa\-\_\-malloc.\-c} }{\pageref{numa__malloc_8c}}{}
\item\contentsline{section}{src/gmxlib/thread\-\_\-mpi/\hyperlink{once_8c}{once.\-c} }{\pageref{once_8c}}{}
\item\contentsline{section}{src/gmxlib/thread\-\_\-mpi/\hyperlink{p2p_8h}{p2p.\-h} }{\pageref{p2p_8h}}{}
\item\contentsline{section}{src/gmxlib/thread\-\_\-mpi/\hyperlink{p2p__buffer_8c}{p2p\-\_\-buffer.\-c} }{\pageref{p2p__buffer_8c}}{}
\item\contentsline{section}{src/gmxlib/thread\-\_\-mpi/\hyperlink{p2p__protocol_8c}{p2p\-\_\-protocol.\-c} }{\pageref{p2p__protocol_8c}}{}
\item\contentsline{section}{src/gmxlib/thread\-\_\-mpi/\hyperlink{p2p__send__recv_8c}{p2p\-\_\-send\-\_\-recv.\-c} }{\pageref{p2p__send__recv_8c}}{}
\item\contentsline{section}{src/gmxlib/thread\-\_\-mpi/\hyperlink{p2p__wait_8c}{p2p\-\_\-wait.\-c} }{\pageref{p2p__wait_8c}}{}
\item\contentsline{section}{src/gmxlib/thread\-\_\-mpi/\hyperlink{profile_8c}{profile.\-c} }{\pageref{profile_8c}}{}
\item\contentsline{section}{src/gmxlib/thread\-\_\-mpi/\hyperlink{profile_8h}{profile.\-h} }{\pageref{profile_8h}}{}
\item\contentsline{section}{src/gmxlib/thread\-\_\-mpi/\hyperlink{pthreads_8c}{pthreads.\-c} }{\pageref{pthreads_8c}}{}
\item\contentsline{section}{src/gmxlib/thread\-\_\-mpi/\hyperlink{pthreads_8h}{pthreads.\-h} }{\pageref{pthreads_8h}}{}
\item\contentsline{section}{src/gmxlib/thread\-\_\-mpi/\hyperlink{reduce_8c}{reduce.\-c} }{\pageref{reduce_8c}}{}
\item\contentsline{section}{src/gmxlib/thread\-\_\-mpi/\hyperlink{reduce__fast_8c}{reduce\-\_\-fast.\-c} }{\pageref{reduce__fast_8c}}{}
\item\contentsline{section}{src/gmxlib/thread\-\_\-mpi/\hyperlink{scan_8c}{scan.\-c} }{\pageref{scan_8c}}{}
\item\contentsline{section}{src/gmxlib/thread\-\_\-mpi/\hyperlink{scatter_8c}{scatter.\-c} }{\pageref{scatter_8c}}{}
\item\contentsline{section}{src/gmxlib/thread\-\_\-mpi/\hyperlink{settings_8h}{settings.\-h} }{\pageref{settings_8h}}{}
\item\contentsline{section}{src/gmxlib/thread\-\_\-mpi/\hyperlink{tmpi__init_8c}{tmpi\-\_\-init.\-c} }{\pageref{tmpi__init_8c}}{}
\item\contentsline{section}{src/gmxlib/thread\-\_\-mpi/\hyperlink{tmpi__malloc_8c}{tmpi\-\_\-malloc.\-c} }{\pageref{tmpi__malloc_8c}}{}
\item\contentsline{section}{src/gmxlib/thread\-\_\-mpi/\hyperlink{tmpi__ops_8h}{tmpi\-\_\-ops.\-h} }{\pageref{tmpi__ops_8h}}{}
\item\contentsline{section}{src/gmxlib/thread\-\_\-mpi/\hyperlink{topology_8c}{topology.\-c} }{\pageref{topology_8c}}{}
\item\contentsline{section}{src/gmxlib/thread\-\_\-mpi/\hyperlink{type_8c}{type.\-c} }{\pageref{type_8c}}{}
\item\contentsline{section}{src/gmxlib/thread\-\_\-mpi/\hyperlink{winthreads_8c}{winthreads.\-c} }{\pageref{winthreads_8c}}{}
\item\contentsline{section}{src/gmxlib/thread\-\_\-mpi/\hyperlink{winthreads_8h}{winthreads.\-h} }{\pageref{winthreads_8h}}{}
\item\contentsline{section}{src/gmxlib/trajana/\hyperlink{centerofmass_8c}{centerofmass.\-c} }{\pageref{centerofmass_8c}}{}
\item\contentsline{section}{src/gmxlib/trajana/\hyperlink{displacement_8c}{displacement.\-c} }{\pageref{displacement_8c}}{}
\item\contentsline{section}{src/gmxlib/trajana/\hyperlink{indexutil_8c}{indexutil.\-c} }{\pageref{indexutil_8c}}{}
\item\contentsline{section}{src/gmxlib/trajana/\hyperlink{nbsearch_8c}{nbsearch.\-c} }{\pageref{nbsearch_8c}}{}
\item\contentsline{section}{src/gmxlib/trajana/\hyperlink{poscalc_8c}{poscalc.\-c} }{\pageref{poscalc_8c}}{}
\item\contentsline{section}{src/gmxlib/trajana/\hyperlink{position_8c}{position.\-c} }{\pageref{position_8c}}{}
\item\contentsline{section}{src/gmxlib/trajana/\hyperlink{trajana_8c}{trajana.\-c} }{\pageref{trajana_8c}}{}
\item\contentsline{section}{src/kernel/\hyperlink{add__par_8c}{add\-\_\-par.\-c} }{\pageref{add__par_8c}}{}
\item\contentsline{section}{src/kernel/\hyperlink{add__par_8h}{add\-\_\-par.\-h} }{\pageref{add__par_8h}}{}
\item\contentsline{section}{src/kernel/\hyperlink{calc__verletbuf_8c}{calc\-\_\-verletbuf.\-c} }{\pageref{calc__verletbuf_8c}}{}
\item\contentsline{section}{src/kernel/\hyperlink{calc__verletbuf_8h}{calc\-\_\-verletbuf.\-h} }{\pageref{calc__verletbuf_8h}}{}
\item\contentsline{section}{src/kernel/\hyperlink{compute__io_8c}{compute\-\_\-io.\-c} }{\pageref{compute__io_8c}}{}
\item\contentsline{section}{src/kernel/\hyperlink{compute__io_8h}{compute\-\_\-io.\-h} }{\pageref{compute__io_8h}}{}
\item\contentsline{section}{src/kernel/\hyperlink{convparm_8c}{convparm.\-c} }{\pageref{convparm_8c}}{}
\item\contentsline{section}{src/kernel/\hyperlink{convparm_8h}{convparm.\-h} }{\pageref{convparm_8h}}{}
\item\contentsline{section}{src/kernel/\hyperlink{do__gct_8c}{do\-\_\-gct.\-c} }{\pageref{do__gct_8c}}{}
\item\contentsline{section}{src/kernel/\hyperlink{fflibutil_8c}{fflibutil.\-c} }{\pageref{fflibutil_8c}}{}
\item\contentsline{section}{src/kernel/\hyperlink{fflibutil_8h}{fflibutil.\-h} }{\pageref{fflibutil_8h}}{}
\item\contentsline{section}{src/kernel/\hyperlink{g__luck_8c}{g\-\_\-luck.\-c} }{\pageref{g__luck_8c}}{}
\item\contentsline{section}{src/kernel/\hyperlink{g__protonate_8c}{g\-\_\-protonate.\-c} }{\pageref{g__protonate_8c}}{}
\item\contentsline{section}{src/kernel/\hyperlink{g__x2top_8c}{g\-\_\-x2top.\-c} }{\pageref{g__x2top_8c}}{}
\item\contentsline{section}{src/kernel/\hyperlink{g__x2top_8h}{g\-\_\-x2top.\-h} }{\pageref{g__x2top_8h}}{}
\item\contentsline{section}{src/kernel/\hyperlink{gctio_8c}{gctio.\-c} }{\pageref{gctio_8c}}{}
\item\contentsline{section}{src/kernel/\hyperlink{gen__ad_8c}{gen\-\_\-ad.\-c} }{\pageref{gen__ad_8c}}{}
\item\contentsline{section}{src/kernel/\hyperlink{gen__vsite_8c}{gen\-\_\-vsite.\-c} }{\pageref{gen__vsite_8c}}{}
\item\contentsline{section}{src/kernel/\hyperlink{gen__vsite_8h}{gen\-\_\-vsite.\-h} }{\pageref{gen__vsite_8h}}{}
\item\contentsline{section}{src/kernel/\hyperlink{genalg_8c}{genalg.\-c} }{\pageref{genalg_8c}}{}
\item\contentsline{section}{src/kernel/\hyperlink{genalg_8h}{genalg.\-h} }{\pageref{genalg_8h}}{}
\item\contentsline{section}{src/kernel/\hyperlink{genhydro_8c}{genhydro.\-c} }{\pageref{genhydro_8c}}{}
\item\contentsline{section}{src/kernel/\hyperlink{genhydro_8h}{genhydro.\-h} }{\pageref{genhydro_8h}}{}
\item\contentsline{section}{src/kernel/\hyperlink{gmxcheck_8c}{gmxcheck.\-c} }{\pageref{gmxcheck_8c}}{}
\item\contentsline{section}{src/kernel/\hyperlink{gmxdump_8c}{gmxdump.\-c} }{\pageref{gmxdump_8c}}{}
\item\contentsline{section}{src/kernel/\hyperlink{gpp__atomtype_8c}{gpp\-\_\-atomtype.\-c} }{\pageref{gpp__atomtype_8c}}{}
\item\contentsline{section}{src/kernel/\hyperlink{gpp__bond__atomtype_8c}{gpp\-\_\-bond\-\_\-atomtype.\-c} }{\pageref{gpp__bond__atomtype_8c}}{}
\item\contentsline{section}{src/kernel/\hyperlink{gpp__bond__atomtype_8h}{gpp\-\_\-bond\-\_\-atomtype.\-h} }{\pageref{gpp__bond__atomtype_8h}}{}
\item\contentsline{section}{src/kernel/\hyperlink{gpp__tomorse_8h}{gpp\-\_\-tomorse.\-h} }{\pageref{gpp__tomorse_8h}}{}
\item\contentsline{section}{src/kernel/\hyperlink{grompp_8c}{grompp.\-c} }{\pageref{grompp_8c}}{}
\item\contentsline{section}{src/kernel/\hyperlink{h__db_8c}{h\-\_\-db.\-c} }{\pageref{h__db_8c}}{}
\item\contentsline{section}{src/kernel/\hyperlink{h__db_8h}{h\-\_\-db.\-h} }{\pageref{h__db_8h}}{}
\item\contentsline{section}{src/kernel/\hyperlink{hackblock_8c}{hackblock.\-c} }{\pageref{hackblock_8c}}{}
\item\contentsline{section}{src/kernel/\hyperlink{hizzie_8c}{hizzie.\-c} }{\pageref{hizzie_8c}}{}
\item\contentsline{section}{src/kernel/\hyperlink{hizzie_8h}{hizzie.\-h} }{\pageref{hizzie_8h}}{}
\item\contentsline{section}{src/kernel/\hyperlink{ionize_8c}{ionize.\-c} }{\pageref{ionize_8c}}{}
\item\contentsline{section}{src/kernel/\hyperlink{ionize_8h}{ionize.\-h} }{\pageref{ionize_8h}}{}
\item\contentsline{section}{src/kernel/\hyperlink{kernel_2main_8c}{main.\-c} }{\pageref{kernel_2main_8c}}{}
\item\contentsline{section}{src/kernel/\hyperlink{md_8c}{md.\-c} }{\pageref{md_8c}}{}
\item\contentsline{section}{src/kernel/\hyperlink{mdrun_8c}{mdrun.\-c} }{\pageref{mdrun_8c}}{}
\item\contentsline{section}{src/kernel/\hyperlink{membed_8c}{membed.\-c} }{\pageref{membed_8c}}{}
\item\contentsline{section}{src/kernel/\hyperlink{membed_8h}{membed.\-h} }{\pageref{membed_8h}}{}
\item\contentsline{section}{src/kernel/\hyperlink{nm2type_8c}{nm2type.\-c} }{\pageref{nm2type_8c}}{}
\item\contentsline{section}{src/kernel/\hyperlink{pdb2gmx_8c}{pdb2gmx.\-c} }{\pageref{pdb2gmx_8c}}{}
\item\contentsline{section}{src/kernel/\hyperlink{pdb2top_8c}{pdb2top.\-c} }{\pageref{pdb2top_8c}}{}
\item\contentsline{section}{src/kernel/\hyperlink{pgutil_8c}{pgutil.\-c} }{\pageref{pgutil_8c}}{}
\item\contentsline{section}{src/kernel/\hyperlink{pgutil_8h}{pgutil.\-h} }{\pageref{pgutil_8h}}{}
\item\contentsline{section}{src/kernel/\hyperlink{pme__loadbal_8c}{pme\-\_\-loadbal.\-c} }{\pageref{pme__loadbal_8c}}{}
\item\contentsline{section}{src/kernel/\hyperlink{pme__loadbal_8h}{pme\-\_\-loadbal.\-h} }{\pageref{pme__loadbal_8h}}{}
\item\contentsline{section}{src/kernel/\hyperlink{readadress_8c}{readadress.\-c} }{\pageref{readadress_8c}}{}
\item\contentsline{section}{src/kernel/\hyperlink{readir_8c}{readir.\-c} }{\pageref{readir_8c}}{}
\item\contentsline{section}{src/kernel/\hyperlink{readir_8h}{readir.\-h} }{\pageref{readir_8h}}{}
\item\contentsline{section}{src/kernel/\hyperlink{readpull_8c}{readpull.\-c} }{\pageref{readpull_8c}}{}
\item\contentsline{section}{src/kernel/\hyperlink{readrot_8c}{readrot.\-c} }{\pageref{readrot_8c}}{}
\item\contentsline{section}{src/kernel/\hyperlink{repl__ex_8c}{repl\-\_\-ex.\-c} }{\pageref{repl__ex_8c}}{}
\item\contentsline{section}{src/kernel/\hyperlink{repl__ex_8h}{repl\-\_\-ex.\-h} }{\pageref{repl__ex_8h}}{}
\item\contentsline{section}{src/kernel/\hyperlink{resall_8c}{resall.\-c} }{\pageref{resall_8c}}{}
\item\contentsline{section}{src/kernel/\hyperlink{runner_8c}{runner.\-c} }{\pageref{runner_8c}}{}
\item\contentsline{section}{src/kernel/\hyperlink{sorting_8c}{sorting.\-c} }{\pageref{sorting_8c}}{}
\item\contentsline{section}{src/kernel/\hyperlink{sorting_8h}{sorting.\-h} }{\pageref{sorting_8h}}{}
\item\contentsline{section}{src/kernel/\hyperlink{specbond_8c}{specbond.\-c} }{\pageref{specbond_8c}}{}
\item\contentsline{section}{src/kernel/\hyperlink{specbond_8h}{specbond.\-h} }{\pageref{specbond_8h}}{}
\item\contentsline{section}{src/kernel/\hyperlink{ter__db_8c}{ter\-\_\-db.\-c} }{\pageref{ter__db_8c}}{}
\item\contentsline{section}{src/kernel/\hyperlink{ter__db_8h}{ter\-\_\-db.\-h} }{\pageref{ter__db_8h}}{}
\item\contentsline{section}{src/kernel/\hyperlink{tomorse_8c}{tomorse.\-c} }{\pageref{tomorse_8c}}{}
\item\contentsline{section}{src/kernel/\hyperlink{topdef_8h}{topdef.\-h} }{\pageref{topdef_8h}}{}
\item\contentsline{section}{src/kernel/\hyperlink{topdirs_8c}{topdirs.\-c} }{\pageref{topdirs_8c}}{}
\item\contentsline{section}{src/kernel/\hyperlink{topdirs_8h}{topdirs.\-h} }{\pageref{topdirs_8h}}{}
\item\contentsline{section}{src/kernel/\hyperlink{topexcl_8c}{topexcl.\-c} }{\pageref{topexcl_8c}}{}
\item\contentsline{section}{src/kernel/\hyperlink{topexcl_8h}{topexcl.\-h} }{\pageref{topexcl_8h}}{}
\item\contentsline{section}{src/kernel/\hyperlink{topio_8c}{topio.\-c} }{\pageref{topio_8c}}{}
\item\contentsline{section}{src/kernel/\hyperlink{topio_8h}{topio.\-h} }{\pageref{topio_8h}}{}
\item\contentsline{section}{src/kernel/\hyperlink{toppush_8c}{toppush.\-c} }{\pageref{toppush_8c}}{}
\item\contentsline{section}{src/kernel/\hyperlink{toppush_8h}{toppush.\-h} }{\pageref{toppush_8h}}{}
\item\contentsline{section}{src/kernel/\hyperlink{topshake_8c}{topshake.\-c} }{\pageref{topshake_8c}}{}
\item\contentsline{section}{src/kernel/\hyperlink{topshake_8h}{topshake.\-h} }{\pageref{topshake_8h}}{}
\item\contentsline{section}{src/kernel/\hyperlink{toputil_8c}{toputil.\-c} }{\pageref{toputil_8c}}{}
\item\contentsline{section}{src/kernel/\hyperlink{tpbcmp_8c}{tpbcmp.\-c} }{\pageref{tpbcmp_8c}}{}
\item\contentsline{section}{src/kernel/\hyperlink{tpbcmp_8h}{tpbcmp.\-h} }{\pageref{tpbcmp_8h}}{}
\item\contentsline{section}{src/kernel/\hyperlink{tpbconv_8c}{tpbconv.\-c} }{\pageref{tpbconv_8c}}{}
\item\contentsline{section}{src/kernel/\hyperlink{vsite__parm_8c}{vsite\-\_\-parm.\-c} }{\pageref{vsite__parm_8c}}{}
\item\contentsline{section}{src/kernel/\hyperlink{vsite__parm_8h}{vsite\-\_\-parm.\-h} }{\pageref{vsite__parm_8h}}{}
\item\contentsline{section}{src/kernel/\hyperlink{xlate_8c}{xlate.\-c} }{\pageref{xlate_8c}}{}
\item\contentsline{section}{src/kernel/\hyperlink{xlate_8h}{xlate.\-h} }{\pageref{xlate_8h}}{}
\item\contentsline{section}{src/kernel/\hyperlink{xmdrun_8h}{xmdrun.\-h} }{\pageref{xmdrun_8h}}{}
\item\contentsline{section}{src/kernel/\hyperlink{xutils_8c}{xutils.\-c} }{\pageref{xutils_8c}}{}
\item\contentsline{section}{src/mdlib/\hyperlink{adress_8c}{adress.\-c} }{\pageref{adress_8c}}{}
\item\contentsline{section}{src/mdlib/\hyperlink{adress_8h}{adress.\-h} \\*\-Implementation of the \-Ad\-Res\-S method }{\pageref{adress_8h}}{}
\item\contentsline{section}{src/mdlib/\hyperlink{calcmu_8c}{calcmu.\-c} }{\pageref{calcmu_8c}}{}
\item\contentsline{section}{src/mdlib/\hyperlink{calcvir_8c}{calcvir.\-c} }{\pageref{calcvir_8c}}{}
\item\contentsline{section}{src/mdlib/\hyperlink{clincs_8c}{clincs.\-c} }{\pageref{clincs_8c}}{}
\item\contentsline{section}{src/mdlib/\hyperlink{constr_8c}{constr.\-c} }{\pageref{constr_8c}}{}
\item\contentsline{section}{src/mdlib/\hyperlink{coupling_8c}{coupling.\-c} }{\pageref{coupling_8c}}{}
\item\contentsline{section}{src/mdlib/\hyperlink{csettle_8c}{csettle.\-c} }{\pageref{csettle_8c}}{}
\item\contentsline{section}{src/mdlib/\hyperlink{domdec_8c}{domdec.\-c} }{\pageref{domdec_8c}}{}
\item\contentsline{section}{src/mdlib/\hyperlink{domdec__box_8c}{domdec\-\_\-box.\-c} }{\pageref{domdec__box_8c}}{}
\item\contentsline{section}{src/mdlib/\hyperlink{domdec__con_8c}{domdec\-\_\-con.\-c} }{\pageref{domdec__con_8c}}{}
\item\contentsline{section}{src/mdlib/\hyperlink{domdec__network_8c}{domdec\-\_\-network.\-c} }{\pageref{domdec__network_8c}}{}
\item\contentsline{section}{src/mdlib/\hyperlink{domdec__setup_8c}{domdec\-\_\-setup.\-c} }{\pageref{domdec__setup_8c}}{}
\item\contentsline{section}{src/mdlib/\hyperlink{domdec__top_8c}{domdec\-\_\-top.\-c} }{\pageref{domdec__top_8c}}{}
\item\contentsline{section}{src/mdlib/\hyperlink{ebin_8c}{ebin.\-c} }{\pageref{ebin_8c}}{}
\item\contentsline{section}{src/mdlib/\hyperlink{edsam_8c}{edsam.\-c} }{\pageref{edsam_8c}}{}
\item\contentsline{section}{src/mdlib/\hyperlink{ewald_8c}{ewald.\-c} }{\pageref{ewald_8c}}{}
\item\contentsline{section}{src/mdlib/\hyperlink{expanded_8c}{expanded.\-c} }{\pageref{expanded_8c}}{}
\item\contentsline{section}{src/mdlib/\hyperlink{fft5d_8c}{fft5d.\-c} }{\pageref{fft5d_8c}}{}
\item\contentsline{section}{src/mdlib/\hyperlink{fft5d_8h}{fft5d.\-h} }{\pageref{fft5d_8h}}{}
\item\contentsline{section}{src/mdlib/\hyperlink{fftpack_8c}{fftpack.\-c} }{\pageref{fftpack_8c}}{}
\item\contentsline{section}{src/mdlib/\hyperlink{fftpack_8h}{fftpack.\-h} }{\pageref{fftpack_8h}}{}
\item\contentsline{section}{src/mdlib/\hyperlink{force_8c}{force.\-c} }{\pageref{force_8c}}{}
\item\contentsline{section}{src/mdlib/\hyperlink{forcerec_8c}{forcerec.\-c} }{\pageref{forcerec_8c}}{}
\item\contentsline{section}{src/mdlib/\hyperlink{genborn_8c}{genborn.\-c} }{\pageref{genborn_8c}}{}
\item\contentsline{section}{src/mdlib/\hyperlink{genborn__allvsall_8c}{genborn\-\_\-allvsall.\-c} }{\pageref{genborn__allvsall_8c}}{}
\item\contentsline{section}{src/mdlib/\hyperlink{genborn__allvsall_8h}{genborn\-\_\-allvsall.\-h} }{\pageref{genborn__allvsall_8h}}{}
\item\contentsline{section}{src/mdlib/\hyperlink{genborn__allvsall__sse2__double_8c}{genborn\-\_\-allvsall\-\_\-sse2\-\_\-double.\-c} }{\pageref{genborn__allvsall__sse2__double_8c}}{}
\item\contentsline{section}{src/mdlib/\hyperlink{genborn__allvsall__sse2__double_8h}{genborn\-\_\-allvsall\-\_\-sse2\-\_\-double.\-h} }{\pageref{genborn__allvsall__sse2__double_8h}}{}
\item\contentsline{section}{src/mdlib/\hyperlink{genborn__allvsall__sse2__single_8c}{genborn\-\_\-allvsall\-\_\-sse2\-\_\-single.\-c} }{\pageref{genborn__allvsall__sse2__single_8c}}{}
\item\contentsline{section}{src/mdlib/\hyperlink{genborn__allvsall__sse2__single_8h}{genborn\-\_\-allvsall\-\_\-sse2\-\_\-single.\-h} }{\pageref{genborn__allvsall__sse2__single_8h}}{}
\item\contentsline{section}{src/mdlib/\hyperlink{genborn__sse2__double_8c}{genborn\-\_\-sse2\-\_\-double.\-c} }{\pageref{genborn__sse2__double_8c}}{}
\item\contentsline{section}{src/mdlib/\hyperlink{genborn__sse2__double_8h}{genborn\-\_\-sse2\-\_\-double.\-h} }{\pageref{genborn__sse2__double_8h}}{}
\item\contentsline{section}{src/mdlib/\hyperlink{genborn__sse2__single_8c}{genborn\-\_\-sse2\-\_\-single.\-c} }{\pageref{genborn__sse2__single_8c}}{}
\item\contentsline{section}{src/mdlib/\hyperlink{genborn__sse2__single_8h}{genborn\-\_\-sse2\-\_\-single.\-h} }{\pageref{genborn__sse2__single_8h}}{}
\item\contentsline{section}{src/mdlib/\hyperlink{gmx__fft_8c}{gmx\-\_\-fft.\-c} }{\pageref{gmx__fft_8c}}{}
\item\contentsline{section}{src/mdlib/\hyperlink{gmx__fft__acml_8c}{gmx\-\_\-fft\-\_\-acml.\-c} }{\pageref{gmx__fft__acml_8c}}{}
\item\contentsline{section}{src/mdlib/\hyperlink{gmx__fft__fftpack_8c}{gmx\-\_\-fft\-\_\-fftpack.\-c} }{\pageref{gmx__fft__fftpack_8c}}{}
\item\contentsline{section}{src/mdlib/\hyperlink{gmx__fft__fftw3_8c}{gmx\-\_\-fft\-\_\-fftw3.\-c} }{\pageref{gmx__fft__fftw3_8c}}{}
\item\contentsline{section}{src/mdlib/\hyperlink{gmx__fft__mkl_8c}{gmx\-\_\-fft\-\_\-mkl.\-c} }{\pageref{gmx__fft__mkl_8c}}{}
\item\contentsline{section}{src/mdlib/\hyperlink{gmx__parallel__3dfft_8c}{gmx\-\_\-parallel\-\_\-3dfft.\-c} }{\pageref{gmx__parallel__3dfft_8c}}{}
\item\contentsline{section}{src/mdlib/\hyperlink{gmx__wallcycle_8c}{gmx\-\_\-wallcycle.\-c} }{\pageref{gmx__wallcycle_8c}}{}
\item\contentsline{section}{src/mdlib/\hyperlink{groupcoord_8c}{groupcoord.\-c} }{\pageref{groupcoord_8c}}{}
\item\contentsline{section}{src/mdlib/\hyperlink{groupcoord_8h}{groupcoord.\-h} \\*\-Assemble atom positions for comparison with a reference set }{\pageref{groupcoord_8h}}{}
\item\contentsline{section}{src/mdlib/\hyperlink{init_8c}{init.\-c} }{\pageref{init_8c}}{}
\item\contentsline{section}{src/mdlib/\hyperlink{iteratedconstraints_8c}{iteratedconstraints.\-c} }{\pageref{iteratedconstraints_8c}}{}
\item\contentsline{section}{src/mdlib/\hyperlink{md__support_8c}{md\-\_\-support.\-c} }{\pageref{md__support_8c}}{}
\item\contentsline{section}{src/mdlib/\hyperlink{mdatom_8c}{mdatom.\-c} }{\pageref{mdatom_8c}}{}
\item\contentsline{section}{src/mdlib/\hyperlink{mdebin_8c}{mdebin.\-c} }{\pageref{mdebin_8c}}{}
\item\contentsline{section}{src/mdlib/\hyperlink{mdebin__bar_8c}{mdebin\-\_\-bar.\-c} }{\pageref{mdebin__bar_8c}}{}
\item\contentsline{section}{src/mdlib/\hyperlink{mdebin__bar_8h}{mdebin\-\_\-bar.\-h} }{\pageref{mdebin__bar_8h}}{}
\item\contentsline{section}{src/mdlib/\hyperlink{minimize_8c}{minimize.\-c} }{\pageref{minimize_8c}}{}
\item\contentsline{section}{src/mdlib/\hyperlink{mvxvf_8c}{mvxvf.\-c} }{\pageref{mvxvf_8c}}{}
\item\contentsline{section}{src/mdlib/\hyperlink{nbnxn__atomdata_8c}{nbnxn\-\_\-atomdata.\-c} }{\pageref{nbnxn__atomdata_8c}}{}
\item\contentsline{section}{src/mdlib/\hyperlink{nbnxn__atomdata_8h}{nbnxn\-\_\-atomdata.\-h} }{\pageref{nbnxn__atomdata_8h}}{}
\item\contentsline{section}{src/mdlib/\hyperlink{nbnxn__consts_8h}{nbnxn\-\_\-consts.\-h} }{\pageref{nbnxn__consts_8h}}{}
\item\contentsline{section}{src/mdlib/\hyperlink{nbnxn__internal_8h}{nbnxn\-\_\-internal.\-h} }{\pageref{nbnxn__internal_8h}}{}
\item\contentsline{section}{src/mdlib/\hyperlink{nbnxn__search_8c}{nbnxn\-\_\-search.\-c} }{\pageref{nbnxn__search_8c}}{}
\item\contentsline{section}{src/mdlib/\hyperlink{nbnxn__search_8h}{nbnxn\-\_\-search.\-h} }{\pageref{nbnxn__search_8h}}{}
\item\contentsline{section}{src/mdlib/\hyperlink{nbnxn__search__simd__2xnn_8h}{nbnxn\-\_\-search\-\_\-simd\-\_\-2xnn.\-h} }{\pageref{nbnxn__search__simd__2xnn_8h}}{}
\item\contentsline{section}{src/mdlib/\hyperlink{nbnxn__search__simd__4xn_8h}{nbnxn\-\_\-search\-\_\-simd\-\_\-4xn.\-h} }{\pageref{nbnxn__search__simd__4xn_8h}}{}
\item\contentsline{section}{src/mdlib/\hyperlink{nlistheuristics_8c}{nlistheuristics.\-c} }{\pageref{nlistheuristics_8c}}{}
\item\contentsline{section}{src/mdlib/\hyperlink{ns_8c}{ns.\-c} }{\pageref{ns_8c}}{}
\item\contentsline{section}{src/mdlib/\hyperlink{nsgrid_8c}{nsgrid.\-c} }{\pageref{nsgrid_8c}}{}
\item\contentsline{section}{src/mdlib/\hyperlink{partdec_8c}{partdec.\-c} }{\pageref{partdec_8c}}{}
\item\contentsline{section}{src/mdlib/\hyperlink{perf__est_8c}{perf\-\_\-est.\-c} }{\pageref{perf__est_8c}}{}
\item\contentsline{section}{src/mdlib/\hyperlink{pme_8c}{pme.\-c} }{\pageref{pme_8c}}{}
\item\contentsline{section}{src/mdlib/\hyperlink{pme__pp_8c}{pme\-\_\-pp.\-c} }{\pageref{pme__pp_8c}}{}
\item\contentsline{section}{src/mdlib/\hyperlink{pme__se_8h}{pme\-\_\-se.\-h} }{\pageref{pme__se_8h}}{}
\item\contentsline{section}{src/mdlib/\hyperlink{pme__sse__single_8h}{pme\-\_\-sse\-\_\-single.\-h} }{\pageref{pme__sse__single_8h}}{}
\item\contentsline{section}{src/mdlib/\hyperlink{pull_8c}{pull.\-c} }{\pageref{pull_8c}}{}
\item\contentsline{section}{src/mdlib/\hyperlink{pull__rotation_8c}{pull\-\_\-rotation.\-c} }{\pageref{pull__rotation_8c}}{}
\item\contentsline{section}{src/mdlib/\hyperlink{pullutil_8c}{pullutil.\-c} }{\pageref{pullutil_8c}}{}
\item\contentsline{section}{src/mdlib/\hyperlink{qm__gamess_8c}{qm\-\_\-gamess.\-c} }{\pageref{qm__gamess_8c}}{}
\item\contentsline{section}{src/mdlib/\hyperlink{qm__gaussian_8c}{qm\-\_\-gaussian.\-c} }{\pageref{qm__gaussian_8c}}{}
\item\contentsline{section}{src/mdlib/\hyperlink{qm__mopac_8c}{qm\-\_\-mopac.\-c} }{\pageref{qm__mopac_8c}}{}
\item\contentsline{section}{src/mdlib/\hyperlink{qm__orca_8c}{qm\-\_\-orca.\-c} }{\pageref{qm__orca_8c}}{}
\item\contentsline{section}{src/mdlib/\hyperlink{qmmm_8c}{qmmm.\-c} }{\pageref{qmmm_8c}}{}
\item\contentsline{section}{src/mdlib/\hyperlink{rf__util_8c}{rf\-\_\-util.\-c} }{\pageref{rf__util_8c}}{}
\item\contentsline{section}{src/mdlib/\hyperlink{SE__fgg_8h}{\-S\-E\-\_\-fgg.\-h} }{\pageref{SE__fgg_8h}}{}
\item\contentsline{section}{src/mdlib/\hyperlink{shakef_8c}{shakef.\-c} }{\pageref{shakef_8c}}{}
\item\contentsline{section}{src/mdlib/\hyperlink{shellfc_8c}{shellfc.\-c} }{\pageref{shellfc_8c}}{}
\item\contentsline{section}{src/mdlib/\hyperlink{sim__util_8c}{sim\-\_\-util.\-c} }{\pageref{sim__util_8c}}{}
\item\contentsline{section}{src/mdlib/\hyperlink{stat_8c}{stat.\-c} }{\pageref{stat_8c}}{}
\item\contentsline{section}{src/mdlib/\hyperlink{tables_8c}{tables.\-c} }{\pageref{tables_8c}}{}
\item\contentsline{section}{src/mdlib/\hyperlink{tgroup_8c}{tgroup.\-c} }{\pageref{tgroup_8c}}{}
\item\contentsline{section}{src/mdlib/\hyperlink{tpi_8c}{tpi.\-c} }{\pageref{tpi_8c}}{}
\item\contentsline{section}{src/mdlib/\hyperlink{update_8c}{update.\-c} }{\pageref{update_8c}}{}
\item\contentsline{section}{src/mdlib/\hyperlink{vcm_8c}{vcm.\-c} }{\pageref{vcm_8c}}{}
\item\contentsline{section}{src/mdlib/\hyperlink{vsite_8c}{vsite.\-c} }{\pageref{vsite_8c}}{}
\item\contentsline{section}{src/mdlib/\hyperlink{wall_8c}{wall.\-c} }{\pageref{wall_8c}}{}
\item\contentsline{section}{src/mdlib/\hyperlink{wnblist_8c}{wnblist.\-c} }{\pageref{wnblist_8c}}{}
\item\contentsline{section}{src/mdlib/\-C\-Make\-Files/\-Compiler\-Id\-C/\hyperlink{src_2mdlib_2CMakeFiles_2CompilerIdC_2CMakeCCompilerId_8c}{\-C\-Make\-C\-Compiler\-Id.\-c} }{\pageref{src_2mdlib_2CMakeFiles_2CompilerIdC_2CMakeCCompilerId_8c}}{}
\item\contentsline{section}{src/mdlib/nbnxn\-\_\-cuda/\hyperlink{nbnxn__cuda_8h}{nbnxn\-\_\-cuda.\-h} }{\pageref{nbnxn__cuda_8h}}{}
\item\contentsline{section}{src/mdlib/nbnxn\-\_\-cuda/\hyperlink{nbnxn__cuda__types_8h}{nbnxn\-\_\-cuda\-\_\-types.\-h} }{\pageref{nbnxn__cuda__types_8h}}{}
\item\contentsline{section}{src/mdlib/nbnxn\-\_\-kernels/\hyperlink{nbnxn__kernel__common_8c}{nbnxn\-\_\-kernel\-\_\-common.\-c} }{\pageref{nbnxn__kernel__common_8c}}{}
\item\contentsline{section}{src/mdlib/nbnxn\-\_\-kernels/\hyperlink{nbnxn__kernel__common_8h}{nbnxn\-\_\-kernel\-\_\-common.\-h} }{\pageref{nbnxn__kernel__common_8h}}{}
\item\contentsline{section}{src/mdlib/nbnxn\-\_\-kernels/\hyperlink{nbnxn__kernel__gpu__ref_8c}{nbnxn\-\_\-kernel\-\_\-gpu\-\_\-ref.\-c} }{\pageref{nbnxn__kernel__gpu__ref_8c}}{}
\item\contentsline{section}{src/mdlib/nbnxn\-\_\-kernels/\hyperlink{nbnxn__kernel__gpu__ref_8h}{nbnxn\-\_\-kernel\-\_\-gpu\-\_\-ref.\-h} }{\pageref{nbnxn__kernel__gpu__ref_8h}}{}
\item\contentsline{section}{src/mdlib/nbnxn\-\_\-kernels/\hyperlink{nbnxn__kernel__ref_8c}{nbnxn\-\_\-kernel\-\_\-ref.\-c} }{\pageref{nbnxn__kernel__ref_8c}}{}
\item\contentsline{section}{src/mdlib/nbnxn\-\_\-kernels/\hyperlink{nbnxn__kernel__ref_8h}{nbnxn\-\_\-kernel\-\_\-ref.\-h} }{\pageref{nbnxn__kernel__ref_8h}}{}
\item\contentsline{section}{src/mdlib/nbnxn\-\_\-kernels/\hyperlink{nbnxn__kernel__ref__inner_8h}{nbnxn\-\_\-kernel\-\_\-ref\-\_\-inner.\-h} }{\pageref{nbnxn__kernel__ref__inner_8h}}{}
\item\contentsline{section}{src/mdlib/nbnxn\-\_\-kernels/\hyperlink{nbnxn__kernel__ref__outer_8h}{nbnxn\-\_\-kernel\-\_\-ref\-\_\-outer.\-h} }{\pageref{nbnxn__kernel__ref__outer_8h}}{}
\item\contentsline{section}{src/mdlib/nbnxn\-\_\-kernels/\hyperlink{nbnxn__kernel__simd__2xnn_8c}{nbnxn\-\_\-kernel\-\_\-simd\-\_\-2xnn.\-c} }{\pageref{nbnxn__kernel__simd__2xnn_8c}}{}
\item\contentsline{section}{src/mdlib/nbnxn\-\_\-kernels/\hyperlink{nbnxn__kernel__simd__2xnn_8h}{nbnxn\-\_\-kernel\-\_\-simd\-\_\-2xnn.\-h} }{\pageref{nbnxn__kernel__simd__2xnn_8h}}{}
\item\contentsline{section}{src/mdlib/nbnxn\-\_\-kernels/\hyperlink{nbnxn__kernel__simd__2xnn__includes_8h}{nbnxn\-\_\-kernel\-\_\-simd\-\_\-2xnn\-\_\-includes.\-h} }{\pageref{nbnxn__kernel__simd__2xnn__includes_8h}}{}
\item\contentsline{section}{src/mdlib/nbnxn\-\_\-kernels/\hyperlink{nbnxn__kernel__simd__2xnn__inner_8h}{nbnxn\-\_\-kernel\-\_\-simd\-\_\-2xnn\-\_\-inner.\-h} }{\pageref{nbnxn__kernel__simd__2xnn__inner_8h}}{}
\item\contentsline{section}{src/mdlib/nbnxn\-\_\-kernels/\hyperlink{nbnxn__kernel__simd__2xnn__outer_8h}{nbnxn\-\_\-kernel\-\_\-simd\-\_\-2xnn\-\_\-outer.\-h} }{\pageref{nbnxn__kernel__simd__2xnn__outer_8h}}{}
\item\contentsline{section}{src/mdlib/nbnxn\-\_\-kernels/\hyperlink{nbnxn__kernel__simd__4xn_8c}{nbnxn\-\_\-kernel\-\_\-simd\-\_\-4xn.\-c} }{\pageref{nbnxn__kernel__simd__4xn_8c}}{}
\item\contentsline{section}{src/mdlib/nbnxn\-\_\-kernels/\hyperlink{nbnxn__kernel__simd__4xn_8h}{nbnxn\-\_\-kernel\-\_\-simd\-\_\-4xn.\-h} }{\pageref{nbnxn__kernel__simd__4xn_8h}}{}
\item\contentsline{section}{src/mdlib/nbnxn\-\_\-kernels/\hyperlink{nbnxn__kernel__simd__4xn__includes_8h}{nbnxn\-\_\-kernel\-\_\-simd\-\_\-4xn\-\_\-includes.\-h} }{\pageref{nbnxn__kernel__simd__4xn__includes_8h}}{}
\item\contentsline{section}{src/mdlib/nbnxn\-\_\-kernels/\hyperlink{nbnxn__kernel__simd__4xn__inner_8h}{nbnxn\-\_\-kernel\-\_\-simd\-\_\-4xn\-\_\-inner.\-h} }{\pageref{nbnxn__kernel__simd__4xn__inner_8h}}{}
\item\contentsline{section}{src/mdlib/nbnxn\-\_\-kernels/\hyperlink{nbnxn__kernel__simd__4xn__outer_8h}{nbnxn\-\_\-kernel\-\_\-simd\-\_\-4xn\-\_\-outer.\-h} }{\pageref{nbnxn__kernel__simd__4xn__outer_8h}}{}
\item\contentsline{section}{src/mdlib/nbnxn\-\_\-kernels/\hyperlink{nbnxn__kernel__simd__utils_8h}{nbnxn\-\_\-kernel\-\_\-simd\-\_\-utils.\-h} }{\pageref{nbnxn__kernel__simd__utils_8h}}{}
\item\contentsline{section}{src/ngmx/\hyperlink{buttons_8c}{buttons.\-c} }{\pageref{buttons_8c}}{}
\item\contentsline{section}{src/ngmx/\hyperlink{buttons_8h}{buttons.\-h} }{\pageref{buttons_8h}}{}
\item\contentsline{section}{src/ngmx/\hyperlink{dialogs_8c}{dialogs.\-c} }{\pageref{dialogs_8c}}{}
\item\contentsline{section}{src/ngmx/\hyperlink{dialogs_8h}{dialogs.\-h} }{\pageref{dialogs_8h}}{}
\item\contentsline{section}{src/ngmx/\hyperlink{fgrid_8c}{fgrid.\-c} }{\pageref{fgrid_8c}}{}
\item\contentsline{section}{src/ngmx/\hyperlink{fgrid_8h}{fgrid.\-h} }{\pageref{fgrid_8h}}{}
\item\contentsline{section}{src/ngmx/\hyperlink{filter_8c}{filter.\-c} }{\pageref{filter_8c}}{}
\item\contentsline{section}{src/ngmx/\hyperlink{g__highway_8c}{g\-\_\-highway.\-c} }{\pageref{g__highway_8c}}{}
\item\contentsline{section}{src/ngmx/\hyperlink{g__logo_8c}{g\-\_\-logo.\-c} }{\pageref{g__logo_8c}}{}
\item\contentsline{section}{src/ngmx/\hyperlink{g__showcol_8c}{g\-\_\-showcol.\-c} }{\pageref{g__showcol_8c}}{}
\item\contentsline{section}{src/ngmx/\hyperlink{g__xrama_8c}{g\-\_\-xrama.\-c} }{\pageref{g__xrama_8c}}{}
\item\contentsline{section}{src/ngmx/\hyperlink{logo_8c}{logo.\-c} }{\pageref{logo_8c}}{}
\item\contentsline{section}{src/ngmx/\hyperlink{logo_8h}{logo.\-h} }{\pageref{logo_8h}}{}
\item\contentsline{section}{src/ngmx/\hyperlink{manager_8c}{manager.\-c} }{\pageref{manager_8c}}{}
\item\contentsline{section}{src/ngmx/\hyperlink{manager_8h}{manager.\-h} }{\pageref{manager_8h}}{}
\item\contentsline{section}{src/ngmx/\hyperlink{molps_8c}{molps.\-c} }{\pageref{molps_8c}}{}
\item\contentsline{section}{src/ngmx/\hyperlink{molps_8h}{molps.\-h} }{\pageref{molps_8h}}{}
\item\contentsline{section}{src/ngmx/\hyperlink{ngmx_8c}{ngmx.\-c} }{\pageref{ngmx_8c}}{}
\item\contentsline{section}{src/ngmx/\hyperlink{nleg_8c}{nleg.\-c} }{\pageref{nleg_8c}}{}
\item\contentsline{section}{src/ngmx/\hyperlink{nleg_8h}{nleg.\-h} }{\pageref{nleg_8h}}{}
\item\contentsline{section}{src/ngmx/\hyperlink{nmol_8c}{nmol.\-c} }{\pageref{nmol_8c}}{}
\item\contentsline{section}{src/ngmx/\hyperlink{nmol_8h}{nmol.\-h} }{\pageref{nmol_8h}}{}
\item\contentsline{section}{src/ngmx/\hyperlink{popup_8c}{popup.\-c} }{\pageref{popup_8c}}{}
\item\contentsline{section}{src/ngmx/\hyperlink{popup_8h}{popup.\-h} }{\pageref{popup_8h}}{}
\item\contentsline{section}{src/ngmx/\hyperlink{pulldown_8c}{pulldown.\-c} }{\pageref{pulldown_8c}}{}
\item\contentsline{section}{src/ngmx/\hyperlink{pulldown_8h}{pulldown.\-h} }{\pageref{pulldown_8h}}{}
\item\contentsline{section}{src/ngmx/\hyperlink{test__ngmx__dialog_8c}{test\-\_\-ngmx\-\_\-dialog.\-c} }{\pageref{test__ngmx__dialog_8c}}{}
\item\contentsline{section}{src/ngmx/\hyperlink{vbox_8c}{vbox.\-c} }{\pageref{vbox_8c}}{}
\item\contentsline{section}{src/ngmx/\hyperlink{x11_8c}{x11.\-c} }{\pageref{x11_8c}}{}
\item\contentsline{section}{src/ngmx/\hyperlink{x11_8h}{x11.\-h} }{\pageref{x11_8h}}{}
\item\contentsline{section}{src/ngmx/\hyperlink{xdlg_8c}{xdlg.\-c} }{\pageref{xdlg_8c}}{}
\item\contentsline{section}{src/ngmx/\hyperlink{xdlg_8h}{xdlg.\-h} }{\pageref{xdlg_8h}}{}
\item\contentsline{section}{src/ngmx/\hyperlink{xdlghi_8c}{xdlghi.\-c} }{\pageref{xdlghi_8c}}{}
\item\contentsline{section}{src/ngmx/\hyperlink{xdlghi_8h}{xdlghi.\-h} }{\pageref{xdlghi_8h}}{}
\item\contentsline{section}{src/ngmx/\hyperlink{xdlgitem_8c}{xdlgitem.\-c} }{\pageref{xdlgitem_8c}}{}
\item\contentsline{section}{src/ngmx/\hyperlink{xdlgitem_8h}{xdlgitem.\-h} }{\pageref{xdlgitem_8h}}{}
\item\contentsline{section}{src/ngmx/\hyperlink{xmb_8c}{xmb.\-c} }{\pageref{xmb_8c}}{}
\item\contentsline{section}{src/ngmx/\hyperlink{xmb_8h}{xmb.\-h} }{\pageref{xmb_8h}}{}
\item\contentsline{section}{src/ngmx/\hyperlink{xstat_8c}{xstat.\-c} }{\pageref{xstat_8c}}{}
\item\contentsline{section}{src/ngmx/\hyperlink{Xstuff_8h}{\-Xstuff.\-h} }{\pageref{Xstuff_8h}}{}
\item\contentsline{section}{src/ngmx/\hyperlink{xutil_8c}{xutil.\-c} }{\pageref{xutil_8c}}{}
\item\contentsline{section}{src/ngmx/\hyperlink{xutil_8h}{xutil.\-h} }{\pageref{xutil_8h}}{}
\item\contentsline{section}{src/tools/\hyperlink{addconf_8c}{addconf.\-c} }{\pageref{addconf_8c}}{}
\item\contentsline{section}{src/tools/\hyperlink{addconf_8h}{addconf.\-h} }{\pageref{addconf_8h}}{}
\item\contentsline{section}{src/tools/\hyperlink{anadih_8c}{anadih.\-c} }{\pageref{anadih_8c}}{}
\item\contentsline{section}{src/tools/\hyperlink{angstat_8h}{angstat.\-h} }{\pageref{angstat_8h}}{}
\item\contentsline{section}{src/tools/\hyperlink{autocorr_8c}{autocorr.\-c} }{\pageref{autocorr_8c}}{}
\item\contentsline{section}{src/tools/\hyperlink{binsearch_8c}{binsearch.\-c} }{\pageref{binsearch_8c}}{}
\item\contentsline{section}{src/tools/\hyperlink{binsearch_8h}{binsearch.\-h} }{\pageref{binsearch_8h}}{}
\item\contentsline{section}{src/tools/\hyperlink{calcpot_8c}{calcpot.\-c} }{\pageref{calcpot_8c}}{}
\item\contentsline{section}{src/tools/\hyperlink{calcpot_8h}{calcpot.\-h} }{\pageref{calcpot_8h}}{}
\item\contentsline{section}{src/tools/\hyperlink{cmat_8c}{cmat.\-c} }{\pageref{cmat_8c}}{}
\item\contentsline{section}{src/tools/\hyperlink{cmat_8h}{cmat.\-h} }{\pageref{cmat_8h}}{}
\item\contentsline{section}{src/tools/\hyperlink{correl_8c}{correl.\-c} }{\pageref{correl_8c}}{}
\item\contentsline{section}{src/tools/\hyperlink{correl_8h}{correl.\-h} }{\pageref{correl_8h}}{}
\item\contentsline{section}{src/tools/\hyperlink{dens__filter_8c}{dens\-\_\-filter.\-c} }{\pageref{dens__filter_8c}}{}
\item\contentsline{section}{src/tools/\hyperlink{dens__filter_8h}{dens\-\_\-filter.\-h} }{\pageref{dens__filter_8h}}{}
\item\contentsline{section}{src/tools/\hyperlink{dlist_8c}{dlist.\-c} }{\pageref{dlist_8c}}{}
\item\contentsline{section}{src/tools/\hyperlink{do__dssp_8c}{do\-\_\-dssp.\-c} }{\pageref{do__dssp_8c}}{}
\item\contentsline{section}{src/tools/\hyperlink{editconf_8c}{editconf.\-c} }{\pageref{editconf_8c}}{}
\item\contentsline{section}{src/tools/\hyperlink{edittop_8c}{edittop.\-c} }{\pageref{edittop_8c}}{}
\item\contentsline{section}{src/tools/\hyperlink{eigensolver_8c}{eigensolver.\-c} }{\pageref{eigensolver_8c}}{}
\item\contentsline{section}{src/tools/\hyperlink{eigensolver_8h}{eigensolver.\-h} }{\pageref{eigensolver_8h}}{}
\item\contentsline{section}{src/tools/\hyperlink{eigio_8c}{eigio.\-c} }{\pageref{eigio_8c}}{}
\item\contentsline{section}{src/tools/\hyperlink{eigio_8h}{eigio.\-h} }{\pageref{eigio_8h}}{}
\item\contentsline{section}{src/tools/\hyperlink{eneconv_8c}{eneconv.\-c} }{\pageref{eneconv_8c}}{}
\item\contentsline{section}{src/tools/\hyperlink{expfit_8c}{expfit.\-c} }{\pageref{expfit_8c}}{}
\item\contentsline{section}{src/tools/\hyperlink{fitahx_8c}{fitahx.\-c} }{\pageref{fitahx_8c}}{}
\item\contentsline{section}{src/tools/\hyperlink{fitahx_8h}{fitahx.\-h} }{\pageref{fitahx_8h}}{}
\item\contentsline{section}{src/tools/\hyperlink{g__anadock_8c}{g\-\_\-anadock.\-c} }{\pageref{g__anadock_8c}}{}
\item\contentsline{section}{src/tools/\hyperlink{g__anaeig_8c}{g\-\_\-anaeig.\-c} }{\pageref{g__anaeig_8c}}{}
\item\contentsline{section}{src/tools/\hyperlink{g__analyze_8c}{g\-\_\-analyze.\-c} }{\pageref{g__analyze_8c}}{}
\item\contentsline{section}{src/tools/\hyperlink{g__angle_8c}{g\-\_\-angle.\-c} }{\pageref{g__angle_8c}}{}
\item\contentsline{section}{src/tools/\hyperlink{g__bar_8c}{g\-\_\-bar.\-c} }{\pageref{g__bar_8c}}{}
\item\contentsline{section}{src/tools/\hyperlink{g__bond_8c}{g\-\_\-bond.\-c} }{\pageref{g__bond_8c}}{}
\item\contentsline{section}{src/tools/\hyperlink{g__bundle_8c}{g\-\_\-bundle.\-c} }{\pageref{g__bundle_8c}}{}
\item\contentsline{section}{src/tools/\hyperlink{g__chi_8c}{g\-\_\-chi.\-c} }{\pageref{g__chi_8c}}{}
\item\contentsline{section}{src/tools/\hyperlink{g__cluster_8c}{g\-\_\-cluster.\-c} }{\pageref{g__cluster_8c}}{}
\item\contentsline{section}{src/tools/\hyperlink{g__clustsize_8c}{g\-\_\-clustsize.\-c} }{\pageref{g__clustsize_8c}}{}
\item\contentsline{section}{src/tools/\hyperlink{g__confrms_8c}{g\-\_\-confrms.\-c} }{\pageref{g__confrms_8c}}{}
\item\contentsline{section}{src/tools/\hyperlink{g__covar_8c}{g\-\_\-covar.\-c} }{\pageref{g__covar_8c}}{}
\item\contentsline{section}{src/tools/\hyperlink{g__current_8c}{g\-\_\-current.\-c} }{\pageref{g__current_8c}}{}
\item\contentsline{section}{src/tools/\hyperlink{g__density_8c}{g\-\_\-density.\-c} }{\pageref{g__density_8c}}{}
\item\contentsline{section}{src/tools/\hyperlink{g__densmap_8c}{g\-\_\-densmap.\-c} }{\pageref{g__densmap_8c}}{}
\item\contentsline{section}{src/tools/\hyperlink{g__densorder_8c}{g\-\_\-densorder.\-c} }{\pageref{g__densorder_8c}}{}
\item\contentsline{section}{src/tools/\hyperlink{g__dielectric_8c}{g\-\_\-dielectric.\-c} }{\pageref{g__dielectric_8c}}{}
\item\contentsline{section}{src/tools/\hyperlink{g__dipoles_8c}{g\-\_\-dipoles.\-c} }{\pageref{g__dipoles_8c}}{}
\item\contentsline{section}{src/tools/\hyperlink{g__disre_8c}{g\-\_\-disre.\-c} }{\pageref{g__disre_8c}}{}
\item\contentsline{section}{src/tools/\hyperlink{g__dist_8c}{g\-\_\-dist.\-c} }{\pageref{g__dist_8c}}{}
\item\contentsline{section}{src/tools/\hyperlink{g__dos_8c}{g\-\_\-dos.\-c} }{\pageref{g__dos_8c}}{}
\item\contentsline{section}{src/tools/\hyperlink{g__dyecoupl_8c}{g\-\_\-dyecoupl.\-c} }{\pageref{g__dyecoupl_8c}}{}
\item\contentsline{section}{src/tools/\hyperlink{g__dyndom_8c}{g\-\_\-dyndom.\-c} }{\pageref{g__dyndom_8c}}{}
\item\contentsline{section}{src/tools/\hyperlink{g__enemat_8c}{g\-\_\-enemat.\-c} }{\pageref{g__enemat_8c}}{}
\item\contentsline{section}{src/tools/\hyperlink{g__energy_8c}{g\-\_\-energy.\-c} }{\pageref{g__energy_8c}}{}
\item\contentsline{section}{src/tools/\hyperlink{g__filter_8c}{g\-\_\-filter.\-c} }{\pageref{g__filter_8c}}{}
\item\contentsline{section}{src/tools/\hyperlink{g__gyrate_8c}{g\-\_\-gyrate.\-c} }{\pageref{g__gyrate_8c}}{}
\item\contentsline{section}{src/tools/\hyperlink{g__h2order_8c}{g\-\_\-h2order.\-c} }{\pageref{g__h2order_8c}}{}
\item\contentsline{section}{src/tools/\hyperlink{g__hbond_8c}{g\-\_\-hbond.\-c} }{\pageref{g__hbond_8c}}{}
\item\contentsline{section}{src/tools/\hyperlink{g__helix_8c}{g\-\_\-helix.\-c} }{\pageref{g__helix_8c}}{}
\item\contentsline{section}{src/tools/\hyperlink{g__helixorient_8c}{g\-\_\-helixorient.\-c} }{\pageref{g__helixorient_8c}}{}
\item\contentsline{section}{src/tools/\hyperlink{g__hydorder_8c}{g\-\_\-hydorder.\-c} }{\pageref{g__hydorder_8c}}{}
\item\contentsline{section}{src/tools/\hyperlink{g__kinetics_8c}{g\-\_\-kinetics.\-c} }{\pageref{g__kinetics_8c}}{}
\item\contentsline{section}{src/tools/\hyperlink{g__lie_8c}{g\-\_\-lie.\-c} }{\pageref{g__lie_8c}}{}
\item\contentsline{section}{src/tools/\hyperlink{g__mdmat_8c}{g\-\_\-mdmat.\-c} }{\pageref{g__mdmat_8c}}{}
\item\contentsline{section}{src/tools/\hyperlink{g__membed_8c}{g\-\_\-membed.\-c} }{\pageref{g__membed_8c}}{}
\item\contentsline{section}{src/tools/\hyperlink{g__mindist_8c}{g\-\_\-mindist.\-c} }{\pageref{g__mindist_8c}}{}
\item\contentsline{section}{src/tools/\hyperlink{g__morph_8c}{g\-\_\-morph.\-c} }{\pageref{g__morph_8c}}{}
\item\contentsline{section}{src/tools/\hyperlink{g__msd_8c}{g\-\_\-msd.\-c} }{\pageref{g__msd_8c}}{}
\item\contentsline{section}{src/tools/\hyperlink{g__nmeig_8c}{g\-\_\-nmeig.\-c} }{\pageref{g__nmeig_8c}}{}
\item\contentsline{section}{src/tools/\hyperlink{g__nmens_8c}{g\-\_\-nmens.\-c} }{\pageref{g__nmens_8c}}{}
\item\contentsline{section}{src/tools/\hyperlink{g__nmtraj_8c}{g\-\_\-nmtraj.\-c} }{\pageref{g__nmtraj_8c}}{}
\item\contentsline{section}{src/tools/\hyperlink{g__options_8c}{g\-\_\-options.\-c} }{\pageref{g__options_8c}}{}
\item\contentsline{section}{src/tools/\hyperlink{g__order_8c}{g\-\_\-order.\-c} }{\pageref{g__order_8c}}{}
\item\contentsline{section}{src/tools/\hyperlink{g__pme__error_8c}{g\-\_\-pme\-\_\-error.\-c} }{\pageref{g__pme__error_8c}}{}
\item\contentsline{section}{src/tools/\hyperlink{g__polystat_8c}{g\-\_\-polystat.\-c} }{\pageref{g__polystat_8c}}{}
\item\contentsline{section}{src/tools/\hyperlink{g__potential_8c}{g\-\_\-potential.\-c} }{\pageref{g__potential_8c}}{}
\item\contentsline{section}{src/tools/\hyperlink{g__principal_8c}{g\-\_\-principal.\-c} }{\pageref{g__principal_8c}}{}
\item\contentsline{section}{src/tools/\hyperlink{g__rama_8c}{g\-\_\-rama.\-c} }{\pageref{g__rama_8c}}{}
\item\contentsline{section}{src/tools/\hyperlink{g__rdf_8c}{g\-\_\-rdf.\-c} }{\pageref{g__rdf_8c}}{}
\item\contentsline{section}{src/tools/\hyperlink{g__rms_8c}{g\-\_\-rms.\-c} }{\pageref{g__rms_8c}}{}
\item\contentsline{section}{src/tools/\hyperlink{g__rmsdist_8c}{g\-\_\-rmsdist.\-c} }{\pageref{g__rmsdist_8c}}{}
\item\contentsline{section}{src/tools/\hyperlink{g__rmsf_8c}{g\-\_\-rmsf.\-c} }{\pageref{g__rmsf_8c}}{}
\item\contentsline{section}{src/tools/\hyperlink{g__rotacf_8c}{g\-\_\-rotacf.\-c} }{\pageref{g__rotacf_8c}}{}
\item\contentsline{section}{src/tools/\hyperlink{g__rotmat_8c}{g\-\_\-rotmat.\-c} }{\pageref{g__rotmat_8c}}{}
\item\contentsline{section}{src/tools/\hyperlink{g__saltbr_8c}{g\-\_\-saltbr.\-c} }{\pageref{g__saltbr_8c}}{}
\item\contentsline{section}{src/tools/\hyperlink{g__sans_8c}{g\-\_\-sans.\-c} }{\pageref{g__sans_8c}}{}
\item\contentsline{section}{src/tools/\hyperlink{g__sas_8c}{g\-\_\-sas.\-c} }{\pageref{g__sas_8c}}{}
\item\contentsline{section}{src/tools/\hyperlink{g__select_8c}{g\-\_\-select.\-c} \\*\-Utility program for writing out basic data for selections }{\pageref{g__select_8c}}{}
\item\contentsline{section}{src/tools/\hyperlink{g__sgangle_8c}{g\-\_\-sgangle.\-c} }{\pageref{g__sgangle_8c}}{}
\item\contentsline{section}{src/tools/\hyperlink{g__sham_8c}{g\-\_\-sham.\-c} }{\pageref{g__sham_8c}}{}
\item\contentsline{section}{src/tools/\hyperlink{g__sigeps_8c}{g\-\_\-sigeps.\-c} }{\pageref{g__sigeps_8c}}{}
\item\contentsline{section}{src/tools/\hyperlink{g__sorient_8c}{g\-\_\-sorient.\-c} }{\pageref{g__sorient_8c}}{}
\item\contentsline{section}{src/tools/\hyperlink{g__spatial_8c}{g\-\_\-spatial.\-c} }{\pageref{g__spatial_8c}}{}
\item\contentsline{section}{src/tools/\hyperlink{g__spol_8c}{g\-\_\-spol.\-c} }{\pageref{g__spol_8c}}{}
\item\contentsline{section}{src/tools/\hyperlink{g__tcaf_8c}{g\-\_\-tcaf.\-c} }{\pageref{g__tcaf_8c}}{}
\item\contentsline{section}{src/tools/\hyperlink{g__traj_8c}{g\-\_\-traj.\-c} }{\pageref{g__traj_8c}}{}
\item\contentsline{section}{src/tools/\hyperlink{g__tune__pme_8c}{g\-\_\-tune\-\_\-pme.\-c} }{\pageref{g__tune__pme_8c}}{}
\item\contentsline{section}{src/tools/\hyperlink{g__vanhove_8c}{g\-\_\-vanhove.\-c} }{\pageref{g__vanhove_8c}}{}
\item\contentsline{section}{src/tools/\hyperlink{g__velacc_8c}{g\-\_\-velacc.\-c} }{\pageref{g__velacc_8c}}{}
\item\contentsline{section}{src/tools/\hyperlink{g__wham_8c}{g\-\_\-wham.\-c} }{\pageref{g__wham_8c}}{}
\item\contentsline{section}{src/tools/\hyperlink{g__wheel_8c}{g\-\_\-wheel.\-c} }{\pageref{g__wheel_8c}}{}
\item\contentsline{section}{src/tools/\hyperlink{geminate_8c}{geminate.\-c} }{\pageref{geminate_8c}}{}
\item\contentsline{section}{src/tools/\hyperlink{geminate_8h}{geminate.\-h} }{\pageref{geminate_8h}}{}
\item\contentsline{section}{src/tools/\hyperlink{genbox_8c}{genbox.\-c} }{\pageref{genbox_8c}}{}
\item\contentsline{section}{src/tools/\hyperlink{genconf_8c}{genconf.\-c} }{\pageref{genconf_8c}}{}
\item\contentsline{section}{src/tools/\hyperlink{genion_8c}{genion.\-c} }{\pageref{genion_8c}}{}
\item\contentsline{section}{src/tools/\hyperlink{genrestr_8c}{genrestr.\-c} }{\pageref{genrestr_8c}}{}
\item\contentsline{section}{src/tools/\hyperlink{gmx__anadock_8c}{gmx\-\_\-anadock.\-c} }{\pageref{gmx__anadock_8c}}{}
\item\contentsline{section}{src/tools/\hyperlink{gmx__anaeig_8c}{gmx\-\_\-anaeig.\-c} }{\pageref{gmx__anaeig_8c}}{}
\item\contentsline{section}{src/tools/\hyperlink{gmx__analyze_8c}{gmx\-\_\-analyze.\-c} }{\pageref{gmx__analyze_8c}}{}
\item\contentsline{section}{src/tools/\hyperlink{gmx__angle_8c}{gmx\-\_\-angle.\-c} }{\pageref{gmx__angle_8c}}{}
\item\contentsline{section}{src/tools/\hyperlink{gmx__bar_8c}{gmx\-\_\-bar.\-c} }{\pageref{gmx__bar_8c}}{}
\item\contentsline{section}{src/tools/\hyperlink{gmx__bond_8c}{gmx\-\_\-bond.\-c} }{\pageref{gmx__bond_8c}}{}
\item\contentsline{section}{src/tools/\hyperlink{gmx__bundle_8c}{gmx\-\_\-bundle.\-c} }{\pageref{gmx__bundle_8c}}{}
\item\contentsline{section}{src/tools/\hyperlink{gmx__chi_8c}{gmx\-\_\-chi.\-c} }{\pageref{gmx__chi_8c}}{}
\item\contentsline{section}{src/tools/\hyperlink{gmx__cluster_8c}{gmx\-\_\-cluster.\-c} }{\pageref{gmx__cluster_8c}}{}
\item\contentsline{section}{src/tools/\hyperlink{gmx__clustsize_8c}{gmx\-\_\-clustsize.\-c} }{\pageref{gmx__clustsize_8c}}{}
\item\contentsline{section}{src/tools/\hyperlink{gmx__confrms_8c}{gmx\-\_\-confrms.\-c} }{\pageref{gmx__confrms_8c}}{}
\item\contentsline{section}{src/tools/\hyperlink{gmx__covar_8c}{gmx\-\_\-covar.\-c} }{\pageref{gmx__covar_8c}}{}
\item\contentsline{section}{src/tools/\hyperlink{gmx__current_8c}{gmx\-\_\-current.\-c} }{\pageref{gmx__current_8c}}{}
\item\contentsline{section}{src/tools/\hyperlink{gmx__density_8c}{gmx\-\_\-density.\-c} }{\pageref{gmx__density_8c}}{}
\item\contentsline{section}{src/tools/\hyperlink{gmx__densmap_8c}{gmx\-\_\-densmap.\-c} }{\pageref{gmx__densmap_8c}}{}
\item\contentsline{section}{src/tools/\hyperlink{gmx__densorder_8c}{gmx\-\_\-densorder.\-c} }{\pageref{gmx__densorder_8c}}{}
\item\contentsline{section}{src/tools/\hyperlink{gmx__dielectric_8c}{gmx\-\_\-dielectric.\-c} }{\pageref{gmx__dielectric_8c}}{}
\item\contentsline{section}{src/tools/\hyperlink{gmx__dipoles_8c}{gmx\-\_\-dipoles.\-c} }{\pageref{gmx__dipoles_8c}}{}
\item\contentsline{section}{src/tools/\hyperlink{gmx__disre_8c}{gmx\-\_\-disre.\-c} }{\pageref{gmx__disre_8c}}{}
\item\contentsline{section}{src/tools/\hyperlink{gmx__dist_8c}{gmx\-\_\-dist.\-c} }{\pageref{gmx__dist_8c}}{}
\item\contentsline{section}{src/tools/\hyperlink{gmx__do__dssp_8c}{gmx\-\_\-do\-\_\-dssp.\-c} }{\pageref{gmx__do__dssp_8c}}{}
\item\contentsline{section}{src/tools/\hyperlink{gmx__dos_8c}{gmx\-\_\-dos.\-c} }{\pageref{gmx__dos_8c}}{}
\item\contentsline{section}{src/tools/\hyperlink{gmx__dyecoupl_8c}{gmx\-\_\-dyecoupl.\-c} }{\pageref{gmx__dyecoupl_8c}}{}
\item\contentsline{section}{src/tools/\hyperlink{gmx__dyndom_8c}{gmx\-\_\-dyndom.\-c} }{\pageref{gmx__dyndom_8c}}{}
\item\contentsline{section}{src/tools/\hyperlink{gmx__editconf_8c}{gmx\-\_\-editconf.\-c} }{\pageref{gmx__editconf_8c}}{}
\item\contentsline{section}{src/tools/\hyperlink{gmx__eneconv_8c}{gmx\-\_\-eneconv.\-c} }{\pageref{gmx__eneconv_8c}}{}
\item\contentsline{section}{src/tools/\hyperlink{gmx__enemat_8c}{gmx\-\_\-enemat.\-c} }{\pageref{gmx__enemat_8c}}{}
\item\contentsline{section}{src/tools/\hyperlink{gmx__energy_8c}{gmx\-\_\-energy.\-c} }{\pageref{gmx__energy_8c}}{}
\item\contentsline{section}{src/tools/\hyperlink{gmx__filter_8c}{gmx\-\_\-filter.\-c} }{\pageref{gmx__filter_8c}}{}
\item\contentsline{section}{src/tools/\hyperlink{gmx__genbox_8c}{gmx\-\_\-genbox.\-c} }{\pageref{gmx__genbox_8c}}{}
\item\contentsline{section}{src/tools/\hyperlink{gmx__genconf_8c}{gmx\-\_\-genconf.\-c} }{\pageref{gmx__genconf_8c}}{}
\item\contentsline{section}{src/tools/\hyperlink{gmx__genion_8c}{gmx\-\_\-genion.\-c} }{\pageref{gmx__genion_8c}}{}
\item\contentsline{section}{src/tools/\hyperlink{gmx__genpr_8c}{gmx\-\_\-genpr.\-c} }{\pageref{gmx__genpr_8c}}{}
\item\contentsline{section}{src/tools/\hyperlink{gmx__gyrate_8c}{gmx\-\_\-gyrate.\-c} }{\pageref{gmx__gyrate_8c}}{}
\item\contentsline{section}{src/tools/\hyperlink{gmx__h2order_8c}{gmx\-\_\-h2order.\-c} }{\pageref{gmx__h2order_8c}}{}
\item\contentsline{section}{src/tools/\hyperlink{gmx__hbond_8c}{gmx\-\_\-hbond.\-c} }{\pageref{gmx__hbond_8c}}{}
\item\contentsline{section}{src/tools/\hyperlink{gmx__helix_8c}{gmx\-\_\-helix.\-c} }{\pageref{gmx__helix_8c}}{}
\item\contentsline{section}{src/tools/\hyperlink{gmx__helixorient_8c}{gmx\-\_\-helixorient.\-c} }{\pageref{gmx__helixorient_8c}}{}
\item\contentsline{section}{src/tools/\hyperlink{gmx__hydorder_8c}{gmx\-\_\-hydorder.\-c} }{\pageref{gmx__hydorder_8c}}{}
\item\contentsline{section}{src/tools/\hyperlink{gmx__kinetics_8c}{gmx\-\_\-kinetics.\-c} }{\pageref{gmx__kinetics_8c}}{}
\item\contentsline{section}{src/tools/\hyperlink{gmx__lie_8c}{gmx\-\_\-lie.\-c} }{\pageref{gmx__lie_8c}}{}
\item\contentsline{section}{src/tools/\hyperlink{gmx__make__edi_8c}{gmx\-\_\-make\-\_\-edi.\-c} }{\pageref{gmx__make__edi_8c}}{}
\item\contentsline{section}{src/tools/\hyperlink{gmx__make__ndx_8c}{gmx\-\_\-make\-\_\-ndx.\-c} }{\pageref{gmx__make__ndx_8c}}{}
\item\contentsline{section}{src/tools/\hyperlink{gmx__mdmat_8c}{gmx\-\_\-mdmat.\-c} }{\pageref{gmx__mdmat_8c}}{}
\item\contentsline{section}{src/tools/\hyperlink{gmx__membed_8c}{gmx\-\_\-membed.\-c} }{\pageref{gmx__membed_8c}}{}
\item\contentsline{section}{src/tools/\hyperlink{gmx__mindist_8c}{gmx\-\_\-mindist.\-c} }{\pageref{gmx__mindist_8c}}{}
\item\contentsline{section}{src/tools/\hyperlink{gmx__mk__angndx_8c}{gmx\-\_\-mk\-\_\-angndx.\-c} }{\pageref{gmx__mk__angndx_8c}}{}
\item\contentsline{section}{src/tools/\hyperlink{gmx__morph_8c}{gmx\-\_\-morph.\-c} }{\pageref{gmx__morph_8c}}{}
\item\contentsline{section}{src/tools/\hyperlink{gmx__msd_8c}{gmx\-\_\-msd.\-c} }{\pageref{gmx__msd_8c}}{}
\item\contentsline{section}{src/tools/\hyperlink{gmx__nmeig_8c}{gmx\-\_\-nmeig.\-c} }{\pageref{gmx__nmeig_8c}}{}
\item\contentsline{section}{src/tools/\hyperlink{gmx__nmens_8c}{gmx\-\_\-nmens.\-c} }{\pageref{gmx__nmens_8c}}{}
\item\contentsline{section}{src/tools/\hyperlink{gmx__nmtraj_8c}{gmx\-\_\-nmtraj.\-c} }{\pageref{gmx__nmtraj_8c}}{}
\item\contentsline{section}{src/tools/\hyperlink{gmx__options_8c}{gmx\-\_\-options.\-c} }{\pageref{gmx__options_8c}}{}
\item\contentsline{section}{src/tools/\hyperlink{gmx__order_8c}{gmx\-\_\-order.\-c} }{\pageref{gmx__order_8c}}{}
\item\contentsline{section}{src/tools/\hyperlink{gmx__pme__error_8c}{gmx\-\_\-pme\-\_\-error.\-c} }{\pageref{gmx__pme__error_8c}}{}
\item\contentsline{section}{src/tools/\hyperlink{gmx__polystat_8c}{gmx\-\_\-polystat.\-c} }{\pageref{gmx__polystat_8c}}{}
\item\contentsline{section}{src/tools/\hyperlink{gmx__potential_8c}{gmx\-\_\-potential.\-c} }{\pageref{gmx__potential_8c}}{}
\item\contentsline{section}{src/tools/\hyperlink{gmx__principal_8c}{gmx\-\_\-principal.\-c} }{\pageref{gmx__principal_8c}}{}
\item\contentsline{section}{src/tools/\hyperlink{gmx__rama_8c}{gmx\-\_\-rama.\-c} }{\pageref{gmx__rama_8c}}{}
\item\contentsline{section}{src/tools/\hyperlink{gmx__rdf_8c}{gmx\-\_\-rdf.\-c} }{\pageref{gmx__rdf_8c}}{}
\item\contentsline{section}{src/tools/\hyperlink{gmx__rms_8c}{gmx\-\_\-rms.\-c} }{\pageref{gmx__rms_8c}}{}
\item\contentsline{section}{src/tools/\hyperlink{gmx__rmsdist_8c}{gmx\-\_\-rmsdist.\-c} }{\pageref{gmx__rmsdist_8c}}{}
\item\contentsline{section}{src/tools/\hyperlink{gmx__rmsf_8c}{gmx\-\_\-rmsf.\-c} }{\pageref{gmx__rmsf_8c}}{}
\item\contentsline{section}{src/tools/\hyperlink{gmx__rotacf_8c}{gmx\-\_\-rotacf.\-c} }{\pageref{gmx__rotacf_8c}}{}
\item\contentsline{section}{src/tools/\hyperlink{gmx__rotmat_8c}{gmx\-\_\-rotmat.\-c} }{\pageref{gmx__rotmat_8c}}{}
\item\contentsline{section}{src/tools/\hyperlink{gmx__saltbr_8c}{gmx\-\_\-saltbr.\-c} }{\pageref{gmx__saltbr_8c}}{}
\item\contentsline{section}{src/tools/\hyperlink{gmx__sans_8c}{gmx\-\_\-sans.\-c} }{\pageref{gmx__sans_8c}}{}
\item\contentsline{section}{src/tools/\hyperlink{gmx__sas_8c}{gmx\-\_\-sas.\-c} }{\pageref{gmx__sas_8c}}{}
\item\contentsline{section}{src/tools/\hyperlink{gmx__select_8c}{gmx\-\_\-select.\-c} \\*\-Utility program for writing out basic data for selections }{\pageref{gmx__select_8c}}{}
\item\contentsline{section}{src/tools/\hyperlink{gmx__sgangle_8c}{gmx\-\_\-sgangle.\-c} }{\pageref{gmx__sgangle_8c}}{}
\item\contentsline{section}{src/tools/\hyperlink{gmx__sham_8c}{gmx\-\_\-sham.\-c} }{\pageref{gmx__sham_8c}}{}
\item\contentsline{section}{src/tools/\hyperlink{gmx__sigeps_8c}{gmx\-\_\-sigeps.\-c} }{\pageref{gmx__sigeps_8c}}{}
\item\contentsline{section}{src/tools/\hyperlink{gmx__sorient_8c}{gmx\-\_\-sorient.\-c} }{\pageref{gmx__sorient_8c}}{}
\item\contentsline{section}{src/tools/\hyperlink{gmx__spatial_8c}{gmx\-\_\-spatial.\-c} }{\pageref{gmx__spatial_8c}}{}
\item\contentsline{section}{src/tools/\hyperlink{gmx__spol_8c}{gmx\-\_\-spol.\-c} }{\pageref{gmx__spol_8c}}{}
\item\contentsline{section}{src/tools/\hyperlink{gmx__tcaf_8c}{gmx\-\_\-tcaf.\-c} }{\pageref{gmx__tcaf_8c}}{}
\item\contentsline{section}{src/tools/\hyperlink{gmx__traj_8c}{gmx\-\_\-traj.\-c} }{\pageref{gmx__traj_8c}}{}
\item\contentsline{section}{src/tools/\hyperlink{gmx__trjcat_8c}{gmx\-\_\-trjcat.\-c} }{\pageref{gmx__trjcat_8c}}{}
\item\contentsline{section}{src/tools/\hyperlink{gmx__trjconv_8c}{gmx\-\_\-trjconv.\-c} }{\pageref{gmx__trjconv_8c}}{}
\item\contentsline{section}{src/tools/\hyperlink{gmx__trjorder_8c}{gmx\-\_\-trjorder.\-c} }{\pageref{gmx__trjorder_8c}}{}
\item\contentsline{section}{src/tools/\hyperlink{gmx__tune__pme_8c}{gmx\-\_\-tune\-\_\-pme.\-c} }{\pageref{gmx__tune__pme_8c}}{}
\item\contentsline{section}{src/tools/\hyperlink{gmx__vanhove_8c}{gmx\-\_\-vanhove.\-c} }{\pageref{gmx__vanhove_8c}}{}
\item\contentsline{section}{src/tools/\hyperlink{gmx__velacc_8c}{gmx\-\_\-velacc.\-c} }{\pageref{gmx__velacc_8c}}{}
\item\contentsline{section}{src/tools/\hyperlink{gmx__wham_8c}{gmx\-\_\-wham.\-c} }{\pageref{gmx__wham_8c}}{}
\item\contentsline{section}{src/tools/\hyperlink{gmx__wheel_8c}{gmx\-\_\-wheel.\-c} }{\pageref{gmx__wheel_8c}}{}
\item\contentsline{section}{src/tools/\hyperlink{gmx__xpm2ps_8c}{gmx\-\_\-xpm2ps.\-c} }{\pageref{gmx__xpm2ps_8c}}{}
\item\contentsline{section}{src/tools/\hyperlink{hxprops_8c}{hxprops.\-c} }{\pageref{hxprops_8c}}{}
\item\contentsline{section}{src/tools/\hyperlink{hxprops_8h}{hxprops.\-h} }{\pageref{hxprops_8h}}{}
\item\contentsline{section}{src/tools/\hyperlink{interf_8h}{interf.\-h} }{\pageref{interf_8h}}{}
\item\contentsline{section}{src/tools/\hyperlink{levenmar_8c}{levenmar.\-c} }{\pageref{levenmar_8c}}{}
\item\contentsline{section}{src/tools/\hyperlink{make__edi_8c}{make\-\_\-edi.\-c} }{\pageref{make__edi_8c}}{}
\item\contentsline{section}{src/tools/\hyperlink{make__ndx_8c}{make\-\_\-ndx.\-c} }{\pageref{make__ndx_8c}}{}
\item\contentsline{section}{src/tools/\hyperlink{mk__angndx_8c}{mk\-\_\-angndx.\-c} }{\pageref{mk__angndx_8c}}{}
\item\contentsline{section}{src/tools/\hyperlink{nsc_8c}{nsc.\-c} }{\pageref{nsc_8c}}{}
\item\contentsline{section}{src/tools/\hyperlink{nsc_8h}{nsc.\-h} }{\pageref{nsc_8h}}{}
\item\contentsline{section}{src/tools/\hyperlink{nsfactor_8c}{nsfactor.\-c} }{\pageref{nsfactor_8c}}{}
\item\contentsline{section}{src/tools/\hyperlink{nsfactor_8h}{nsfactor.\-h} }{\pageref{nsfactor_8h}}{}
\item\contentsline{section}{src/tools/\hyperlink{polynomials_8c}{polynomials.\-c} }{\pageref{polynomials_8c}}{}
\item\contentsline{section}{src/tools/\hyperlink{powerspect_8c}{powerspect.\-c} }{\pageref{powerspect_8c}}{}
\item\contentsline{section}{src/tools/\hyperlink{powerspect_8h}{powerspect.\-h} }{\pageref{powerspect_8h}}{}
\item\contentsline{section}{src/tools/\hyperlink{pp2shift_8c}{pp2shift.\-c} }{\pageref{pp2shift_8c}}{}
\item\contentsline{section}{src/tools/\hyperlink{pp2shift_8h}{pp2shift.\-h} }{\pageref{pp2shift_8h}}{}
\item\contentsline{section}{src/tools/\hyperlink{trjcat_8c}{trjcat.\-c} }{\pageref{trjcat_8c}}{}
\item\contentsline{section}{src/tools/\hyperlink{trjconv_8c}{trjconv.\-c} }{\pageref{trjconv_8c}}{}
\item\contentsline{section}{src/tools/\hyperlink{trjorder_8c}{trjorder.\-c} }{\pageref{trjorder_8c}}{}
\item\contentsline{section}{src/tools/\hyperlink{xpm2ps_8c}{xpm2ps.\-c} }{\pageref{xpm2ps_8c}}{}
\end{DoxyCompactList}


\section{Run Parameters\swapindexquiet{run}{parameter}}
% TODO Check this is up to date when things stabilize
The descriptions of {\tt .mdp} parameters can be found at
\url{http://manual.gromacs.org/current/mdp-options.html}
or in your installation at {\tt share/gromacs/html/mdp-options.html}

% LocalWords:  online html GMXRC MANPATH grompp xdr NFS
